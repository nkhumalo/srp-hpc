\documentclass[pra]{revtex4-1}

\usepackage{hyperref}
\usepackage[version=3]{mhchem}

\begin{document}
\title{Reply to "Comment on 'Generalization of the Kohn-Sham system that can
       represent arbitrary one-electron density matrices'"}
\author{Hubertus J. J. van Dam}
\affiliation{Brookhaven National Laboratory, Upton, NY 11973-5000}
\date{February 20, 2022}

\begin{abstract}
Our paper [Phys. Rev. A {\bf 93}, 052512 (2016)], proposing a novel form of
single configuration wave function that admits non-idempotent 1-electron
density matrices, has recently received a Comment [Phys. Rev. A {\bf 96},
046501 (2017)] suggesting a number of flaws:
\begin{enumerate}
\item The form of the 1-electron density matrix that we proposed is deemed
      invalid;
\item None of the currently known functionals are given in terms of the
      1-electron density matrix but known only in the basis where the
      density matrix is diagonal;
\item In NOFT the energy is not invariant with respect to unitary
      transformations of the orbitals;
\item The M{\"u}ller functional we used suffers from serious shortcomings;
\item In NOFT the detachment energies should be obtained from the extended
      Koopmans theorem.
\end{enumerate}
Below we will address these criticisms in sequence.
\end{abstract}

\maketitle

\section{Point 1}

In our paper~\cite{vanDam:2016} we proposed a single configuration wavefunction
that can generate the exact 1-electron density maxtrix (1RDM) in the believe
that this is a novel way to view the wavefunction in quantum mechanics.
We showed that using the usual anti-symmetry for the permutation of 1-electron
states, but employing an extended formulation of the 1-electron probability
density, was sufficient to obtain correct 1-electron density matrices
according to our Eq.(16). 

Piris and Pernal [Phys. Rev. A {\bf 96}, 046501 (2017)] claim that my Eq.(16) is
a mistake and invalid, instead they suggest that their Eq.(5) should be used.

Recently, it has been brought to our attention that the form of the wavefunction
we proposed is actually not new. In fact our formulation of the 1-particle 
density matrix was first proposed by Araki and Woods in 1963~\cite{Araki_1963},
see Eq.(???) for bosons. Later it formed the basis for a finite temperature
quantum field theory by the name of Thermo Field Dynamics (TFD) developed by
Takahashi and Umezawa in 1975~\cite{TAKAHASHI_1975} (the original publication
is no longer availabe but a more recent reprint is~\cite{TAKAHASHI_1996}).
A report by Das~\cite{das2000topics} matches our approach even more closely in
that Eq.(44) is almost identical to our Eq.(16).


\bibliographystyle{unsrt}
\bibliography{WFN1C,WFN1C_alt}
\end{document}
