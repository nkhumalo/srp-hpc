\documentclass[pra]{revtex4-1}

\usepackage{hyperref}
\usepackage[version=3]{mhchem}
\usepackage{xcolor}

\DeclareMathOperator{\tr}{Tr}
\newcommand{\eria}[4]{\left\langle #1^\alpha #2^\beta \right.\left| #3^\alpha #4^\beta \right\rangle}
\newcommand{\eris}[4]{\left\langle #1^\sigma #2^{\sigma'} \right.\left| #3^\sigma #4^{\sigma'} \right\rangle}
\newcommand{\reria}[4]{\left( #1^\alpha #3^\alpha \right.\left| #2^\beta #4^\beta \right)}

\begin{document}

\title{Electron correlation in thermo field dynamics based effective one-electron models}
\author{Hubertus J. J. van Dam}
\affiliation{Brookhaven National Laboratory, Upton, NY 11973-5000}
\date{October 27, 2019}

\begin{abstract}
\end{abstract}

\maketitle

\section{Introduction}

In a recent paper it was shown that it is possible to formulate a 
single-configuration wave function that can generate arbitrary one-electron
density matrices~\cite{van_Dam_2016}. Since the publication of that paper
three developments have taken place:
\begin{itemize}
\item The paper has been heavily criticized by Piris and Pernal~\cite{Piris_2017}.
\item The author has learned that the approach of~\cite{van_Dam_2016} is
      essentially a form of Thermo Field
      Dynamics~\cite{TAKAHASHI_1996, Umezawa:1982, das2000topics}
      but applied only to the 
      one-electron density matrix. Thermo Field Dynamics, however, 
      is a wave function method and therefore generates an N-representable
      energy expression consisting of 1- and 2-electron density matrix terms.
\item Combining elements from Thermo Field Dynamics and adding a natural
      orbital functional a correlated electronic structure method has been
      constructed.
\end{itemize}
As the paper by van Dam~\cite{van_Dam_2016} as well as the criticism thereof by
Piris and Pernal~\cite{Piris_2017} demonstrated a clear unfamiliarity with the
field theory of Thermo Field Dynamics it would seem reasonable to provide some
context. 

That paper sought to deploy this wave function
in the context of density functional theory (DFT) or reduced density matrix
functional theory (RDMFT). In particular for RDMFT an energy functional in
terms of the one-electron density matrix is required. A number of such 
functionals have been proposed~\cite{M_ller_1984,Goedecker_1998,Cs_nyi_2000,
BUIJSE_2002,Mazziotti_2001,Gritsenko_2005,Marques_2008,Piris_2012} but thus far
there does not seem to be a clear favorite. In addition in a recent publication
Wang et al.~\cite{Wang_2015} argued that there are multiple two-electron density
matrices that produce different energies but project to the same one-electron
density matrix. This raises questions about the feasibility of formulating
practical one-electron density matrix functionals. It would seem that in any
case some information about the two-electron density matrix is required as was
also noted by Wang et al.~\cite{Wang_2015}. This situation suggested taking a
closer look at the density matrices generated by the wave function proposed in
our previous paper. 

\section{Proof that the exact energy is an orbital functional}

Given that we can proof that the exact energy is a functional of the electron density,
and also that it is an functional of the one-electron density matrix, it should be
possible to also prove that it is a functional of the orbitals. This results from the
fact that the orbitals feed into the construction of the density matrix which in turn
feeds into the construction of the density. 

The idea behind presenting such a proof is that although the theorems at all three
levels are valid the requirement imposed on the associated functional is different.
For the density functional we have found that the exact energy is practice very 
difficult to attain. In fact going from the Thomas-Fermi approach to Kohn-Sham the 
density matrix was inserted and later with hybrid functionals (think also about Hubbard
U) the functionals that are most successful today are already in fact orbital
functionals. For reduced density matrix functional theory we have a similar problem in
that the reconstruction approaches to two-electron density matrices do not really work.
The hope is that by taking one step further back we can use orbitals which have even
more information about the system and maybe it is possible to formulate accurate 
functionals starting from the level.

\section{Arbitrary one-electron and N-representable two-electron density
         matrices from a single-configuration wave function}

In a recent paper~\cite{van_Dam_2016} we proposed a new form of single-configuration
wave function. We showed that this wave function can generate any arbitrary
N-representable one-electron density matrix even non-idem-potent ones. Here
it is shown that the two-electron density matrix generated from this wave
function is also N-representable, as should be expected. 

The wave function proposed in the previous paper is~\cite{van_Dam_2016}
\begin{equation}
   \Psi(r_1,\ldots,r_{n_e})
   = \left|G_1(r_1)G_2(r_2)\ldots G_{n_e}(r_{n_e})\right|
   \label{Eq:wavefunction}
\end{equation}
Here the functions $G$ are generalized orbitals expressed in terms of natural
orbitals $N$ and correlation functions $C$ as
\begin{eqnarray}
   \left|G_s(r)\right\rangle
   &=& \sum_{a,i=1}^{n_b} N_{ai}C_{is}\left|\chi_a(r)\right\rangle
   \label{Eq:genorb}
\end{eqnarray}
The natural orbitals form an orthonormal set as do the correlation functions.
As the correlation functions are expressed in terms of the natural orbitals
they are expressed in an orthonormal basis. Hence the matric $C$ is a
unitary matrix.

Furthermore we postulated that the one-electron density matrix be given by
\begin{eqnarray}
  \sum_{a,b=1}^{n_b}\left|\chi_a(r_1)\right\rangle D_{ab}
                    \left\langle\chi_b(r'_1)\right|
  &=& n_e \int \Psi(r_1,r_2,\ldots,r_{n_e})\Psi^*(r'_1,r_2,\ldots,r_{n_e})
      \mathrm{d}r_2 \ldots \mathrm{d}r_{n_e} \\
  D_{ab}
  &=& \sum_{i=1}^{n_b}\sum_{s=1}^{n_e} N_{ai}C_{is}C^*_{is}N^*_{bi}
      \label{Eq:1densmat}
\end{eqnarray}

From this wave function the two-electron density matrix can be obtained as
\begin{eqnarray}
  \sum_{a,b,c,d=1}^{n_b}\left|\chi_a(r_1)\chi_b(r_2)\right\rangle
                    \Gamma_{abcd}
                    \left\langle\chi_c(r'_1)\chi_d(r'_2)\right|
  &=& \frac{n_e(n_e-1)}{2}
      \int \Psi(r_1,r_2,\ldots,r_{n_e})\Psi^*(r'_1,r'_2,\ldots,r_{n_e})
      \mathrm{d}r_3 \ldots \mathrm{d}r_{n_e} \\
  \Gamma_{abcd}
  &=& \frac{1}{2}\left[
      \sum_{s,t=1}^{n_e}
      \left(\sum_{i=1}^{n_b} N_{ai}C_{is}C^*_{is}N^*_{ci}\right)
      \left(\sum_{j=1}^{n_b} N_{bj}C_{jt}C^*_{jt}N^*_{dj}\right)
      \right.
      \nonumber \\
  &&- \left.
      \sum_{s,t=1}^{n_e}
      \left(\sum_{i=1}^{n_b} N_{ai}C_{is}C^*_{it}N^*_{ci}\right)
      \left(\sum_{j=1}^{n_b} N_{bj}C_{jt}C^*_{js}N^*_{dj}\right)
      \right]
      \label{Eq:2densmat}
\end{eqnarray}
\begin{eqnarray}
  & \sum_{a,b,c,d=1}^{n_b}&\left|\chi_a(r_1)\chi_b(r_2)\right\rangle
                    \Gamma_{abcd}
                    \left\langle\chi_c(r'_1)\chi_d(r'_2)\right| \nonumber\\
  &=& \frac{n_e(n_e-1)}{2}
      \int \Psi(r_1,r_2,\ldots,r_{n_e})\Psi^*(r'_1,r'_2,\ldots,r_{n_e})
      \mathrm{d}r_3 \ldots \mathrm{d}r_{n_e} \\
  &=& \frac{n_e(n_e-1)}{2}
      \sum_{a,b,c,d=1}^{n_b}
      \left|\chi_a(r'_1)\chi_b(r'_2)\right\rangle
      \left\langle\chi_c(r"_1)\chi_d(r"_2)\right| \nonumber\\
  &&  \int 
      \left\langle\chi_a(r_1)\chi_b(r_2)\right|
      \Psi(r_1,r_2,\ldots,r_{n_e})\Psi^*(r_1,r_2,\ldots,r_{n_e})
      \left|\chi_c(r_1)\chi_d(r_2)\right\rangle
      \mathrm{d}r_1 \ldots \mathrm{d}r_{n_e} \\
  \Gamma_{abcd}
  &=& \frac{1}{4}\left\{
      \left[
      \sum_{s,t=1}^{n_e}
      \left(\sum_{i=1}^{n_b} N_{ai}C_{is}C^*_{is}N^*_{ci}\right)
      \left(\sum_{j=1}^{n_b} N_{bj}C_{jt}C^*_{jt}N^*_{dj}\right)
      \right.\right.
      \nonumber \\
  &&- \left.
      \sum_{s,t=1}^{n_e}
      \left(\sum_{i=1}^{n_b} N_{ai}C_{is}C^*_{it}N^*_{ci}\right)
      \left(\sum_{j=1}^{n_b} N_{bj}C_{jt}C^*_{js}N^*_{dj}\right)
      \right] \nonumber \\
  &+& \left[ 
      \sum_{s,t=1}^{n_e}
      \left(\sum_{i=1}^{n_b} N_{bi}C_{is}C^*_{is}N^*_{di}\right)
      \left(\sum_{j=1}^{n_b} N_{aj}C_{jt}C^*_{jt}N^*_{cj}\right)
      \right.
      \nonumber \\
  &&- \left.
      \sum_{s,t=1}^{n_e}
      \left(\sum_{i=1}^{n_b} N_{bi}C_{is}C^*_{it}N^*_{di}\right)
      \left(\sum_{j=1}^{n_b} N_{aj}C_{jt}C^*_{js}N^*_{cj}\right)
      \right] \nonumber \\
  &-& \left[
      \sum_{s,t=1}^{n_e}
      \left(\sum_{i=1}^{n_b} N_{ai}C_{is}C^*_{is}N^*_{di}\right)
      \left(\sum_{j=1}^{n_b} N_{bj}C_{jt}C^*_{jt}N^*_{cj}\right)
      \right.
      \nonumber \\
  &&- \left.
      \sum_{s,t=1}^{n_e}
      \left(\sum_{i=1}^{n_b} N_{ai}C_{is}C^*_{it}N^*_{di}\right)
      \left(\sum_{j=1}^{n_b} N_{bj}C_{jt}C^*_{js}N^*_{cj}\right)
      \right] \nonumber \\
  &-& \left[
      \sum_{s,t=1}^{n_e}
      \left(\sum_{i=1}^{n_b} N_{bi}C_{is}C^*_{is}N^*_{ci}\right)
      \left(\sum_{j=1}^{n_b} N_{aj}C_{jt}C^*_{jt}N^*_{dj}\right)
      \right.
      \nonumber \\
  &&- \left.\left.
      \sum_{s,t=1}^{n_e}
      \left(\sum_{i=1}^{n_b} N_{bi}C_{is}C^*_{it}N^*_{ci}\right)
      \left(\sum_{j=1}^{n_b} N_{aj}C_{jt}C^*_{js}N^*_{dj}\right)
      \right]\right\} \nonumber \\
      \label{Eq:2densmat_2}
\end{eqnarray}
In Eq.~\ref{Eq:2densmat_2} we have used that both
$\left|\chi_a(r_1)\chi_b(r_2)\right\rangle$ and 
$\left\langle\chi_c(r_1)\chi_d(r_2)\right|$ are anti-symmetric functions.
Note that Eq.~\ref{Eq:2densmat} can be used for all different blocks of the
two-electron density matrix. The case where generalized orbitals $s$ and $t$ 
are both $\alpha$ orbitals or both are $\beta$ orbitals is directly represented
in the equation. The case where $s$ and $t$ are different spin orbitals the
spin integration eliminates the exchange term and only the first term survives.

The expression of Eq.~\ref{Eq:2densmat_2} can be simplified by collapsing the 
2-electron density matrix occupation numbers as
\begin{eqnarray}
  d_{ij}
  &=& 
      \sum_{s,t=1}^{n_e}
      C_{is}C^*_{is}
      C_{jt}C^*_{jt}
    - C_{is}C^*_{it}
      C_{jt}C^*_{js}
      \label{Eq:2occupation_3} \\
  \Gamma_{abcd}
  &=& \frac{1}{4}
      \left[
      \left(\sum_{i,j=1}^{n_b} N_{ai}N^*_{ci}N_{bj}N^*_{dj}d_{ij}\right)
      \right.
      \nonumber \\
  &&+ \left(\sum_{i,j=1}^{n_b} N_{bi}N^*_{di}N_{aj}N^*_{cj}d_{ij}\right)
      \nonumber \\
  &&- \left(\sum_{i,j=1}^{n_b} N_{ai}N^*_{di}N_{bj}N^*_{cj}d_{ij}\right)
      \nonumber \\
  &&- \left.
      \left(\sum_{i,j=1}^{n_b} N_{bi}N^*_{ci}N_{aj}N^*_{dj}d_{ij}\right)
      \right]
      \label{Eq:2densmat_3}
\end{eqnarray}


Note that in Eq.~\ref{Eq:2densmat} the expression for the probability density
had to be generalized somewhat. Here we use a transition density matrix
for a pair of generalized orbitals defined as
\begin{eqnarray}
  D^{st}_{ab} &=& \sum_{i=1}^{n_b}N_{ai}C_{is}C^*_{it}N^*_{bi}
  \label{Eq:1orbmat}
\end{eqnarray}
The case where $s$ equals $t$ then is a special case which generates a regular
single-orbital one-electron density matrix that generates the corresponding
probability density.

In order for this two-electron density matrix to be N-representable 
three requirements have to be met: 1) it should be anti-symmetric under
permutations of two orbitals in either the wave function or its complex 
conjugate; 2) it should be an non-negative matrix; 3) the trace of the matrix
should equal the number of electron pairs.

The anti-symmetry requirement translates into the condition that
\begin{eqnarray}
   \Gamma_{abcd} &=& - \Gamma_{bacd} \\
                 &=& - \Gamma_{abdc}
\end{eqnarray}
In order for this to hold it is necessary to define the permutation such that
it involves the complete generalized orbital. I.e. it is not permisible to just
interchange $N_{ai}$ and $N_{bj}$ in Eq.~\ref{Eq:2densmat}. Instead we have to
postulate that the permutation involves $N_{ai}C_{is}$ and $N_{bj}C_{jt}$ as
together they form the coefficients of the generalized orbitals. Also for this
permutation to safisfy Eq.~\ref{Eq:1orbmat} the indeces $i$ and $j$ need to be
relabeled on the quantities being interchanged. It is trivial to verify that
if we postulate that permutation of indeces of $\Gamma$ be performed this way
that then $\Gamma$ is indeed anti-symmetric.

The fact that the matrix is non-negative can be proven by making use of the
Gauchy-Schwarz inequality which states that
\begin{eqnarray}
  \left|\sum_{i=1}^n x_i y_i^*\right| 
  &\leq& \left(\sum_{j=1}^n|x_j|^2\right)\left(\sum_{k=1}^n|y_k|^2\right)
  \label{Eq:GauchySchwarz}
\end{eqnarray}
For simplicity we prove that the first term in Eq.~\ref{Eq:2densmat} is larger
than the second term for every combination of $i$ and $j$. In this argument we
discard the natural orbital factors as they are the same in both terms for a
given pair of $i$ and $j$ and focus just on the correlation functions.
I.e. we have to show that 
\begin{eqnarray}
  \sum_{s,t=1}^{n_e} C_{is}C^*_{it}C_{jt}C^*_{js} &\leq&
  \sum_{s,t=1}^{n_e} C_{is}C^*_{is}C_{jt}C^*_{jt} 
\end{eqnarray}
By trivially rearranging the factors and the summations this expression can be
rewritten as
\begin{eqnarray}
  \left(\sum_{s=1}^{n_e} C_{is}C^*_{js}\right)
  \left(\sum_{t=1}^{n_e}C_{jt}C^*_{it}\right) &\leq&
  \left(\sum_{s=1}^{n_e} C_{is}C^*_{is}\right)
  \left(\sum_{t=1}^{n_e}C_{jt}C^*_{jt}\right) 
\end{eqnarray}
which is the same as Eq.~\ref{Eq:GauchySchwarz} if the vector $y$ is equated to
the $j$-th row of $C$ and vector $x$ is equated to the $i$-th row of $C$.
Furthermore as the first term in Eq.~\ref{Eq:2densmat} is obviously 
non-negative it immediately follows that $\Gamma_{abcd}$ is a non-negative
density matrix.

Finally we need to show that $\Gamma_{abcd}$ is the correct trace. Once again
we focus on the factors related to the correlation functions. Furthermore
we exploit the orthonormality of the correlation functions. Starting from
\begin{eqnarray}
  &&\frac{1}{2}\left[\sum_{s,t=1}^{n_e}
  \left(\sum_{i=1}^{n_b}C_{is}C^*_{is}\right)
  \left(\sum_{j=1}^{n_b}C_{jt}C^*_{jt}\right)
  -\sum_{s,t=1}^{n_e}
  \left(\sum_{i=1}^{n_b}C_{is}C^*_{it}\right)
  \left(\sum_{j=1}^{n_b}C_{jt}C^*_{js}\right)\right] \nonumber \\
  &&=
  \left[\sum_{\begin{array}{c}s,t=1\\s > t\end{array}}^{n_e}
  \left(\sum_{i=1}^{n_b}C_{is}C^*_{is}\right)
  \left(\sum_{j=1}^{n_b}C_{jt}C^*_{jt}\right)
  -\sum_{\begin{array}{c}s,t=1\\s > t\end{array}}^{n_e}
  \left(\sum_{i=1}^{n_b}C_{is}C^*_{it}\right)
  \left(\sum_{j=1}^{n_b}C_{jt}C^*_{js}\right)\right] \\
  &&= 
  \left[\sum_{\begin{array}{c}s,t=1\\s > t\end{array}}^{n_e}
  1\cdot 1
  -\sum_{\begin{array}{c}s,t=1\\s > t\end{array}}^{n_e}
  0 \cdot 0
  \right] \\
  &&= n_e(n_e-1)
\end{eqnarray}
Therefore we have that the trace equals the number of electron pairs.

Thus we have shown that it is possible to formulate a single-determinant
wave function
in terms of generalized orbitals and a suitably chosen formalism such that it
can generate the exact one-electron density matrix and an N-representable
two-electron density matrix. In particular the two-electron density matrix
is N-representable even if the one-electron density matrix for the same state
is not idem-potent. Of course the two-electron density matrix is not exact as
it corresponds to a single-determinant wave function and therefore no 
correlation effects are accounted for. 

Furthermore the two-electron density matrix is orbital dependent due to the
exchange terms. This raises questions about the extent to which it is possible
to reconstruct the two-electron density matrix from the one-electron density
matrix. It seems that this is highly non-trivial especially if the 
resulting two-electron density matrix is to meet the N-representability 
conditions. So if two-electron density matrix reconstruction approaches might
not be the right path to proceed to find a way to represent electron correlation
effects then what alternatives can we pursue?

One alternative approach is to consider an effective one-electron model as 
a reduced dimensionality model obtained by coarse graining the full many-body 
problem. Consider that an effective one-electron models is essentially a 
mean-field approximation. The full many-body problem in addition contains
dynamical correlation, effects that result from the interaction between 
fluctuations around the mean-field electron distribution. Against this notion
there there are many ideas that suggest a coarse graining approach. 
The separation between slow modes (mean-field) and fast modes (dynamic
correlation) referred to by Gorban~\cite{Gorban_2006}. The {\bf more stuff}...

\section{Coarse graining}

In coarse graining approaches the aim is to reduce the number of degrees of 
freedom while maintaining the statistical properties of the full system. 
Typically this requires adding terms such that the free energy of the 
reduced dimensionality system is as close as possible to the free energy
of the full system. As in this work we are interested in representing electron
correlation in molecules in vacuum the Helmholtz free energy is the appropriate
quantity to preserve~\cite{Warren_1996,Weinert_1992,Gillan_1989,Mermin_1965}
\begin{equation}
  F = U - TS
  \label{Eq:Helmholtz}
\end{equation}
where $U$ is the internal energy which equates the expectation value of the
Hamiltonian in the Schr{\"o}dinger equation, $T$ is the temperature and $S$ the
entropy. 

In the following two subsections we consider how to formulate the temperature
and the entropy in the context where we want to describe electron correlation
as a coarse graining consequence. Important in this discussion will be three
levels of theory: effective one-electron approaches, electron-pair
models, and the Full-CI approach. From the outset it should be clear that the
Full-CI energy and the two-electron density matrix energy are both exact
without the $TS$ term. Only in the effective one-electron approach is the 
$TS$ non-zero. The theory proposed will bear this out.

\subsection{The temperature}

In this section the temperature is considered in detail. Typically discussions
about temperature involve a system coupled to a bath. In this context 
Eq.~\ref{Eq:Helmholtz} has already been deployed in electronic structure
theory~\cite{Warren_1996,Weinert_1992,Gillan_1989,Mermin_1965}. The fractional
occupation numbers resulting in this case are not representative of electron
correlation resulting from electron-electron interactions. Although we are
interested in the latter it is still illustrative to start with the traditional
view of a system coupled to a bath as a reference point for the discussion.

When a system, a collection of particles, is coupled to a heat bath and the
system is in equilibrium with the bath the temperature of the particles
is the same as that of the bath. In this case if the bath has zero temperature
this means that the particles in the system are in their lowest accessible
energy states. Raising the temperature of the bath causes energy to be deposited
in the various degrees of freedom of the particles until equilibrium is once
again obtained. For system of, for example, noble gas atoms the temperature
is closely related to the average kinetic energy per degree of freedom. The
reason for that is that the kinetic energy is most easily changed, excitations
of electronic states typically require substantially more energy to access.
The important point is that the temperature is proportional to the raise in
energy of a particle due to the interaction with the bath.

With the idea of the previous paragraph in mind we turn to the temperature
of electrons in effective one-electron models $T_1$, effective electron-pair
models $T_2$, and true many-body models $T_{n_e}$. In effective one-electron
models the electron would occupy the lowest one-electron energy states if it
wasn't for the electron-electron interaction. This interaction drives the
electrons out of the lowest possible states and raises their energy to that
of the orbital eigenvalues of the effective one-electron model. Hence $T_1$ is
proportional to the electron-electron repulsion per electron.

By a similar consideration the electron-pairs of an effective electron-pair
model cannot occupy the lower pair states due to the pair-pair interactions. In
this case of course in the intra-pair electron-electron interaction is not
contributing to the temperature as this property is integral to the properties
of the electron pair. The inter-pair interaction on the contrary is contributing
to the temperature. In fact the pair temperature $T_2$ is directly propertional
to the total pair-pair interaction energy per electron pair.  

Finally considering the many-body, i.e. Full-CI $n_e$-electron, temperature
we realize that for a system of $n_e$ electrons there are no other $n_e$
electron states to interact with. Hence the lowest $n_e$ electron state is
accessible without constraint and the corresponding temperature $T_{n_e}$ is
zero.

Note that as the temperature for higher order states is concerned increasingly
more interactions are considered internal to the states of that order and
fewer interactions are inter-state interactions it is
reasonable that $T_1 \ge T_2 \ge \ldots \ge T_{n_e} = 0$. In remainder of the
paper it is important to note in particular that in general $T_2 > 0$. The 
latter fact is important in the discussion of the entropy.

\subsection{The entropy}

Having settled on an interpretation for the temperature the entropy is the
remaining quantity needed to define the free energy. Already in 1926 the
entropy of an electron distribution over available one-electron states was
worked out independently by Fermi and Dirac~\cite{Fermi_1926,Dirac_1926}.
The expression they found
\begin{eqnarray}
   S = \sum_{i=1}^{n_b} \left(d_i\ln(d_i) + (1-d_i)\ln(1-d_i)\right)
   \label{Eq:Entropy}
\end{eqnarray}
follows directly from the expression for the number of ways a given number
of electrons can be distributed over a set of orbitals of infinite degeneracy, 
and applying the Sterling approximation to the logarithm of the factorials.
Hence it would seem that the one-electron picture is obvious.

The $n_e$-electron picture is also obvious. The density matrix of the many-body
ground state wave function has one eigenvalue of value $1$ corresponding to that
state. All other eigenvalues are $0$. As a result the entropy of the 
$n_e$-electron wave function is $0$. 

The entropies for the intermediate density matrices can in principle be worked
out on the basis of information theoretical ideas. In information theory an 
increase in entropy is equated to a loss of information about the system in 
question. Hence when we start from the $n_e$ electron density matrix and one
electron coordinate is integrated out to produce a $(n_e-1)$-electron density
matrix some information about the system is lost and hence the entropy 
increases. This suggests that the entropy for all density matrices from the
one-electron density matrix to the $(n_e-1)$-electron density matrix is
non-zero.
This causes a problem because in the previous subsection it was already found
that the electron-pair temperature $T_2$ is non-zero. Combined with a non-zero
two-electron density matrix entropy there would be a non-zero entropic term to
the free energy, but it is clear that the internal energy already gives the
exact free energy. Hence we face a paradox.

When considering the paradox found it is clear that either the temperature
or the entropy is not correct. The electron-pair temperature as defined in the
previous subsection as given by the pair-pair interactions per pair clearly is
non-zero. Hence in order for the entropic term to vanish the two-electron 
density matrix entropy should be zero. The way this can be explained is by
considering the order of the interactions and the associated physics. Electrons
interact by potentials that involve at most two-electrons. Hence a
three-electron density matrix has no physical relevance as there are no 
physical phenomena that are directly related to it. Therefore when an 
electron coordinate is integrated out of a three-electron density matrix to 
obtain a two-electron density matrix mathematically information about the system
is lost. However, physically that information is of no consequence. The same
argument holds for all density matrices of orders higher than two. Only when
the two-electron density matrix is projected to generate the one-electron
density matrix is actually physically relevant information lost. Hence the
zero point of physically relevant information is shifted with respect to the
mathematical zero point. Based on this argument we postulate that the entropy
be chosen such that the entropy of the two-electron density matrix is zero
by definition. This means, however, that the one-electron density matrix 
entropy must be corrected by the shift of the zero-point of the two-electron
density matrix entropy. As a result the correct one-electron entropy is smaller
than expected on the basis of the conventional entropy expression.

Instead of the entropy of Eq.~\ref{Eq:Entropy} we propose the expression
\begin{eqnarray}
   S &=& S^{(1)} - S^{(2)}
   \label{Eq:Entropy12} \\
   S^{(1)} &=& 
     -\sum_{i=1}^{n_b}
         \left(d_i\ln(d_i) + (1-d_i)\ln(1-d_i) \right)
   \label{Eq:Entropy1} \\
   S^{(2)} &=& 
     -   \frac{1}{n_e-1}\sum_{i,j=1}^{n_b}
         \left(d_{ij}\ln(d_{ij}) + (1-d_{ij})\ln(1-d_{ij}) \right)
   \label{Eq:Entropy2}
\end{eqnarray}
where $d_i$ are the occupation numbers of the one-electron density matrix and
$d_{ij}$ are the occupation numbers of the two-electron density matrix.
The scale factor for the two-electron density matrix term stems from the fact
that the corresponding entropy grows with the number of electron pairs. Clearly
such an electron-pair entropy cannot be compared directly against an 
one-electron entropy. Hence the scale factor is needed to obtain an 
electron-pair entropy that is comparable to an one-electron entropy.
Note that the term for the two-electron density matrix can be used in the
present context. In the general case the occupation numbers of the two matrix
can exceed $1$ unlike in the case of the one-electron density matrix. However,
this can occur only in the case of multi-determinantal wave functions. In the
case of a single-determinant wave function the occupation numbers of the 
two-electron density matrix satisfy the same constraints as those of the
one-electron density as is clear from Eq.~\ref{Eq:2densmat}.

{\bf Possibly $TS$ can be broken up into $T_iS_i$, i.e. a free energy term 
per natural orbital. Although there does not seem to be any justification for
doing that, it would certainly ensure that the free energy of a system 
consisting of non-interacting subsystems is the sum of free energies of all 
subsystems. It is definetely non-obvious that Eq.~\ref{Eq:Entropy12} would
do that. 

Even when we write the free energy contribution as $T_iS_i$ there is yet 
another problem. If there are two interacting electrons 1 and 2 then there
is no reason for the entropy associated with both electrons to be the same.
As a result we arrive at another paradox, namely that the strength of the
interaction of electron 1 with 2 could be different from the strength
of the interaction of electron 2 with 1. This clearly violates a rather
fundamental principle in physics.}

\subsection{Formulating the energy expression}

Based on the arguments above the overall expression for the free energy can
be formulated as
\begin{eqnarray}
   F &=& U-TS \\
     &=& H+W-TS
   \label{Eq:TS}
\end{eqnarray}
where $H$ is the one-electron internal energy, i.e. the kinetic energy and the
nuclear attraction energies, $W$ is the two-electron repulsion energy, 
$T=W/n_e$ is the temperature and $S$ is the entropy. Obviously if the division
by the number of electrons is moved out of the temperature and into the 
entropy this equation can be simplified a bit to
\begin{eqnarray}
   F &=& H+W(1-S/n_e)
   \label{Eq:Coulombquenching}
\end{eqnarray}
Hence, electron correlation in this view arises as an entropy driven quenching
of the electron-electron repulsion. Unfortunately the picture of 
Eq.~\ref{Eq:Coulombquenching} cannot be right as it relies on the average entropy
of a system. This fact causes a problem when studing systems that consist of 
sub-systems without interaction. For example consider a \ce{Be} atom and a 
\ce{H2O} molecule separated by a large distance. Clearly the overall free energy
of this system should be the same as the sum of the free energy of a \ce{Be} atom in
isolation and the free energy of a \ce{H2O} molecule in isolation. However, because 
an average entropy is used the entropy for the \ce{Be} electrons is modified by the
sheer existence of a \ce{H2O} molecule, irrespective how remote that molecule is.
Solving this problem requires replacing the entropic term in Eq.~\ref{Eq:TS} by 
something else that associates an entropic term with each electron, i.e. something
of the form
\begin{eqnarray}
   F &=& H+W-\sum_{r=1}^{n_e}T_rS_r
   \label{Eq:TrSr}
\end{eqnarray}
is needed.

To complete the definition of the free energy expression in Eq.~\ref{Eq:TrSr} a 
re-interpretation of the temperature and the entropy is needed. The interpretation
of the temperature of a single electron follows from the temperature definition 
chosen earlier. If the total electron-electron repulsion per electron is the 
temperature of the system then the temperature of an electron in generalized 
orbital $r$ is given by
\begin{eqnarray}
   \Gamma_{abcd}^{(r)} &=& \sum_{s=1}^{n_e}\Gamma_{abcd}^{(rs)} 
   \label{Eq:Gr} \\
   T_{r} &=& \sum_{a,b,c,d=1}^{n_b}(ab|cd)\Gamma_{abcd}^{(r)}
   \label{Eq:Tr}
\end{eqnarray}.
I.e. the repulsion energy that an electron in orbital $r$ experiences due to the
interaction with all other electrons.

To deal with the entropy it is first of all important to recognize that the entropy
depends on the behavior of all electrons. I.e. if the entropy of the one-electron
density matrix is given by
\begin{eqnarray}
   S^{(1)} &=& \sum_{i=1}^{n_b}\left\{d_i\ln(d_i)+(1-d_i)\ln(1-d_i)\right\}
\end{eqnarray}
then it makes no sense to define the entropy of an electron in orbital $r$ as
\begin{eqnarray}
   S^{(1)}_r &=& \sum_{i=1}^{n_b}
           \left\{d_i^{(r)}\ln(d_i^{(r)})+(1-d_i^{(r)})\ln(1-d_i^{(r)})\right\}
   \label{Eq:nonsense}
\end{eqnarray}
where $d_i^{(r)}$ is the occupation number of natural orbital $i$ generated by
generalized orbital $r$. The reason this makes no sense is that if we consider two
natural orbitals which are fully occupied then their contribution to the entropy is
zero. However if we choose to distribute each electron half-and-half over both natural
orbitals then the total density matrix would remain the same but Eq.~\ref{Eq:nonsense}
would assign a non-zero entropy to both electrons.

Therefore it is clear that the entropy itself must be evaluated on the density matrix
of all electrons but the resulting entropy must be divided over all the electrons.
Hence we chose to partition the entropy prorata with respect to the fraction of the
occupation number that a given orbital contributes. I.e. for the entropy of the 
one-electron density matrix
\begin{eqnarray}
   S^{(1)}_r &=& -\sum_{i=1}^{n_b}\frac{d_i^{(r)}}{d_i}
   \left\{d_i\ln(d_i)+(1-d_i)\ln(1-d_i)\right\}
   \label{Eq:entropy1part}
\end{eqnarray}
For the entropy of the two-electron density matrix we have
\begin{eqnarray}
   S^{(2)}_r &=& -\sum_{i,j=1}^{n_b}
           \left\{\frac{\sum_{s=1}^{n_e}d_{ij}^{(rs)}}{d_{ij}(n_e-1)}\right\}
           \left\{d_{ij}\ln(d_{ij})+(1-d_{ij})\ln(1-d_{ij})\right\}
   \label{Eq:entropy2part}
\end{eqnarray}
where $d_{ij}^{(rs)}$ is occupation number for the $i, j$ natural orbital pair
generated by the $r, s$ generalized orbital pair. Clearly these definitions have
the property that 
\begin{equation}
    S = \sum_{r=1}^{n_e}\left(S^{(1)}_r-S^{(2)}_r\right)
\end{equation}
which seems a reasonably property for dividing an entropy over $n_e$ electrons.

Even after splitting the entropy up over different electrons there is still
another problem. In Eq.~\ref{Eq:TrSr} for a given pair of electrons the fact
that the entropy of electron 1 may be different from the entropy of electron 2
leads to an unphysical result. Namely the result that the interaction of 
electron 1 with electron 2 is of a different strength than the interaction
of electron 2 with electron 1. This clearly violates the third law of Newton's
laws of motion. This serious problem can be addressed by rewriting Eq.~\ref{Eq:TrSr}
as 
\begin{eqnarray}
   F &=& H+W-\sum_{r,s=1}^{n_e}T_{rs}S_{rs}
   \label{Eq:TrsSrs}
\end{eqnarray}
whereby it is used that the temperature is a linear function that sums just 
the electron pair interactions. Hence an effective temperature contribution
can be assigned to every electron pair. Based on Eq.~\ref{Eq:TrsSrs} 
we need to evaluate an entropy for every electron pair. The corresponding pair
entropy has to include the entropies of both individual electrons as well
as the entropy of the 2-electron density matrix for the pair of electrons.
In part this expression exploits the fact that the temperature is a linear
expression in terms of the electron pair interations.
In the above expression the pair entropy $S_{rs}$ is given by
\begin{eqnarray}
  S_{rs} &=& S^{(1)}_r + S^{(1)}_s - S^{(2)}_{rs}
\end{eqnarray}
Alternatively we may write Eq.~\ref{Eq:TrsSrs} as 
\begin{eqnarray}
   F &=& H+\frac{1}{2}\sum_{r,s=1}^{n_e}W_{rs}(1-S_{rs})
   \label{Eq:CoulombQuench_rs}
\end{eqnarray}
Note that the 2-electron entropy may involve different expressions for different
blocks of the 2-electron density matrix. In the light of these changes the entropy
definitions need to be modified a bit to give
\begin{eqnarray}
   S^{(1)}_r &=& -\sum_{i=1}^{n_b}\frac{d_i^{(r)}}{d_i}S^{(1)}(d_i)
   \label{Eq:entropy1partfin} \\
   S^{(1)}(d_i) &=&
   \left\{d_i\ln(d_i)+(1-d_i)\ln(1-d_i)\right\} \\
   S^{(2)}_{r\bar{s}} &=& -\sum_{i,\bar{j}=1}^{n_b}
           \left\{\frac{d_{i}^{(r)}d_{\bar{j}}^{(\bar{s})}}{d_{i}d_{\bar{j}}}\right\}
           S^{(2)}(d_{i},d_{\bar{j}})
   \label{Eq:entropy2partfinab} \\
   S^{(2)}(d_{i},d_{\bar{j}}) &=&
           \left\{d_{i}d_{\bar{j}}\ln(d_{i}d_{\bar{j}})
           +(1-d_{i})(1-d_{\bar{j}})\ln((1-d_{i})(1-d_{\bar{j}}))\right\} \\
   S^{(2)}_{rs} &=& -\sum_{i,\bar{j}=1}^{n_b}
           \left\{\frac{d_{ij}^{(rs)}}{d_{ij}}\right\}
           S^{(2)}(d_{ij})
   \label{Eq:entropy2partfinaa} \\
   S^{(2)}(d_{ij}) &=&
           \left\{\sqrt{d_{ij}}\sqrt{d_{ij}}\ln(\sqrt{d_{ij}}\sqrt{d_{ij}})
           +(1-\sqrt{d_{ij}})(1-\sqrt{d_{ij}})
            \ln((1-\sqrt{d_{ij}})(1-\sqrt{d_{ij}}))\right\}
\end{eqnarray}
In these equations the bar is used to indicate labels of opposite spin. Note that in 
the 2-electron entropies the division by the number of electrons has been removed. 
This is a consequence of handling both the 1-electron entropy and the 2-electron 
entropy on a per electron pair basis.

In practice we need to deal with a few more problems. 
\begin{itemize}
\item The factors used to attribute fractions of the entropy to individual electrons
      will cause numerical issues when differentiated. Hence it would be better
      to get rid of them.
\item The entropy for a pair of states is supposed to describe correlation. That
      means the entropy should vanish if one of the two states is not correlated.
      I.e. $S_{rs}$ should be 0 if $d_i$ or $d_j$ equals 0 or 1.
\item The Stirling approximation used to get terms as $d_i\ln(d_i)$ has serious
      artifacts. Hence it would be better to use $\ln(d_i!)$ instead.
\end{itemize}
To deal with the first two points we may replace $S_{rs}$ with $S_{ij}$ where we 
define
\begin{eqnarray}
  S_{ij} &=& S_{i}^{(1)}+S_{j}^{(1)}-S_{ij}^{(2)} \\
  S_{i}^{(1)} &=& \ln(d_i!) + \ln([1-d_i]!) \label{Eq:S1}
\end{eqnarray}
The third point requires to reformulate the $S^{(2)}$ term. The expression
\begin{eqnarray}
  S_{ij}^{(2)} &=& \ln([d_id_j]!)+\ln([(1-d_i)d_j]!)
                +  \ln([d_i(1-d_j)]!)+\ln([(1-d_i)(1-d_j)]!) \label{Eq:S2}
\end{eqnarray}
For example if $d_i=0$ then
\begin{eqnarray}
  S_{ij}^{(2)} &=& 0+\ln(d_j!)+0+\ln([1-d_j]!)
\end{eqnarray}
which exactly cancels $S_{j}^{(1)}$. The resulting function is shown in~\ref{Fig1}.
Clearly this function has the characteristics needed. If any of the natural orbital
are uncorrelated the entropy is 0. If both natural orbitals are half occupied the
correlation is maximal.

\begin{figure}
% GNUPLOT: LaTeX picture
\setlength{\unitlength}{0.240900pt}
\ifx\plotpoint\undefined\newsavebox{\plotpoint}\fi
\sbox{\plotpoint}{\rule[-0.200pt]{0.400pt}{0.400pt}}%
\begin{picture}(1500,900)(0,0)
\sbox{\plotpoint}{\rule[-0.200pt]{0.400pt}{0.400pt}}%
\multiput(212.00,251.58)(1.089,0.500){359}{\rule{0.971pt}{0.120pt}}
\multiput(212.00,250.17)(391.985,181.000){2}{\rule{0.485pt}{0.400pt}}
\multiput(1276.80,327.58)(-3.256,0.499){207}{\rule{2.698pt}{0.120pt}}
\multiput(1282.40,326.17)(-676.400,105.000){2}{\rule{1.349pt}{0.400pt}}
\put(212.0,251.0){\rule[-0.200pt]{0.400pt}{87.206pt}}
\put(606.0,432.0){\rule[-0.200pt]{0.400pt}{28.908pt}}
\put(1288.0,327.0){\rule[-0.200pt]{0.400pt}{29.149pt}}
\multiput(212.00,251.59)(1.220,0.488){13}{\rule{1.050pt}{0.117pt}}
\multiput(212.00,250.17)(16.821,8.000){2}{\rule{0.525pt}{0.400pt}}
\put(193,233){\makebox(0,0)[r]{ 0}}
\multiput(602.08,430.93)(-1.077,-0.489){15}{\rule{0.944pt}{0.118pt}}
\multiput(604.04,431.17)(-17.040,-9.000){2}{\rule{0.472pt}{0.400pt}}
\multiput(349.00,230.59)(1.220,0.488){13}{\rule{1.050pt}{0.117pt}}
\multiput(349.00,229.17)(16.821,8.000){2}{\rule{0.525pt}{0.400pt}}
\put(330,212){\makebox(0,0)[r]{ 0.2}}
\multiput(738.64,409.93)(-1.220,-0.488){13}{\rule{1.050pt}{0.117pt}}
\multiput(740.82,410.17)(-16.821,-8.000){2}{\rule{0.525pt}{0.400pt}}
\multiput(485.00,209.59)(1.220,0.488){13}{\rule{1.050pt}{0.117pt}}
\multiput(485.00,208.17)(16.821,8.000){2}{\rule{0.525pt}{0.400pt}}
\put(466,191){\makebox(0,0)[r]{ 0.4}}
\multiput(873.64,388.93)(-1.220,-0.488){13}{\rule{1.050pt}{0.117pt}}
\multiput(875.82,389.17)(-16.821,-8.000){2}{\rule{0.525pt}{0.400pt}}
\multiput(622.00,188.59)(1.077,0.489){15}{\rule{0.944pt}{0.118pt}}
\multiput(622.00,187.17)(17.040,9.000){2}{\rule{0.472pt}{0.400pt}}
\put(603,170){\makebox(0,0)[r]{ 0.6}}
\multiput(1010.64,367.93)(-1.220,-0.488){13}{\rule{1.050pt}{0.117pt}}
\multiput(1012.82,368.17)(-16.821,-8.000){2}{\rule{0.525pt}{0.400pt}}
\multiput(757.00,167.59)(1.077,0.489){15}{\rule{0.944pt}{0.118pt}}
\multiput(757.00,166.17)(17.040,9.000){2}{\rule{0.472pt}{0.400pt}}
\put(739,149){\makebox(0,0)[r]{ 0.8}}
\multiput(1146.64,346.93)(-1.220,-0.488){13}{\rule{1.050pt}{0.117pt}}
\multiput(1148.82,347.17)(-16.821,-8.000){2}{\rule{0.525pt}{0.400pt}}
\multiput(894.00,146.59)(1.077,0.489){15}{\rule{0.944pt}{0.118pt}}
\multiput(894.00,145.17)(17.040,9.000){2}{\rule{0.472pt}{0.400pt}}
\put(875,128){\makebox(0,0)[r]{ 1}}
\multiput(1283.64,325.93)(-1.220,-0.488){13}{\rule{1.050pt}{0.117pt}}
\multiput(1285.82,326.17)(-16.821,-8.000){2}{\rule{0.525pt}{0.400pt}}
\multiput(883.62,146.61)(-3.811,0.447){3}{\rule{2.500pt}{0.108pt}}
\multiput(888.81,145.17)(-12.811,3.000){2}{\rule{1.250pt}{0.400pt}}
\put(911,141){\makebox(0,0){ 0}}
\multiput(212.00,249.95)(3.811,-0.447){3}{\rule{2.500pt}{0.108pt}}
\multiput(212.00,250.17)(12.811,-3.000){2}{\rule{1.250pt}{0.400pt}}
\multiput(962.62,182.61)(-3.811,0.447){3}{\rule{2.500pt}{0.108pt}}
\multiput(967.81,181.17)(-12.811,3.000){2}{\rule{1.250pt}{0.400pt}}
\put(990,177){\makebox(0,0){ 0.2}}
\multiput(291.00,285.95)(3.588,-0.447){3}{\rule{2.367pt}{0.108pt}}
\multiput(291.00,286.17)(12.088,-3.000){2}{\rule{1.183pt}{0.400pt}}
\put(1034,219.17){\rule{3.500pt}{0.400pt}}
\multiput(1043.74,218.17)(-9.736,2.000){2}{\rule{1.750pt}{0.400pt}}
\put(1069,213){\makebox(0,0){ 0.4}}
\put(370,321.17){\rule{3.500pt}{0.400pt}}
\multiput(370.00,322.17)(9.736,-2.000){2}{\rule{1.750pt}{0.400pt}}
\put(1113,255.17){\rule{3.500pt}{0.400pt}}
\multiput(1122.74,254.17)(-9.736,2.000){2}{\rule{1.750pt}{0.400pt}}
\put(1147,249){\makebox(0,0){ 0.6}}
\multiput(449.00,358.95)(3.588,-0.447){3}{\rule{2.367pt}{0.108pt}}
\multiput(449.00,359.17)(12.088,-3.000){2}{\rule{1.183pt}{0.400pt}}
\multiput(1199.18,291.61)(-3.588,0.447){3}{\rule{2.367pt}{0.108pt}}
\multiput(1204.09,290.17)(-12.088,3.000){2}{\rule{1.183pt}{0.400pt}}
\put(1226,286){\makebox(0,0){ 0.8}}
\multiput(527.00,394.95)(3.811,-0.447){3}{\rule{2.500pt}{0.108pt}}
\multiput(527.00,395.17)(12.811,-3.000){2}{\rule{1.250pt}{0.400pt}}
\multiput(1277.62,327.61)(-3.811,0.447){3}{\rule{2.500pt}{0.108pt}}
\multiput(1282.81,326.17)(-12.811,3.000){2}{\rule{1.250pt}{0.400pt}}
\put(1305,322){\makebox(0,0){ 1}}
\put(606,430.17){\rule{3.700pt}{0.400pt}}
\multiput(606.00,431.17)(10.320,-2.000){2}{\rule{1.850pt}{0.400pt}}
\put(172,372){\makebox(0,0)[r]{ 0}}
\put(212.0,372.0){\rule[-0.200pt]{4.818pt}{0.400pt}}
\put(172,396){\makebox(0,0)[r]{ 0.1}}
\put(212.0,396.0){\rule[-0.200pt]{4.818pt}{0.400pt}}
\put(172,420){\makebox(0,0)[r]{ 0.2}}
\put(212.0,420.0){\rule[-0.200pt]{4.818pt}{0.400pt}}
\put(172,444){\makebox(0,0)[r]{ 0.3}}
\put(212.0,444.0){\rule[-0.200pt]{4.818pt}{0.400pt}}
\put(172,468){\makebox(0,0)[r]{ 0.4}}
\put(212.0,468.0){\rule[-0.200pt]{4.818pt}{0.400pt}}
\put(172,492){\makebox(0,0)[r]{ 0.5}}
\put(212.0,492.0){\rule[-0.200pt]{4.818pt}{0.400pt}}
\put(172,516){\makebox(0,0)[r]{ 0.6}}
\put(212.0,516.0){\rule[-0.200pt]{4.818pt}{0.400pt}}
\put(172,540){\makebox(0,0)[r]{ 0.7}}
\put(212.0,540.0){\rule[-0.200pt]{4.818pt}{0.400pt}}
\put(172,564){\makebox(0,0)[r]{ 0.8}}
\put(212.0,564.0){\rule[-0.200pt]{4.818pt}{0.400pt}}
\put(172,588){\makebox(0,0)[r]{ 0.9}}
\put(212.0,588.0){\rule[-0.200pt]{4.818pt}{0.400pt}}
\put(172,613){\makebox(0,0)[r]{ 1}}
\put(212.0,613.0){\rule[-0.200pt]{4.818pt}{0.400pt}}
\put(1300,756){\makebox(0,0)[r]{s(x,y)/s(0.5,0.5)}}
\put(211,369.67){\rule{1.686pt}{0.400pt}}
\multiput(211.00,370.17)(3.500,-1.000){2}{\rule{0.843pt}{0.400pt}}
\put(218,368.67){\rule{1.686pt}{0.400pt}}
\multiput(218.00,369.17)(3.500,-1.000){2}{\rule{0.843pt}{0.400pt}}
\put(225,367.67){\rule{1.686pt}{0.400pt}}
\multiput(225.00,368.17)(3.500,-1.000){2}{\rule{0.843pt}{0.400pt}}
\put(232,366.17){\rule{1.500pt}{0.400pt}}
\multiput(232.00,367.17)(3.887,-2.000){2}{\rule{0.750pt}{0.400pt}}
\put(239,364.67){\rule{1.686pt}{0.400pt}}
\multiput(239.00,365.17)(3.500,-1.000){2}{\rule{0.843pt}{0.400pt}}
\put(246,363.67){\rule{1.686pt}{0.400pt}}
\multiput(246.00,364.17)(3.500,-1.000){2}{\rule{0.843pt}{0.400pt}}
\put(253,362.67){\rule{1.686pt}{0.400pt}}
\multiput(253.00,363.17)(3.500,-1.000){2}{\rule{0.843pt}{0.400pt}}
\put(260,361.67){\rule{1.445pt}{0.400pt}}
\multiput(260.00,362.17)(3.000,-1.000){2}{\rule{0.723pt}{0.400pt}}
\put(266,360.67){\rule{1.686pt}{0.400pt}}
\multiput(266.00,361.17)(3.500,-1.000){2}{\rule{0.843pt}{0.400pt}}
\put(273,359.67){\rule{1.686pt}{0.400pt}}
\multiput(273.00,360.17)(3.500,-1.000){2}{\rule{0.843pt}{0.400pt}}
\put(280,358.67){\rule{1.686pt}{0.400pt}}
\multiput(280.00,359.17)(3.500,-1.000){2}{\rule{0.843pt}{0.400pt}}
\put(287,357.67){\rule{1.686pt}{0.400pt}}
\multiput(287.00,358.17)(3.500,-1.000){2}{\rule{0.843pt}{0.400pt}}
\put(294,356.67){\rule{1.686pt}{0.400pt}}
\multiput(294.00,357.17)(3.500,-1.000){2}{\rule{0.843pt}{0.400pt}}
\put(301,355.67){\rule{1.686pt}{0.400pt}}
\multiput(301.00,356.17)(3.500,-1.000){2}{\rule{0.843pt}{0.400pt}}
\put(308,354.67){\rule{1.686pt}{0.400pt}}
\multiput(308.00,355.17)(3.500,-1.000){2}{\rule{0.843pt}{0.400pt}}
\put(315,353.67){\rule{1.686pt}{0.400pt}}
\multiput(315.00,354.17)(3.500,-1.000){2}{\rule{0.843pt}{0.400pt}}
\put(322,352.67){\rule{1.445pt}{0.400pt}}
\multiput(322.00,353.17)(3.000,-1.000){2}{\rule{0.723pt}{0.400pt}}
\put(328,351.67){\rule{1.686pt}{0.400pt}}
\multiput(328.00,352.17)(3.500,-1.000){2}{\rule{0.843pt}{0.400pt}}
\put(335,350.67){\rule{1.686pt}{0.400pt}}
\multiput(335.00,351.17)(3.500,-1.000){2}{\rule{0.843pt}{0.400pt}}
\put(342,349.67){\rule{1.686pt}{0.400pt}}
\multiput(342.00,350.17)(3.500,-1.000){2}{\rule{0.843pt}{0.400pt}}
\put(349,348.17){\rule{1.500pt}{0.400pt}}
\multiput(349.00,349.17)(3.887,-2.000){2}{\rule{0.750pt}{0.400pt}}
\put(356,346.67){\rule{1.686pt}{0.400pt}}
\multiput(356.00,347.17)(3.500,-1.000){2}{\rule{0.843pt}{0.400pt}}
\put(363,345.67){\rule{1.686pt}{0.400pt}}
\multiput(363.00,346.17)(3.500,-1.000){2}{\rule{0.843pt}{0.400pt}}
\put(370,344.67){\rule{1.686pt}{0.400pt}}
\multiput(370.00,345.17)(3.500,-1.000){2}{\rule{0.843pt}{0.400pt}}
\put(377,343.67){\rule{1.686pt}{0.400pt}}
\multiput(377.00,344.17)(3.500,-1.000){2}{\rule{0.843pt}{0.400pt}}
\put(384,342.67){\rule{1.686pt}{0.400pt}}
\multiput(384.00,343.17)(3.500,-1.000){2}{\rule{0.843pt}{0.400pt}}
\put(391,341.67){\rule{1.445pt}{0.400pt}}
\multiput(391.00,342.17)(3.000,-1.000){2}{\rule{0.723pt}{0.400pt}}
\put(397,340.67){\rule{1.686pt}{0.400pt}}
\multiput(397.00,341.17)(3.500,-1.000){2}{\rule{0.843pt}{0.400pt}}
\put(404,339.67){\rule{1.686pt}{0.400pt}}
\multiput(404.00,340.17)(3.500,-1.000){2}{\rule{0.843pt}{0.400pt}}
\put(411,338.67){\rule{1.686pt}{0.400pt}}
\multiput(411.00,339.17)(3.500,-1.000){2}{\rule{0.843pt}{0.400pt}}
\put(418,337.67){\rule{1.686pt}{0.400pt}}
\multiput(418.00,338.17)(3.500,-1.000){2}{\rule{0.843pt}{0.400pt}}
\put(425,336.67){\rule{1.686pt}{0.400pt}}
\multiput(425.00,337.17)(3.500,-1.000){2}{\rule{0.843pt}{0.400pt}}
\put(432,335.67){\rule{1.686pt}{0.400pt}}
\multiput(432.00,336.17)(3.500,-1.000){2}{\rule{0.843pt}{0.400pt}}
\put(439,334.67){\rule{1.686pt}{0.400pt}}
\multiput(439.00,335.17)(3.500,-1.000){2}{\rule{0.843pt}{0.400pt}}
\put(446,333.67){\rule{1.686pt}{0.400pt}}
\multiput(446.00,334.17)(3.500,-1.000){2}{\rule{0.843pt}{0.400pt}}
\put(453,332.67){\rule{1.445pt}{0.400pt}}
\multiput(453.00,333.17)(3.000,-1.000){2}{\rule{0.723pt}{0.400pt}}
\put(459,331.67){\rule{1.686pt}{0.400pt}}
\multiput(459.00,332.17)(3.500,-1.000){2}{\rule{0.843pt}{0.400pt}}
\put(466,330.17){\rule{1.500pt}{0.400pt}}
\multiput(466.00,331.17)(3.887,-2.000){2}{\rule{0.750pt}{0.400pt}}
\put(473,328.67){\rule{1.686pt}{0.400pt}}
\multiput(473.00,329.17)(3.500,-1.000){2}{\rule{0.843pt}{0.400pt}}
\put(480,327.67){\rule{1.686pt}{0.400pt}}
\multiput(480.00,328.17)(3.500,-1.000){2}{\rule{0.843pt}{0.400pt}}
\put(487,326.67){\rule{1.686pt}{0.400pt}}
\multiput(487.00,327.17)(3.500,-1.000){2}{\rule{0.843pt}{0.400pt}}
\put(494,325.67){\rule{1.686pt}{0.400pt}}
\multiput(494.00,326.17)(3.500,-1.000){2}{\rule{0.843pt}{0.400pt}}
\put(501,324.67){\rule{1.686pt}{0.400pt}}
\multiput(501.00,325.17)(3.500,-1.000){2}{\rule{0.843pt}{0.400pt}}
\put(508,323.67){\rule{1.686pt}{0.400pt}}
\multiput(508.00,324.17)(3.500,-1.000){2}{\rule{0.843pt}{0.400pt}}
\put(515,322.67){\rule{1.445pt}{0.400pt}}
\multiput(515.00,323.17)(3.000,-1.000){2}{\rule{0.723pt}{0.400pt}}
\put(521,321.67){\rule{1.686pt}{0.400pt}}
\multiput(521.00,322.17)(3.500,-1.000){2}{\rule{0.843pt}{0.400pt}}
\put(528,320.67){\rule{1.686pt}{0.400pt}}
\multiput(528.00,321.17)(3.500,-1.000){2}{\rule{0.843pt}{0.400pt}}
\put(535,319.67){\rule{1.686pt}{0.400pt}}
\multiput(535.00,320.17)(3.500,-1.000){2}{\rule{0.843pt}{0.400pt}}
\put(542,318.67){\rule{1.686pt}{0.400pt}}
\multiput(542.00,319.17)(3.500,-1.000){2}{\rule{0.843pt}{0.400pt}}
\put(549,317.67){\rule{1.686pt}{0.400pt}}
\multiput(549.00,318.17)(3.500,-1.000){2}{\rule{0.843pt}{0.400pt}}
\put(556,316.67){\rule{1.686pt}{0.400pt}}
\multiput(556.00,317.17)(3.500,-1.000){2}{\rule{0.843pt}{0.400pt}}
\put(563,315.67){\rule{1.686pt}{0.400pt}}
\multiput(563.00,316.17)(3.500,-1.000){2}{\rule{0.843pt}{0.400pt}}
\put(570,314.67){\rule{1.686pt}{0.400pt}}
\multiput(570.00,315.17)(3.500,-1.000){2}{\rule{0.843pt}{0.400pt}}
\put(577,313.67){\rule{1.686pt}{0.400pt}}
\multiput(577.00,314.17)(3.500,-1.000){2}{\rule{0.843pt}{0.400pt}}
\put(584,312.17){\rule{1.300pt}{0.400pt}}
\multiput(584.00,313.17)(3.302,-2.000){2}{\rule{0.650pt}{0.400pt}}
\put(590,310.67){\rule{1.686pt}{0.400pt}}
\multiput(590.00,311.17)(3.500,-1.000){2}{\rule{0.843pt}{0.400pt}}
\put(597,309.67){\rule{1.686pt}{0.400pt}}
\multiput(597.00,310.17)(3.500,-1.000){2}{\rule{0.843pt}{0.400pt}}
\put(604,308.67){\rule{1.686pt}{0.400pt}}
\multiput(604.00,309.17)(3.500,-1.000){2}{\rule{0.843pt}{0.400pt}}
\put(611,307.67){\rule{1.686pt}{0.400pt}}
\multiput(611.00,308.17)(3.500,-1.000){2}{\rule{0.843pt}{0.400pt}}
\put(618,306.67){\rule{1.686pt}{0.400pt}}
\multiput(618.00,307.17)(3.500,-1.000){2}{\rule{0.843pt}{0.400pt}}
\put(625,305.67){\rule{1.686pt}{0.400pt}}
\multiput(625.00,306.17)(3.500,-1.000){2}{\rule{0.843pt}{0.400pt}}
\put(632,304.67){\rule{1.686pt}{0.400pt}}
\multiput(632.00,305.17)(3.500,-1.000){2}{\rule{0.843pt}{0.400pt}}
\put(639,303.67){\rule{1.686pt}{0.400pt}}
\multiput(639.00,304.17)(3.500,-1.000){2}{\rule{0.843pt}{0.400pt}}
\put(646,302.67){\rule{1.445pt}{0.400pt}}
\multiput(646.00,303.17)(3.000,-1.000){2}{\rule{0.723pt}{0.400pt}}
\put(652,301.67){\rule{1.686pt}{0.400pt}}
\multiput(652.00,302.17)(3.500,-1.000){2}{\rule{0.843pt}{0.400pt}}
\put(659,300.67){\rule{1.686pt}{0.400pt}}
\multiput(659.00,301.17)(3.500,-1.000){2}{\rule{0.843pt}{0.400pt}}
\put(666,299.67){\rule{1.686pt}{0.400pt}}
\multiput(666.00,300.17)(3.500,-1.000){2}{\rule{0.843pt}{0.400pt}}
\put(673,298.67){\rule{1.686pt}{0.400pt}}
\multiput(673.00,299.17)(3.500,-1.000){2}{\rule{0.843pt}{0.400pt}}
\put(680,297.67){\rule{1.686pt}{0.400pt}}
\multiput(680.00,298.17)(3.500,-1.000){2}{\rule{0.843pt}{0.400pt}}
\put(687,296.67){\rule{1.686pt}{0.400pt}}
\multiput(687.00,297.17)(3.500,-1.000){2}{\rule{0.843pt}{0.400pt}}
\put(694,295.67){\rule{1.686pt}{0.400pt}}
\multiput(694.00,296.17)(3.500,-1.000){2}{\rule{0.843pt}{0.400pt}}
\put(701,294.17){\rule{1.500pt}{0.400pt}}
\multiput(701.00,295.17)(3.887,-2.000){2}{\rule{0.750pt}{0.400pt}}
\put(708,292.67){\rule{1.445pt}{0.400pt}}
\multiput(708.00,293.17)(3.000,-1.000){2}{\rule{0.723pt}{0.400pt}}
\put(714,291.67){\rule{1.686pt}{0.400pt}}
\multiput(714.00,292.17)(3.500,-1.000){2}{\rule{0.843pt}{0.400pt}}
\put(721,290.67){\rule{1.686pt}{0.400pt}}
\multiput(721.00,291.17)(3.500,-1.000){2}{\rule{0.843pt}{0.400pt}}
\put(728,289.67){\rule{1.686pt}{0.400pt}}
\multiput(728.00,290.17)(3.500,-1.000){2}{\rule{0.843pt}{0.400pt}}
\put(735,288.67){\rule{1.686pt}{0.400pt}}
\multiput(735.00,289.17)(3.500,-1.000){2}{\rule{0.843pt}{0.400pt}}
\put(742,287.67){\rule{1.686pt}{0.400pt}}
\multiput(742.00,288.17)(3.500,-1.000){2}{\rule{0.843pt}{0.400pt}}
\put(749,286.67){\rule{1.686pt}{0.400pt}}
\multiput(749.00,287.17)(3.500,-1.000){2}{\rule{0.843pt}{0.400pt}}
\put(756,285.67){\rule{1.686pt}{0.400pt}}
\multiput(756.00,286.17)(3.500,-1.000){2}{\rule{0.843pt}{0.400pt}}
\put(763,284.67){\rule{1.686pt}{0.400pt}}
\multiput(763.00,285.17)(3.500,-1.000){2}{\rule{0.843pt}{0.400pt}}
\put(770,283.67){\rule{1.686pt}{0.400pt}}
\multiput(770.00,284.17)(3.500,-1.000){2}{\rule{0.843pt}{0.400pt}}
\put(777,282.67){\rule{1.445pt}{0.400pt}}
\multiput(777.00,283.17)(3.000,-1.000){2}{\rule{0.723pt}{0.400pt}}
\put(783,281.67){\rule{1.686pt}{0.400pt}}
\multiput(783.00,282.17)(3.500,-1.000){2}{\rule{0.843pt}{0.400pt}}
\put(790,280.67){\rule{1.686pt}{0.400pt}}
\multiput(790.00,281.17)(3.500,-1.000){2}{\rule{0.843pt}{0.400pt}}
\put(797,279.67){\rule{1.686pt}{0.400pt}}
\multiput(797.00,280.17)(3.500,-1.000){2}{\rule{0.843pt}{0.400pt}}
\put(804,278.67){\rule{1.686pt}{0.400pt}}
\multiput(804.00,279.17)(3.500,-1.000){2}{\rule{0.843pt}{0.400pt}}
\put(811,277.67){\rule{1.686pt}{0.400pt}}
\multiput(811.00,278.17)(3.500,-1.000){2}{\rule{0.843pt}{0.400pt}}
\put(818,276.67){\rule{1.686pt}{0.400pt}}
\multiput(818.00,277.17)(3.500,-1.000){2}{\rule{0.843pt}{0.400pt}}
\put(825,275.17){\rule{1.500pt}{0.400pt}}
\multiput(825.00,276.17)(3.887,-2.000){2}{\rule{0.750pt}{0.400pt}}
\put(832,273.67){\rule{1.686pt}{0.400pt}}
\multiput(832.00,274.17)(3.500,-1.000){2}{\rule{0.843pt}{0.400pt}}
\put(839,272.67){\rule{1.445pt}{0.400pt}}
\multiput(839.00,273.17)(3.000,-1.000){2}{\rule{0.723pt}{0.400pt}}
\put(845,271.67){\rule{1.686pt}{0.400pt}}
\multiput(845.00,272.17)(3.500,-1.000){2}{\rule{0.843pt}{0.400pt}}
\put(852,270.67){\rule{1.686pt}{0.400pt}}
\multiput(852.00,271.17)(3.500,-1.000){2}{\rule{0.843pt}{0.400pt}}
\put(859,269.67){\rule{1.686pt}{0.400pt}}
\multiput(859.00,270.17)(3.500,-1.000){2}{\rule{0.843pt}{0.400pt}}
\put(866,268.67){\rule{1.686pt}{0.400pt}}
\multiput(866.00,269.17)(3.500,-1.000){2}{\rule{0.843pt}{0.400pt}}
\put(873,267.67){\rule{1.686pt}{0.400pt}}
\multiput(873.00,268.17)(3.500,-1.000){2}{\rule{0.843pt}{0.400pt}}
\put(880,266.67){\rule{1.686pt}{0.400pt}}
\multiput(880.00,267.17)(3.500,-1.000){2}{\rule{0.843pt}{0.400pt}}
\put(887,265.67){\rule{1.686pt}{0.400pt}}
\multiput(887.00,266.17)(3.500,-1.000){2}{\rule{0.843pt}{0.400pt}}
\put(232,379.67){\rule{1.686pt}{0.400pt}}
\multiput(232.00,379.17)(3.500,1.000){2}{\rule{0.843pt}{0.400pt}}
\put(239,380.67){\rule{1.686pt}{0.400pt}}
\multiput(239.00,380.17)(3.500,1.000){2}{\rule{0.843pt}{0.400pt}}
\put(246,381.67){\rule{1.686pt}{0.400pt}}
\multiput(246.00,381.17)(3.500,1.000){2}{\rule{0.843pt}{0.400pt}}
\put(253,382.67){\rule{1.686pt}{0.400pt}}
\multiput(253.00,382.17)(3.500,1.000){2}{\rule{0.843pt}{0.400pt}}
\put(260,383.67){\rule{1.445pt}{0.400pt}}
\multiput(260.00,383.17)(3.000,1.000){2}{\rule{0.723pt}{0.400pt}}
\put(266,384.67){\rule{1.686pt}{0.400pt}}
\multiput(266.00,384.17)(3.500,1.000){2}{\rule{0.843pt}{0.400pt}}
\put(273,385.67){\rule{1.686pt}{0.400pt}}
\multiput(273.00,385.17)(3.500,1.000){2}{\rule{0.843pt}{0.400pt}}
\put(280,386.67){\rule{1.686pt}{0.400pt}}
\multiput(280.00,386.17)(3.500,1.000){2}{\rule{0.843pt}{0.400pt}}
\put(1320.0,756.0){\rule[-0.200pt]{24.090pt}{0.400pt}}
\put(294,387.67){\rule{1.686pt}{0.400pt}}
\multiput(294.00,387.17)(3.500,1.000){2}{\rule{0.843pt}{0.400pt}}
\put(301,388.67){\rule{1.686pt}{0.400pt}}
\multiput(301.00,388.17)(3.500,1.000){2}{\rule{0.843pt}{0.400pt}}
\put(287.0,388.0){\rule[-0.200pt]{1.686pt}{0.400pt}}
\put(315,389.67){\rule{1.686pt}{0.400pt}}
\multiput(315.00,389.17)(3.500,1.000){2}{\rule{0.843pt}{0.400pt}}
\put(308.0,390.0){\rule[-0.200pt]{1.686pt}{0.400pt}}
\put(322.0,391.0){\rule[-0.200pt]{1.686pt}{0.400pt}}
\put(335,390.67){\rule{1.686pt}{0.400pt}}
\multiput(335.00,390.17)(3.500,1.000){2}{\rule{0.843pt}{0.400pt}}
\put(329.0,391.0){\rule[-0.200pt]{1.445pt}{0.400pt}}
\put(342.0,392.0){\rule[-0.200pt]{1.686pt}{0.400pt}}
\put(356,391.67){\rule{1.686pt}{0.400pt}}
\multiput(356.00,391.17)(3.500,1.000){2}{\rule{0.843pt}{0.400pt}}
\put(349.0,392.0){\rule[-0.200pt]{1.686pt}{0.400pt}}
\put(363.0,393.0){\rule[-0.200pt]{1.686pt}{0.400pt}}
\put(370.0,393.0){\rule[-0.200pt]{1.686pt}{0.400pt}}
\put(377.0,393.0){\rule[-0.200pt]{1.686pt}{0.400pt}}
\put(384.0,393.0){\rule[-0.200pt]{1.686pt}{0.400pt}}
\put(391.0,393.0){\rule[-0.200pt]{1.445pt}{0.400pt}}
\put(397.0,393.0){\rule[-0.200pt]{1.686pt}{0.400pt}}
\put(404.0,393.0){\rule[-0.200pt]{1.686pt}{0.400pt}}
\put(418,391.67){\rule{1.686pt}{0.400pt}}
\multiput(418.00,392.17)(3.500,-1.000){2}{\rule{0.843pt}{0.400pt}}
\put(411.0,393.0){\rule[-0.200pt]{1.686pt}{0.400pt}}
\put(425.0,392.0){\rule[-0.200pt]{1.686pt}{0.400pt}}
\put(432.0,392.0){\rule[-0.200pt]{1.686pt}{0.400pt}}
\put(446,390.67){\rule{1.686pt}{0.400pt}}
\multiput(446.00,391.17)(3.500,-1.000){2}{\rule{0.843pt}{0.400pt}}
\put(439.0,392.0){\rule[-0.200pt]{1.686pt}{0.400pt}}
\put(453.0,391.0){\rule[-0.200pt]{1.445pt}{0.400pt}}
\put(466,389.67){\rule{1.686pt}{0.400pt}}
\multiput(466.00,390.17)(3.500,-1.000){2}{\rule{0.843pt}{0.400pt}}
\put(459.0,391.0){\rule[-0.200pt]{1.686pt}{0.400pt}}
\put(480,388.67){\rule{1.686pt}{0.400pt}}
\multiput(480.00,389.17)(3.500,-1.000){2}{\rule{0.843pt}{0.400pt}}
\put(487,387.67){\rule{1.686pt}{0.400pt}}
\multiput(487.00,388.17)(3.500,-1.000){2}{\rule{0.843pt}{0.400pt}}
\put(473.0,390.0){\rule[-0.200pt]{1.686pt}{0.400pt}}
\put(501,386.67){\rule{1.686pt}{0.400pt}}
\multiput(501.00,387.17)(3.500,-1.000){2}{\rule{0.843pt}{0.400pt}}
\put(508,385.67){\rule{1.686pt}{0.400pt}}
\multiput(508.00,386.17)(3.500,-1.000){2}{\rule{0.843pt}{0.400pt}}
\put(494.0,388.0){\rule[-0.200pt]{1.686pt}{0.400pt}}
\put(522,384.67){\rule{1.445pt}{0.400pt}}
\multiput(522.00,385.17)(3.000,-1.000){2}{\rule{0.723pt}{0.400pt}}
\put(528,383.67){\rule{1.686pt}{0.400pt}}
\multiput(528.00,384.17)(3.500,-1.000){2}{\rule{0.843pt}{0.400pt}}
\put(535,382.67){\rule{1.686pt}{0.400pt}}
\multiput(535.00,383.17)(3.500,-1.000){2}{\rule{0.843pt}{0.400pt}}
\put(542,381.67){\rule{1.686pt}{0.400pt}}
\multiput(542.00,382.17)(3.500,-1.000){2}{\rule{0.843pt}{0.400pt}}
\put(549,380.67){\rule{1.686pt}{0.400pt}}
\multiput(549.00,381.17)(3.500,-1.000){2}{\rule{0.843pt}{0.400pt}}
\put(556,379.67){\rule{1.686pt}{0.400pt}}
\multiput(556.00,380.17)(3.500,-1.000){2}{\rule{0.843pt}{0.400pt}}
\put(563,378.67){\rule{1.686pt}{0.400pt}}
\multiput(563.00,379.17)(3.500,-1.000){2}{\rule{0.843pt}{0.400pt}}
\put(570,377.67){\rule{1.686pt}{0.400pt}}
\multiput(570.00,378.17)(3.500,-1.000){2}{\rule{0.843pt}{0.400pt}}
\put(577,376.67){\rule{1.686pt}{0.400pt}}
\multiput(577.00,377.17)(3.500,-1.000){2}{\rule{0.843pt}{0.400pt}}
\put(584,375.67){\rule{1.445pt}{0.400pt}}
\multiput(584.00,376.17)(3.000,-1.000){2}{\rule{0.723pt}{0.400pt}}
\put(590,374.67){\rule{1.686pt}{0.400pt}}
\multiput(590.00,375.17)(3.500,-1.000){2}{\rule{0.843pt}{0.400pt}}
\put(597,373.67){\rule{1.686pt}{0.400pt}}
\multiput(597.00,374.17)(3.500,-1.000){2}{\rule{0.843pt}{0.400pt}}
\put(604,372.17){\rule{1.500pt}{0.400pt}}
\multiput(604.00,373.17)(3.887,-2.000){2}{\rule{0.750pt}{0.400pt}}
\put(611,370.67){\rule{1.686pt}{0.400pt}}
\multiput(611.00,371.17)(3.500,-1.000){2}{\rule{0.843pt}{0.400pt}}
\put(618,369.67){\rule{1.686pt}{0.400pt}}
\multiput(618.00,370.17)(3.500,-1.000){2}{\rule{0.843pt}{0.400pt}}
\put(625,368.17){\rule{1.500pt}{0.400pt}}
\multiput(625.00,369.17)(3.887,-2.000){2}{\rule{0.750pt}{0.400pt}}
\put(632,366.67){\rule{1.686pt}{0.400pt}}
\multiput(632.00,367.17)(3.500,-1.000){2}{\rule{0.843pt}{0.400pt}}
\put(639,365.67){\rule{1.686pt}{0.400pt}}
\multiput(639.00,366.17)(3.500,-1.000){2}{\rule{0.843pt}{0.400pt}}
\put(646,364.17){\rule{1.500pt}{0.400pt}}
\multiput(646.00,365.17)(3.887,-2.000){2}{\rule{0.750pt}{0.400pt}}
\put(653,362.67){\rule{1.445pt}{0.400pt}}
\multiput(653.00,363.17)(3.000,-1.000){2}{\rule{0.723pt}{0.400pt}}
\put(659,361.17){\rule{1.500pt}{0.400pt}}
\multiput(659.00,362.17)(3.887,-2.000){2}{\rule{0.750pt}{0.400pt}}
\put(666,359.17){\rule{1.500pt}{0.400pt}}
\multiput(666.00,360.17)(3.887,-2.000){2}{\rule{0.750pt}{0.400pt}}
\put(673,357.67){\rule{1.686pt}{0.400pt}}
\multiput(673.00,358.17)(3.500,-1.000){2}{\rule{0.843pt}{0.400pt}}
\put(680,356.17){\rule{1.500pt}{0.400pt}}
\multiput(680.00,357.17)(3.887,-2.000){2}{\rule{0.750pt}{0.400pt}}
\put(687,354.17){\rule{1.500pt}{0.400pt}}
\multiput(687.00,355.17)(3.887,-2.000){2}{\rule{0.750pt}{0.400pt}}
\put(694,352.67){\rule{1.686pt}{0.400pt}}
\multiput(694.00,353.17)(3.500,-1.000){2}{\rule{0.843pt}{0.400pt}}
\put(701,351.17){\rule{1.500pt}{0.400pt}}
\multiput(701.00,352.17)(3.887,-2.000){2}{\rule{0.750pt}{0.400pt}}
\put(708,349.17){\rule{1.500pt}{0.400pt}}
\multiput(708.00,350.17)(3.887,-2.000){2}{\rule{0.750pt}{0.400pt}}
\put(715,347.17){\rule{1.300pt}{0.400pt}}
\multiput(715.00,348.17)(3.302,-2.000){2}{\rule{0.650pt}{0.400pt}}
\put(721,345.17){\rule{1.500pt}{0.400pt}}
\multiput(721.00,346.17)(3.887,-2.000){2}{\rule{0.750pt}{0.400pt}}
\put(728,343.17){\rule{1.500pt}{0.400pt}}
\multiput(728.00,344.17)(3.887,-2.000){2}{\rule{0.750pt}{0.400pt}}
\put(735,341.17){\rule{1.500pt}{0.400pt}}
\multiput(735.00,342.17)(3.887,-2.000){2}{\rule{0.750pt}{0.400pt}}
\put(742,339.17){\rule{1.500pt}{0.400pt}}
\multiput(742.00,340.17)(3.887,-2.000){2}{\rule{0.750pt}{0.400pt}}
\put(749,337.17){\rule{1.500pt}{0.400pt}}
\multiput(749.00,338.17)(3.887,-2.000){2}{\rule{0.750pt}{0.400pt}}
\put(756,335.17){\rule{1.500pt}{0.400pt}}
\multiput(756.00,336.17)(3.887,-2.000){2}{\rule{0.750pt}{0.400pt}}
\multiput(763.00,333.95)(1.355,-0.447){3}{\rule{1.033pt}{0.108pt}}
\multiput(763.00,334.17)(4.855,-3.000){2}{\rule{0.517pt}{0.400pt}}
\put(770,330.17){\rule{1.500pt}{0.400pt}}
\multiput(770.00,331.17)(3.887,-2.000){2}{\rule{0.750pt}{0.400pt}}
\put(777,328.17){\rule{1.300pt}{0.400pt}}
\multiput(777.00,329.17)(3.302,-2.000){2}{\rule{0.650pt}{0.400pt}}
\put(783,326.17){\rule{1.500pt}{0.400pt}}
\multiput(783.00,327.17)(3.887,-2.000){2}{\rule{0.750pt}{0.400pt}}
\multiput(790.00,324.95)(1.355,-0.447){3}{\rule{1.033pt}{0.108pt}}
\multiput(790.00,325.17)(4.855,-3.000){2}{\rule{0.517pt}{0.400pt}}
\put(797,321.17){\rule{1.500pt}{0.400pt}}
\multiput(797.00,322.17)(3.887,-2.000){2}{\rule{0.750pt}{0.400pt}}
\multiput(804.00,319.95)(1.355,-0.447){3}{\rule{1.033pt}{0.108pt}}
\multiput(804.00,320.17)(4.855,-3.000){2}{\rule{0.517pt}{0.400pt}}
\put(811,316.17){\rule{1.500pt}{0.400pt}}
\multiput(811.00,317.17)(3.887,-2.000){2}{\rule{0.750pt}{0.400pt}}
\multiput(818.00,314.95)(1.355,-0.447){3}{\rule{1.033pt}{0.108pt}}
\multiput(818.00,315.17)(4.855,-3.000){2}{\rule{0.517pt}{0.400pt}}
\put(825,311.17){\rule{1.500pt}{0.400pt}}
\multiput(825.00,312.17)(3.887,-2.000){2}{\rule{0.750pt}{0.400pt}}
\multiput(832.00,309.95)(1.355,-0.447){3}{\rule{1.033pt}{0.108pt}}
\multiput(832.00,310.17)(4.855,-3.000){2}{\rule{0.517pt}{0.400pt}}
\multiput(839.00,306.95)(1.355,-0.447){3}{\rule{1.033pt}{0.108pt}}
\multiput(839.00,307.17)(4.855,-3.000){2}{\rule{0.517pt}{0.400pt}}
\put(846,303.17){\rule{1.300pt}{0.400pt}}
\multiput(846.00,304.17)(3.302,-2.000){2}{\rule{0.650pt}{0.400pt}}
\multiput(852.00,301.95)(1.355,-0.447){3}{\rule{1.033pt}{0.108pt}}
\multiput(852.00,302.17)(4.855,-3.000){2}{\rule{0.517pt}{0.400pt}}
\multiput(859.00,298.95)(1.355,-0.447){3}{\rule{1.033pt}{0.108pt}}
\multiput(859.00,299.17)(4.855,-3.000){2}{\rule{0.517pt}{0.400pt}}
\multiput(866.00,295.95)(1.355,-0.447){3}{\rule{1.033pt}{0.108pt}}
\multiput(866.00,296.17)(4.855,-3.000){2}{\rule{0.517pt}{0.400pt}}
\multiput(873.00,292.95)(1.355,-0.447){3}{\rule{1.033pt}{0.108pt}}
\multiput(873.00,293.17)(4.855,-3.000){2}{\rule{0.517pt}{0.400pt}}
\multiput(880.00,289.95)(1.355,-0.447){3}{\rule{1.033pt}{0.108pt}}
\multiput(880.00,290.17)(4.855,-3.000){2}{\rule{0.517pt}{0.400pt}}
\multiput(887.00,286.95)(1.355,-0.447){3}{\rule{1.033pt}{0.108pt}}
\multiput(887.00,287.17)(4.855,-3.000){2}{\rule{0.517pt}{0.400pt}}
\multiput(894.00,283.95)(1.355,-0.447){3}{\rule{1.033pt}{0.108pt}}
\multiput(894.00,284.17)(4.855,-3.000){2}{\rule{0.517pt}{0.400pt}}
\multiput(901.00,280.95)(1.355,-0.447){3}{\rule{1.033pt}{0.108pt}}
\multiput(901.00,281.17)(4.855,-3.000){2}{\rule{0.517pt}{0.400pt}}
\multiput(908.00,277.94)(0.774,-0.468){5}{\rule{0.700pt}{0.113pt}}
\multiput(908.00,278.17)(4.547,-4.000){2}{\rule{0.350pt}{0.400pt}}
\multiput(253.00,390.61)(1.355,0.447){3}{\rule{1.033pt}{0.108pt}}
\multiput(253.00,389.17)(4.855,3.000){2}{\rule{0.517pt}{0.400pt}}
\multiput(260.00,393.61)(1.355,0.447){3}{\rule{1.033pt}{0.108pt}}
\multiput(260.00,392.17)(4.855,3.000){2}{\rule{0.517pt}{0.400pt}}
\put(267,396.17){\rule{1.300pt}{0.400pt}}
\multiput(267.00,395.17)(3.302,2.000){2}{\rule{0.650pt}{0.400pt}}
\multiput(273.00,398.61)(1.355,0.447){3}{\rule{1.033pt}{0.108pt}}
\multiput(273.00,397.17)(4.855,3.000){2}{\rule{0.517pt}{0.400pt}}
\multiput(280.00,401.61)(1.355,0.447){3}{\rule{1.033pt}{0.108pt}}
\multiput(280.00,400.17)(4.855,3.000){2}{\rule{0.517pt}{0.400pt}}
\put(287,404.17){\rule{1.500pt}{0.400pt}}
\multiput(287.00,403.17)(3.887,2.000){2}{\rule{0.750pt}{0.400pt}}
\multiput(294.00,406.61)(1.355,0.447){3}{\rule{1.033pt}{0.108pt}}
\multiput(294.00,405.17)(4.855,3.000){2}{\rule{0.517pt}{0.400pt}}
\put(301,409.17){\rule{1.500pt}{0.400pt}}
\multiput(301.00,408.17)(3.887,2.000){2}{\rule{0.750pt}{0.400pt}}
\put(308,411.17){\rule{1.500pt}{0.400pt}}
\multiput(308.00,410.17)(3.887,2.000){2}{\rule{0.750pt}{0.400pt}}
\put(315,413.17){\rule{1.500pt}{0.400pt}}
\multiput(315.00,412.17)(3.887,2.000){2}{\rule{0.750pt}{0.400pt}}
\put(322,415.17){\rule{1.500pt}{0.400pt}}
\multiput(322.00,414.17)(3.887,2.000){2}{\rule{0.750pt}{0.400pt}}
\put(329,417.17){\rule{1.300pt}{0.400pt}}
\multiput(329.00,416.17)(3.302,2.000){2}{\rule{0.650pt}{0.400pt}}
\put(335,419.17){\rule{1.500pt}{0.400pt}}
\multiput(335.00,418.17)(3.887,2.000){2}{\rule{0.750pt}{0.400pt}}
\put(342,421.17){\rule{1.500pt}{0.400pt}}
\multiput(342.00,420.17)(3.887,2.000){2}{\rule{0.750pt}{0.400pt}}
\put(349,422.67){\rule{1.686pt}{0.400pt}}
\multiput(349.00,422.17)(3.500,1.000){2}{\rule{0.843pt}{0.400pt}}
\put(356,424.17){\rule{1.500pt}{0.400pt}}
\multiput(356.00,423.17)(3.887,2.000){2}{\rule{0.750pt}{0.400pt}}
\put(363,425.67){\rule{1.686pt}{0.400pt}}
\multiput(363.00,425.17)(3.500,1.000){2}{\rule{0.843pt}{0.400pt}}
\put(370,427.17){\rule{1.500pt}{0.400pt}}
\multiput(370.00,426.17)(3.887,2.000){2}{\rule{0.750pt}{0.400pt}}
\put(377,428.67){\rule{1.686pt}{0.400pt}}
\multiput(377.00,428.17)(3.500,1.000){2}{\rule{0.843pt}{0.400pt}}
\put(384,429.67){\rule{1.686pt}{0.400pt}}
\multiput(384.00,429.17)(3.500,1.000){2}{\rule{0.843pt}{0.400pt}}
\put(391,430.67){\rule{1.686pt}{0.400pt}}
\multiput(391.00,430.17)(3.500,1.000){2}{\rule{0.843pt}{0.400pt}}
\put(398,432.17){\rule{1.300pt}{0.400pt}}
\multiput(398.00,431.17)(3.302,2.000){2}{\rule{0.650pt}{0.400pt}}
\put(404,433.67){\rule{1.686pt}{0.400pt}}
\multiput(404.00,433.17)(3.500,1.000){2}{\rule{0.843pt}{0.400pt}}
\put(515.0,386.0){\rule[-0.200pt]{1.686pt}{0.400pt}}
\put(418,434.67){\rule{1.686pt}{0.400pt}}
\multiput(418.00,434.17)(3.500,1.000){2}{\rule{0.843pt}{0.400pt}}
\put(425,435.67){\rule{1.686pt}{0.400pt}}
\multiput(425.00,435.17)(3.500,1.000){2}{\rule{0.843pt}{0.400pt}}
\put(432,436.67){\rule{1.686pt}{0.400pt}}
\multiput(432.00,436.17)(3.500,1.000){2}{\rule{0.843pt}{0.400pt}}
\put(411.0,435.0){\rule[-0.200pt]{1.686pt}{0.400pt}}
\put(446,437.67){\rule{1.686pt}{0.400pt}}
\multiput(446.00,437.17)(3.500,1.000){2}{\rule{0.843pt}{0.400pt}}
\put(439.0,438.0){\rule[-0.200pt]{1.686pt}{0.400pt}}
\put(460,438.67){\rule{1.445pt}{0.400pt}}
\multiput(460.00,438.17)(3.000,1.000){2}{\rule{0.723pt}{0.400pt}}
\put(453.0,439.0){\rule[-0.200pt]{1.686pt}{0.400pt}}
\put(466.0,440.0){\rule[-0.200pt]{1.686pt}{0.400pt}}
\put(473.0,440.0){\rule[-0.200pt]{1.686pt}{0.400pt}}
\put(480.0,440.0){\rule[-0.200pt]{1.686pt}{0.400pt}}
\put(487.0,440.0){\rule[-0.200pt]{1.686pt}{0.400pt}}
\put(494.0,440.0){\rule[-0.200pt]{1.686pt}{0.400pt}}
\put(501.0,440.0){\rule[-0.200pt]{1.686pt}{0.400pt}}
\put(508.0,440.0){\rule[-0.200pt]{1.686pt}{0.400pt}}
\put(522,438.67){\rule{1.445pt}{0.400pt}}
\multiput(522.00,439.17)(3.000,-1.000){2}{\rule{0.723pt}{0.400pt}}
\put(515.0,440.0){\rule[-0.200pt]{1.686pt}{0.400pt}}
\put(535,437.67){\rule{1.686pt}{0.400pt}}
\multiput(535.00,438.17)(3.500,-1.000){2}{\rule{0.843pt}{0.400pt}}
\put(528.0,439.0){\rule[-0.200pt]{1.686pt}{0.400pt}}
\put(549,436.67){\rule{1.686pt}{0.400pt}}
\multiput(549.00,437.17)(3.500,-1.000){2}{\rule{0.843pt}{0.400pt}}
\put(542.0,438.0){\rule[-0.200pt]{1.686pt}{0.400pt}}
\put(563,435.67){\rule{1.686pt}{0.400pt}}
\multiput(563.00,436.17)(3.500,-1.000){2}{\rule{0.843pt}{0.400pt}}
\put(570,434.67){\rule{1.686pt}{0.400pt}}
\multiput(570.00,435.17)(3.500,-1.000){2}{\rule{0.843pt}{0.400pt}}
\put(577,433.67){\rule{1.686pt}{0.400pt}}
\multiput(577.00,434.17)(3.500,-1.000){2}{\rule{0.843pt}{0.400pt}}
\put(584,432.67){\rule{1.686pt}{0.400pt}}
\multiput(584.00,433.17)(3.500,-1.000){2}{\rule{0.843pt}{0.400pt}}
\put(591,431.67){\rule{1.445pt}{0.400pt}}
\multiput(591.00,432.17)(3.000,-1.000){2}{\rule{0.723pt}{0.400pt}}
\put(597,430.67){\rule{1.686pt}{0.400pt}}
\multiput(597.00,431.17)(3.500,-1.000){2}{\rule{0.843pt}{0.400pt}}
\put(604,429.67){\rule{1.686pt}{0.400pt}}
\multiput(604.00,430.17)(3.500,-1.000){2}{\rule{0.843pt}{0.400pt}}
\put(611,428.17){\rule{1.500pt}{0.400pt}}
\multiput(611.00,429.17)(3.887,-2.000){2}{\rule{0.750pt}{0.400pt}}
\put(618,426.67){\rule{1.686pt}{0.400pt}}
\multiput(618.00,427.17)(3.500,-1.000){2}{\rule{0.843pt}{0.400pt}}
\put(625,425.67){\rule{1.686pt}{0.400pt}}
\multiput(625.00,426.17)(3.500,-1.000){2}{\rule{0.843pt}{0.400pt}}
\put(632,424.17){\rule{1.500pt}{0.400pt}}
\multiput(632.00,425.17)(3.887,-2.000){2}{\rule{0.750pt}{0.400pt}}
\put(639,422.67){\rule{1.686pt}{0.400pt}}
\multiput(639.00,423.17)(3.500,-1.000){2}{\rule{0.843pt}{0.400pt}}
\put(646,421.17){\rule{1.500pt}{0.400pt}}
\multiput(646.00,422.17)(3.887,-2.000){2}{\rule{0.750pt}{0.400pt}}
\put(653,419.17){\rule{1.300pt}{0.400pt}}
\multiput(653.00,420.17)(3.302,-2.000){2}{\rule{0.650pt}{0.400pt}}
\put(659,417.17){\rule{1.500pt}{0.400pt}}
\multiput(659.00,418.17)(3.887,-2.000){2}{\rule{0.750pt}{0.400pt}}
\put(666,415.67){\rule{1.686pt}{0.400pt}}
\multiput(666.00,416.17)(3.500,-1.000){2}{\rule{0.843pt}{0.400pt}}
\put(673,414.17){\rule{1.500pt}{0.400pt}}
\multiput(673.00,415.17)(3.887,-2.000){2}{\rule{0.750pt}{0.400pt}}
\put(680,412.17){\rule{1.500pt}{0.400pt}}
\multiput(680.00,413.17)(3.887,-2.000){2}{\rule{0.750pt}{0.400pt}}
\multiput(687.00,410.95)(1.355,-0.447){3}{\rule{1.033pt}{0.108pt}}
\multiput(687.00,411.17)(4.855,-3.000){2}{\rule{0.517pt}{0.400pt}}
\put(694,407.17){\rule{1.500pt}{0.400pt}}
\multiput(694.00,408.17)(3.887,-2.000){2}{\rule{0.750pt}{0.400pt}}
\put(701,405.17){\rule{1.500pt}{0.400pt}}
\multiput(701.00,406.17)(3.887,-2.000){2}{\rule{0.750pt}{0.400pt}}
\put(708,403.17){\rule{1.500pt}{0.400pt}}
\multiput(708.00,404.17)(3.887,-2.000){2}{\rule{0.750pt}{0.400pt}}
\multiput(715.00,401.95)(1.132,-0.447){3}{\rule{0.900pt}{0.108pt}}
\multiput(715.00,402.17)(4.132,-3.000){2}{\rule{0.450pt}{0.400pt}}
\put(721,398.17){\rule{1.500pt}{0.400pt}}
\multiput(721.00,399.17)(3.887,-2.000){2}{\rule{0.750pt}{0.400pt}}
\multiput(728.00,396.95)(1.355,-0.447){3}{\rule{1.033pt}{0.108pt}}
\multiput(728.00,397.17)(4.855,-3.000){2}{\rule{0.517pt}{0.400pt}}
\put(735,393.17){\rule{1.500pt}{0.400pt}}
\multiput(735.00,394.17)(3.887,-2.000){2}{\rule{0.750pt}{0.400pt}}
\multiput(742.00,391.95)(1.355,-0.447){3}{\rule{1.033pt}{0.108pt}}
\multiput(742.00,392.17)(4.855,-3.000){2}{\rule{0.517pt}{0.400pt}}
\multiput(749.00,388.95)(1.355,-0.447){3}{\rule{1.033pt}{0.108pt}}
\multiput(749.00,389.17)(4.855,-3.000){2}{\rule{0.517pt}{0.400pt}}
\multiput(756.00,385.95)(1.355,-0.447){3}{\rule{1.033pt}{0.108pt}}
\multiput(756.00,386.17)(4.855,-3.000){2}{\rule{0.517pt}{0.400pt}}
\multiput(763.00,382.95)(1.355,-0.447){3}{\rule{1.033pt}{0.108pt}}
\multiput(763.00,383.17)(4.855,-3.000){2}{\rule{0.517pt}{0.400pt}}
\multiput(770.00,379.95)(1.355,-0.447){3}{\rule{1.033pt}{0.108pt}}
\multiput(770.00,380.17)(4.855,-3.000){2}{\rule{0.517pt}{0.400pt}}
\multiput(777.00,376.95)(1.355,-0.447){3}{\rule{1.033pt}{0.108pt}}
\multiput(777.00,377.17)(4.855,-3.000){2}{\rule{0.517pt}{0.400pt}}
\multiput(784.00,373.95)(1.132,-0.447){3}{\rule{0.900pt}{0.108pt}}
\multiput(784.00,374.17)(4.132,-3.000){2}{\rule{0.450pt}{0.400pt}}
\multiput(790.00,370.95)(1.355,-0.447){3}{\rule{1.033pt}{0.108pt}}
\multiput(790.00,371.17)(4.855,-3.000){2}{\rule{0.517pt}{0.400pt}}
\multiput(797.00,367.95)(1.355,-0.447){3}{\rule{1.033pt}{0.108pt}}
\multiput(797.00,368.17)(4.855,-3.000){2}{\rule{0.517pt}{0.400pt}}
\multiput(804.00,364.94)(0.920,-0.468){5}{\rule{0.800pt}{0.113pt}}
\multiput(804.00,365.17)(5.340,-4.000){2}{\rule{0.400pt}{0.400pt}}
\multiput(811.00,360.95)(1.355,-0.447){3}{\rule{1.033pt}{0.108pt}}
\multiput(811.00,361.17)(4.855,-3.000){2}{\rule{0.517pt}{0.400pt}}
\multiput(818.00,357.94)(0.920,-0.468){5}{\rule{0.800pt}{0.113pt}}
\multiput(818.00,358.17)(5.340,-4.000){2}{\rule{0.400pt}{0.400pt}}
\multiput(825.00,353.94)(0.920,-0.468){5}{\rule{0.800pt}{0.113pt}}
\multiput(825.00,354.17)(5.340,-4.000){2}{\rule{0.400pt}{0.400pt}}
\multiput(832.00,349.95)(1.355,-0.447){3}{\rule{1.033pt}{0.108pt}}
\multiput(832.00,350.17)(4.855,-3.000){2}{\rule{0.517pt}{0.400pt}}
\multiput(839.00,346.94)(0.920,-0.468){5}{\rule{0.800pt}{0.113pt}}
\multiput(839.00,347.17)(5.340,-4.000){2}{\rule{0.400pt}{0.400pt}}
\multiput(846.00,342.94)(0.774,-0.468){5}{\rule{0.700pt}{0.113pt}}
\multiput(846.00,343.17)(4.547,-4.000){2}{\rule{0.350pt}{0.400pt}}
\multiput(852.00,338.94)(0.920,-0.468){5}{\rule{0.800pt}{0.113pt}}
\multiput(852.00,339.17)(5.340,-4.000){2}{\rule{0.400pt}{0.400pt}}
\multiput(859.00,334.94)(0.920,-0.468){5}{\rule{0.800pt}{0.113pt}}
\multiput(859.00,335.17)(5.340,-4.000){2}{\rule{0.400pt}{0.400pt}}
\multiput(866.00,330.93)(0.710,-0.477){7}{\rule{0.660pt}{0.115pt}}
\multiput(866.00,331.17)(5.630,-5.000){2}{\rule{0.330pt}{0.400pt}}
\multiput(873.00,325.94)(0.920,-0.468){5}{\rule{0.800pt}{0.113pt}}
\multiput(873.00,326.17)(5.340,-4.000){2}{\rule{0.400pt}{0.400pt}}
\multiput(880.00,321.94)(0.920,-0.468){5}{\rule{0.800pt}{0.113pt}}
\multiput(880.00,322.17)(5.340,-4.000){2}{\rule{0.400pt}{0.400pt}}
\multiput(887.00,317.93)(0.710,-0.477){7}{\rule{0.660pt}{0.115pt}}
\multiput(887.00,318.17)(5.630,-5.000){2}{\rule{0.330pt}{0.400pt}}
\multiput(894.00,312.94)(0.920,-0.468){5}{\rule{0.800pt}{0.113pt}}
\multiput(894.00,313.17)(5.340,-4.000){2}{\rule{0.400pt}{0.400pt}}
\multiput(901.00,308.93)(0.710,-0.477){7}{\rule{0.660pt}{0.115pt}}
\multiput(901.00,309.17)(5.630,-5.000){2}{\rule{0.330pt}{0.400pt}}
\multiput(908.00,303.93)(0.710,-0.477){7}{\rule{0.660pt}{0.115pt}}
\multiput(908.00,304.17)(5.630,-5.000){2}{\rule{0.330pt}{0.400pt}}
\multiput(915.00,298.93)(0.599,-0.477){7}{\rule{0.580pt}{0.115pt}}
\multiput(915.00,299.17)(4.796,-5.000){2}{\rule{0.290pt}{0.400pt}}
\multiput(921.00,293.93)(0.710,-0.477){7}{\rule{0.660pt}{0.115pt}}
\multiput(921.00,294.17)(5.630,-5.000){2}{\rule{0.330pt}{0.400pt}}
\multiput(928.00,288.93)(0.710,-0.477){7}{\rule{0.660pt}{0.115pt}}
\multiput(928.00,289.17)(5.630,-5.000){2}{\rule{0.330pt}{0.400pt}}
\multiput(273.00,399.59)(0.710,0.477){7}{\rule{0.660pt}{0.115pt}}
\multiput(273.00,398.17)(5.630,5.000){2}{\rule{0.330pt}{0.400pt}}
\multiput(280.00,404.60)(0.920,0.468){5}{\rule{0.800pt}{0.113pt}}
\multiput(280.00,403.17)(5.340,4.000){2}{\rule{0.400pt}{0.400pt}}
\multiput(287.00,408.59)(0.710,0.477){7}{\rule{0.660pt}{0.115pt}}
\multiput(287.00,407.17)(5.630,5.000){2}{\rule{0.330pt}{0.400pt}}
\multiput(294.00,413.60)(0.920,0.468){5}{\rule{0.800pt}{0.113pt}}
\multiput(294.00,412.17)(5.340,4.000){2}{\rule{0.400pt}{0.400pt}}
\multiput(301.00,417.60)(0.920,0.468){5}{\rule{0.800pt}{0.113pt}}
\multiput(301.00,416.17)(5.340,4.000){2}{\rule{0.400pt}{0.400pt}}
\multiput(308.00,421.60)(0.920,0.468){5}{\rule{0.800pt}{0.113pt}}
\multiput(308.00,420.17)(5.340,4.000){2}{\rule{0.400pt}{0.400pt}}
\multiput(315.00,425.60)(0.920,0.468){5}{\rule{0.800pt}{0.113pt}}
\multiput(315.00,424.17)(5.340,4.000){2}{\rule{0.400pt}{0.400pt}}
\multiput(322.00,429.61)(1.355,0.447){3}{\rule{1.033pt}{0.108pt}}
\multiput(322.00,428.17)(4.855,3.000){2}{\rule{0.517pt}{0.400pt}}
\multiput(329.00,432.60)(0.920,0.468){5}{\rule{0.800pt}{0.113pt}}
\multiput(329.00,431.17)(5.340,4.000){2}{\rule{0.400pt}{0.400pt}}
\multiput(336.00,436.61)(1.132,0.447){3}{\rule{0.900pt}{0.108pt}}
\multiput(336.00,435.17)(4.132,3.000){2}{\rule{0.450pt}{0.400pt}}
\multiput(342.00,439.61)(1.355,0.447){3}{\rule{1.033pt}{0.108pt}}
\multiput(342.00,438.17)(4.855,3.000){2}{\rule{0.517pt}{0.400pt}}
\multiput(349.00,442.61)(1.355,0.447){3}{\rule{1.033pt}{0.108pt}}
\multiput(349.00,441.17)(4.855,3.000){2}{\rule{0.517pt}{0.400pt}}
\multiput(356.00,445.61)(1.355,0.447){3}{\rule{1.033pt}{0.108pt}}
\multiput(356.00,444.17)(4.855,3.000){2}{\rule{0.517pt}{0.400pt}}
\multiput(363.00,448.61)(1.355,0.447){3}{\rule{1.033pt}{0.108pt}}
\multiput(363.00,447.17)(4.855,3.000){2}{\rule{0.517pt}{0.400pt}}
\multiput(370.00,451.61)(1.355,0.447){3}{\rule{1.033pt}{0.108pt}}
\multiput(370.00,450.17)(4.855,3.000){2}{\rule{0.517pt}{0.400pt}}
\multiput(377.00,454.61)(1.355,0.447){3}{\rule{1.033pt}{0.108pt}}
\multiput(377.00,453.17)(4.855,3.000){2}{\rule{0.517pt}{0.400pt}}
\put(384,457.17){\rule{1.500pt}{0.400pt}}
\multiput(384.00,456.17)(3.887,2.000){2}{\rule{0.750pt}{0.400pt}}
\put(391,459.17){\rule{1.500pt}{0.400pt}}
\multiput(391.00,458.17)(3.887,2.000){2}{\rule{0.750pt}{0.400pt}}
\multiput(398.00,461.61)(1.132,0.447){3}{\rule{0.900pt}{0.108pt}}
\multiput(398.00,460.17)(4.132,3.000){2}{\rule{0.450pt}{0.400pt}}
\put(404,464.17){\rule{1.500pt}{0.400pt}}
\multiput(404.00,463.17)(3.887,2.000){2}{\rule{0.750pt}{0.400pt}}
\put(411,466.17){\rule{1.500pt}{0.400pt}}
\multiput(411.00,465.17)(3.887,2.000){2}{\rule{0.750pt}{0.400pt}}
\put(418,468.17){\rule{1.500pt}{0.400pt}}
\multiput(418.00,467.17)(3.887,2.000){2}{\rule{0.750pt}{0.400pt}}
\put(425,469.67){\rule{1.686pt}{0.400pt}}
\multiput(425.00,469.17)(3.500,1.000){2}{\rule{0.843pt}{0.400pt}}
\put(432,471.17){\rule{1.500pt}{0.400pt}}
\multiput(432.00,470.17)(3.887,2.000){2}{\rule{0.750pt}{0.400pt}}
\put(439,473.17){\rule{1.500pt}{0.400pt}}
\multiput(439.00,472.17)(3.887,2.000){2}{\rule{0.750pt}{0.400pt}}
\put(446,474.67){\rule{1.686pt}{0.400pt}}
\multiput(446.00,474.17)(3.500,1.000){2}{\rule{0.843pt}{0.400pt}}
\put(453,476.17){\rule{1.500pt}{0.400pt}}
\multiput(453.00,475.17)(3.887,2.000){2}{\rule{0.750pt}{0.400pt}}
\put(460,477.67){\rule{1.686pt}{0.400pt}}
\multiput(460.00,477.17)(3.500,1.000){2}{\rule{0.843pt}{0.400pt}}
\put(467,478.67){\rule{1.445pt}{0.400pt}}
\multiput(467.00,478.17)(3.000,1.000){2}{\rule{0.723pt}{0.400pt}}
\put(473,479.67){\rule{1.686pt}{0.400pt}}
\multiput(473.00,479.17)(3.500,1.000){2}{\rule{0.843pt}{0.400pt}}
\put(480,480.67){\rule{1.686pt}{0.400pt}}
\multiput(480.00,480.17)(3.500,1.000){2}{\rule{0.843pt}{0.400pt}}
\put(487,481.67){\rule{1.686pt}{0.400pt}}
\multiput(487.00,481.17)(3.500,1.000){2}{\rule{0.843pt}{0.400pt}}
\put(556.0,437.0){\rule[-0.200pt]{1.686pt}{0.400pt}}
\put(501,482.67){\rule{1.686pt}{0.400pt}}
\multiput(501.00,482.17)(3.500,1.000){2}{\rule{0.843pt}{0.400pt}}
\put(494.0,483.0){\rule[-0.200pt]{1.686pt}{0.400pt}}
\put(515,483.67){\rule{1.686pt}{0.400pt}}
\multiput(515.00,483.17)(3.500,1.000){2}{\rule{0.843pt}{0.400pt}}
\put(508.0,484.0){\rule[-0.200pt]{1.686pt}{0.400pt}}
\put(522.0,485.0){\rule[-0.200pt]{1.686pt}{0.400pt}}
\put(529.0,485.0){\rule[-0.200pt]{1.445pt}{0.400pt}}
\put(535.0,485.0){\rule[-0.200pt]{1.686pt}{0.400pt}}
\put(542.0,485.0){\rule[-0.200pt]{1.686pt}{0.400pt}}
\put(549.0,485.0){\rule[-0.200pt]{1.686pt}{0.400pt}}
\put(556.0,485.0){\rule[-0.200pt]{1.686pt}{0.400pt}}
\put(570,483.67){\rule{1.686pt}{0.400pt}}
\multiput(570.00,484.17)(3.500,-1.000){2}{\rule{0.843pt}{0.400pt}}
\put(563.0,485.0){\rule[-0.200pt]{1.686pt}{0.400pt}}
\put(584,482.67){\rule{1.686pt}{0.400pt}}
\multiput(584.00,483.17)(3.500,-1.000){2}{\rule{0.843pt}{0.400pt}}
\put(591,481.67){\rule{1.445pt}{0.400pt}}
\multiput(591.00,482.17)(3.000,-1.000){2}{\rule{0.723pt}{0.400pt}}
\put(597,480.67){\rule{1.686pt}{0.400pt}}
\multiput(597.00,481.17)(3.500,-1.000){2}{\rule{0.843pt}{0.400pt}}
\put(604,479.67){\rule{1.686pt}{0.400pt}}
\multiput(604.00,480.17)(3.500,-1.000){2}{\rule{0.843pt}{0.400pt}}
\put(611,478.67){\rule{1.686pt}{0.400pt}}
\multiput(611.00,479.17)(3.500,-1.000){2}{\rule{0.843pt}{0.400pt}}
\put(618,477.67){\rule{1.686pt}{0.400pt}}
\multiput(618.00,478.17)(3.500,-1.000){2}{\rule{0.843pt}{0.400pt}}
\put(625,476.67){\rule{1.686pt}{0.400pt}}
\multiput(625.00,477.17)(3.500,-1.000){2}{\rule{0.843pt}{0.400pt}}
\put(632,475.17){\rule{1.500pt}{0.400pt}}
\multiput(632.00,476.17)(3.887,-2.000){2}{\rule{0.750pt}{0.400pt}}
\put(639,473.67){\rule{1.686pt}{0.400pt}}
\multiput(639.00,474.17)(3.500,-1.000){2}{\rule{0.843pt}{0.400pt}}
\put(646,472.17){\rule{1.500pt}{0.400pt}}
\multiput(646.00,473.17)(3.887,-2.000){2}{\rule{0.750pt}{0.400pt}}
\put(653,470.67){\rule{1.686pt}{0.400pt}}
\multiput(653.00,471.17)(3.500,-1.000){2}{\rule{0.843pt}{0.400pt}}
\put(660,469.17){\rule{1.300pt}{0.400pt}}
\multiput(660.00,470.17)(3.302,-2.000){2}{\rule{0.650pt}{0.400pt}}
\put(666,467.17){\rule{1.500pt}{0.400pt}}
\multiput(666.00,468.17)(3.887,-2.000){2}{\rule{0.750pt}{0.400pt}}
\put(673,465.17){\rule{1.500pt}{0.400pt}}
\multiput(673.00,466.17)(3.887,-2.000){2}{\rule{0.750pt}{0.400pt}}
\put(680,463.17){\rule{1.500pt}{0.400pt}}
\multiput(680.00,464.17)(3.887,-2.000){2}{\rule{0.750pt}{0.400pt}}
\put(687,461.17){\rule{1.500pt}{0.400pt}}
\multiput(687.00,462.17)(3.887,-2.000){2}{\rule{0.750pt}{0.400pt}}
\put(694,459.17){\rule{1.500pt}{0.400pt}}
\multiput(694.00,460.17)(3.887,-2.000){2}{\rule{0.750pt}{0.400pt}}
\multiput(701.00,457.95)(1.355,-0.447){3}{\rule{1.033pt}{0.108pt}}
\multiput(701.00,458.17)(4.855,-3.000){2}{\rule{0.517pt}{0.400pt}}
\put(708,454.17){\rule{1.500pt}{0.400pt}}
\multiput(708.00,455.17)(3.887,-2.000){2}{\rule{0.750pt}{0.400pt}}
\multiput(715.00,452.95)(1.355,-0.447){3}{\rule{1.033pt}{0.108pt}}
\multiput(715.00,453.17)(4.855,-3.000){2}{\rule{0.517pt}{0.400pt}}
\multiput(722.00,449.95)(1.132,-0.447){3}{\rule{0.900pt}{0.108pt}}
\multiput(722.00,450.17)(4.132,-3.000){2}{\rule{0.450pt}{0.400pt}}
\put(728,446.17){\rule{1.500pt}{0.400pt}}
\multiput(728.00,447.17)(3.887,-2.000){2}{\rule{0.750pt}{0.400pt}}
\multiput(735.00,444.95)(1.355,-0.447){3}{\rule{1.033pt}{0.108pt}}
\multiput(735.00,445.17)(4.855,-3.000){2}{\rule{0.517pt}{0.400pt}}
\multiput(742.00,441.95)(1.355,-0.447){3}{\rule{1.033pt}{0.108pt}}
\multiput(742.00,442.17)(4.855,-3.000){2}{\rule{0.517pt}{0.400pt}}
\multiput(749.00,438.95)(1.355,-0.447){3}{\rule{1.033pt}{0.108pt}}
\multiput(749.00,439.17)(4.855,-3.000){2}{\rule{0.517pt}{0.400pt}}
\multiput(756.00,435.94)(0.920,-0.468){5}{\rule{0.800pt}{0.113pt}}
\multiput(756.00,436.17)(5.340,-4.000){2}{\rule{0.400pt}{0.400pt}}
\multiput(763.00,431.95)(1.355,-0.447){3}{\rule{1.033pt}{0.108pt}}
\multiput(763.00,432.17)(4.855,-3.000){2}{\rule{0.517pt}{0.400pt}}
\multiput(770.00,428.94)(0.920,-0.468){5}{\rule{0.800pt}{0.113pt}}
\multiput(770.00,429.17)(5.340,-4.000){2}{\rule{0.400pt}{0.400pt}}
\multiput(777.00,424.95)(1.355,-0.447){3}{\rule{1.033pt}{0.108pt}}
\multiput(777.00,425.17)(4.855,-3.000){2}{\rule{0.517pt}{0.400pt}}
\multiput(784.00,421.94)(0.774,-0.468){5}{\rule{0.700pt}{0.113pt}}
\multiput(784.00,422.17)(4.547,-4.000){2}{\rule{0.350pt}{0.400pt}}
\multiput(790.00,417.94)(0.920,-0.468){5}{\rule{0.800pt}{0.113pt}}
\multiput(790.00,418.17)(5.340,-4.000){2}{\rule{0.400pt}{0.400pt}}
\multiput(797.00,413.95)(1.355,-0.447){3}{\rule{1.033pt}{0.108pt}}
\multiput(797.00,414.17)(4.855,-3.000){2}{\rule{0.517pt}{0.400pt}}
\multiput(804.00,410.93)(0.710,-0.477){7}{\rule{0.660pt}{0.115pt}}
\multiput(804.00,411.17)(5.630,-5.000){2}{\rule{0.330pt}{0.400pt}}
\multiput(811.00,405.94)(0.920,-0.468){5}{\rule{0.800pt}{0.113pt}}
\multiput(811.00,406.17)(5.340,-4.000){2}{\rule{0.400pt}{0.400pt}}
\multiput(818.00,401.94)(0.920,-0.468){5}{\rule{0.800pt}{0.113pt}}
\multiput(818.00,402.17)(5.340,-4.000){2}{\rule{0.400pt}{0.400pt}}
\multiput(825.00,397.94)(0.920,-0.468){5}{\rule{0.800pt}{0.113pt}}
\multiput(825.00,398.17)(5.340,-4.000){2}{\rule{0.400pt}{0.400pt}}
\multiput(832.00,393.93)(0.710,-0.477){7}{\rule{0.660pt}{0.115pt}}
\multiput(832.00,394.17)(5.630,-5.000){2}{\rule{0.330pt}{0.400pt}}
\multiput(839.00,388.94)(0.920,-0.468){5}{\rule{0.800pt}{0.113pt}}
\multiput(839.00,389.17)(5.340,-4.000){2}{\rule{0.400pt}{0.400pt}}
\multiput(846.00,384.93)(0.710,-0.477){7}{\rule{0.660pt}{0.115pt}}
\multiput(846.00,385.17)(5.630,-5.000){2}{\rule{0.330pt}{0.400pt}}
\multiput(853.00,379.93)(0.599,-0.477){7}{\rule{0.580pt}{0.115pt}}
\multiput(853.00,380.17)(4.796,-5.000){2}{\rule{0.290pt}{0.400pt}}
\multiput(859.00,374.93)(0.710,-0.477){7}{\rule{0.660pt}{0.115pt}}
\multiput(859.00,375.17)(5.630,-5.000){2}{\rule{0.330pt}{0.400pt}}
\multiput(866.00,369.93)(0.710,-0.477){7}{\rule{0.660pt}{0.115pt}}
\multiput(866.00,370.17)(5.630,-5.000){2}{\rule{0.330pt}{0.400pt}}
\multiput(873.00,364.93)(0.710,-0.477){7}{\rule{0.660pt}{0.115pt}}
\multiput(873.00,365.17)(5.630,-5.000){2}{\rule{0.330pt}{0.400pt}}
\multiput(880.00,359.93)(0.581,-0.482){9}{\rule{0.567pt}{0.116pt}}
\multiput(880.00,360.17)(5.824,-6.000){2}{\rule{0.283pt}{0.400pt}}
\multiput(887.00,353.93)(0.710,-0.477){7}{\rule{0.660pt}{0.115pt}}
\multiput(887.00,354.17)(5.630,-5.000){2}{\rule{0.330pt}{0.400pt}}
\multiput(894.00,348.93)(0.581,-0.482){9}{\rule{0.567pt}{0.116pt}}
\multiput(894.00,349.17)(5.824,-6.000){2}{\rule{0.283pt}{0.400pt}}
\multiput(901.00,342.93)(0.710,-0.477){7}{\rule{0.660pt}{0.115pt}}
\multiput(901.00,343.17)(5.630,-5.000){2}{\rule{0.330pt}{0.400pt}}
\multiput(908.00,337.93)(0.581,-0.482){9}{\rule{0.567pt}{0.116pt}}
\multiput(908.00,338.17)(5.824,-6.000){2}{\rule{0.283pt}{0.400pt}}
\multiput(915.00,331.93)(0.491,-0.482){9}{\rule{0.500pt}{0.116pt}}
\multiput(915.00,332.17)(4.962,-6.000){2}{\rule{0.250pt}{0.400pt}}
\multiput(921.00,325.93)(0.581,-0.482){9}{\rule{0.567pt}{0.116pt}}
\multiput(921.00,326.17)(5.824,-6.000){2}{\rule{0.283pt}{0.400pt}}
\multiput(928.00,319.93)(0.492,-0.485){11}{\rule{0.500pt}{0.117pt}}
\multiput(928.00,320.17)(5.962,-7.000){2}{\rule{0.250pt}{0.400pt}}
\multiput(935.00,312.93)(0.581,-0.482){9}{\rule{0.567pt}{0.116pt}}
\multiput(935.00,313.17)(5.824,-6.000){2}{\rule{0.283pt}{0.400pt}}
\multiput(942.00,306.93)(0.492,-0.485){11}{\rule{0.500pt}{0.117pt}}
\multiput(942.00,307.17)(5.962,-7.000){2}{\rule{0.250pt}{0.400pt}}
\multiput(949.00,299.93)(0.581,-0.482){9}{\rule{0.567pt}{0.116pt}}
\multiput(949.00,300.17)(5.824,-6.000){2}{\rule{0.283pt}{0.400pt}}
\multiput(294.00,409.59)(0.581,0.482){9}{\rule{0.567pt}{0.116pt}}
\multiput(294.00,408.17)(5.824,6.000){2}{\rule{0.283pt}{0.400pt}}
\multiput(301.00,415.59)(0.581,0.482){9}{\rule{0.567pt}{0.116pt}}
\multiput(301.00,414.17)(5.824,6.000){2}{\rule{0.283pt}{0.400pt}}
\multiput(308.00,421.59)(0.710,0.477){7}{\rule{0.660pt}{0.115pt}}
\multiput(308.00,420.17)(5.630,5.000){2}{\rule{0.330pt}{0.400pt}}
\multiput(315.00,426.59)(0.581,0.482){9}{\rule{0.567pt}{0.116pt}}
\multiput(315.00,425.17)(5.824,6.000){2}{\rule{0.283pt}{0.400pt}}
\multiput(322.00,432.59)(0.710,0.477){7}{\rule{0.660pt}{0.115pt}}
\multiput(322.00,431.17)(5.630,5.000){2}{\rule{0.330pt}{0.400pt}}
\multiput(329.00,437.59)(0.710,0.477){7}{\rule{0.660pt}{0.115pt}}
\multiput(329.00,436.17)(5.630,5.000){2}{\rule{0.330pt}{0.400pt}}
\multiput(336.00,442.59)(0.599,0.477){7}{\rule{0.580pt}{0.115pt}}
\multiput(336.00,441.17)(4.796,5.000){2}{\rule{0.290pt}{0.400pt}}
\multiput(342.00,447.59)(0.710,0.477){7}{\rule{0.660pt}{0.115pt}}
\multiput(342.00,446.17)(5.630,5.000){2}{\rule{0.330pt}{0.400pt}}
\multiput(349.00,452.60)(0.920,0.468){5}{\rule{0.800pt}{0.113pt}}
\multiput(349.00,451.17)(5.340,4.000){2}{\rule{0.400pt}{0.400pt}}
\multiput(356.00,456.59)(0.710,0.477){7}{\rule{0.660pt}{0.115pt}}
\multiput(356.00,455.17)(5.630,5.000){2}{\rule{0.330pt}{0.400pt}}
\multiput(363.00,461.60)(0.920,0.468){5}{\rule{0.800pt}{0.113pt}}
\multiput(363.00,460.17)(5.340,4.000){2}{\rule{0.400pt}{0.400pt}}
\multiput(370.00,465.60)(0.920,0.468){5}{\rule{0.800pt}{0.113pt}}
\multiput(370.00,464.17)(5.340,4.000){2}{\rule{0.400pt}{0.400pt}}
\multiput(377.00,469.60)(0.920,0.468){5}{\rule{0.800pt}{0.113pt}}
\multiput(377.00,468.17)(5.340,4.000){2}{\rule{0.400pt}{0.400pt}}
\multiput(384.00,473.60)(0.920,0.468){5}{\rule{0.800pt}{0.113pt}}
\multiput(384.00,472.17)(5.340,4.000){2}{\rule{0.400pt}{0.400pt}}
\multiput(391.00,477.61)(1.355,0.447){3}{\rule{1.033pt}{0.108pt}}
\multiput(391.00,476.17)(4.855,3.000){2}{\rule{0.517pt}{0.400pt}}
\multiput(398.00,480.60)(0.920,0.468){5}{\rule{0.800pt}{0.113pt}}
\multiput(398.00,479.17)(5.340,4.000){2}{\rule{0.400pt}{0.400pt}}
\multiput(405.00,484.61)(1.132,0.447){3}{\rule{0.900pt}{0.108pt}}
\multiput(405.00,483.17)(4.132,3.000){2}{\rule{0.450pt}{0.400pt}}
\multiput(411.00,487.61)(1.355,0.447){3}{\rule{1.033pt}{0.108pt}}
\multiput(411.00,486.17)(4.855,3.000){2}{\rule{0.517pt}{0.400pt}}
\multiput(418.00,490.61)(1.355,0.447){3}{\rule{1.033pt}{0.108pt}}
\multiput(418.00,489.17)(4.855,3.000){2}{\rule{0.517pt}{0.400pt}}
\multiput(425.00,493.61)(1.355,0.447){3}{\rule{1.033pt}{0.108pt}}
\multiput(425.00,492.17)(4.855,3.000){2}{\rule{0.517pt}{0.400pt}}
\multiput(432.00,496.61)(1.355,0.447){3}{\rule{1.033pt}{0.108pt}}
\multiput(432.00,495.17)(4.855,3.000){2}{\rule{0.517pt}{0.400pt}}
\multiput(439.00,499.61)(1.355,0.447){3}{\rule{1.033pt}{0.108pt}}
\multiput(439.00,498.17)(4.855,3.000){2}{\rule{0.517pt}{0.400pt}}
\put(446,502.17){\rule{1.500pt}{0.400pt}}
\multiput(446.00,501.17)(3.887,2.000){2}{\rule{0.750pt}{0.400pt}}
\put(453,504.17){\rule{1.500pt}{0.400pt}}
\multiput(453.00,503.17)(3.887,2.000){2}{\rule{0.750pt}{0.400pt}}
\multiput(460.00,506.61)(1.355,0.447){3}{\rule{1.033pt}{0.108pt}}
\multiput(460.00,505.17)(4.855,3.000){2}{\rule{0.517pt}{0.400pt}}
\put(467,509.17){\rule{1.300pt}{0.400pt}}
\multiput(467.00,508.17)(3.302,2.000){2}{\rule{0.650pt}{0.400pt}}
\put(473,511.17){\rule{1.500pt}{0.400pt}}
\multiput(473.00,510.17)(3.887,2.000){2}{\rule{0.750pt}{0.400pt}}
\put(480,512.67){\rule{1.686pt}{0.400pt}}
\multiput(480.00,512.17)(3.500,1.000){2}{\rule{0.843pt}{0.400pt}}
\put(487,514.17){\rule{1.500pt}{0.400pt}}
\multiput(487.00,513.17)(3.887,2.000){2}{\rule{0.750pt}{0.400pt}}
\put(494,516.17){\rule{1.500pt}{0.400pt}}
\multiput(494.00,515.17)(3.887,2.000){2}{\rule{0.750pt}{0.400pt}}
\put(501,517.67){\rule{1.686pt}{0.400pt}}
\multiput(501.00,517.17)(3.500,1.000){2}{\rule{0.843pt}{0.400pt}}
\put(508,518.67){\rule{1.686pt}{0.400pt}}
\multiput(508.00,518.17)(3.500,1.000){2}{\rule{0.843pt}{0.400pt}}
\put(515,519.67){\rule{1.686pt}{0.400pt}}
\multiput(515.00,519.17)(3.500,1.000){2}{\rule{0.843pt}{0.400pt}}
\put(522,520.67){\rule{1.686pt}{0.400pt}}
\multiput(522.00,520.17)(3.500,1.000){2}{\rule{0.843pt}{0.400pt}}
\put(529,521.67){\rule{1.445pt}{0.400pt}}
\multiput(529.00,521.17)(3.000,1.000){2}{\rule{0.723pt}{0.400pt}}
\put(535,522.67){\rule{1.686pt}{0.400pt}}
\multiput(535.00,522.17)(3.500,1.000){2}{\rule{0.843pt}{0.400pt}}
\put(577.0,484.0){\rule[-0.200pt]{1.686pt}{0.400pt}}
\put(549,523.67){\rule{1.686pt}{0.400pt}}
\multiput(549.00,523.17)(3.500,1.000){2}{\rule{0.843pt}{0.400pt}}
\put(542.0,524.0){\rule[-0.200pt]{1.686pt}{0.400pt}}
\put(556.0,525.0){\rule[-0.200pt]{1.686pt}{0.400pt}}
\put(563.0,525.0){\rule[-0.200pt]{1.686pt}{0.400pt}}
\put(570.0,525.0){\rule[-0.200pt]{1.686pt}{0.400pt}}
\put(577.0,525.0){\rule[-0.200pt]{1.686pt}{0.400pt}}
\put(584.0,525.0){\rule[-0.200pt]{1.686pt}{0.400pt}}
\put(591.0,525.0){\rule[-0.200pt]{1.686pt}{0.400pt}}
\put(604,523.67){\rule{1.686pt}{0.400pt}}
\multiput(604.00,524.17)(3.500,-1.000){2}{\rule{0.843pt}{0.400pt}}
\put(611,522.67){\rule{1.686pt}{0.400pt}}
\multiput(611.00,523.17)(3.500,-1.000){2}{\rule{0.843pt}{0.400pt}}
\put(618,521.67){\rule{1.686pt}{0.400pt}}
\multiput(618.00,522.17)(3.500,-1.000){2}{\rule{0.843pt}{0.400pt}}
\put(598.0,525.0){\rule[-0.200pt]{1.445pt}{0.400pt}}
\put(632,520.17){\rule{1.500pt}{0.400pt}}
\multiput(632.00,521.17)(3.887,-2.000){2}{\rule{0.750pt}{0.400pt}}
\put(639,518.67){\rule{1.686pt}{0.400pt}}
\multiput(639.00,519.17)(3.500,-1.000){2}{\rule{0.843pt}{0.400pt}}
\put(646,517.67){\rule{1.686pt}{0.400pt}}
\multiput(646.00,518.17)(3.500,-1.000){2}{\rule{0.843pt}{0.400pt}}
\put(653,516.67){\rule{1.686pt}{0.400pt}}
\multiput(653.00,517.17)(3.500,-1.000){2}{\rule{0.843pt}{0.400pt}}
\put(660,515.17){\rule{1.300pt}{0.400pt}}
\multiput(660.00,516.17)(3.302,-2.000){2}{\rule{0.650pt}{0.400pt}}
\put(666,513.17){\rule{1.500pt}{0.400pt}}
\multiput(666.00,514.17)(3.887,-2.000){2}{\rule{0.750pt}{0.400pt}}
\put(673,511.17){\rule{1.500pt}{0.400pt}}
\multiput(673.00,512.17)(3.887,-2.000){2}{\rule{0.750pt}{0.400pt}}
\put(680,509.67){\rule{1.686pt}{0.400pt}}
\multiput(680.00,510.17)(3.500,-1.000){2}{\rule{0.843pt}{0.400pt}}
\multiput(687.00,508.95)(1.355,-0.447){3}{\rule{1.033pt}{0.108pt}}
\multiput(687.00,509.17)(4.855,-3.000){2}{\rule{0.517pt}{0.400pt}}
\put(694,505.17){\rule{1.500pt}{0.400pt}}
\multiput(694.00,506.17)(3.887,-2.000){2}{\rule{0.750pt}{0.400pt}}
\put(701,503.17){\rule{1.500pt}{0.400pt}}
\multiput(701.00,504.17)(3.887,-2.000){2}{\rule{0.750pt}{0.400pt}}
\put(708,501.17){\rule{1.500pt}{0.400pt}}
\multiput(708.00,502.17)(3.887,-2.000){2}{\rule{0.750pt}{0.400pt}}
\multiput(715.00,499.95)(1.355,-0.447){3}{\rule{1.033pt}{0.108pt}}
\multiput(715.00,500.17)(4.855,-3.000){2}{\rule{0.517pt}{0.400pt}}
\multiput(722.00,496.95)(1.355,-0.447){3}{\rule{1.033pt}{0.108pt}}
\multiput(722.00,497.17)(4.855,-3.000){2}{\rule{0.517pt}{0.400pt}}
\multiput(729.00,493.95)(1.132,-0.447){3}{\rule{0.900pt}{0.108pt}}
\multiput(729.00,494.17)(4.132,-3.000){2}{\rule{0.450pt}{0.400pt}}
\multiput(735.00,490.95)(1.355,-0.447){3}{\rule{1.033pt}{0.108pt}}
\multiput(735.00,491.17)(4.855,-3.000){2}{\rule{0.517pt}{0.400pt}}
\multiput(742.00,487.95)(1.355,-0.447){3}{\rule{1.033pt}{0.108pt}}
\multiput(742.00,488.17)(4.855,-3.000){2}{\rule{0.517pt}{0.400pt}}
\multiput(749.00,484.95)(1.355,-0.447){3}{\rule{1.033pt}{0.108pt}}
\multiput(749.00,485.17)(4.855,-3.000){2}{\rule{0.517pt}{0.400pt}}
\multiput(756.00,481.95)(1.355,-0.447){3}{\rule{1.033pt}{0.108pt}}
\multiput(756.00,482.17)(4.855,-3.000){2}{\rule{0.517pt}{0.400pt}}
\multiput(763.00,478.94)(0.920,-0.468){5}{\rule{0.800pt}{0.113pt}}
\multiput(763.00,479.17)(5.340,-4.000){2}{\rule{0.400pt}{0.400pt}}
\multiput(770.00,474.95)(1.355,-0.447){3}{\rule{1.033pt}{0.108pt}}
\multiput(770.00,475.17)(4.855,-3.000){2}{\rule{0.517pt}{0.400pt}}
\multiput(777.00,471.94)(0.920,-0.468){5}{\rule{0.800pt}{0.113pt}}
\multiput(777.00,472.17)(5.340,-4.000){2}{\rule{0.400pt}{0.400pt}}
\multiput(784.00,467.94)(0.920,-0.468){5}{\rule{0.800pt}{0.113pt}}
\multiput(784.00,468.17)(5.340,-4.000){2}{\rule{0.400pt}{0.400pt}}
\multiput(791.00,463.94)(0.774,-0.468){5}{\rule{0.700pt}{0.113pt}}
\multiput(791.00,464.17)(4.547,-4.000){2}{\rule{0.350pt}{0.400pt}}
\multiput(797.00,459.94)(0.920,-0.468){5}{\rule{0.800pt}{0.113pt}}
\multiput(797.00,460.17)(5.340,-4.000){2}{\rule{0.400pt}{0.400pt}}
\multiput(804.00,455.93)(0.710,-0.477){7}{\rule{0.660pt}{0.115pt}}
\multiput(804.00,456.17)(5.630,-5.000){2}{\rule{0.330pt}{0.400pt}}
\multiput(811.00,450.94)(0.920,-0.468){5}{\rule{0.800pt}{0.113pt}}
\multiput(811.00,451.17)(5.340,-4.000){2}{\rule{0.400pt}{0.400pt}}
\multiput(818.00,446.93)(0.710,-0.477){7}{\rule{0.660pt}{0.115pt}}
\multiput(818.00,447.17)(5.630,-5.000){2}{\rule{0.330pt}{0.400pt}}
\multiput(825.00,441.94)(0.920,-0.468){5}{\rule{0.800pt}{0.113pt}}
\multiput(825.00,442.17)(5.340,-4.000){2}{\rule{0.400pt}{0.400pt}}
\multiput(832.00,437.93)(0.710,-0.477){7}{\rule{0.660pt}{0.115pt}}
\multiput(832.00,438.17)(5.630,-5.000){2}{\rule{0.330pt}{0.400pt}}
\multiput(839.00,432.93)(0.710,-0.477){7}{\rule{0.660pt}{0.115pt}}
\multiput(839.00,433.17)(5.630,-5.000){2}{\rule{0.330pt}{0.400pt}}
\multiput(846.00,427.93)(0.710,-0.477){7}{\rule{0.660pt}{0.115pt}}
\multiput(846.00,428.17)(5.630,-5.000){2}{\rule{0.330pt}{0.400pt}}
\multiput(853.00,422.93)(0.491,-0.482){9}{\rule{0.500pt}{0.116pt}}
\multiput(853.00,423.17)(4.962,-6.000){2}{\rule{0.250pt}{0.400pt}}
\multiput(859.00,416.93)(0.710,-0.477){7}{\rule{0.660pt}{0.115pt}}
\multiput(859.00,417.17)(5.630,-5.000){2}{\rule{0.330pt}{0.400pt}}
\multiput(866.00,411.93)(0.581,-0.482){9}{\rule{0.567pt}{0.116pt}}
\multiput(866.00,412.17)(5.824,-6.000){2}{\rule{0.283pt}{0.400pt}}
\multiput(873.00,405.93)(0.710,-0.477){7}{\rule{0.660pt}{0.115pt}}
\multiput(873.00,406.17)(5.630,-5.000){2}{\rule{0.330pt}{0.400pt}}
\multiput(880.00,400.93)(0.581,-0.482){9}{\rule{0.567pt}{0.116pt}}
\multiput(880.00,401.17)(5.824,-6.000){2}{\rule{0.283pt}{0.400pt}}
\multiput(887.00,394.93)(0.581,-0.482){9}{\rule{0.567pt}{0.116pt}}
\multiput(887.00,395.17)(5.824,-6.000){2}{\rule{0.283pt}{0.400pt}}
\multiput(894.00,388.93)(0.492,-0.485){11}{\rule{0.500pt}{0.117pt}}
\multiput(894.00,389.17)(5.962,-7.000){2}{\rule{0.250pt}{0.400pt}}
\multiput(901.00,381.93)(0.581,-0.482){9}{\rule{0.567pt}{0.116pt}}
\multiput(901.00,382.17)(5.824,-6.000){2}{\rule{0.283pt}{0.400pt}}
\multiput(908.00,375.93)(0.492,-0.485){11}{\rule{0.500pt}{0.117pt}}
\multiput(908.00,376.17)(5.962,-7.000){2}{\rule{0.250pt}{0.400pt}}
\multiput(915.00,368.93)(0.581,-0.482){9}{\rule{0.567pt}{0.116pt}}
\multiput(915.00,369.17)(5.824,-6.000){2}{\rule{0.283pt}{0.400pt}}
\multiput(922.59,361.65)(0.482,-0.581){9}{\rule{0.116pt}{0.567pt}}
\multiput(921.17,362.82)(6.000,-5.824){2}{\rule{0.400pt}{0.283pt}}
\multiput(928.00,355.93)(0.492,-0.485){11}{\rule{0.500pt}{0.117pt}}
\multiput(928.00,356.17)(5.962,-7.000){2}{\rule{0.250pt}{0.400pt}}
\multiput(935.00,348.93)(0.492,-0.485){11}{\rule{0.500pt}{0.117pt}}
\multiput(935.00,349.17)(5.962,-7.000){2}{\rule{0.250pt}{0.400pt}}
\multiput(942.59,340.69)(0.485,-0.569){11}{\rule{0.117pt}{0.557pt}}
\multiput(941.17,341.84)(7.000,-6.844){2}{\rule{0.400pt}{0.279pt}}
\multiput(949.00,333.93)(0.492,-0.485){11}{\rule{0.500pt}{0.117pt}}
\multiput(949.00,334.17)(5.962,-7.000){2}{\rule{0.250pt}{0.400pt}}
\multiput(956.59,325.69)(0.485,-0.569){11}{\rule{0.117pt}{0.557pt}}
\multiput(955.17,326.84)(7.000,-6.844){2}{\rule{0.400pt}{0.279pt}}
\multiput(963.59,317.69)(0.485,-0.569){11}{\rule{0.117pt}{0.557pt}}
\multiput(962.17,318.84)(7.000,-6.844){2}{\rule{0.400pt}{0.279pt}}
\multiput(970.59,309.69)(0.485,-0.569){11}{\rule{0.117pt}{0.557pt}}
\multiput(969.17,310.84)(7.000,-6.844){2}{\rule{0.400pt}{0.279pt}}
\multiput(315.59,418.00)(0.485,0.569){11}{\rule{0.117pt}{0.557pt}}
\multiput(314.17,418.00)(7.000,6.844){2}{\rule{0.400pt}{0.279pt}}
\multiput(322.00,426.59)(0.581,0.482){9}{\rule{0.567pt}{0.116pt}}
\multiput(322.00,425.17)(5.824,6.000){2}{\rule{0.283pt}{0.400pt}}
\multiput(329.00,432.59)(0.492,0.485){11}{\rule{0.500pt}{0.117pt}}
\multiput(329.00,431.17)(5.962,7.000){2}{\rule{0.250pt}{0.400pt}}
\multiput(336.00,439.59)(0.581,0.482){9}{\rule{0.567pt}{0.116pt}}
\multiput(336.00,438.17)(5.824,6.000){2}{\rule{0.283pt}{0.400pt}}
\multiput(343.59,445.00)(0.482,0.581){9}{\rule{0.116pt}{0.567pt}}
\multiput(342.17,445.00)(6.000,5.824){2}{\rule{0.400pt}{0.283pt}}
\multiput(349.00,452.59)(0.581,0.482){9}{\rule{0.567pt}{0.116pt}}
\multiput(349.00,451.17)(5.824,6.000){2}{\rule{0.283pt}{0.400pt}}
\multiput(356.00,458.59)(0.581,0.482){9}{\rule{0.567pt}{0.116pt}}
\multiput(356.00,457.17)(5.824,6.000){2}{\rule{0.283pt}{0.400pt}}
\multiput(363.00,464.59)(0.710,0.477){7}{\rule{0.660pt}{0.115pt}}
\multiput(363.00,463.17)(5.630,5.000){2}{\rule{0.330pt}{0.400pt}}
\multiput(370.00,469.59)(0.581,0.482){9}{\rule{0.567pt}{0.116pt}}
\multiput(370.00,468.17)(5.824,6.000){2}{\rule{0.283pt}{0.400pt}}
\multiput(377.00,475.59)(0.710,0.477){7}{\rule{0.660pt}{0.115pt}}
\multiput(377.00,474.17)(5.630,5.000){2}{\rule{0.330pt}{0.400pt}}
\multiput(384.00,480.59)(0.710,0.477){7}{\rule{0.660pt}{0.115pt}}
\multiput(384.00,479.17)(5.630,5.000){2}{\rule{0.330pt}{0.400pt}}
\multiput(391.00,485.59)(0.710,0.477){7}{\rule{0.660pt}{0.115pt}}
\multiput(391.00,484.17)(5.630,5.000){2}{\rule{0.330pt}{0.400pt}}
\multiput(398.00,490.59)(0.710,0.477){7}{\rule{0.660pt}{0.115pt}}
\multiput(398.00,489.17)(5.630,5.000){2}{\rule{0.330pt}{0.400pt}}
\multiput(405.00,495.60)(0.774,0.468){5}{\rule{0.700pt}{0.113pt}}
\multiput(405.00,494.17)(4.547,4.000){2}{\rule{0.350pt}{0.400pt}}
\multiput(411.00,499.59)(0.710,0.477){7}{\rule{0.660pt}{0.115pt}}
\multiput(411.00,498.17)(5.630,5.000){2}{\rule{0.330pt}{0.400pt}}
\multiput(418.00,504.60)(0.920,0.468){5}{\rule{0.800pt}{0.113pt}}
\multiput(418.00,503.17)(5.340,4.000){2}{\rule{0.400pt}{0.400pt}}
\multiput(425.00,508.60)(0.920,0.468){5}{\rule{0.800pt}{0.113pt}}
\multiput(425.00,507.17)(5.340,4.000){2}{\rule{0.400pt}{0.400pt}}
\multiput(432.00,512.60)(0.920,0.468){5}{\rule{0.800pt}{0.113pt}}
\multiput(432.00,511.17)(5.340,4.000){2}{\rule{0.400pt}{0.400pt}}
\multiput(439.00,516.61)(1.355,0.447){3}{\rule{1.033pt}{0.108pt}}
\multiput(439.00,515.17)(4.855,3.000){2}{\rule{0.517pt}{0.400pt}}
\multiput(446.00,519.60)(0.920,0.468){5}{\rule{0.800pt}{0.113pt}}
\multiput(446.00,518.17)(5.340,4.000){2}{\rule{0.400pt}{0.400pt}}
\multiput(453.00,523.61)(1.355,0.447){3}{\rule{1.033pt}{0.108pt}}
\multiput(453.00,522.17)(4.855,3.000){2}{\rule{0.517pt}{0.400pt}}
\multiput(460.00,526.60)(0.920,0.468){5}{\rule{0.800pt}{0.113pt}}
\multiput(460.00,525.17)(5.340,4.000){2}{\rule{0.400pt}{0.400pt}}
\multiput(467.00,530.61)(1.355,0.447){3}{\rule{1.033pt}{0.108pt}}
\multiput(467.00,529.17)(4.855,3.000){2}{\rule{0.517pt}{0.400pt}}
\put(474,533.17){\rule{1.300pt}{0.400pt}}
\multiput(474.00,532.17)(3.302,2.000){2}{\rule{0.650pt}{0.400pt}}
\multiput(480.00,535.61)(1.355,0.447){3}{\rule{1.033pt}{0.108pt}}
\multiput(480.00,534.17)(4.855,3.000){2}{\rule{0.517pt}{0.400pt}}
\multiput(487.00,538.61)(1.355,0.447){3}{\rule{1.033pt}{0.108pt}}
\multiput(487.00,537.17)(4.855,3.000){2}{\rule{0.517pt}{0.400pt}}
\put(494,541.17){\rule{1.500pt}{0.400pt}}
\multiput(494.00,540.17)(3.887,2.000){2}{\rule{0.750pt}{0.400pt}}
\put(501,543.17){\rule{1.500pt}{0.400pt}}
\multiput(501.00,542.17)(3.887,2.000){2}{\rule{0.750pt}{0.400pt}}
\put(508,545.17){\rule{1.500pt}{0.400pt}}
\multiput(508.00,544.17)(3.887,2.000){2}{\rule{0.750pt}{0.400pt}}
\put(515,547.17){\rule{1.500pt}{0.400pt}}
\multiput(515.00,546.17)(3.887,2.000){2}{\rule{0.750pt}{0.400pt}}
\put(522,549.17){\rule{1.500pt}{0.400pt}}
\multiput(522.00,548.17)(3.887,2.000){2}{\rule{0.750pt}{0.400pt}}
\put(529,551.17){\rule{1.500pt}{0.400pt}}
\multiput(529.00,550.17)(3.887,2.000){2}{\rule{0.750pt}{0.400pt}}
\put(536,552.67){\rule{1.445pt}{0.400pt}}
\multiput(536.00,552.17)(3.000,1.000){2}{\rule{0.723pt}{0.400pt}}
\put(542,553.67){\rule{1.686pt}{0.400pt}}
\multiput(542.00,553.17)(3.500,1.000){2}{\rule{0.843pt}{0.400pt}}
\put(549,555.17){\rule{1.500pt}{0.400pt}}
\multiput(549.00,554.17)(3.887,2.000){2}{\rule{0.750pt}{0.400pt}}
\put(556,556.67){\rule{1.686pt}{0.400pt}}
\multiput(556.00,556.17)(3.500,1.000){2}{\rule{0.843pt}{0.400pt}}
\put(625.0,522.0){\rule[-0.200pt]{1.686pt}{0.400pt}}
\put(570,557.67){\rule{1.686pt}{0.400pt}}
\multiput(570.00,557.17)(3.500,1.000){2}{\rule{0.843pt}{0.400pt}}
\put(577,558.67){\rule{1.686pt}{0.400pt}}
\multiput(577.00,558.17)(3.500,1.000){2}{\rule{0.843pt}{0.400pt}}
\put(563.0,558.0){\rule[-0.200pt]{1.686pt}{0.400pt}}
\put(584.0,560.0){\rule[-0.200pt]{1.686pt}{0.400pt}}
\put(591.0,560.0){\rule[-0.200pt]{1.686pt}{0.400pt}}
\put(598.0,560.0){\rule[-0.200pt]{1.445pt}{0.400pt}}
\put(604.0,560.0){\rule[-0.200pt]{1.686pt}{0.400pt}}
\put(611.0,560.0){\rule[-0.200pt]{1.686pt}{0.400pt}}
\put(625,558.67){\rule{1.686pt}{0.400pt}}
\multiput(625.00,559.17)(3.500,-1.000){2}{\rule{0.843pt}{0.400pt}}
\put(618.0,560.0){\rule[-0.200pt]{1.686pt}{0.400pt}}
\put(639,557.67){\rule{1.686pt}{0.400pt}}
\multiput(639.00,558.17)(3.500,-1.000){2}{\rule{0.843pt}{0.400pt}}
\put(646,556.67){\rule{1.686pt}{0.400pt}}
\multiput(646.00,557.17)(3.500,-1.000){2}{\rule{0.843pt}{0.400pt}}
\put(653,555.67){\rule{1.686pt}{0.400pt}}
\multiput(653.00,556.17)(3.500,-1.000){2}{\rule{0.843pt}{0.400pt}}
\put(660,554.67){\rule{1.686pt}{0.400pt}}
\multiput(660.00,555.17)(3.500,-1.000){2}{\rule{0.843pt}{0.400pt}}
\put(667,553.17){\rule{1.300pt}{0.400pt}}
\multiput(667.00,554.17)(3.302,-2.000){2}{\rule{0.650pt}{0.400pt}}
\put(673,551.67){\rule{1.686pt}{0.400pt}}
\multiput(673.00,552.17)(3.500,-1.000){2}{\rule{0.843pt}{0.400pt}}
\put(680,550.17){\rule{1.500pt}{0.400pt}}
\multiput(680.00,551.17)(3.887,-2.000){2}{\rule{0.750pt}{0.400pt}}
\put(687,548.67){\rule{1.686pt}{0.400pt}}
\multiput(687.00,549.17)(3.500,-1.000){2}{\rule{0.843pt}{0.400pt}}
\put(694,547.17){\rule{1.500pt}{0.400pt}}
\multiput(694.00,548.17)(3.887,-2.000){2}{\rule{0.750pt}{0.400pt}}
\put(701,545.17){\rule{1.500pt}{0.400pt}}
\multiput(701.00,546.17)(3.887,-2.000){2}{\rule{0.750pt}{0.400pt}}
\multiput(708.00,543.95)(1.355,-0.447){3}{\rule{1.033pt}{0.108pt}}
\multiput(708.00,544.17)(4.855,-3.000){2}{\rule{0.517pt}{0.400pt}}
\put(715,540.17){\rule{1.500pt}{0.400pt}}
\multiput(715.00,541.17)(3.887,-2.000){2}{\rule{0.750pt}{0.400pt}}
\multiput(722.00,538.95)(1.355,-0.447){3}{\rule{1.033pt}{0.108pt}}
\multiput(722.00,539.17)(4.855,-3.000){2}{\rule{0.517pt}{0.400pt}}
\put(729,535.17){\rule{1.300pt}{0.400pt}}
\multiput(729.00,536.17)(3.302,-2.000){2}{\rule{0.650pt}{0.400pt}}
\multiput(735.00,533.95)(1.355,-0.447){3}{\rule{1.033pt}{0.108pt}}
\multiput(735.00,534.17)(4.855,-3.000){2}{\rule{0.517pt}{0.400pt}}
\multiput(742.00,530.95)(1.355,-0.447){3}{\rule{1.033pt}{0.108pt}}
\multiput(742.00,531.17)(4.855,-3.000){2}{\rule{0.517pt}{0.400pt}}
\multiput(749.00,527.95)(1.355,-0.447){3}{\rule{1.033pt}{0.108pt}}
\multiput(749.00,528.17)(4.855,-3.000){2}{\rule{0.517pt}{0.400pt}}
\multiput(756.00,524.95)(1.355,-0.447){3}{\rule{1.033pt}{0.108pt}}
\multiput(756.00,525.17)(4.855,-3.000){2}{\rule{0.517pt}{0.400pt}}
\multiput(763.00,521.94)(0.920,-0.468){5}{\rule{0.800pt}{0.113pt}}
\multiput(763.00,522.17)(5.340,-4.000){2}{\rule{0.400pt}{0.400pt}}
\multiput(770.00,517.95)(1.355,-0.447){3}{\rule{1.033pt}{0.108pt}}
\multiput(770.00,518.17)(4.855,-3.000){2}{\rule{0.517pt}{0.400pt}}
\multiput(777.00,514.94)(0.920,-0.468){5}{\rule{0.800pt}{0.113pt}}
\multiput(777.00,515.17)(5.340,-4.000){2}{\rule{0.400pt}{0.400pt}}
\multiput(784.00,510.94)(0.920,-0.468){5}{\rule{0.800pt}{0.113pt}}
\multiput(784.00,511.17)(5.340,-4.000){2}{\rule{0.400pt}{0.400pt}}
\multiput(791.00,506.94)(0.774,-0.468){5}{\rule{0.700pt}{0.113pt}}
\multiput(791.00,507.17)(4.547,-4.000){2}{\rule{0.350pt}{0.400pt}}
\multiput(797.00,502.94)(0.920,-0.468){5}{\rule{0.800pt}{0.113pt}}
\multiput(797.00,503.17)(5.340,-4.000){2}{\rule{0.400pt}{0.400pt}}
\multiput(804.00,498.93)(0.710,-0.477){7}{\rule{0.660pt}{0.115pt}}
\multiput(804.00,499.17)(5.630,-5.000){2}{\rule{0.330pt}{0.400pt}}
\multiput(811.00,493.94)(0.920,-0.468){5}{\rule{0.800pt}{0.113pt}}
\multiput(811.00,494.17)(5.340,-4.000){2}{\rule{0.400pt}{0.400pt}}
\multiput(818.00,489.93)(0.710,-0.477){7}{\rule{0.660pt}{0.115pt}}
\multiput(818.00,490.17)(5.630,-5.000){2}{\rule{0.330pt}{0.400pt}}
\multiput(825.00,484.93)(0.710,-0.477){7}{\rule{0.660pt}{0.115pt}}
\multiput(825.00,485.17)(5.630,-5.000){2}{\rule{0.330pt}{0.400pt}}
\multiput(832.00,479.93)(0.710,-0.477){7}{\rule{0.660pt}{0.115pt}}
\multiput(832.00,480.17)(5.630,-5.000){2}{\rule{0.330pt}{0.400pt}}
\multiput(839.00,474.93)(0.710,-0.477){7}{\rule{0.660pt}{0.115pt}}
\multiput(839.00,475.17)(5.630,-5.000){2}{\rule{0.330pt}{0.400pt}}
\multiput(846.00,469.93)(0.710,-0.477){7}{\rule{0.660pt}{0.115pt}}
\multiput(846.00,470.17)(5.630,-5.000){2}{\rule{0.330pt}{0.400pt}}
\multiput(853.00,464.93)(0.710,-0.477){7}{\rule{0.660pt}{0.115pt}}
\multiput(853.00,465.17)(5.630,-5.000){2}{\rule{0.330pt}{0.400pt}}
\multiput(860.00,459.93)(0.491,-0.482){9}{\rule{0.500pt}{0.116pt}}
\multiput(860.00,460.17)(4.962,-6.000){2}{\rule{0.250pt}{0.400pt}}
\multiput(866.00,453.93)(0.581,-0.482){9}{\rule{0.567pt}{0.116pt}}
\multiput(866.00,454.17)(5.824,-6.000){2}{\rule{0.283pt}{0.400pt}}
\multiput(873.00,447.93)(0.581,-0.482){9}{\rule{0.567pt}{0.116pt}}
\multiput(873.00,448.17)(5.824,-6.000){2}{\rule{0.283pt}{0.400pt}}
\multiput(880.00,441.93)(0.581,-0.482){9}{\rule{0.567pt}{0.116pt}}
\multiput(880.00,442.17)(5.824,-6.000){2}{\rule{0.283pt}{0.400pt}}
\multiput(887.00,435.93)(0.581,-0.482){9}{\rule{0.567pt}{0.116pt}}
\multiput(887.00,436.17)(5.824,-6.000){2}{\rule{0.283pt}{0.400pt}}
\multiput(894.00,429.93)(0.492,-0.485){11}{\rule{0.500pt}{0.117pt}}
\multiput(894.00,430.17)(5.962,-7.000){2}{\rule{0.250pt}{0.400pt}}
\multiput(901.00,422.93)(0.581,-0.482){9}{\rule{0.567pt}{0.116pt}}
\multiput(901.00,423.17)(5.824,-6.000){2}{\rule{0.283pt}{0.400pt}}
\multiput(908.00,416.93)(0.492,-0.485){11}{\rule{0.500pt}{0.117pt}}
\multiput(908.00,417.17)(5.962,-7.000){2}{\rule{0.250pt}{0.400pt}}
\multiput(915.00,409.93)(0.492,-0.485){11}{\rule{0.500pt}{0.117pt}}
\multiput(915.00,410.17)(5.962,-7.000){2}{\rule{0.250pt}{0.400pt}}
\multiput(922.59,401.37)(0.482,-0.671){9}{\rule{0.116pt}{0.633pt}}
\multiput(921.17,402.69)(6.000,-6.685){2}{\rule{0.400pt}{0.317pt}}
\multiput(928.00,394.93)(0.492,-0.485){11}{\rule{0.500pt}{0.117pt}}
\multiput(928.00,395.17)(5.962,-7.000){2}{\rule{0.250pt}{0.400pt}}
\multiput(935.59,386.69)(0.485,-0.569){11}{\rule{0.117pt}{0.557pt}}
\multiput(934.17,387.84)(7.000,-6.844){2}{\rule{0.400pt}{0.279pt}}
\multiput(942.00,379.93)(0.492,-0.485){11}{\rule{0.500pt}{0.117pt}}
\multiput(942.00,380.17)(5.962,-7.000){2}{\rule{0.250pt}{0.400pt}}
\multiput(949.59,371.69)(0.485,-0.569){11}{\rule{0.117pt}{0.557pt}}
\multiput(948.17,372.84)(7.000,-6.844){2}{\rule{0.400pt}{0.279pt}}
\multiput(956.59,363.69)(0.485,-0.569){11}{\rule{0.117pt}{0.557pt}}
\multiput(955.17,364.84)(7.000,-6.844){2}{\rule{0.400pt}{0.279pt}}
\multiput(963.59,355.45)(0.485,-0.645){11}{\rule{0.117pt}{0.614pt}}
\multiput(962.17,356.73)(7.000,-7.725){2}{\rule{0.400pt}{0.307pt}}
\multiput(970.59,346.69)(0.485,-0.569){11}{\rule{0.117pt}{0.557pt}}
\multiput(969.17,347.84)(7.000,-6.844){2}{\rule{0.400pt}{0.279pt}}
\multiput(977.59,338.45)(0.485,-0.645){11}{\rule{0.117pt}{0.614pt}}
\multiput(976.17,339.73)(7.000,-7.725){2}{\rule{0.400pt}{0.307pt}}
\multiput(984.59,329.45)(0.485,-0.645){11}{\rule{0.117pt}{0.614pt}}
\multiput(983.17,330.73)(7.000,-7.725){2}{\rule{0.400pt}{0.307pt}}
\multiput(991.59,320.09)(0.482,-0.762){9}{\rule{0.116pt}{0.700pt}}
\multiput(990.17,321.55)(6.000,-7.547){2}{\rule{0.400pt}{0.350pt}}
\multiput(336.59,428.00)(0.485,0.569){11}{\rule{0.117pt}{0.557pt}}
\multiput(335.17,428.00)(7.000,6.844){2}{\rule{0.400pt}{0.279pt}}
\multiput(343.59,436.00)(0.482,0.671){9}{\rule{0.116pt}{0.633pt}}
\multiput(342.17,436.00)(6.000,6.685){2}{\rule{0.400pt}{0.317pt}}
\multiput(349.00,444.59)(0.492,0.485){11}{\rule{0.500pt}{0.117pt}}
\multiput(349.00,443.17)(5.962,7.000){2}{\rule{0.250pt}{0.400pt}}
\multiput(356.00,451.59)(0.492,0.485){11}{\rule{0.500pt}{0.117pt}}
\multiput(356.00,450.17)(5.962,7.000){2}{\rule{0.250pt}{0.400pt}}
\multiput(363.00,458.59)(0.492,0.485){11}{\rule{0.500pt}{0.117pt}}
\multiput(363.00,457.17)(5.962,7.000){2}{\rule{0.250pt}{0.400pt}}
\multiput(370.00,465.59)(0.492,0.485){11}{\rule{0.500pt}{0.117pt}}
\multiput(370.00,464.17)(5.962,7.000){2}{\rule{0.250pt}{0.400pt}}
\multiput(377.00,472.59)(0.492,0.485){11}{\rule{0.500pt}{0.117pt}}
\multiput(377.00,471.17)(5.962,7.000){2}{\rule{0.250pt}{0.400pt}}
\multiput(384.00,479.59)(0.581,0.482){9}{\rule{0.567pt}{0.116pt}}
\multiput(384.00,478.17)(5.824,6.000){2}{\rule{0.283pt}{0.400pt}}
\multiput(391.00,485.59)(0.581,0.482){9}{\rule{0.567pt}{0.116pt}}
\multiput(391.00,484.17)(5.824,6.000){2}{\rule{0.283pt}{0.400pt}}
\multiput(398.00,491.59)(0.581,0.482){9}{\rule{0.567pt}{0.116pt}}
\multiput(398.00,490.17)(5.824,6.000){2}{\rule{0.283pt}{0.400pt}}
\multiput(405.00,497.59)(0.581,0.482){9}{\rule{0.567pt}{0.116pt}}
\multiput(405.00,496.17)(5.824,6.000){2}{\rule{0.283pt}{0.400pt}}
\multiput(412.00,503.59)(0.491,0.482){9}{\rule{0.500pt}{0.116pt}}
\multiput(412.00,502.17)(4.962,6.000){2}{\rule{0.250pt}{0.400pt}}
\multiput(418.00,509.59)(0.710,0.477){7}{\rule{0.660pt}{0.115pt}}
\multiput(418.00,508.17)(5.630,5.000){2}{\rule{0.330pt}{0.400pt}}
\multiput(425.00,514.59)(0.710,0.477){7}{\rule{0.660pt}{0.115pt}}
\multiput(425.00,513.17)(5.630,5.000){2}{\rule{0.330pt}{0.400pt}}
\multiput(432.00,519.59)(0.710,0.477){7}{\rule{0.660pt}{0.115pt}}
\multiput(432.00,518.17)(5.630,5.000){2}{\rule{0.330pt}{0.400pt}}
\multiput(439.00,524.59)(0.710,0.477){7}{\rule{0.660pt}{0.115pt}}
\multiput(439.00,523.17)(5.630,5.000){2}{\rule{0.330pt}{0.400pt}}
\multiput(446.00,529.59)(0.710,0.477){7}{\rule{0.660pt}{0.115pt}}
\multiput(446.00,528.17)(5.630,5.000){2}{\rule{0.330pt}{0.400pt}}
\multiput(453.00,534.60)(0.920,0.468){5}{\rule{0.800pt}{0.113pt}}
\multiput(453.00,533.17)(5.340,4.000){2}{\rule{0.400pt}{0.400pt}}
\multiput(460.00,538.60)(0.920,0.468){5}{\rule{0.800pt}{0.113pt}}
\multiput(460.00,537.17)(5.340,4.000){2}{\rule{0.400pt}{0.400pt}}
\multiput(467.00,542.60)(0.920,0.468){5}{\rule{0.800pt}{0.113pt}}
\multiput(467.00,541.17)(5.340,4.000){2}{\rule{0.400pt}{0.400pt}}
\multiput(474.00,546.60)(0.774,0.468){5}{\rule{0.700pt}{0.113pt}}
\multiput(474.00,545.17)(4.547,4.000){2}{\rule{0.350pt}{0.400pt}}
\multiput(480.00,550.60)(0.920,0.468){5}{\rule{0.800pt}{0.113pt}}
\multiput(480.00,549.17)(5.340,4.000){2}{\rule{0.400pt}{0.400pt}}
\multiput(487.00,554.61)(1.355,0.447){3}{\rule{1.033pt}{0.108pt}}
\multiput(487.00,553.17)(4.855,3.000){2}{\rule{0.517pt}{0.400pt}}
\multiput(494.00,557.61)(1.355,0.447){3}{\rule{1.033pt}{0.108pt}}
\multiput(494.00,556.17)(4.855,3.000){2}{\rule{0.517pt}{0.400pt}}
\multiput(501.00,560.61)(1.355,0.447){3}{\rule{1.033pt}{0.108pt}}
\multiput(501.00,559.17)(4.855,3.000){2}{\rule{0.517pt}{0.400pt}}
\multiput(508.00,563.61)(1.355,0.447){3}{\rule{1.033pt}{0.108pt}}
\multiput(508.00,562.17)(4.855,3.000){2}{\rule{0.517pt}{0.400pt}}
\multiput(515.00,566.61)(1.355,0.447){3}{\rule{1.033pt}{0.108pt}}
\multiput(515.00,565.17)(4.855,3.000){2}{\rule{0.517pt}{0.400pt}}
\multiput(522.00,569.61)(1.355,0.447){3}{\rule{1.033pt}{0.108pt}}
\multiput(522.00,568.17)(4.855,3.000){2}{\rule{0.517pt}{0.400pt}}
\put(529,572.17){\rule{1.500pt}{0.400pt}}
\multiput(529.00,571.17)(3.887,2.000){2}{\rule{0.750pt}{0.400pt}}
\put(536,574.17){\rule{1.500pt}{0.400pt}}
\multiput(536.00,573.17)(3.887,2.000){2}{\rule{0.750pt}{0.400pt}}
\put(543,576.17){\rule{1.300pt}{0.400pt}}
\multiput(543.00,575.17)(3.302,2.000){2}{\rule{0.650pt}{0.400pt}}
\put(549,578.17){\rule{1.500pt}{0.400pt}}
\multiput(549.00,577.17)(3.887,2.000){2}{\rule{0.750pt}{0.400pt}}
\put(556,580.17){\rule{1.500pt}{0.400pt}}
\multiput(556.00,579.17)(3.887,2.000){2}{\rule{0.750pt}{0.400pt}}
\put(563,582.17){\rule{1.500pt}{0.400pt}}
\multiput(563.00,581.17)(3.887,2.000){2}{\rule{0.750pt}{0.400pt}}
\put(570,583.67){\rule{1.686pt}{0.400pt}}
\multiput(570.00,583.17)(3.500,1.000){2}{\rule{0.843pt}{0.400pt}}
\put(577,584.67){\rule{1.686pt}{0.400pt}}
\multiput(577.00,584.17)(3.500,1.000){2}{\rule{0.843pt}{0.400pt}}
\put(584,585.67){\rule{1.686pt}{0.400pt}}
\multiput(584.00,585.17)(3.500,1.000){2}{\rule{0.843pt}{0.400pt}}
\put(591,586.67){\rule{1.686pt}{0.400pt}}
\multiput(591.00,586.17)(3.500,1.000){2}{\rule{0.843pt}{0.400pt}}
\put(598,587.67){\rule{1.686pt}{0.400pt}}
\multiput(598.00,587.17)(3.500,1.000){2}{\rule{0.843pt}{0.400pt}}
\put(632.0,559.0){\rule[-0.200pt]{1.686pt}{0.400pt}}
\put(611,588.67){\rule{1.686pt}{0.400pt}}
\multiput(611.00,588.17)(3.500,1.000){2}{\rule{0.843pt}{0.400pt}}
\put(605.0,589.0){\rule[-0.200pt]{1.445pt}{0.400pt}}
\put(618.0,590.0){\rule[-0.200pt]{1.686pt}{0.400pt}}
\put(625.0,590.0){\rule[-0.200pt]{1.686pt}{0.400pt}}
\put(632.0,590.0){\rule[-0.200pt]{1.686pt}{0.400pt}}
\put(646,588.67){\rule{1.686pt}{0.400pt}}
\multiput(646.00,589.17)(3.500,-1.000){2}{\rule{0.843pt}{0.400pt}}
\put(639.0,590.0){\rule[-0.200pt]{1.686pt}{0.400pt}}
\put(660,587.67){\rule{1.686pt}{0.400pt}}
\multiput(660.00,588.17)(3.500,-1.000){2}{\rule{0.843pt}{0.400pt}}
\put(667,586.67){\rule{1.445pt}{0.400pt}}
\multiput(667.00,587.17)(3.000,-1.000){2}{\rule{0.723pt}{0.400pt}}
\put(673,585.67){\rule{1.686pt}{0.400pt}}
\multiput(673.00,586.17)(3.500,-1.000){2}{\rule{0.843pt}{0.400pt}}
\put(680,584.67){\rule{1.686pt}{0.400pt}}
\multiput(680.00,585.17)(3.500,-1.000){2}{\rule{0.843pt}{0.400pt}}
\put(687,583.17){\rule{1.500pt}{0.400pt}}
\multiput(687.00,584.17)(3.887,-2.000){2}{\rule{0.750pt}{0.400pt}}
\put(694,581.67){\rule{1.686pt}{0.400pt}}
\multiput(694.00,582.17)(3.500,-1.000){2}{\rule{0.843pt}{0.400pt}}
\put(701,580.17){\rule{1.500pt}{0.400pt}}
\multiput(701.00,581.17)(3.887,-2.000){2}{\rule{0.750pt}{0.400pt}}
\put(708,578.17){\rule{1.500pt}{0.400pt}}
\multiput(708.00,579.17)(3.887,-2.000){2}{\rule{0.750pt}{0.400pt}}
\put(715,576.17){\rule{1.500pt}{0.400pt}}
\multiput(715.00,577.17)(3.887,-2.000){2}{\rule{0.750pt}{0.400pt}}
\put(722,574.17){\rule{1.500pt}{0.400pt}}
\multiput(722.00,575.17)(3.887,-2.000){2}{\rule{0.750pt}{0.400pt}}
\put(729,572.17){\rule{1.500pt}{0.400pt}}
\multiput(729.00,573.17)(3.887,-2.000){2}{\rule{0.750pt}{0.400pt}}
\multiput(736.00,570.95)(1.132,-0.447){3}{\rule{0.900pt}{0.108pt}}
\multiput(736.00,571.17)(4.132,-3.000){2}{\rule{0.450pt}{0.400pt}}
\put(742,567.17){\rule{1.500pt}{0.400pt}}
\multiput(742.00,568.17)(3.887,-2.000){2}{\rule{0.750pt}{0.400pt}}
\multiput(749.00,565.95)(1.355,-0.447){3}{\rule{1.033pt}{0.108pt}}
\multiput(749.00,566.17)(4.855,-3.000){2}{\rule{0.517pt}{0.400pt}}
\multiput(756.00,562.95)(1.355,-0.447){3}{\rule{1.033pt}{0.108pt}}
\multiput(756.00,563.17)(4.855,-3.000){2}{\rule{0.517pt}{0.400pt}}
\multiput(763.00,559.95)(1.355,-0.447){3}{\rule{1.033pt}{0.108pt}}
\multiput(763.00,560.17)(4.855,-3.000){2}{\rule{0.517pt}{0.400pt}}
\multiput(770.00,556.94)(0.920,-0.468){5}{\rule{0.800pt}{0.113pt}}
\multiput(770.00,557.17)(5.340,-4.000){2}{\rule{0.400pt}{0.400pt}}
\multiput(777.00,552.95)(1.355,-0.447){3}{\rule{1.033pt}{0.108pt}}
\multiput(777.00,553.17)(4.855,-3.000){2}{\rule{0.517pt}{0.400pt}}
\multiput(784.00,549.94)(0.920,-0.468){5}{\rule{0.800pt}{0.113pt}}
\multiput(784.00,550.17)(5.340,-4.000){2}{\rule{0.400pt}{0.400pt}}
\multiput(791.00,545.94)(0.920,-0.468){5}{\rule{0.800pt}{0.113pt}}
\multiput(791.00,546.17)(5.340,-4.000){2}{\rule{0.400pt}{0.400pt}}
\multiput(798.00,541.94)(0.774,-0.468){5}{\rule{0.700pt}{0.113pt}}
\multiput(798.00,542.17)(4.547,-4.000){2}{\rule{0.350pt}{0.400pt}}
\multiput(804.00,537.94)(0.920,-0.468){5}{\rule{0.800pt}{0.113pt}}
\multiput(804.00,538.17)(5.340,-4.000){2}{\rule{0.400pt}{0.400pt}}
\multiput(811.00,533.94)(0.920,-0.468){5}{\rule{0.800pt}{0.113pt}}
\multiput(811.00,534.17)(5.340,-4.000){2}{\rule{0.400pt}{0.400pt}}
\multiput(818.00,529.93)(0.710,-0.477){7}{\rule{0.660pt}{0.115pt}}
\multiput(818.00,530.17)(5.630,-5.000){2}{\rule{0.330pt}{0.400pt}}
\multiput(825.00,524.93)(0.710,-0.477){7}{\rule{0.660pt}{0.115pt}}
\multiput(825.00,525.17)(5.630,-5.000){2}{\rule{0.330pt}{0.400pt}}
\multiput(832.00,519.94)(0.920,-0.468){5}{\rule{0.800pt}{0.113pt}}
\multiput(832.00,520.17)(5.340,-4.000){2}{\rule{0.400pt}{0.400pt}}
\multiput(839.00,515.93)(0.710,-0.477){7}{\rule{0.660pt}{0.115pt}}
\multiput(839.00,516.17)(5.630,-5.000){2}{\rule{0.330pt}{0.400pt}}
\multiput(846.00,510.93)(0.581,-0.482){9}{\rule{0.567pt}{0.116pt}}
\multiput(846.00,511.17)(5.824,-6.000){2}{\rule{0.283pt}{0.400pt}}
\multiput(853.00,504.93)(0.710,-0.477){7}{\rule{0.660pt}{0.115pt}}
\multiput(853.00,505.17)(5.630,-5.000){2}{\rule{0.330pt}{0.400pt}}
\multiput(860.00,499.93)(0.491,-0.482){9}{\rule{0.500pt}{0.116pt}}
\multiput(860.00,500.17)(4.962,-6.000){2}{\rule{0.250pt}{0.400pt}}
\multiput(866.00,493.93)(0.710,-0.477){7}{\rule{0.660pt}{0.115pt}}
\multiput(866.00,494.17)(5.630,-5.000){2}{\rule{0.330pt}{0.400pt}}
\multiput(873.00,488.93)(0.581,-0.482){9}{\rule{0.567pt}{0.116pt}}
\multiput(873.00,489.17)(5.824,-6.000){2}{\rule{0.283pt}{0.400pt}}
\multiput(880.00,482.93)(0.581,-0.482){9}{\rule{0.567pt}{0.116pt}}
\multiput(880.00,483.17)(5.824,-6.000){2}{\rule{0.283pt}{0.400pt}}
\multiput(887.00,476.93)(0.492,-0.485){11}{\rule{0.500pt}{0.117pt}}
\multiput(887.00,477.17)(5.962,-7.000){2}{\rule{0.250pt}{0.400pt}}
\multiput(894.00,469.93)(0.581,-0.482){9}{\rule{0.567pt}{0.116pt}}
\multiput(894.00,470.17)(5.824,-6.000){2}{\rule{0.283pt}{0.400pt}}
\multiput(901.00,463.93)(0.492,-0.485){11}{\rule{0.500pt}{0.117pt}}
\multiput(901.00,464.17)(5.962,-7.000){2}{\rule{0.250pt}{0.400pt}}
\multiput(908.00,456.93)(0.492,-0.485){11}{\rule{0.500pt}{0.117pt}}
\multiput(908.00,457.17)(5.962,-7.000){2}{\rule{0.250pt}{0.400pt}}
\multiput(915.00,449.93)(0.492,-0.485){11}{\rule{0.500pt}{0.117pt}}
\multiput(915.00,450.17)(5.962,-7.000){2}{\rule{0.250pt}{0.400pt}}
\multiput(922.00,442.93)(0.492,-0.485){11}{\rule{0.500pt}{0.117pt}}
\multiput(922.00,443.17)(5.962,-7.000){2}{\rule{0.250pt}{0.400pt}}
\multiput(929.59,434.37)(0.482,-0.671){9}{\rule{0.116pt}{0.633pt}}
\multiput(928.17,435.69)(6.000,-6.685){2}{\rule{0.400pt}{0.317pt}}
\multiput(935.00,427.93)(0.492,-0.485){11}{\rule{0.500pt}{0.117pt}}
\multiput(935.00,428.17)(5.962,-7.000){2}{\rule{0.250pt}{0.400pt}}
\multiput(942.59,419.69)(0.485,-0.569){11}{\rule{0.117pt}{0.557pt}}
\multiput(941.17,420.84)(7.000,-6.844){2}{\rule{0.400pt}{0.279pt}}
\multiput(949.59,411.69)(0.485,-0.569){11}{\rule{0.117pt}{0.557pt}}
\multiput(948.17,412.84)(7.000,-6.844){2}{\rule{0.400pt}{0.279pt}}
\multiput(956.59,403.45)(0.485,-0.645){11}{\rule{0.117pt}{0.614pt}}
\multiput(955.17,404.73)(7.000,-7.725){2}{\rule{0.400pt}{0.307pt}}
\multiput(963.59,394.69)(0.485,-0.569){11}{\rule{0.117pt}{0.557pt}}
\multiput(962.17,395.84)(7.000,-6.844){2}{\rule{0.400pt}{0.279pt}}
\multiput(970.59,386.45)(0.485,-0.645){11}{\rule{0.117pt}{0.614pt}}
\multiput(969.17,387.73)(7.000,-7.725){2}{\rule{0.400pt}{0.307pt}}
\multiput(977.59,377.45)(0.485,-0.645){11}{\rule{0.117pt}{0.614pt}}
\multiput(976.17,378.73)(7.000,-7.725){2}{\rule{0.400pt}{0.307pt}}
\multiput(984.59,368.45)(0.485,-0.645){11}{\rule{0.117pt}{0.614pt}}
\multiput(983.17,369.73)(7.000,-7.725){2}{\rule{0.400pt}{0.307pt}}
\multiput(991.59,359.09)(0.482,-0.762){9}{\rule{0.116pt}{0.700pt}}
\multiput(990.17,360.55)(6.000,-7.547){2}{\rule{0.400pt}{0.350pt}}
\multiput(997.59,350.21)(0.485,-0.721){11}{\rule{0.117pt}{0.671pt}}
\multiput(996.17,351.61)(7.000,-8.606){2}{\rule{0.400pt}{0.336pt}}
\multiput(1004.59,340.21)(0.485,-0.721){11}{\rule{0.117pt}{0.671pt}}
\multiput(1003.17,341.61)(7.000,-8.606){2}{\rule{0.400pt}{0.336pt}}
\multiput(1011.59,330.21)(0.485,-0.721){11}{\rule{0.117pt}{0.671pt}}
\multiput(1010.17,331.61)(7.000,-8.606){2}{\rule{0.400pt}{0.336pt}}
\multiput(356.59,438.00)(0.485,0.569){11}{\rule{0.117pt}{0.557pt}}
\multiput(355.17,438.00)(7.000,6.844){2}{\rule{0.400pt}{0.279pt}}
\multiput(363.59,446.00)(0.485,0.569){11}{\rule{0.117pt}{0.557pt}}
\multiput(362.17,446.00)(7.000,6.844){2}{\rule{0.400pt}{0.279pt}}
\multiput(370.59,454.00)(0.485,0.645){11}{\rule{0.117pt}{0.614pt}}
\multiput(369.17,454.00)(7.000,7.725){2}{\rule{0.400pt}{0.307pt}}
\multiput(377.00,463.59)(0.492,0.485){11}{\rule{0.500pt}{0.117pt}}
\multiput(377.00,462.17)(5.962,7.000){2}{\rule{0.250pt}{0.400pt}}
\multiput(384.59,470.00)(0.485,0.569){11}{\rule{0.117pt}{0.557pt}}
\multiput(383.17,470.00)(7.000,6.844){2}{\rule{0.400pt}{0.279pt}}
\multiput(391.00,478.59)(0.492,0.485){11}{\rule{0.500pt}{0.117pt}}
\multiput(391.00,477.17)(5.962,7.000){2}{\rule{0.250pt}{0.400pt}}
\multiput(398.59,485.00)(0.485,0.569){11}{\rule{0.117pt}{0.557pt}}
\multiput(397.17,485.00)(7.000,6.844){2}{\rule{0.400pt}{0.279pt}}
\multiput(405.00,493.59)(0.492,0.485){11}{\rule{0.500pt}{0.117pt}}
\multiput(405.00,492.17)(5.962,7.000){2}{\rule{0.250pt}{0.400pt}}
\multiput(412.00,500.59)(0.491,0.482){9}{\rule{0.500pt}{0.116pt}}
\multiput(412.00,499.17)(4.962,6.000){2}{\rule{0.250pt}{0.400pt}}
\multiput(418.00,506.59)(0.492,0.485){11}{\rule{0.500pt}{0.117pt}}
\multiput(418.00,505.17)(5.962,7.000){2}{\rule{0.250pt}{0.400pt}}
\multiput(425.00,513.59)(0.581,0.482){9}{\rule{0.567pt}{0.116pt}}
\multiput(425.00,512.17)(5.824,6.000){2}{\rule{0.283pt}{0.400pt}}
\multiput(432.00,519.59)(0.581,0.482){9}{\rule{0.567pt}{0.116pt}}
\multiput(432.00,518.17)(5.824,6.000){2}{\rule{0.283pt}{0.400pt}}
\multiput(439.00,525.59)(0.581,0.482){9}{\rule{0.567pt}{0.116pt}}
\multiput(439.00,524.17)(5.824,6.000){2}{\rule{0.283pt}{0.400pt}}
\multiput(446.00,531.59)(0.710,0.477){7}{\rule{0.660pt}{0.115pt}}
\multiput(446.00,530.17)(5.630,5.000){2}{\rule{0.330pt}{0.400pt}}
\multiput(453.00,536.59)(0.581,0.482){9}{\rule{0.567pt}{0.116pt}}
\multiput(453.00,535.17)(5.824,6.000){2}{\rule{0.283pt}{0.400pt}}
\multiput(460.00,542.59)(0.710,0.477){7}{\rule{0.660pt}{0.115pt}}
\multiput(460.00,541.17)(5.630,5.000){2}{\rule{0.330pt}{0.400pt}}
\multiput(467.00,547.59)(0.710,0.477){7}{\rule{0.660pt}{0.115pt}}
\multiput(467.00,546.17)(5.630,5.000){2}{\rule{0.330pt}{0.400pt}}
\multiput(474.00,552.59)(0.710,0.477){7}{\rule{0.660pt}{0.115pt}}
\multiput(474.00,551.17)(5.630,5.000){2}{\rule{0.330pt}{0.400pt}}
\multiput(481.00,557.60)(0.774,0.468){5}{\rule{0.700pt}{0.113pt}}
\multiput(481.00,556.17)(4.547,4.000){2}{\rule{0.350pt}{0.400pt}}
\multiput(487.00,561.59)(0.710,0.477){7}{\rule{0.660pt}{0.115pt}}
\multiput(487.00,560.17)(5.630,5.000){2}{\rule{0.330pt}{0.400pt}}
\multiput(494.00,566.60)(0.920,0.468){5}{\rule{0.800pt}{0.113pt}}
\multiput(494.00,565.17)(5.340,4.000){2}{\rule{0.400pt}{0.400pt}}
\multiput(501.00,570.60)(0.920,0.468){5}{\rule{0.800pt}{0.113pt}}
\multiput(501.00,569.17)(5.340,4.000){2}{\rule{0.400pt}{0.400pt}}
\multiput(508.00,574.60)(0.920,0.468){5}{\rule{0.800pt}{0.113pt}}
\multiput(508.00,573.17)(5.340,4.000){2}{\rule{0.400pt}{0.400pt}}
\multiput(515.00,578.61)(1.355,0.447){3}{\rule{1.033pt}{0.108pt}}
\multiput(515.00,577.17)(4.855,3.000){2}{\rule{0.517pt}{0.400pt}}
\multiput(522.00,581.60)(0.920,0.468){5}{\rule{0.800pt}{0.113pt}}
\multiput(522.00,580.17)(5.340,4.000){2}{\rule{0.400pt}{0.400pt}}
\multiput(529.00,585.61)(1.355,0.447){3}{\rule{1.033pt}{0.108pt}}
\multiput(529.00,584.17)(4.855,3.000){2}{\rule{0.517pt}{0.400pt}}
\multiput(536.00,588.61)(1.355,0.447){3}{\rule{1.033pt}{0.108pt}}
\multiput(536.00,587.17)(4.855,3.000){2}{\rule{0.517pt}{0.400pt}}
\multiput(543.00,591.61)(1.132,0.447){3}{\rule{0.900pt}{0.108pt}}
\multiput(543.00,590.17)(4.132,3.000){2}{\rule{0.450pt}{0.400pt}}
\put(549,594.17){\rule{1.500pt}{0.400pt}}
\multiput(549.00,593.17)(3.887,2.000){2}{\rule{0.750pt}{0.400pt}}
\multiput(556.00,596.61)(1.355,0.447){3}{\rule{1.033pt}{0.108pt}}
\multiput(556.00,595.17)(4.855,3.000){2}{\rule{0.517pt}{0.400pt}}
\put(563,599.17){\rule{1.500pt}{0.400pt}}
\multiput(563.00,598.17)(3.887,2.000){2}{\rule{0.750pt}{0.400pt}}
\put(570,601.17){\rule{1.500pt}{0.400pt}}
\multiput(570.00,600.17)(3.887,2.000){2}{\rule{0.750pt}{0.400pt}}
\put(577,603.17){\rule{1.500pt}{0.400pt}}
\multiput(577.00,602.17)(3.887,2.000){2}{\rule{0.750pt}{0.400pt}}
\put(584,605.17){\rule{1.500pt}{0.400pt}}
\multiput(584.00,604.17)(3.887,2.000){2}{\rule{0.750pt}{0.400pt}}
\put(591,606.67){\rule{1.686pt}{0.400pt}}
\multiput(591.00,606.17)(3.500,1.000){2}{\rule{0.843pt}{0.400pt}}
\put(598,608.17){\rule{1.500pt}{0.400pt}}
\multiput(598.00,607.17)(3.887,2.000){2}{\rule{0.750pt}{0.400pt}}
\put(605,609.67){\rule{1.445pt}{0.400pt}}
\multiput(605.00,609.17)(3.000,1.000){2}{\rule{0.723pt}{0.400pt}}
\put(611,610.67){\rule{1.686pt}{0.400pt}}
\multiput(611.00,610.17)(3.500,1.000){2}{\rule{0.843pt}{0.400pt}}
\put(618,611.67){\rule{1.686pt}{0.400pt}}
\multiput(618.00,611.17)(3.500,1.000){2}{\rule{0.843pt}{0.400pt}}
\put(625,612.67){\rule{1.686pt}{0.400pt}}
\multiput(625.00,612.17)(3.500,1.000){2}{\rule{0.843pt}{0.400pt}}
\put(653.0,589.0){\rule[-0.200pt]{1.686pt}{0.400pt}}
\put(632.0,614.0){\rule[-0.200pt]{1.686pt}{0.400pt}}
\put(646,613.67){\rule{1.686pt}{0.400pt}}
\multiput(646.00,613.17)(3.500,1.000){2}{\rule{0.843pt}{0.400pt}}
\put(639.0,614.0){\rule[-0.200pt]{1.686pt}{0.400pt}}
\put(660,613.67){\rule{1.686pt}{0.400pt}}
\multiput(660.00,614.17)(3.500,-1.000){2}{\rule{0.843pt}{0.400pt}}
\put(653.0,615.0){\rule[-0.200pt]{1.686pt}{0.400pt}}
\put(674,612.67){\rule{1.445pt}{0.400pt}}
\multiput(674.00,613.17)(3.000,-1.000){2}{\rule{0.723pt}{0.400pt}}
\put(667.0,614.0){\rule[-0.200pt]{1.686pt}{0.400pt}}
\put(687,611.67){\rule{1.686pt}{0.400pt}}
\multiput(687.00,612.17)(3.500,-1.000){2}{\rule{0.843pt}{0.400pt}}
\put(694,610.67){\rule{1.686pt}{0.400pt}}
\multiput(694.00,611.17)(3.500,-1.000){2}{\rule{0.843pt}{0.400pt}}
\put(701,609.67){\rule{1.686pt}{0.400pt}}
\multiput(701.00,610.17)(3.500,-1.000){2}{\rule{0.843pt}{0.400pt}}
\put(708,608.17){\rule{1.500pt}{0.400pt}}
\multiput(708.00,609.17)(3.887,-2.000){2}{\rule{0.750pt}{0.400pt}}
\put(715,606.67){\rule{1.686pt}{0.400pt}}
\multiput(715.00,607.17)(3.500,-1.000){2}{\rule{0.843pt}{0.400pt}}
\put(722,605.17){\rule{1.500pt}{0.400pt}}
\multiput(722.00,606.17)(3.887,-2.000){2}{\rule{0.750pt}{0.400pt}}
\put(729,603.17){\rule{1.500pt}{0.400pt}}
\multiput(729.00,604.17)(3.887,-2.000){2}{\rule{0.750pt}{0.400pt}}
\put(736,601.17){\rule{1.300pt}{0.400pt}}
\multiput(736.00,602.17)(3.302,-2.000){2}{\rule{0.650pt}{0.400pt}}
\put(742,599.17){\rule{1.500pt}{0.400pt}}
\multiput(742.00,600.17)(3.887,-2.000){2}{\rule{0.750pt}{0.400pt}}
\multiput(749.00,597.95)(1.355,-0.447){3}{\rule{1.033pt}{0.108pt}}
\multiput(749.00,598.17)(4.855,-3.000){2}{\rule{0.517pt}{0.400pt}}
\put(756,594.17){\rule{1.500pt}{0.400pt}}
\multiput(756.00,595.17)(3.887,-2.000){2}{\rule{0.750pt}{0.400pt}}
\multiput(763.00,592.95)(1.355,-0.447){3}{\rule{1.033pt}{0.108pt}}
\multiput(763.00,593.17)(4.855,-3.000){2}{\rule{0.517pt}{0.400pt}}
\multiput(770.00,589.95)(1.355,-0.447){3}{\rule{1.033pt}{0.108pt}}
\multiput(770.00,590.17)(4.855,-3.000){2}{\rule{0.517pt}{0.400pt}}
\multiput(777.00,586.95)(1.355,-0.447){3}{\rule{1.033pt}{0.108pt}}
\multiput(777.00,587.17)(4.855,-3.000){2}{\rule{0.517pt}{0.400pt}}
\multiput(784.00,583.94)(0.920,-0.468){5}{\rule{0.800pt}{0.113pt}}
\multiput(784.00,584.17)(5.340,-4.000){2}{\rule{0.400pt}{0.400pt}}
\multiput(791.00,579.95)(1.355,-0.447){3}{\rule{1.033pt}{0.108pt}}
\multiput(791.00,580.17)(4.855,-3.000){2}{\rule{0.517pt}{0.400pt}}
\multiput(798.00,576.94)(0.920,-0.468){5}{\rule{0.800pt}{0.113pt}}
\multiput(798.00,577.17)(5.340,-4.000){2}{\rule{0.400pt}{0.400pt}}
\multiput(805.00,572.94)(0.774,-0.468){5}{\rule{0.700pt}{0.113pt}}
\multiput(805.00,573.17)(4.547,-4.000){2}{\rule{0.350pt}{0.400pt}}
\multiput(811.00,568.94)(0.920,-0.468){5}{\rule{0.800pt}{0.113pt}}
\multiput(811.00,569.17)(5.340,-4.000){2}{\rule{0.400pt}{0.400pt}}
\multiput(818.00,564.94)(0.920,-0.468){5}{\rule{0.800pt}{0.113pt}}
\multiput(818.00,565.17)(5.340,-4.000){2}{\rule{0.400pt}{0.400pt}}
\multiput(825.00,560.94)(0.920,-0.468){5}{\rule{0.800pt}{0.113pt}}
\multiput(825.00,561.17)(5.340,-4.000){2}{\rule{0.400pt}{0.400pt}}
\multiput(832.00,556.93)(0.710,-0.477){7}{\rule{0.660pt}{0.115pt}}
\multiput(832.00,557.17)(5.630,-5.000){2}{\rule{0.330pt}{0.400pt}}
\multiput(839.00,551.93)(0.710,-0.477){7}{\rule{0.660pt}{0.115pt}}
\multiput(839.00,552.17)(5.630,-5.000){2}{\rule{0.330pt}{0.400pt}}
\multiput(846.00,546.93)(0.710,-0.477){7}{\rule{0.660pt}{0.115pt}}
\multiput(846.00,547.17)(5.630,-5.000){2}{\rule{0.330pt}{0.400pt}}
\multiput(853.00,541.93)(0.710,-0.477){7}{\rule{0.660pt}{0.115pt}}
\multiput(853.00,542.17)(5.630,-5.000){2}{\rule{0.330pt}{0.400pt}}
\multiput(860.00,536.93)(0.710,-0.477){7}{\rule{0.660pt}{0.115pt}}
\multiput(860.00,537.17)(5.630,-5.000){2}{\rule{0.330pt}{0.400pt}}
\multiput(867.00,531.93)(0.491,-0.482){9}{\rule{0.500pt}{0.116pt}}
\multiput(867.00,532.17)(4.962,-6.000){2}{\rule{0.250pt}{0.400pt}}
\multiput(873.00,525.93)(0.710,-0.477){7}{\rule{0.660pt}{0.115pt}}
\multiput(873.00,526.17)(5.630,-5.000){2}{\rule{0.330pt}{0.400pt}}
\multiput(880.00,520.93)(0.581,-0.482){9}{\rule{0.567pt}{0.116pt}}
\multiput(880.00,521.17)(5.824,-6.000){2}{\rule{0.283pt}{0.400pt}}
\multiput(887.00,514.93)(0.581,-0.482){9}{\rule{0.567pt}{0.116pt}}
\multiput(887.00,515.17)(5.824,-6.000){2}{\rule{0.283pt}{0.400pt}}
\multiput(894.00,508.93)(0.492,-0.485){11}{\rule{0.500pt}{0.117pt}}
\multiput(894.00,509.17)(5.962,-7.000){2}{\rule{0.250pt}{0.400pt}}
\multiput(901.00,501.93)(0.581,-0.482){9}{\rule{0.567pt}{0.116pt}}
\multiput(901.00,502.17)(5.824,-6.000){2}{\rule{0.283pt}{0.400pt}}
\multiput(908.00,495.93)(0.492,-0.485){11}{\rule{0.500pt}{0.117pt}}
\multiput(908.00,496.17)(5.962,-7.000){2}{\rule{0.250pt}{0.400pt}}
\multiput(915.00,488.93)(0.492,-0.485){11}{\rule{0.500pt}{0.117pt}}
\multiput(915.00,489.17)(5.962,-7.000){2}{\rule{0.250pt}{0.400pt}}
\multiput(922.00,481.93)(0.492,-0.485){11}{\rule{0.500pt}{0.117pt}}
\multiput(922.00,482.17)(5.962,-7.000){2}{\rule{0.250pt}{0.400pt}}
\multiput(929.59,473.65)(0.482,-0.581){9}{\rule{0.116pt}{0.567pt}}
\multiput(928.17,474.82)(6.000,-5.824){2}{\rule{0.400pt}{0.283pt}}
\multiput(935.59,466.69)(0.485,-0.569){11}{\rule{0.117pt}{0.557pt}}
\multiput(934.17,467.84)(7.000,-6.844){2}{\rule{0.400pt}{0.279pt}}
\multiput(942.00,459.93)(0.492,-0.485){11}{\rule{0.500pt}{0.117pt}}
\multiput(942.00,460.17)(5.962,-7.000){2}{\rule{0.250pt}{0.400pt}}
\multiput(949.59,451.69)(0.485,-0.569){11}{\rule{0.117pt}{0.557pt}}
\multiput(948.17,452.84)(7.000,-6.844){2}{\rule{0.400pt}{0.279pt}}
\multiput(956.59,443.69)(0.485,-0.569){11}{\rule{0.117pt}{0.557pt}}
\multiput(955.17,444.84)(7.000,-6.844){2}{\rule{0.400pt}{0.279pt}}
\multiput(963.59,435.45)(0.485,-0.645){11}{\rule{0.117pt}{0.614pt}}
\multiput(962.17,436.73)(7.000,-7.725){2}{\rule{0.400pt}{0.307pt}}
\multiput(970.59,426.69)(0.485,-0.569){11}{\rule{0.117pt}{0.557pt}}
\multiput(969.17,427.84)(7.000,-6.844){2}{\rule{0.400pt}{0.279pt}}
\multiput(977.59,418.45)(0.485,-0.645){11}{\rule{0.117pt}{0.614pt}}
\multiput(976.17,419.73)(7.000,-7.725){2}{\rule{0.400pt}{0.307pt}}
\multiput(984.59,409.45)(0.485,-0.645){11}{\rule{0.117pt}{0.614pt}}
\multiput(983.17,410.73)(7.000,-7.725){2}{\rule{0.400pt}{0.307pt}}
\multiput(991.59,400.21)(0.485,-0.721){11}{\rule{0.117pt}{0.671pt}}
\multiput(990.17,401.61)(7.000,-8.606){2}{\rule{0.400pt}{0.336pt}}
\multiput(998.59,390.09)(0.482,-0.762){9}{\rule{0.116pt}{0.700pt}}
\multiput(997.17,391.55)(6.000,-7.547){2}{\rule{0.400pt}{0.350pt}}
\multiput(1004.59,381.21)(0.485,-0.721){11}{\rule{0.117pt}{0.671pt}}
\multiput(1003.17,382.61)(7.000,-8.606){2}{\rule{0.400pt}{0.336pt}}
\multiput(1011.59,371.21)(0.485,-0.721){11}{\rule{0.117pt}{0.671pt}}
\multiput(1010.17,372.61)(7.000,-8.606){2}{\rule{0.400pt}{0.336pt}}
\multiput(1018.59,361.21)(0.485,-0.721){11}{\rule{0.117pt}{0.671pt}}
\multiput(1017.17,362.61)(7.000,-8.606){2}{\rule{0.400pt}{0.336pt}}
\multiput(1025.59,351.21)(0.485,-0.721){11}{\rule{0.117pt}{0.671pt}}
\multiput(1024.17,352.61)(7.000,-8.606){2}{\rule{0.400pt}{0.336pt}}
\multiput(1032.59,340.98)(0.485,-0.798){11}{\rule{0.117pt}{0.729pt}}
\multiput(1031.17,342.49)(7.000,-9.488){2}{\rule{0.400pt}{0.364pt}}
\multiput(377.59,447.00)(0.485,0.645){11}{\rule{0.117pt}{0.614pt}}
\multiput(376.17,447.00)(7.000,7.725){2}{\rule{0.400pt}{0.307pt}}
\multiput(384.59,456.00)(0.485,0.645){11}{\rule{0.117pt}{0.614pt}}
\multiput(383.17,456.00)(7.000,7.725){2}{\rule{0.400pt}{0.307pt}}
\multiput(391.59,465.00)(0.485,0.569){11}{\rule{0.117pt}{0.557pt}}
\multiput(390.17,465.00)(7.000,6.844){2}{\rule{0.400pt}{0.279pt}}
\multiput(398.59,473.00)(0.485,0.645){11}{\rule{0.117pt}{0.614pt}}
\multiput(397.17,473.00)(7.000,7.725){2}{\rule{0.400pt}{0.307pt}}
\multiput(405.59,482.00)(0.485,0.569){11}{\rule{0.117pt}{0.557pt}}
\multiput(404.17,482.00)(7.000,6.844){2}{\rule{0.400pt}{0.279pt}}
\multiput(412.00,490.59)(0.492,0.485){11}{\rule{0.500pt}{0.117pt}}
\multiput(412.00,489.17)(5.962,7.000){2}{\rule{0.250pt}{0.400pt}}
\multiput(419.59,497.00)(0.482,0.671){9}{\rule{0.116pt}{0.633pt}}
\multiput(418.17,497.00)(6.000,6.685){2}{\rule{0.400pt}{0.317pt}}
\multiput(425.00,505.59)(0.492,0.485){11}{\rule{0.500pt}{0.117pt}}
\multiput(425.00,504.17)(5.962,7.000){2}{\rule{0.250pt}{0.400pt}}
\multiput(432.00,512.59)(0.492,0.485){11}{\rule{0.500pt}{0.117pt}}
\multiput(432.00,511.17)(5.962,7.000){2}{\rule{0.250pt}{0.400pt}}
\multiput(439.00,519.59)(0.492,0.485){11}{\rule{0.500pt}{0.117pt}}
\multiput(439.00,518.17)(5.962,7.000){2}{\rule{0.250pt}{0.400pt}}
\multiput(446.00,526.59)(0.492,0.485){11}{\rule{0.500pt}{0.117pt}}
\multiput(446.00,525.17)(5.962,7.000){2}{\rule{0.250pt}{0.400pt}}
\multiput(453.00,533.59)(0.581,0.482){9}{\rule{0.567pt}{0.116pt}}
\multiput(453.00,532.17)(5.824,6.000){2}{\rule{0.283pt}{0.400pt}}
\multiput(460.00,539.59)(0.581,0.482){9}{\rule{0.567pt}{0.116pt}}
\multiput(460.00,538.17)(5.824,6.000){2}{\rule{0.283pt}{0.400pt}}
\multiput(467.00,545.59)(0.581,0.482){9}{\rule{0.567pt}{0.116pt}}
\multiput(467.00,544.17)(5.824,6.000){2}{\rule{0.283pt}{0.400pt}}
\multiput(474.00,551.59)(0.581,0.482){9}{\rule{0.567pt}{0.116pt}}
\multiput(474.00,550.17)(5.824,6.000){2}{\rule{0.283pt}{0.400pt}}
\multiput(481.00,557.59)(0.599,0.477){7}{\rule{0.580pt}{0.115pt}}
\multiput(481.00,556.17)(4.796,5.000){2}{\rule{0.290pt}{0.400pt}}
\multiput(487.00,562.59)(0.710,0.477){7}{\rule{0.660pt}{0.115pt}}
\multiput(487.00,561.17)(5.630,5.000){2}{\rule{0.330pt}{0.400pt}}
\multiput(494.00,567.59)(0.581,0.482){9}{\rule{0.567pt}{0.116pt}}
\multiput(494.00,566.17)(5.824,6.000){2}{\rule{0.283pt}{0.400pt}}
\multiput(501.00,573.60)(0.920,0.468){5}{\rule{0.800pt}{0.113pt}}
\multiput(501.00,572.17)(5.340,4.000){2}{\rule{0.400pt}{0.400pt}}
\multiput(508.00,577.59)(0.710,0.477){7}{\rule{0.660pt}{0.115pt}}
\multiput(508.00,576.17)(5.630,5.000){2}{\rule{0.330pt}{0.400pt}}
\multiput(515.00,582.60)(0.920,0.468){5}{\rule{0.800pt}{0.113pt}}
\multiput(515.00,581.17)(5.340,4.000){2}{\rule{0.400pt}{0.400pt}}
\multiput(522.00,586.59)(0.710,0.477){7}{\rule{0.660pt}{0.115pt}}
\multiput(522.00,585.17)(5.630,5.000){2}{\rule{0.330pt}{0.400pt}}
\multiput(529.00,591.60)(0.920,0.468){5}{\rule{0.800pt}{0.113pt}}
\multiput(529.00,590.17)(5.340,4.000){2}{\rule{0.400pt}{0.400pt}}
\multiput(536.00,595.61)(1.355,0.447){3}{\rule{1.033pt}{0.108pt}}
\multiput(536.00,594.17)(4.855,3.000){2}{\rule{0.517pt}{0.400pt}}
\multiput(543.00,598.60)(0.920,0.468){5}{\rule{0.800pt}{0.113pt}}
\multiput(543.00,597.17)(5.340,4.000){2}{\rule{0.400pt}{0.400pt}}
\multiput(550.00,602.61)(1.132,0.447){3}{\rule{0.900pt}{0.108pt}}
\multiput(550.00,601.17)(4.132,3.000){2}{\rule{0.450pt}{0.400pt}}
\multiput(556.00,605.60)(0.920,0.468){5}{\rule{0.800pt}{0.113pt}}
\multiput(556.00,604.17)(5.340,4.000){2}{\rule{0.400pt}{0.400pt}}
\multiput(563.00,609.61)(1.355,0.447){3}{\rule{1.033pt}{0.108pt}}
\multiput(563.00,608.17)(4.855,3.000){2}{\rule{0.517pt}{0.400pt}}
\put(570,612.17){\rule{1.500pt}{0.400pt}}
\multiput(570.00,611.17)(3.887,2.000){2}{\rule{0.750pt}{0.400pt}}
\multiput(577.00,614.61)(1.355,0.447){3}{\rule{1.033pt}{0.108pt}}
\multiput(577.00,613.17)(4.855,3.000){2}{\rule{0.517pt}{0.400pt}}
\put(584,617.17){\rule{1.500pt}{0.400pt}}
\multiput(584.00,616.17)(3.887,2.000){2}{\rule{0.750pt}{0.400pt}}
\multiput(591.00,619.61)(1.355,0.447){3}{\rule{1.033pt}{0.108pt}}
\multiput(591.00,618.17)(4.855,3.000){2}{\rule{0.517pt}{0.400pt}}
\put(598,622.17){\rule{1.500pt}{0.400pt}}
\multiput(598.00,621.17)(3.887,2.000){2}{\rule{0.750pt}{0.400pt}}
\put(605,624.17){\rule{1.500pt}{0.400pt}}
\multiput(605.00,623.17)(3.887,2.000){2}{\rule{0.750pt}{0.400pt}}
\put(612,625.67){\rule{1.445pt}{0.400pt}}
\multiput(612.00,625.17)(3.000,1.000){2}{\rule{0.723pt}{0.400pt}}
\put(618,627.17){\rule{1.500pt}{0.400pt}}
\multiput(618.00,626.17)(3.887,2.000){2}{\rule{0.750pt}{0.400pt}}
\put(625,628.67){\rule{1.686pt}{0.400pt}}
\multiput(625.00,628.17)(3.500,1.000){2}{\rule{0.843pt}{0.400pt}}
\put(632,629.67){\rule{1.686pt}{0.400pt}}
\multiput(632.00,629.17)(3.500,1.000){2}{\rule{0.843pt}{0.400pt}}
\put(639,630.67){\rule{1.686pt}{0.400pt}}
\multiput(639.00,630.17)(3.500,1.000){2}{\rule{0.843pt}{0.400pt}}
\put(646,631.67){\rule{1.686pt}{0.400pt}}
\multiput(646.00,631.17)(3.500,1.000){2}{\rule{0.843pt}{0.400pt}}
\put(680.0,613.0){\rule[-0.200pt]{1.686pt}{0.400pt}}
\put(660,632.67){\rule{1.686pt}{0.400pt}}
\multiput(660.00,632.17)(3.500,1.000){2}{\rule{0.843pt}{0.400pt}}
\put(653.0,633.0){\rule[-0.200pt]{1.686pt}{0.400pt}}
\put(667.0,634.0){\rule[-0.200pt]{1.686pt}{0.400pt}}
\put(674.0,634.0){\rule[-0.200pt]{1.445pt}{0.400pt}}
\put(680.0,634.0){\rule[-0.200pt]{1.686pt}{0.400pt}}
\put(694,632.67){\rule{1.686pt}{0.400pt}}
\multiput(694.00,633.17)(3.500,-1.000){2}{\rule{0.843pt}{0.400pt}}
\put(701,631.67){\rule{1.686pt}{0.400pt}}
\multiput(701.00,632.17)(3.500,-1.000){2}{\rule{0.843pt}{0.400pt}}
\put(708,630.67){\rule{1.686pt}{0.400pt}}
\multiput(708.00,631.17)(3.500,-1.000){2}{\rule{0.843pt}{0.400pt}}
\put(715,629.67){\rule{1.686pt}{0.400pt}}
\multiput(715.00,630.17)(3.500,-1.000){2}{\rule{0.843pt}{0.400pt}}
\put(722,628.67){\rule{1.686pt}{0.400pt}}
\multiput(722.00,629.17)(3.500,-1.000){2}{\rule{0.843pt}{0.400pt}}
\put(729,627.67){\rule{1.686pt}{0.400pt}}
\multiput(729.00,628.17)(3.500,-1.000){2}{\rule{0.843pt}{0.400pt}}
\put(736,626.17){\rule{1.500pt}{0.400pt}}
\multiput(736.00,627.17)(3.887,-2.000){2}{\rule{0.750pt}{0.400pt}}
\put(743,624.17){\rule{1.300pt}{0.400pt}}
\multiput(743.00,625.17)(3.302,-2.000){2}{\rule{0.650pt}{0.400pt}}
\put(749,622.17){\rule{1.500pt}{0.400pt}}
\multiput(749.00,623.17)(3.887,-2.000){2}{\rule{0.750pt}{0.400pt}}
\put(756,620.17){\rule{1.500pt}{0.400pt}}
\multiput(756.00,621.17)(3.887,-2.000){2}{\rule{0.750pt}{0.400pt}}
\put(763,618.17){\rule{1.500pt}{0.400pt}}
\multiput(763.00,619.17)(3.887,-2.000){2}{\rule{0.750pt}{0.400pt}}
\multiput(770.00,616.95)(1.355,-0.447){3}{\rule{1.033pt}{0.108pt}}
\multiput(770.00,617.17)(4.855,-3.000){2}{\rule{0.517pt}{0.400pt}}
\put(777,613.17){\rule{1.500pt}{0.400pt}}
\multiput(777.00,614.17)(3.887,-2.000){2}{\rule{0.750pt}{0.400pt}}
\multiput(784.00,611.95)(1.355,-0.447){3}{\rule{1.033pt}{0.108pt}}
\multiput(784.00,612.17)(4.855,-3.000){2}{\rule{0.517pt}{0.400pt}}
\multiput(791.00,608.95)(1.355,-0.447){3}{\rule{1.033pt}{0.108pt}}
\multiput(791.00,609.17)(4.855,-3.000){2}{\rule{0.517pt}{0.400pt}}
\multiput(798.00,605.95)(1.355,-0.447){3}{\rule{1.033pt}{0.108pt}}
\multiput(798.00,606.17)(4.855,-3.000){2}{\rule{0.517pt}{0.400pt}}
\multiput(805.00,602.94)(0.774,-0.468){5}{\rule{0.700pt}{0.113pt}}
\multiput(805.00,603.17)(4.547,-4.000){2}{\rule{0.350pt}{0.400pt}}
\multiput(811.00,598.95)(1.355,-0.447){3}{\rule{1.033pt}{0.108pt}}
\multiput(811.00,599.17)(4.855,-3.000){2}{\rule{0.517pt}{0.400pt}}
\multiput(818.00,595.94)(0.920,-0.468){5}{\rule{0.800pt}{0.113pt}}
\multiput(818.00,596.17)(5.340,-4.000){2}{\rule{0.400pt}{0.400pt}}
\multiput(825.00,591.94)(0.920,-0.468){5}{\rule{0.800pt}{0.113pt}}
\multiput(825.00,592.17)(5.340,-4.000){2}{\rule{0.400pt}{0.400pt}}
\multiput(832.00,587.94)(0.920,-0.468){5}{\rule{0.800pt}{0.113pt}}
\multiput(832.00,588.17)(5.340,-4.000){2}{\rule{0.400pt}{0.400pt}}
\multiput(839.00,583.93)(0.710,-0.477){7}{\rule{0.660pt}{0.115pt}}
\multiput(839.00,584.17)(5.630,-5.000){2}{\rule{0.330pt}{0.400pt}}
\multiput(846.00,578.94)(0.920,-0.468){5}{\rule{0.800pt}{0.113pt}}
\multiput(846.00,579.17)(5.340,-4.000){2}{\rule{0.400pt}{0.400pt}}
\multiput(853.00,574.93)(0.710,-0.477){7}{\rule{0.660pt}{0.115pt}}
\multiput(853.00,575.17)(5.630,-5.000){2}{\rule{0.330pt}{0.400pt}}
\multiput(860.00,569.93)(0.710,-0.477){7}{\rule{0.660pt}{0.115pt}}
\multiput(860.00,570.17)(5.630,-5.000){2}{\rule{0.330pt}{0.400pt}}
\multiput(867.00,564.93)(0.599,-0.477){7}{\rule{0.580pt}{0.115pt}}
\multiput(867.00,565.17)(4.796,-5.000){2}{\rule{0.290pt}{0.400pt}}
\multiput(873.00,559.93)(0.710,-0.477){7}{\rule{0.660pt}{0.115pt}}
\multiput(873.00,560.17)(5.630,-5.000){2}{\rule{0.330pt}{0.400pt}}
\multiput(880.00,554.93)(0.581,-0.482){9}{\rule{0.567pt}{0.116pt}}
\multiput(880.00,555.17)(5.824,-6.000){2}{\rule{0.283pt}{0.400pt}}
\multiput(887.00,548.93)(0.581,-0.482){9}{\rule{0.567pt}{0.116pt}}
\multiput(887.00,549.17)(5.824,-6.000){2}{\rule{0.283pt}{0.400pt}}
\multiput(894.00,542.93)(0.710,-0.477){7}{\rule{0.660pt}{0.115pt}}
\multiput(894.00,543.17)(5.630,-5.000){2}{\rule{0.330pt}{0.400pt}}
\multiput(901.00,537.93)(0.492,-0.485){11}{\rule{0.500pt}{0.117pt}}
\multiput(901.00,538.17)(5.962,-7.000){2}{\rule{0.250pt}{0.400pt}}
\multiput(908.00,530.93)(0.581,-0.482){9}{\rule{0.567pt}{0.116pt}}
\multiput(908.00,531.17)(5.824,-6.000){2}{\rule{0.283pt}{0.400pt}}
\multiput(915.00,524.93)(0.581,-0.482){9}{\rule{0.567pt}{0.116pt}}
\multiput(915.00,525.17)(5.824,-6.000){2}{\rule{0.283pt}{0.400pt}}
\multiput(922.00,518.93)(0.492,-0.485){11}{\rule{0.500pt}{0.117pt}}
\multiput(922.00,519.17)(5.962,-7.000){2}{\rule{0.250pt}{0.400pt}}
\multiput(929.00,511.93)(0.492,-0.485){11}{\rule{0.500pt}{0.117pt}}
\multiput(929.00,512.17)(5.962,-7.000){2}{\rule{0.250pt}{0.400pt}}
\multiput(936.59,503.65)(0.482,-0.581){9}{\rule{0.116pt}{0.567pt}}
\multiput(935.17,504.82)(6.000,-5.824){2}{\rule{0.400pt}{0.283pt}}
\multiput(942.59,496.69)(0.485,-0.569){11}{\rule{0.117pt}{0.557pt}}
\multiput(941.17,497.84)(7.000,-6.844){2}{\rule{0.400pt}{0.279pt}}
\multiput(949.00,489.93)(0.492,-0.485){11}{\rule{0.500pt}{0.117pt}}
\multiput(949.00,490.17)(5.962,-7.000){2}{\rule{0.250pt}{0.400pt}}
\multiput(956.59,481.69)(0.485,-0.569){11}{\rule{0.117pt}{0.557pt}}
\multiput(955.17,482.84)(7.000,-6.844){2}{\rule{0.400pt}{0.279pt}}
\multiput(963.59,473.69)(0.485,-0.569){11}{\rule{0.117pt}{0.557pt}}
\multiput(962.17,474.84)(7.000,-6.844){2}{\rule{0.400pt}{0.279pt}}
\multiput(970.59,465.69)(0.485,-0.569){11}{\rule{0.117pt}{0.557pt}}
\multiput(969.17,466.84)(7.000,-6.844){2}{\rule{0.400pt}{0.279pt}}
\multiput(977.59,457.45)(0.485,-0.645){11}{\rule{0.117pt}{0.614pt}}
\multiput(976.17,458.73)(7.000,-7.725){2}{\rule{0.400pt}{0.307pt}}
\multiput(984.59,448.69)(0.485,-0.569){11}{\rule{0.117pt}{0.557pt}}
\multiput(983.17,449.84)(7.000,-6.844){2}{\rule{0.400pt}{0.279pt}}
\multiput(991.59,440.45)(0.485,-0.645){11}{\rule{0.117pt}{0.614pt}}
\multiput(990.17,441.73)(7.000,-7.725){2}{\rule{0.400pt}{0.307pt}}
\multiput(998.59,430.82)(0.482,-0.852){9}{\rule{0.116pt}{0.767pt}}
\multiput(997.17,432.41)(6.000,-8.409){2}{\rule{0.400pt}{0.383pt}}
\multiput(1004.59,421.45)(0.485,-0.645){11}{\rule{0.117pt}{0.614pt}}
\multiput(1003.17,422.73)(7.000,-7.725){2}{\rule{0.400pt}{0.307pt}}
\multiput(1011.59,412.21)(0.485,-0.721){11}{\rule{0.117pt}{0.671pt}}
\multiput(1010.17,413.61)(7.000,-8.606){2}{\rule{0.400pt}{0.336pt}}
\multiput(1018.59,402.45)(0.485,-0.645){11}{\rule{0.117pt}{0.614pt}}
\multiput(1017.17,403.73)(7.000,-7.725){2}{\rule{0.400pt}{0.307pt}}
\multiput(1025.59,392.98)(0.485,-0.798){11}{\rule{0.117pt}{0.729pt}}
\multiput(1024.17,394.49)(7.000,-9.488){2}{\rule{0.400pt}{0.364pt}}
\multiput(1032.59,382.21)(0.485,-0.721){11}{\rule{0.117pt}{0.671pt}}
\multiput(1031.17,383.61)(7.000,-8.606){2}{\rule{0.400pt}{0.336pt}}
\multiput(1039.59,371.98)(0.485,-0.798){11}{\rule{0.117pt}{0.729pt}}
\multiput(1038.17,373.49)(7.000,-9.488){2}{\rule{0.400pt}{0.364pt}}
\multiput(1046.59,360.98)(0.485,-0.798){11}{\rule{0.117pt}{0.729pt}}
\multiput(1045.17,362.49)(7.000,-9.488){2}{\rule{0.400pt}{0.364pt}}
\multiput(1053.59,349.98)(0.485,-0.798){11}{\rule{0.117pt}{0.729pt}}
\multiput(1052.17,351.49)(7.000,-9.488){2}{\rule{0.400pt}{0.364pt}}
\multiput(398.59,457.00)(0.485,0.645){11}{\rule{0.117pt}{0.614pt}}
\multiput(397.17,457.00)(7.000,7.725){2}{\rule{0.400pt}{0.307pt}}
\multiput(405.59,466.00)(0.485,0.645){11}{\rule{0.117pt}{0.614pt}}
\multiput(404.17,466.00)(7.000,7.725){2}{\rule{0.400pt}{0.307pt}}
\multiput(412.59,475.00)(0.485,0.645){11}{\rule{0.117pt}{0.614pt}}
\multiput(411.17,475.00)(7.000,7.725){2}{\rule{0.400pt}{0.307pt}}
\multiput(419.59,484.00)(0.482,0.671){9}{\rule{0.116pt}{0.633pt}}
\multiput(418.17,484.00)(6.000,6.685){2}{\rule{0.400pt}{0.317pt}}
\multiput(425.59,492.00)(0.485,0.569){11}{\rule{0.117pt}{0.557pt}}
\multiput(424.17,492.00)(7.000,6.844){2}{\rule{0.400pt}{0.279pt}}
\multiput(432.59,500.00)(0.485,0.569){11}{\rule{0.117pt}{0.557pt}}
\multiput(431.17,500.00)(7.000,6.844){2}{\rule{0.400pt}{0.279pt}}
\multiput(439.59,508.00)(0.485,0.569){11}{\rule{0.117pt}{0.557pt}}
\multiput(438.17,508.00)(7.000,6.844){2}{\rule{0.400pt}{0.279pt}}
\multiput(446.00,516.59)(0.492,0.485){11}{\rule{0.500pt}{0.117pt}}
\multiput(446.00,515.17)(5.962,7.000){2}{\rule{0.250pt}{0.400pt}}
\multiput(453.59,523.00)(0.485,0.569){11}{\rule{0.117pt}{0.557pt}}
\multiput(452.17,523.00)(7.000,6.844){2}{\rule{0.400pt}{0.279pt}}
\multiput(460.00,531.59)(0.492,0.485){11}{\rule{0.500pt}{0.117pt}}
\multiput(460.00,530.17)(5.962,7.000){2}{\rule{0.250pt}{0.400pt}}
\multiput(467.00,538.59)(0.581,0.482){9}{\rule{0.567pt}{0.116pt}}
\multiput(467.00,537.17)(5.824,6.000){2}{\rule{0.283pt}{0.400pt}}
\multiput(474.00,544.59)(0.492,0.485){11}{\rule{0.500pt}{0.117pt}}
\multiput(474.00,543.17)(5.962,7.000){2}{\rule{0.250pt}{0.400pt}}
\multiput(481.00,551.59)(0.581,0.482){9}{\rule{0.567pt}{0.116pt}}
\multiput(481.00,550.17)(5.824,6.000){2}{\rule{0.283pt}{0.400pt}}
\multiput(488.00,557.59)(0.491,0.482){9}{\rule{0.500pt}{0.116pt}}
\multiput(488.00,556.17)(4.962,6.000){2}{\rule{0.250pt}{0.400pt}}
\multiput(494.00,563.59)(0.581,0.482){9}{\rule{0.567pt}{0.116pt}}
\multiput(494.00,562.17)(5.824,6.000){2}{\rule{0.283pt}{0.400pt}}
\multiput(501.00,569.59)(0.581,0.482){9}{\rule{0.567pt}{0.116pt}}
\multiput(501.00,568.17)(5.824,6.000){2}{\rule{0.283pt}{0.400pt}}
\multiput(508.00,575.59)(0.710,0.477){7}{\rule{0.660pt}{0.115pt}}
\multiput(508.00,574.17)(5.630,5.000){2}{\rule{0.330pt}{0.400pt}}
\multiput(515.00,580.59)(0.710,0.477){7}{\rule{0.660pt}{0.115pt}}
\multiput(515.00,579.17)(5.630,5.000){2}{\rule{0.330pt}{0.400pt}}
\multiput(522.00,585.59)(0.710,0.477){7}{\rule{0.660pt}{0.115pt}}
\multiput(522.00,584.17)(5.630,5.000){2}{\rule{0.330pt}{0.400pt}}
\multiput(529.00,590.59)(0.710,0.477){7}{\rule{0.660pt}{0.115pt}}
\multiput(529.00,589.17)(5.630,5.000){2}{\rule{0.330pt}{0.400pt}}
\multiput(536.00,595.60)(0.920,0.468){5}{\rule{0.800pt}{0.113pt}}
\multiput(536.00,594.17)(5.340,4.000){2}{\rule{0.400pt}{0.400pt}}
\multiput(543.00,599.59)(0.710,0.477){7}{\rule{0.660pt}{0.115pt}}
\multiput(543.00,598.17)(5.630,5.000){2}{\rule{0.330pt}{0.400pt}}
\multiput(550.00,604.60)(0.774,0.468){5}{\rule{0.700pt}{0.113pt}}
\multiput(550.00,603.17)(4.547,4.000){2}{\rule{0.350pt}{0.400pt}}
\multiput(556.00,608.60)(0.920,0.468){5}{\rule{0.800pt}{0.113pt}}
\multiput(556.00,607.17)(5.340,4.000){2}{\rule{0.400pt}{0.400pt}}
\multiput(563.00,612.61)(1.355,0.447){3}{\rule{1.033pt}{0.108pt}}
\multiput(563.00,611.17)(4.855,3.000){2}{\rule{0.517pt}{0.400pt}}
\multiput(570.00,615.60)(0.920,0.468){5}{\rule{0.800pt}{0.113pt}}
\multiput(570.00,614.17)(5.340,4.000){2}{\rule{0.400pt}{0.400pt}}
\multiput(577.00,619.61)(1.355,0.447){3}{\rule{1.033pt}{0.108pt}}
\multiput(577.00,618.17)(4.855,3.000){2}{\rule{0.517pt}{0.400pt}}
\multiput(584.00,622.61)(1.355,0.447){3}{\rule{1.033pt}{0.108pt}}
\multiput(584.00,621.17)(4.855,3.000){2}{\rule{0.517pt}{0.400pt}}
\multiput(591.00,625.61)(1.355,0.447){3}{\rule{1.033pt}{0.108pt}}
\multiput(591.00,624.17)(4.855,3.000){2}{\rule{0.517pt}{0.400pt}}
\multiput(598.00,628.61)(1.355,0.447){3}{\rule{1.033pt}{0.108pt}}
\multiput(598.00,627.17)(4.855,3.000){2}{\rule{0.517pt}{0.400pt}}
\put(605,631.17){\rule{1.500pt}{0.400pt}}
\multiput(605.00,630.17)(3.887,2.000){2}{\rule{0.750pt}{0.400pt}}
\multiput(612.00,633.61)(1.355,0.447){3}{\rule{1.033pt}{0.108pt}}
\multiput(612.00,632.17)(4.855,3.000){2}{\rule{0.517pt}{0.400pt}}
\put(619,636.17){\rule{1.300pt}{0.400pt}}
\multiput(619.00,635.17)(3.302,2.000){2}{\rule{0.650pt}{0.400pt}}
\put(625,638.17){\rule{1.500pt}{0.400pt}}
\multiput(625.00,637.17)(3.887,2.000){2}{\rule{0.750pt}{0.400pt}}
\put(632,639.67){\rule{1.686pt}{0.400pt}}
\multiput(632.00,639.17)(3.500,1.000){2}{\rule{0.843pt}{0.400pt}}
\put(639,641.17){\rule{1.500pt}{0.400pt}}
\multiput(639.00,640.17)(3.887,2.000){2}{\rule{0.750pt}{0.400pt}}
\put(646,642.67){\rule{1.686pt}{0.400pt}}
\multiput(646.00,642.17)(3.500,1.000){2}{\rule{0.843pt}{0.400pt}}
\put(653,644.17){\rule{1.500pt}{0.400pt}}
\multiput(653.00,643.17)(3.887,2.000){2}{\rule{0.750pt}{0.400pt}}
\put(687.0,634.0){\rule[-0.200pt]{1.686pt}{0.400pt}}
\put(667,645.67){\rule{1.686pt}{0.400pt}}
\multiput(667.00,645.17)(3.500,1.000){2}{\rule{0.843pt}{0.400pt}}
\put(674,646.67){\rule{1.686pt}{0.400pt}}
\multiput(674.00,646.17)(3.500,1.000){2}{\rule{0.843pt}{0.400pt}}
\put(660.0,646.0){\rule[-0.200pt]{1.686pt}{0.400pt}}
\put(687,647.67){\rule{1.686pt}{0.400pt}}
\multiput(687.00,647.17)(3.500,1.000){2}{\rule{0.843pt}{0.400pt}}
\put(681.0,648.0){\rule[-0.200pt]{1.445pt}{0.400pt}}
\put(701,647.67){\rule{1.686pt}{0.400pt}}
\multiput(701.00,648.17)(3.500,-1.000){2}{\rule{0.843pt}{0.400pt}}
\put(694.0,649.0){\rule[-0.200pt]{1.686pt}{0.400pt}}
\put(708.0,648.0){\rule[-0.200pt]{1.686pt}{0.400pt}}
\put(722,646.67){\rule{1.686pt}{0.400pt}}
\multiput(722.00,647.17)(3.500,-1.000){2}{\rule{0.843pt}{0.400pt}}
\put(729,645.67){\rule{1.686pt}{0.400pt}}
\multiput(729.00,646.17)(3.500,-1.000){2}{\rule{0.843pt}{0.400pt}}
\put(736,644.67){\rule{1.686pt}{0.400pt}}
\multiput(736.00,645.17)(3.500,-1.000){2}{\rule{0.843pt}{0.400pt}}
\put(743,643.67){\rule{1.445pt}{0.400pt}}
\multiput(743.00,644.17)(3.000,-1.000){2}{\rule{0.723pt}{0.400pt}}
\put(749,642.17){\rule{1.500pt}{0.400pt}}
\multiput(749.00,643.17)(3.887,-2.000){2}{\rule{0.750pt}{0.400pt}}
\put(756,640.67){\rule{1.686pt}{0.400pt}}
\multiput(756.00,641.17)(3.500,-1.000){2}{\rule{0.843pt}{0.400pt}}
\put(763,639.17){\rule{1.500pt}{0.400pt}}
\multiput(763.00,640.17)(3.887,-2.000){2}{\rule{0.750pt}{0.400pt}}
\put(770,637.17){\rule{1.500pt}{0.400pt}}
\multiput(770.00,638.17)(3.887,-2.000){2}{\rule{0.750pt}{0.400pt}}
\put(777,635.17){\rule{1.500pt}{0.400pt}}
\multiput(777.00,636.17)(3.887,-2.000){2}{\rule{0.750pt}{0.400pt}}
\multiput(784.00,633.95)(1.355,-0.447){3}{\rule{1.033pt}{0.108pt}}
\multiput(784.00,634.17)(4.855,-3.000){2}{\rule{0.517pt}{0.400pt}}
\put(791,630.17){\rule{1.500pt}{0.400pt}}
\multiput(791.00,631.17)(3.887,-2.000){2}{\rule{0.750pt}{0.400pt}}
\multiput(798.00,628.95)(1.355,-0.447){3}{\rule{1.033pt}{0.108pt}}
\multiput(798.00,629.17)(4.855,-3.000){2}{\rule{0.517pt}{0.400pt}}
\multiput(805.00,625.95)(1.355,-0.447){3}{\rule{1.033pt}{0.108pt}}
\multiput(805.00,626.17)(4.855,-3.000){2}{\rule{0.517pt}{0.400pt}}
\multiput(812.00,622.95)(1.132,-0.447){3}{\rule{0.900pt}{0.108pt}}
\multiput(812.00,623.17)(4.132,-3.000){2}{\rule{0.450pt}{0.400pt}}
\multiput(818.00,619.95)(1.355,-0.447){3}{\rule{1.033pt}{0.108pt}}
\multiput(818.00,620.17)(4.855,-3.000){2}{\rule{0.517pt}{0.400pt}}
\multiput(825.00,616.94)(0.920,-0.468){5}{\rule{0.800pt}{0.113pt}}
\multiput(825.00,617.17)(5.340,-4.000){2}{\rule{0.400pt}{0.400pt}}
\multiput(832.00,612.95)(1.355,-0.447){3}{\rule{1.033pt}{0.108pt}}
\multiput(832.00,613.17)(4.855,-3.000){2}{\rule{0.517pt}{0.400pt}}
\multiput(839.00,609.94)(0.920,-0.468){5}{\rule{0.800pt}{0.113pt}}
\multiput(839.00,610.17)(5.340,-4.000){2}{\rule{0.400pt}{0.400pt}}
\multiput(846.00,605.94)(0.920,-0.468){5}{\rule{0.800pt}{0.113pt}}
\multiput(846.00,606.17)(5.340,-4.000){2}{\rule{0.400pt}{0.400pt}}
\multiput(853.00,601.94)(0.920,-0.468){5}{\rule{0.800pt}{0.113pt}}
\multiput(853.00,602.17)(5.340,-4.000){2}{\rule{0.400pt}{0.400pt}}
\multiput(860.00,597.93)(0.710,-0.477){7}{\rule{0.660pt}{0.115pt}}
\multiput(860.00,598.17)(5.630,-5.000){2}{\rule{0.330pt}{0.400pt}}
\multiput(867.00,592.94)(0.920,-0.468){5}{\rule{0.800pt}{0.113pt}}
\multiput(867.00,593.17)(5.340,-4.000){2}{\rule{0.400pt}{0.400pt}}
\multiput(874.00,588.93)(0.599,-0.477){7}{\rule{0.580pt}{0.115pt}}
\multiput(874.00,589.17)(4.796,-5.000){2}{\rule{0.290pt}{0.400pt}}
\multiput(880.00,583.93)(0.710,-0.477){7}{\rule{0.660pt}{0.115pt}}
\multiput(880.00,584.17)(5.630,-5.000){2}{\rule{0.330pt}{0.400pt}}
\multiput(887.00,578.93)(0.710,-0.477){7}{\rule{0.660pt}{0.115pt}}
\multiput(887.00,579.17)(5.630,-5.000){2}{\rule{0.330pt}{0.400pt}}
\multiput(894.00,573.93)(0.581,-0.482){9}{\rule{0.567pt}{0.116pt}}
\multiput(894.00,574.17)(5.824,-6.000){2}{\rule{0.283pt}{0.400pt}}
\multiput(901.00,567.93)(0.710,-0.477){7}{\rule{0.660pt}{0.115pt}}
\multiput(901.00,568.17)(5.630,-5.000){2}{\rule{0.330pt}{0.400pt}}
\multiput(908.00,562.93)(0.581,-0.482){9}{\rule{0.567pt}{0.116pt}}
\multiput(908.00,563.17)(5.824,-6.000){2}{\rule{0.283pt}{0.400pt}}
\multiput(915.00,556.93)(0.581,-0.482){9}{\rule{0.567pt}{0.116pt}}
\multiput(915.00,557.17)(5.824,-6.000){2}{\rule{0.283pt}{0.400pt}}
\multiput(922.00,550.93)(0.492,-0.485){11}{\rule{0.500pt}{0.117pt}}
\multiput(922.00,551.17)(5.962,-7.000){2}{\rule{0.250pt}{0.400pt}}
\multiput(929.00,543.93)(0.581,-0.482){9}{\rule{0.567pt}{0.116pt}}
\multiput(929.00,544.17)(5.824,-6.000){2}{\rule{0.283pt}{0.400pt}}
\multiput(936.59,536.65)(0.482,-0.581){9}{\rule{0.116pt}{0.567pt}}
\multiput(935.17,537.82)(6.000,-5.824){2}{\rule{0.400pt}{0.283pt}}
\multiput(942.00,530.93)(0.581,-0.482){9}{\rule{0.567pt}{0.116pt}}
\multiput(942.00,531.17)(5.824,-6.000){2}{\rule{0.283pt}{0.400pt}}
\multiput(949.59,523.69)(0.485,-0.569){11}{\rule{0.117pt}{0.557pt}}
\multiput(948.17,524.84)(7.000,-6.844){2}{\rule{0.400pt}{0.279pt}}
\multiput(956.00,516.93)(0.492,-0.485){11}{\rule{0.500pt}{0.117pt}}
\multiput(956.00,517.17)(5.962,-7.000){2}{\rule{0.250pt}{0.400pt}}
\multiput(963.00,509.93)(0.492,-0.485){11}{\rule{0.500pt}{0.117pt}}
\multiput(963.00,510.17)(5.962,-7.000){2}{\rule{0.250pt}{0.400pt}}
\multiput(970.59,501.69)(0.485,-0.569){11}{\rule{0.117pt}{0.557pt}}
\multiput(969.17,502.84)(7.000,-6.844){2}{\rule{0.400pt}{0.279pt}}
\multiput(977.59,493.69)(0.485,-0.569){11}{\rule{0.117pt}{0.557pt}}
\multiput(976.17,494.84)(7.000,-6.844){2}{\rule{0.400pt}{0.279pt}}
\multiput(984.59,485.69)(0.485,-0.569){11}{\rule{0.117pt}{0.557pt}}
\multiput(983.17,486.84)(7.000,-6.844){2}{\rule{0.400pt}{0.279pt}}
\multiput(991.59,477.69)(0.485,-0.569){11}{\rule{0.117pt}{0.557pt}}
\multiput(990.17,478.84)(7.000,-6.844){2}{\rule{0.400pt}{0.279pt}}
\multiput(998.59,469.45)(0.485,-0.645){11}{\rule{0.117pt}{0.614pt}}
\multiput(997.17,470.73)(7.000,-7.725){2}{\rule{0.400pt}{0.307pt}}
\multiput(1005.59,460.09)(0.482,-0.762){9}{\rule{0.116pt}{0.700pt}}
\multiput(1004.17,461.55)(6.000,-7.547){2}{\rule{0.400pt}{0.350pt}}
\multiput(1011.59,451.45)(0.485,-0.645){11}{\rule{0.117pt}{0.614pt}}
\multiput(1010.17,452.73)(7.000,-7.725){2}{\rule{0.400pt}{0.307pt}}
\multiput(1018.59,442.45)(0.485,-0.645){11}{\rule{0.117pt}{0.614pt}}
\multiput(1017.17,443.73)(7.000,-7.725){2}{\rule{0.400pt}{0.307pt}}
\multiput(1025.59,433.21)(0.485,-0.721){11}{\rule{0.117pt}{0.671pt}}
\multiput(1024.17,434.61)(7.000,-8.606){2}{\rule{0.400pt}{0.336pt}}
\multiput(1032.59,423.21)(0.485,-0.721){11}{\rule{0.117pt}{0.671pt}}
\multiput(1031.17,424.61)(7.000,-8.606){2}{\rule{0.400pt}{0.336pt}}
\multiput(1039.59,413.21)(0.485,-0.721){11}{\rule{0.117pt}{0.671pt}}
\multiput(1038.17,414.61)(7.000,-8.606){2}{\rule{0.400pt}{0.336pt}}
\multiput(1046.59,403.21)(0.485,-0.721){11}{\rule{0.117pt}{0.671pt}}
\multiput(1045.17,404.61)(7.000,-8.606){2}{\rule{0.400pt}{0.336pt}}
\multiput(1053.59,392.98)(0.485,-0.798){11}{\rule{0.117pt}{0.729pt}}
\multiput(1052.17,394.49)(7.000,-9.488){2}{\rule{0.400pt}{0.364pt}}
\multiput(1060.59,381.98)(0.485,-0.798){11}{\rule{0.117pt}{0.729pt}}
\multiput(1059.17,383.49)(7.000,-9.488){2}{\rule{0.400pt}{0.364pt}}
\multiput(1067.59,370.54)(0.482,-0.943){9}{\rule{0.116pt}{0.833pt}}
\multiput(1066.17,372.27)(6.000,-9.270){2}{\rule{0.400pt}{0.417pt}}
\multiput(1073.59,359.98)(0.485,-0.798){11}{\rule{0.117pt}{0.729pt}}
\multiput(1072.17,361.49)(7.000,-9.488){2}{\rule{0.400pt}{0.364pt}}
\multiput(419.59,466.00)(0.485,0.645){11}{\rule{0.117pt}{0.614pt}}
\multiput(418.17,466.00)(7.000,7.725){2}{\rule{0.400pt}{0.307pt}}
\multiput(426.59,475.00)(0.482,0.762){9}{\rule{0.116pt}{0.700pt}}
\multiput(425.17,475.00)(6.000,7.547){2}{\rule{0.400pt}{0.350pt}}
\multiput(432.59,484.00)(0.485,0.645){11}{\rule{0.117pt}{0.614pt}}
\multiput(431.17,484.00)(7.000,7.725){2}{\rule{0.400pt}{0.307pt}}
\multiput(439.59,493.00)(0.485,0.645){11}{\rule{0.117pt}{0.614pt}}
\multiput(438.17,493.00)(7.000,7.725){2}{\rule{0.400pt}{0.307pt}}
\multiput(446.59,502.00)(0.485,0.569){11}{\rule{0.117pt}{0.557pt}}
\multiput(445.17,502.00)(7.000,6.844){2}{\rule{0.400pt}{0.279pt}}
\multiput(453.59,510.00)(0.485,0.569){11}{\rule{0.117pt}{0.557pt}}
\multiput(452.17,510.00)(7.000,6.844){2}{\rule{0.400pt}{0.279pt}}
\multiput(460.00,518.59)(0.492,0.485){11}{\rule{0.500pt}{0.117pt}}
\multiput(460.00,517.17)(5.962,7.000){2}{\rule{0.250pt}{0.400pt}}
\multiput(467.59,525.00)(0.485,0.569){11}{\rule{0.117pt}{0.557pt}}
\multiput(466.17,525.00)(7.000,6.844){2}{\rule{0.400pt}{0.279pt}}
\multiput(474.00,533.59)(0.492,0.485){11}{\rule{0.500pt}{0.117pt}}
\multiput(474.00,532.17)(5.962,7.000){2}{\rule{0.250pt}{0.400pt}}
\multiput(481.00,540.59)(0.492,0.485){11}{\rule{0.500pt}{0.117pt}}
\multiput(481.00,539.17)(5.962,7.000){2}{\rule{0.250pt}{0.400pt}}
\multiput(488.59,547.00)(0.482,0.581){9}{\rule{0.116pt}{0.567pt}}
\multiput(487.17,547.00)(6.000,5.824){2}{\rule{0.400pt}{0.283pt}}
\multiput(494.00,554.59)(0.581,0.482){9}{\rule{0.567pt}{0.116pt}}
\multiput(494.00,553.17)(5.824,6.000){2}{\rule{0.283pt}{0.400pt}}
\multiput(501.00,560.59)(0.492,0.485){11}{\rule{0.500pt}{0.117pt}}
\multiput(501.00,559.17)(5.962,7.000){2}{\rule{0.250pt}{0.400pt}}
\multiput(508.00,567.59)(0.581,0.482){9}{\rule{0.567pt}{0.116pt}}
\multiput(508.00,566.17)(5.824,6.000){2}{\rule{0.283pt}{0.400pt}}
\multiput(515.00,573.59)(0.581,0.482){9}{\rule{0.567pt}{0.116pt}}
\multiput(515.00,572.17)(5.824,6.000){2}{\rule{0.283pt}{0.400pt}}
\multiput(522.00,579.59)(0.710,0.477){7}{\rule{0.660pt}{0.115pt}}
\multiput(522.00,578.17)(5.630,5.000){2}{\rule{0.330pt}{0.400pt}}
\multiput(529.00,584.59)(0.581,0.482){9}{\rule{0.567pt}{0.116pt}}
\multiput(529.00,583.17)(5.824,6.000){2}{\rule{0.283pt}{0.400pt}}
\multiput(536.00,590.59)(0.710,0.477){7}{\rule{0.660pt}{0.115pt}}
\multiput(536.00,589.17)(5.630,5.000){2}{\rule{0.330pt}{0.400pt}}
\multiput(543.00,595.59)(0.710,0.477){7}{\rule{0.660pt}{0.115pt}}
\multiput(543.00,594.17)(5.630,5.000){2}{\rule{0.330pt}{0.400pt}}
\multiput(550.00,600.60)(0.920,0.468){5}{\rule{0.800pt}{0.113pt}}
\multiput(550.00,599.17)(5.340,4.000){2}{\rule{0.400pt}{0.400pt}}
\multiput(557.00,604.59)(0.599,0.477){7}{\rule{0.580pt}{0.115pt}}
\multiput(557.00,603.17)(4.796,5.000){2}{\rule{0.290pt}{0.400pt}}
\multiput(563.00,609.60)(0.920,0.468){5}{\rule{0.800pt}{0.113pt}}
\multiput(563.00,608.17)(5.340,4.000){2}{\rule{0.400pt}{0.400pt}}
\multiput(570.00,613.60)(0.920,0.468){5}{\rule{0.800pt}{0.113pt}}
\multiput(570.00,612.17)(5.340,4.000){2}{\rule{0.400pt}{0.400pt}}
\multiput(577.00,617.60)(0.920,0.468){5}{\rule{0.800pt}{0.113pt}}
\multiput(577.00,616.17)(5.340,4.000){2}{\rule{0.400pt}{0.400pt}}
\multiput(584.00,621.60)(0.920,0.468){5}{\rule{0.800pt}{0.113pt}}
\multiput(584.00,620.17)(5.340,4.000){2}{\rule{0.400pt}{0.400pt}}
\multiput(591.00,625.61)(1.355,0.447){3}{\rule{1.033pt}{0.108pt}}
\multiput(591.00,624.17)(4.855,3.000){2}{\rule{0.517pt}{0.400pt}}
\multiput(598.00,628.60)(0.920,0.468){5}{\rule{0.800pt}{0.113pt}}
\multiput(598.00,627.17)(5.340,4.000){2}{\rule{0.400pt}{0.400pt}}
\multiput(605.00,632.61)(1.355,0.447){3}{\rule{1.033pt}{0.108pt}}
\multiput(605.00,631.17)(4.855,3.000){2}{\rule{0.517pt}{0.400pt}}
\multiput(612.00,635.61)(1.355,0.447){3}{\rule{1.033pt}{0.108pt}}
\multiput(612.00,634.17)(4.855,3.000){2}{\rule{0.517pt}{0.400pt}}
\put(619,638.17){\rule{1.300pt}{0.400pt}}
\multiput(619.00,637.17)(3.302,2.000){2}{\rule{0.650pt}{0.400pt}}
\multiput(625.00,640.61)(1.355,0.447){3}{\rule{1.033pt}{0.108pt}}
\multiput(625.00,639.17)(4.855,3.000){2}{\rule{0.517pt}{0.400pt}}
\put(632,643.17){\rule{1.500pt}{0.400pt}}
\multiput(632.00,642.17)(3.887,2.000){2}{\rule{0.750pt}{0.400pt}}
\put(639,645.17){\rule{1.500pt}{0.400pt}}
\multiput(639.00,644.17)(3.887,2.000){2}{\rule{0.750pt}{0.400pt}}
\put(646,647.17){\rule{1.500pt}{0.400pt}}
\multiput(646.00,646.17)(3.887,2.000){2}{\rule{0.750pt}{0.400pt}}
\put(653,649.17){\rule{1.500pt}{0.400pt}}
\multiput(653.00,648.17)(3.887,2.000){2}{\rule{0.750pt}{0.400pt}}
\put(660,651.17){\rule{1.500pt}{0.400pt}}
\multiput(660.00,650.17)(3.887,2.000){2}{\rule{0.750pt}{0.400pt}}
\put(667,652.67){\rule{1.686pt}{0.400pt}}
\multiput(667.00,652.17)(3.500,1.000){2}{\rule{0.843pt}{0.400pt}}
\put(674,653.67){\rule{1.686pt}{0.400pt}}
\multiput(674.00,653.17)(3.500,1.000){2}{\rule{0.843pt}{0.400pt}}
\put(681,654.67){\rule{1.445pt}{0.400pt}}
\multiput(681.00,654.17)(3.000,1.000){2}{\rule{0.723pt}{0.400pt}}
\put(687,655.67){\rule{1.686pt}{0.400pt}}
\multiput(687.00,655.17)(3.500,1.000){2}{\rule{0.843pt}{0.400pt}}
\put(715.0,648.0){\rule[-0.200pt]{1.686pt}{0.400pt}}
\put(701,656.67){\rule{1.686pt}{0.400pt}}
\multiput(701.00,656.17)(3.500,1.000){2}{\rule{0.843pt}{0.400pt}}
\put(694.0,657.0){\rule[-0.200pt]{1.686pt}{0.400pt}}
\put(708.0,658.0){\rule[-0.200pt]{1.686pt}{0.400pt}}
\put(715.0,658.0){\rule[-0.200pt]{1.686pt}{0.400pt}}
\put(722.0,658.0){\rule[-0.200pt]{1.686pt}{0.400pt}}
\put(736,656.67){\rule{1.686pt}{0.400pt}}
\multiput(736.00,657.17)(3.500,-1.000){2}{\rule{0.843pt}{0.400pt}}
\put(743,655.67){\rule{1.686pt}{0.400pt}}
\multiput(743.00,656.17)(3.500,-1.000){2}{\rule{0.843pt}{0.400pt}}
\put(729.0,658.0){\rule[-0.200pt]{1.686pt}{0.400pt}}
\put(756,654.67){\rule{1.686pt}{0.400pt}}
\multiput(756.00,655.17)(3.500,-1.000){2}{\rule{0.843pt}{0.400pt}}
\put(763,653.17){\rule{1.500pt}{0.400pt}}
\multiput(763.00,654.17)(3.887,-2.000){2}{\rule{0.750pt}{0.400pt}}
\put(770,651.67){\rule{1.686pt}{0.400pt}}
\multiput(770.00,652.17)(3.500,-1.000){2}{\rule{0.843pt}{0.400pt}}
\put(777,650.17){\rule{1.500pt}{0.400pt}}
\multiput(777.00,651.17)(3.887,-2.000){2}{\rule{0.750pt}{0.400pt}}
\put(784,648.17){\rule{1.500pt}{0.400pt}}
\multiput(784.00,649.17)(3.887,-2.000){2}{\rule{0.750pt}{0.400pt}}
\put(791,646.17){\rule{1.500pt}{0.400pt}}
\multiput(791.00,647.17)(3.887,-2.000){2}{\rule{0.750pt}{0.400pt}}
\put(798,644.17){\rule{1.500pt}{0.400pt}}
\multiput(798.00,645.17)(3.887,-2.000){2}{\rule{0.750pt}{0.400pt}}
\put(805,642.17){\rule{1.500pt}{0.400pt}}
\multiput(805.00,643.17)(3.887,-2.000){2}{\rule{0.750pt}{0.400pt}}
\multiput(812.00,640.95)(1.132,-0.447){3}{\rule{0.900pt}{0.108pt}}
\multiput(812.00,641.17)(4.132,-3.000){2}{\rule{0.450pt}{0.400pt}}
\put(818,637.17){\rule{1.500pt}{0.400pt}}
\multiput(818.00,638.17)(3.887,-2.000){2}{\rule{0.750pt}{0.400pt}}
\multiput(825.00,635.95)(1.355,-0.447){3}{\rule{1.033pt}{0.108pt}}
\multiput(825.00,636.17)(4.855,-3.000){2}{\rule{0.517pt}{0.400pt}}
\multiput(832.00,632.95)(1.355,-0.447){3}{\rule{1.033pt}{0.108pt}}
\multiput(832.00,633.17)(4.855,-3.000){2}{\rule{0.517pt}{0.400pt}}
\multiput(839.00,629.94)(0.920,-0.468){5}{\rule{0.800pt}{0.113pt}}
\multiput(839.00,630.17)(5.340,-4.000){2}{\rule{0.400pt}{0.400pt}}
\multiput(846.00,625.95)(1.355,-0.447){3}{\rule{1.033pt}{0.108pt}}
\multiput(846.00,626.17)(4.855,-3.000){2}{\rule{0.517pt}{0.400pt}}
\multiput(853.00,622.94)(0.920,-0.468){5}{\rule{0.800pt}{0.113pt}}
\multiput(853.00,623.17)(5.340,-4.000){2}{\rule{0.400pt}{0.400pt}}
\multiput(860.00,618.94)(0.920,-0.468){5}{\rule{0.800pt}{0.113pt}}
\multiput(860.00,619.17)(5.340,-4.000){2}{\rule{0.400pt}{0.400pt}}
\multiput(867.00,614.94)(0.920,-0.468){5}{\rule{0.800pt}{0.113pt}}
\multiput(867.00,615.17)(5.340,-4.000){2}{\rule{0.400pt}{0.400pt}}
\multiput(874.00,610.94)(0.774,-0.468){5}{\rule{0.700pt}{0.113pt}}
\multiput(874.00,611.17)(4.547,-4.000){2}{\rule{0.350pt}{0.400pt}}
\multiput(880.00,606.94)(0.920,-0.468){5}{\rule{0.800pt}{0.113pt}}
\multiput(880.00,607.17)(5.340,-4.000){2}{\rule{0.400pt}{0.400pt}}
\multiput(887.00,602.93)(0.710,-0.477){7}{\rule{0.660pt}{0.115pt}}
\multiput(887.00,603.17)(5.630,-5.000){2}{\rule{0.330pt}{0.400pt}}
\multiput(894.00,597.93)(0.710,-0.477){7}{\rule{0.660pt}{0.115pt}}
\multiput(894.00,598.17)(5.630,-5.000){2}{\rule{0.330pt}{0.400pt}}
\multiput(901.00,592.93)(0.710,-0.477){7}{\rule{0.660pt}{0.115pt}}
\multiput(901.00,593.17)(5.630,-5.000){2}{\rule{0.330pt}{0.400pt}}
\multiput(908.00,587.93)(0.710,-0.477){7}{\rule{0.660pt}{0.115pt}}
\multiput(908.00,588.17)(5.630,-5.000){2}{\rule{0.330pt}{0.400pt}}
\multiput(915.00,582.93)(0.710,-0.477){7}{\rule{0.660pt}{0.115pt}}
\multiput(915.00,583.17)(5.630,-5.000){2}{\rule{0.330pt}{0.400pt}}
\multiput(922.00,577.93)(0.581,-0.482){9}{\rule{0.567pt}{0.116pt}}
\multiput(922.00,578.17)(5.824,-6.000){2}{\rule{0.283pt}{0.400pt}}
\multiput(929.00,571.93)(0.581,-0.482){9}{\rule{0.567pt}{0.116pt}}
\multiput(929.00,572.17)(5.824,-6.000){2}{\rule{0.283pt}{0.400pt}}
\multiput(936.00,565.93)(0.581,-0.482){9}{\rule{0.567pt}{0.116pt}}
\multiput(936.00,566.17)(5.824,-6.000){2}{\rule{0.283pt}{0.400pt}}
\multiput(943.00,559.93)(0.491,-0.482){9}{\rule{0.500pt}{0.116pt}}
\multiput(943.00,560.17)(4.962,-6.000){2}{\rule{0.250pt}{0.400pt}}
\multiput(949.00,553.93)(0.581,-0.482){9}{\rule{0.567pt}{0.116pt}}
\multiput(949.00,554.17)(5.824,-6.000){2}{\rule{0.283pt}{0.400pt}}
\multiput(956.00,547.93)(0.492,-0.485){11}{\rule{0.500pt}{0.117pt}}
\multiput(956.00,548.17)(5.962,-7.000){2}{\rule{0.250pt}{0.400pt}}
\multiput(963.00,540.93)(0.492,-0.485){11}{\rule{0.500pt}{0.117pt}}
\multiput(963.00,541.17)(5.962,-7.000){2}{\rule{0.250pt}{0.400pt}}
\multiput(970.00,533.93)(0.492,-0.485){11}{\rule{0.500pt}{0.117pt}}
\multiput(970.00,534.17)(5.962,-7.000){2}{\rule{0.250pt}{0.400pt}}
\multiput(977.00,526.93)(0.492,-0.485){11}{\rule{0.500pt}{0.117pt}}
\multiput(977.00,527.17)(5.962,-7.000){2}{\rule{0.250pt}{0.400pt}}
\multiput(984.59,518.69)(0.485,-0.569){11}{\rule{0.117pt}{0.557pt}}
\multiput(983.17,519.84)(7.000,-6.844){2}{\rule{0.400pt}{0.279pt}}
\multiput(991.00,511.93)(0.492,-0.485){11}{\rule{0.500pt}{0.117pt}}
\multiput(991.00,512.17)(5.962,-7.000){2}{\rule{0.250pt}{0.400pt}}
\multiput(998.59,503.69)(0.485,-0.569){11}{\rule{0.117pt}{0.557pt}}
\multiput(997.17,504.84)(7.000,-6.844){2}{\rule{0.400pt}{0.279pt}}
\multiput(1005.59,495.09)(0.482,-0.762){9}{\rule{0.116pt}{0.700pt}}
\multiput(1004.17,496.55)(6.000,-7.547){2}{\rule{0.400pt}{0.350pt}}
\multiput(1011.59,486.69)(0.485,-0.569){11}{\rule{0.117pt}{0.557pt}}
\multiput(1010.17,487.84)(7.000,-6.844){2}{\rule{0.400pt}{0.279pt}}
\multiput(1018.59,478.45)(0.485,-0.645){11}{\rule{0.117pt}{0.614pt}}
\multiput(1017.17,479.73)(7.000,-7.725){2}{\rule{0.400pt}{0.307pt}}
\multiput(1025.59,469.69)(0.485,-0.569){11}{\rule{0.117pt}{0.557pt}}
\multiput(1024.17,470.84)(7.000,-6.844){2}{\rule{0.400pt}{0.279pt}}
\multiput(1032.59,461.21)(0.485,-0.721){11}{\rule{0.117pt}{0.671pt}}
\multiput(1031.17,462.61)(7.000,-8.606){2}{\rule{0.400pt}{0.336pt}}
\multiput(1039.59,451.45)(0.485,-0.645){11}{\rule{0.117pt}{0.614pt}}
\multiput(1038.17,452.73)(7.000,-7.725){2}{\rule{0.400pt}{0.307pt}}
\multiput(1046.59,442.45)(0.485,-0.645){11}{\rule{0.117pt}{0.614pt}}
\multiput(1045.17,443.73)(7.000,-7.725){2}{\rule{0.400pt}{0.307pt}}
\multiput(1053.59,433.21)(0.485,-0.721){11}{\rule{0.117pt}{0.671pt}}
\multiput(1052.17,434.61)(7.000,-8.606){2}{\rule{0.400pt}{0.336pt}}
\multiput(1060.59,423.21)(0.485,-0.721){11}{\rule{0.117pt}{0.671pt}}
\multiput(1059.17,424.61)(7.000,-8.606){2}{\rule{0.400pt}{0.336pt}}
\multiput(1067.59,412.98)(0.485,-0.798){11}{\rule{0.117pt}{0.729pt}}
\multiput(1066.17,414.49)(7.000,-9.488){2}{\rule{0.400pt}{0.364pt}}
\multiput(1074.59,401.82)(0.482,-0.852){9}{\rule{0.116pt}{0.767pt}}
\multiput(1073.17,403.41)(6.000,-8.409){2}{\rule{0.400pt}{0.383pt}}
\multiput(1080.59,391.98)(0.485,-0.798){11}{\rule{0.117pt}{0.729pt}}
\multiput(1079.17,393.49)(7.000,-9.488){2}{\rule{0.400pt}{0.364pt}}
\multiput(1087.59,380.98)(0.485,-0.798){11}{\rule{0.117pt}{0.729pt}}
\multiput(1086.17,382.49)(7.000,-9.488){2}{\rule{0.400pt}{0.364pt}}
\multiput(1094.59,369.74)(0.485,-0.874){11}{\rule{0.117pt}{0.786pt}}
\multiput(1093.17,371.37)(7.000,-10.369){2}{\rule{0.400pt}{0.393pt}}
\multiput(439.59,476.00)(0.485,0.645){11}{\rule{0.117pt}{0.614pt}}
\multiput(438.17,476.00)(7.000,7.725){2}{\rule{0.400pt}{0.307pt}}
\multiput(446.59,485.00)(0.485,0.645){11}{\rule{0.117pt}{0.614pt}}
\multiput(445.17,485.00)(7.000,7.725){2}{\rule{0.400pt}{0.307pt}}
\multiput(453.59,494.00)(0.485,0.569){11}{\rule{0.117pt}{0.557pt}}
\multiput(452.17,494.00)(7.000,6.844){2}{\rule{0.400pt}{0.279pt}}
\multiput(460.59,502.00)(0.485,0.569){11}{\rule{0.117pt}{0.557pt}}
\multiput(459.17,502.00)(7.000,6.844){2}{\rule{0.400pt}{0.279pt}}
\multiput(467.59,510.00)(0.485,0.569){11}{\rule{0.117pt}{0.557pt}}
\multiput(466.17,510.00)(7.000,6.844){2}{\rule{0.400pt}{0.279pt}}
\multiput(474.59,518.00)(0.485,0.569){11}{\rule{0.117pt}{0.557pt}}
\multiput(473.17,518.00)(7.000,6.844){2}{\rule{0.400pt}{0.279pt}}
\multiput(481.59,526.00)(0.485,0.569){11}{\rule{0.117pt}{0.557pt}}
\multiput(480.17,526.00)(7.000,6.844){2}{\rule{0.400pt}{0.279pt}}
\multiput(488.00,534.59)(0.492,0.485){11}{\rule{0.500pt}{0.117pt}}
\multiput(488.00,533.17)(5.962,7.000){2}{\rule{0.250pt}{0.400pt}}
\multiput(495.59,541.00)(0.482,0.581){9}{\rule{0.116pt}{0.567pt}}
\multiput(494.17,541.00)(6.000,5.824){2}{\rule{0.400pt}{0.283pt}}
\multiput(501.00,548.59)(0.492,0.485){11}{\rule{0.500pt}{0.117pt}}
\multiput(501.00,547.17)(5.962,7.000){2}{\rule{0.250pt}{0.400pt}}
\multiput(508.00,555.59)(0.581,0.482){9}{\rule{0.567pt}{0.116pt}}
\multiput(508.00,554.17)(5.824,6.000){2}{\rule{0.283pt}{0.400pt}}
\multiput(515.00,561.59)(0.492,0.485){11}{\rule{0.500pt}{0.117pt}}
\multiput(515.00,560.17)(5.962,7.000){2}{\rule{0.250pt}{0.400pt}}
\multiput(522.00,568.59)(0.581,0.482){9}{\rule{0.567pt}{0.116pt}}
\multiput(522.00,567.17)(5.824,6.000){2}{\rule{0.283pt}{0.400pt}}
\multiput(529.00,574.59)(0.581,0.482){9}{\rule{0.567pt}{0.116pt}}
\multiput(529.00,573.17)(5.824,6.000){2}{\rule{0.283pt}{0.400pt}}
\multiput(536.00,580.59)(0.710,0.477){7}{\rule{0.660pt}{0.115pt}}
\multiput(536.00,579.17)(5.630,5.000){2}{\rule{0.330pt}{0.400pt}}
\multiput(543.00,585.59)(0.581,0.482){9}{\rule{0.567pt}{0.116pt}}
\multiput(543.00,584.17)(5.824,6.000){2}{\rule{0.283pt}{0.400pt}}
\multiput(550.00,591.59)(0.710,0.477){7}{\rule{0.660pt}{0.115pt}}
\multiput(550.00,590.17)(5.630,5.000){2}{\rule{0.330pt}{0.400pt}}
\multiput(557.00,596.59)(0.599,0.477){7}{\rule{0.580pt}{0.115pt}}
\multiput(557.00,595.17)(4.796,5.000){2}{\rule{0.290pt}{0.400pt}}
\multiput(563.00,601.59)(0.710,0.477){7}{\rule{0.660pt}{0.115pt}}
\multiput(563.00,600.17)(5.630,5.000){2}{\rule{0.330pt}{0.400pt}}
\multiput(570.00,606.59)(0.710,0.477){7}{\rule{0.660pt}{0.115pt}}
\multiput(570.00,605.17)(5.630,5.000){2}{\rule{0.330pt}{0.400pt}}
\multiput(577.00,611.60)(0.920,0.468){5}{\rule{0.800pt}{0.113pt}}
\multiput(577.00,610.17)(5.340,4.000){2}{\rule{0.400pt}{0.400pt}}
\multiput(584.00,615.60)(0.920,0.468){5}{\rule{0.800pt}{0.113pt}}
\multiput(584.00,614.17)(5.340,4.000){2}{\rule{0.400pt}{0.400pt}}
\multiput(591.00,619.60)(0.920,0.468){5}{\rule{0.800pt}{0.113pt}}
\multiput(591.00,618.17)(5.340,4.000){2}{\rule{0.400pt}{0.400pt}}
\multiput(598.00,623.60)(0.920,0.468){5}{\rule{0.800pt}{0.113pt}}
\multiput(598.00,622.17)(5.340,4.000){2}{\rule{0.400pt}{0.400pt}}
\multiput(605.00,627.60)(0.920,0.468){5}{\rule{0.800pt}{0.113pt}}
\multiput(605.00,626.17)(5.340,4.000){2}{\rule{0.400pt}{0.400pt}}
\multiput(612.00,631.61)(1.355,0.447){3}{\rule{1.033pt}{0.108pt}}
\multiput(612.00,630.17)(4.855,3.000){2}{\rule{0.517pt}{0.400pt}}
\multiput(619.00,634.61)(1.355,0.447){3}{\rule{1.033pt}{0.108pt}}
\multiput(619.00,633.17)(4.855,3.000){2}{\rule{0.517pt}{0.400pt}}
\multiput(626.00,637.61)(1.132,0.447){3}{\rule{0.900pt}{0.108pt}}
\multiput(626.00,636.17)(4.132,3.000){2}{\rule{0.450pt}{0.400pt}}
\multiput(632.00,640.61)(1.355,0.447){3}{\rule{1.033pt}{0.108pt}}
\multiput(632.00,639.17)(4.855,3.000){2}{\rule{0.517pt}{0.400pt}}
\multiput(639.00,643.61)(1.355,0.447){3}{\rule{1.033pt}{0.108pt}}
\multiput(639.00,642.17)(4.855,3.000){2}{\rule{0.517pt}{0.400pt}}
\put(646,646.17){\rule{1.500pt}{0.400pt}}
\multiput(646.00,645.17)(3.887,2.000){2}{\rule{0.750pt}{0.400pt}}
\put(653,648.17){\rule{1.500pt}{0.400pt}}
\multiput(653.00,647.17)(3.887,2.000){2}{\rule{0.750pt}{0.400pt}}
\put(660,650.17){\rule{1.500pt}{0.400pt}}
\multiput(660.00,649.17)(3.887,2.000){2}{\rule{0.750pt}{0.400pt}}
\put(667,652.17){\rule{1.500pt}{0.400pt}}
\multiput(667.00,651.17)(3.887,2.000){2}{\rule{0.750pt}{0.400pt}}
\put(674,654.17){\rule{1.500pt}{0.400pt}}
\multiput(674.00,653.17)(3.887,2.000){2}{\rule{0.750pt}{0.400pt}}
\put(681,655.67){\rule{1.686pt}{0.400pt}}
\multiput(681.00,655.17)(3.500,1.000){2}{\rule{0.843pt}{0.400pt}}
\put(688,657.17){\rule{1.300pt}{0.400pt}}
\multiput(688.00,656.17)(3.302,2.000){2}{\rule{0.650pt}{0.400pt}}
\put(694,658.67){\rule{1.686pt}{0.400pt}}
\multiput(694.00,658.17)(3.500,1.000){2}{\rule{0.843pt}{0.400pt}}
\put(701,659.67){\rule{1.686pt}{0.400pt}}
\multiput(701.00,659.17)(3.500,1.000){2}{\rule{0.843pt}{0.400pt}}
\put(708,660.67){\rule{1.686pt}{0.400pt}}
\multiput(708.00,660.17)(3.500,1.000){2}{\rule{0.843pt}{0.400pt}}
\put(750.0,656.0){\rule[-0.200pt]{1.445pt}{0.400pt}}
\put(715.0,662.0){\rule[-0.200pt]{1.686pt}{0.400pt}}
\put(729,661.67){\rule{1.686pt}{0.400pt}}
\multiput(729.00,661.17)(3.500,1.000){2}{\rule{0.843pt}{0.400pt}}
\put(722.0,662.0){\rule[-0.200pt]{1.686pt}{0.400pt}}
\put(736.0,663.0){\rule[-0.200pt]{1.686pt}{0.400pt}}
\put(750,661.67){\rule{1.445pt}{0.400pt}}
\multiput(750.00,662.17)(3.000,-1.000){2}{\rule{0.723pt}{0.400pt}}
\put(743.0,663.0){\rule[-0.200pt]{1.686pt}{0.400pt}}
\put(763,660.67){\rule{1.686pt}{0.400pt}}
\multiput(763.00,661.17)(3.500,-1.000){2}{\rule{0.843pt}{0.400pt}}
\put(770,659.67){\rule{1.686pt}{0.400pt}}
\multiput(770.00,660.17)(3.500,-1.000){2}{\rule{0.843pt}{0.400pt}}
\put(777,658.67){\rule{1.686pt}{0.400pt}}
\multiput(777.00,659.17)(3.500,-1.000){2}{\rule{0.843pt}{0.400pt}}
\put(784,657.67){\rule{1.686pt}{0.400pt}}
\multiput(784.00,658.17)(3.500,-1.000){2}{\rule{0.843pt}{0.400pt}}
\put(791,656.17){\rule{1.500pt}{0.400pt}}
\multiput(791.00,657.17)(3.887,-2.000){2}{\rule{0.750pt}{0.400pt}}
\put(798,654.67){\rule{1.686pt}{0.400pt}}
\multiput(798.00,655.17)(3.500,-1.000){2}{\rule{0.843pt}{0.400pt}}
\put(805,653.17){\rule{1.500pt}{0.400pt}}
\multiput(805.00,654.17)(3.887,-2.000){2}{\rule{0.750pt}{0.400pt}}
\put(812,651.17){\rule{1.500pt}{0.400pt}}
\multiput(812.00,652.17)(3.887,-2.000){2}{\rule{0.750pt}{0.400pt}}
\put(819,649.17){\rule{1.300pt}{0.400pt}}
\multiput(819.00,650.17)(3.302,-2.000){2}{\rule{0.650pt}{0.400pt}}
\put(825,647.17){\rule{1.500pt}{0.400pt}}
\multiput(825.00,648.17)(3.887,-2.000){2}{\rule{0.750pt}{0.400pt}}
\multiput(832.00,645.95)(1.355,-0.447){3}{\rule{1.033pt}{0.108pt}}
\multiput(832.00,646.17)(4.855,-3.000){2}{\rule{0.517pt}{0.400pt}}
\multiput(839.00,642.95)(1.355,-0.447){3}{\rule{1.033pt}{0.108pt}}
\multiput(839.00,643.17)(4.855,-3.000){2}{\rule{0.517pt}{0.400pt}}
\put(846,639.17){\rule{1.500pt}{0.400pt}}
\multiput(846.00,640.17)(3.887,-2.000){2}{\rule{0.750pt}{0.400pt}}
\multiput(853.00,637.95)(1.355,-0.447){3}{\rule{1.033pt}{0.108pt}}
\multiput(853.00,638.17)(4.855,-3.000){2}{\rule{0.517pt}{0.400pt}}
\multiput(860.00,634.94)(0.920,-0.468){5}{\rule{0.800pt}{0.113pt}}
\multiput(860.00,635.17)(5.340,-4.000){2}{\rule{0.400pt}{0.400pt}}
\multiput(867.00,630.95)(1.355,-0.447){3}{\rule{1.033pt}{0.108pt}}
\multiput(867.00,631.17)(4.855,-3.000){2}{\rule{0.517pt}{0.400pt}}
\multiput(874.00,627.94)(0.920,-0.468){5}{\rule{0.800pt}{0.113pt}}
\multiput(874.00,628.17)(5.340,-4.000){2}{\rule{0.400pt}{0.400pt}}
\multiput(881.00,623.94)(0.774,-0.468){5}{\rule{0.700pt}{0.113pt}}
\multiput(881.00,624.17)(4.547,-4.000){2}{\rule{0.350pt}{0.400pt}}
\multiput(887.00,619.94)(0.920,-0.468){5}{\rule{0.800pt}{0.113pt}}
\multiput(887.00,620.17)(5.340,-4.000){2}{\rule{0.400pt}{0.400pt}}
\multiput(894.00,615.94)(0.920,-0.468){5}{\rule{0.800pt}{0.113pt}}
\multiput(894.00,616.17)(5.340,-4.000){2}{\rule{0.400pt}{0.400pt}}
\multiput(901.00,611.94)(0.920,-0.468){5}{\rule{0.800pt}{0.113pt}}
\multiput(901.00,612.17)(5.340,-4.000){2}{\rule{0.400pt}{0.400pt}}
\multiput(908.00,607.93)(0.710,-0.477){7}{\rule{0.660pt}{0.115pt}}
\multiput(908.00,608.17)(5.630,-5.000){2}{\rule{0.330pt}{0.400pt}}
\multiput(915.00,602.94)(0.920,-0.468){5}{\rule{0.800pt}{0.113pt}}
\multiput(915.00,603.17)(5.340,-4.000){2}{\rule{0.400pt}{0.400pt}}
\multiput(922.00,598.93)(0.710,-0.477){7}{\rule{0.660pt}{0.115pt}}
\multiput(922.00,599.17)(5.630,-5.000){2}{\rule{0.330pt}{0.400pt}}
\multiput(929.00,593.93)(0.710,-0.477){7}{\rule{0.660pt}{0.115pt}}
\multiput(929.00,594.17)(5.630,-5.000){2}{\rule{0.330pt}{0.400pt}}
\multiput(936.00,588.93)(0.581,-0.482){9}{\rule{0.567pt}{0.116pt}}
\multiput(936.00,589.17)(5.824,-6.000){2}{\rule{0.283pt}{0.400pt}}
\multiput(943.00,582.93)(0.599,-0.477){7}{\rule{0.580pt}{0.115pt}}
\multiput(943.00,583.17)(4.796,-5.000){2}{\rule{0.290pt}{0.400pt}}
\multiput(949.00,577.93)(0.581,-0.482){9}{\rule{0.567pt}{0.116pt}}
\multiput(949.00,578.17)(5.824,-6.000){2}{\rule{0.283pt}{0.400pt}}
\multiput(956.00,571.93)(0.581,-0.482){9}{\rule{0.567pt}{0.116pt}}
\multiput(956.00,572.17)(5.824,-6.000){2}{\rule{0.283pt}{0.400pt}}
\multiput(963.00,565.93)(0.581,-0.482){9}{\rule{0.567pt}{0.116pt}}
\multiput(963.00,566.17)(5.824,-6.000){2}{\rule{0.283pt}{0.400pt}}
\multiput(970.00,559.93)(0.581,-0.482){9}{\rule{0.567pt}{0.116pt}}
\multiput(970.00,560.17)(5.824,-6.000){2}{\rule{0.283pt}{0.400pt}}
\multiput(977.00,553.93)(0.492,-0.485){11}{\rule{0.500pt}{0.117pt}}
\multiput(977.00,554.17)(5.962,-7.000){2}{\rule{0.250pt}{0.400pt}}
\multiput(984.00,546.93)(0.492,-0.485){11}{\rule{0.500pt}{0.117pt}}
\multiput(984.00,547.17)(5.962,-7.000){2}{\rule{0.250pt}{0.400pt}}
\multiput(991.00,539.93)(0.581,-0.482){9}{\rule{0.567pt}{0.116pt}}
\multiput(991.00,540.17)(5.824,-6.000){2}{\rule{0.283pt}{0.400pt}}
\multiput(998.59,532.69)(0.485,-0.569){11}{\rule{0.117pt}{0.557pt}}
\multiput(997.17,533.84)(7.000,-6.844){2}{\rule{0.400pt}{0.279pt}}
\multiput(1005.00,525.93)(0.492,-0.485){11}{\rule{0.500pt}{0.117pt}}
\multiput(1005.00,526.17)(5.962,-7.000){2}{\rule{0.250pt}{0.400pt}}
\multiput(1012.59,517.37)(0.482,-0.671){9}{\rule{0.116pt}{0.633pt}}
\multiput(1011.17,518.69)(6.000,-6.685){2}{\rule{0.400pt}{0.317pt}}
\multiput(1018.00,510.93)(0.492,-0.485){11}{\rule{0.500pt}{0.117pt}}
\multiput(1018.00,511.17)(5.962,-7.000){2}{\rule{0.250pt}{0.400pt}}
\multiput(1025.59,502.69)(0.485,-0.569){11}{\rule{0.117pt}{0.557pt}}
\multiput(1024.17,503.84)(7.000,-6.844){2}{\rule{0.400pt}{0.279pt}}
\multiput(1032.59,494.45)(0.485,-0.645){11}{\rule{0.117pt}{0.614pt}}
\multiput(1031.17,495.73)(7.000,-7.725){2}{\rule{0.400pt}{0.307pt}}
\multiput(1039.59,485.69)(0.485,-0.569){11}{\rule{0.117pt}{0.557pt}}
\multiput(1038.17,486.84)(7.000,-6.844){2}{\rule{0.400pt}{0.279pt}}
\multiput(1046.59,477.45)(0.485,-0.645){11}{\rule{0.117pt}{0.614pt}}
\multiput(1045.17,478.73)(7.000,-7.725){2}{\rule{0.400pt}{0.307pt}}
\multiput(1053.59,468.45)(0.485,-0.645){11}{\rule{0.117pt}{0.614pt}}
\multiput(1052.17,469.73)(7.000,-7.725){2}{\rule{0.400pt}{0.307pt}}
\multiput(1060.59,459.45)(0.485,-0.645){11}{\rule{0.117pt}{0.614pt}}
\multiput(1059.17,460.73)(7.000,-7.725){2}{\rule{0.400pt}{0.307pt}}
\multiput(1067.59,450.45)(0.485,-0.645){11}{\rule{0.117pt}{0.614pt}}
\multiput(1066.17,451.73)(7.000,-7.725){2}{\rule{0.400pt}{0.307pt}}
\multiput(1074.59,440.82)(0.482,-0.852){9}{\rule{0.116pt}{0.767pt}}
\multiput(1073.17,442.41)(6.000,-8.409){2}{\rule{0.400pt}{0.383pt}}
\multiput(1080.59,431.21)(0.485,-0.721){11}{\rule{0.117pt}{0.671pt}}
\multiput(1079.17,432.61)(7.000,-8.606){2}{\rule{0.400pt}{0.336pt}}
\multiput(1087.59,421.21)(0.485,-0.721){11}{\rule{0.117pt}{0.671pt}}
\multiput(1086.17,422.61)(7.000,-8.606){2}{\rule{0.400pt}{0.336pt}}
\multiput(1094.59,411.21)(0.485,-0.721){11}{\rule{0.117pt}{0.671pt}}
\multiput(1093.17,412.61)(7.000,-8.606){2}{\rule{0.400pt}{0.336pt}}
\multiput(1101.59,400.98)(0.485,-0.798){11}{\rule{0.117pt}{0.729pt}}
\multiput(1100.17,402.49)(7.000,-9.488){2}{\rule{0.400pt}{0.364pt}}
\multiput(1108.59,389.98)(0.485,-0.798){11}{\rule{0.117pt}{0.729pt}}
\multiput(1107.17,391.49)(7.000,-9.488){2}{\rule{0.400pt}{0.364pt}}
\multiput(1115.59,378.98)(0.485,-0.798){11}{\rule{0.117pt}{0.729pt}}
\multiput(1114.17,380.49)(7.000,-9.488){2}{\rule{0.400pt}{0.364pt}}
\multiput(460.59,485.00)(0.485,0.645){11}{\rule{0.117pt}{0.614pt}}
\multiput(459.17,485.00)(7.000,7.725){2}{\rule{0.400pt}{0.307pt}}
\multiput(467.59,494.00)(0.485,0.569){11}{\rule{0.117pt}{0.557pt}}
\multiput(466.17,494.00)(7.000,6.844){2}{\rule{0.400pt}{0.279pt}}
\multiput(474.59,502.00)(0.485,0.569){11}{\rule{0.117pt}{0.557pt}}
\multiput(473.17,502.00)(7.000,6.844){2}{\rule{0.400pt}{0.279pt}}
\multiput(481.59,510.00)(0.485,0.569){11}{\rule{0.117pt}{0.557pt}}
\multiput(480.17,510.00)(7.000,6.844){2}{\rule{0.400pt}{0.279pt}}
\multiput(488.59,518.00)(0.485,0.569){11}{\rule{0.117pt}{0.557pt}}
\multiput(487.17,518.00)(7.000,6.844){2}{\rule{0.400pt}{0.279pt}}
\multiput(495.59,526.00)(0.482,0.581){9}{\rule{0.116pt}{0.567pt}}
\multiput(494.17,526.00)(6.000,5.824){2}{\rule{0.400pt}{0.283pt}}
\multiput(501.00,533.59)(0.492,0.485){11}{\rule{0.500pt}{0.117pt}}
\multiput(501.00,532.17)(5.962,7.000){2}{\rule{0.250pt}{0.400pt}}
\multiput(508.00,540.59)(0.492,0.485){11}{\rule{0.500pt}{0.117pt}}
\multiput(508.00,539.17)(5.962,7.000){2}{\rule{0.250pt}{0.400pt}}
\multiput(515.00,547.59)(0.492,0.485){11}{\rule{0.500pt}{0.117pt}}
\multiput(515.00,546.17)(5.962,7.000){2}{\rule{0.250pt}{0.400pt}}
\multiput(522.00,554.59)(0.581,0.482){9}{\rule{0.567pt}{0.116pt}}
\multiput(522.00,553.17)(5.824,6.000){2}{\rule{0.283pt}{0.400pt}}
\multiput(529.00,560.59)(0.492,0.485){11}{\rule{0.500pt}{0.117pt}}
\multiput(529.00,559.17)(5.962,7.000){2}{\rule{0.250pt}{0.400pt}}
\multiput(536.00,567.59)(0.581,0.482){9}{\rule{0.567pt}{0.116pt}}
\multiput(536.00,566.17)(5.824,6.000){2}{\rule{0.283pt}{0.400pt}}
\multiput(543.00,573.59)(0.581,0.482){9}{\rule{0.567pt}{0.116pt}}
\multiput(543.00,572.17)(5.824,6.000){2}{\rule{0.283pt}{0.400pt}}
\multiput(550.00,579.59)(0.710,0.477){7}{\rule{0.660pt}{0.115pt}}
\multiput(550.00,578.17)(5.630,5.000){2}{\rule{0.330pt}{0.400pt}}
\multiput(557.00,584.59)(0.581,0.482){9}{\rule{0.567pt}{0.116pt}}
\multiput(557.00,583.17)(5.824,6.000){2}{\rule{0.283pt}{0.400pt}}
\multiput(564.00,590.59)(0.599,0.477){7}{\rule{0.580pt}{0.115pt}}
\multiput(564.00,589.17)(4.796,5.000){2}{\rule{0.290pt}{0.400pt}}
\multiput(570.00,595.59)(0.710,0.477){7}{\rule{0.660pt}{0.115pt}}
\multiput(570.00,594.17)(5.630,5.000){2}{\rule{0.330pt}{0.400pt}}
\multiput(577.00,600.59)(0.710,0.477){7}{\rule{0.660pt}{0.115pt}}
\multiput(577.00,599.17)(5.630,5.000){2}{\rule{0.330pt}{0.400pt}}
\multiput(584.00,605.60)(0.920,0.468){5}{\rule{0.800pt}{0.113pt}}
\multiput(584.00,604.17)(5.340,4.000){2}{\rule{0.400pt}{0.400pt}}
\multiput(591.00,609.60)(0.920,0.468){5}{\rule{0.800pt}{0.113pt}}
\multiput(591.00,608.17)(5.340,4.000){2}{\rule{0.400pt}{0.400pt}}
\multiput(598.00,613.59)(0.710,0.477){7}{\rule{0.660pt}{0.115pt}}
\multiput(598.00,612.17)(5.630,5.000){2}{\rule{0.330pt}{0.400pt}}
\multiput(605.00,618.60)(0.920,0.468){5}{\rule{0.800pt}{0.113pt}}
\multiput(605.00,617.17)(5.340,4.000){2}{\rule{0.400pt}{0.400pt}}
\multiput(612.00,622.61)(1.355,0.447){3}{\rule{1.033pt}{0.108pt}}
\multiput(612.00,621.17)(4.855,3.000){2}{\rule{0.517pt}{0.400pt}}
\multiput(619.00,625.60)(0.920,0.468){5}{\rule{0.800pt}{0.113pt}}
\multiput(619.00,624.17)(5.340,4.000){2}{\rule{0.400pt}{0.400pt}}
\multiput(626.00,629.61)(1.132,0.447){3}{\rule{0.900pt}{0.108pt}}
\multiput(626.00,628.17)(4.132,3.000){2}{\rule{0.450pt}{0.400pt}}
\multiput(632.00,632.60)(0.920,0.468){5}{\rule{0.800pt}{0.113pt}}
\multiput(632.00,631.17)(5.340,4.000){2}{\rule{0.400pt}{0.400pt}}
\multiput(639.00,636.61)(1.355,0.447){3}{\rule{1.033pt}{0.108pt}}
\multiput(639.00,635.17)(4.855,3.000){2}{\rule{0.517pt}{0.400pt}}
\put(646,639.17){\rule{1.500pt}{0.400pt}}
\multiput(646.00,638.17)(3.887,2.000){2}{\rule{0.750pt}{0.400pt}}
\multiput(653.00,641.61)(1.355,0.447){3}{\rule{1.033pt}{0.108pt}}
\multiput(653.00,640.17)(4.855,3.000){2}{\rule{0.517pt}{0.400pt}}
\multiput(660.00,644.61)(1.355,0.447){3}{\rule{1.033pt}{0.108pt}}
\multiput(660.00,643.17)(4.855,3.000){2}{\rule{0.517pt}{0.400pt}}
\put(667,647.17){\rule{1.500pt}{0.400pt}}
\multiput(667.00,646.17)(3.887,2.000){2}{\rule{0.750pt}{0.400pt}}
\put(674,649.17){\rule{1.500pt}{0.400pt}}
\multiput(674.00,648.17)(3.887,2.000){2}{\rule{0.750pt}{0.400pt}}
\put(681,651.17){\rule{1.500pt}{0.400pt}}
\multiput(681.00,650.17)(3.887,2.000){2}{\rule{0.750pt}{0.400pt}}
\put(688,653.17){\rule{1.300pt}{0.400pt}}
\multiput(688.00,652.17)(3.302,2.000){2}{\rule{0.650pt}{0.400pt}}
\put(694,654.67){\rule{1.686pt}{0.400pt}}
\multiput(694.00,654.17)(3.500,1.000){2}{\rule{0.843pt}{0.400pt}}
\put(701,656.17){\rule{1.500pt}{0.400pt}}
\multiput(701.00,655.17)(3.887,2.000){2}{\rule{0.750pt}{0.400pt}}
\put(708,657.67){\rule{1.686pt}{0.400pt}}
\multiput(708.00,657.17)(3.500,1.000){2}{\rule{0.843pt}{0.400pt}}
\put(715,658.67){\rule{1.686pt}{0.400pt}}
\multiput(715.00,658.17)(3.500,1.000){2}{\rule{0.843pt}{0.400pt}}
\put(722,659.67){\rule{1.686pt}{0.400pt}}
\multiput(722.00,659.17)(3.500,1.000){2}{\rule{0.843pt}{0.400pt}}
\put(756.0,662.0){\rule[-0.200pt]{1.686pt}{0.400pt}}
\put(736,660.67){\rule{1.686pt}{0.400pt}}
\multiput(736.00,660.17)(3.500,1.000){2}{\rule{0.843pt}{0.400pt}}
\put(729.0,661.0){\rule[-0.200pt]{1.686pt}{0.400pt}}
\put(743.0,662.0){\rule[-0.200pt]{1.686pt}{0.400pt}}
\put(750.0,662.0){\rule[-0.200pt]{1.686pt}{0.400pt}}
\put(757.0,662.0){\rule[-0.200pt]{1.445pt}{0.400pt}}
\put(763.0,662.0){\rule[-0.200pt]{1.686pt}{0.400pt}}
\put(777,660.67){\rule{1.686pt}{0.400pt}}
\multiput(777.00,661.17)(3.500,-1.000){2}{\rule{0.843pt}{0.400pt}}
\put(784,659.67){\rule{1.686pt}{0.400pt}}
\multiput(784.00,660.17)(3.500,-1.000){2}{\rule{0.843pt}{0.400pt}}
\put(770.0,662.0){\rule[-0.200pt]{1.686pt}{0.400pt}}
\put(798,658.67){\rule{1.686pt}{0.400pt}}
\multiput(798.00,659.17)(3.500,-1.000){2}{\rule{0.843pt}{0.400pt}}
\put(805,657.17){\rule{1.500pt}{0.400pt}}
\multiput(805.00,658.17)(3.887,-2.000){2}{\rule{0.750pt}{0.400pt}}
\put(812,655.67){\rule{1.686pt}{0.400pt}}
\multiput(812.00,656.17)(3.500,-1.000){2}{\rule{0.843pt}{0.400pt}}
\put(819,654.17){\rule{1.300pt}{0.400pt}}
\multiput(819.00,655.17)(3.302,-2.000){2}{\rule{0.650pt}{0.400pt}}
\put(825,652.67){\rule{1.686pt}{0.400pt}}
\multiput(825.00,653.17)(3.500,-1.000){2}{\rule{0.843pt}{0.400pt}}
\put(832,651.17){\rule{1.500pt}{0.400pt}}
\multiput(832.00,652.17)(3.887,-2.000){2}{\rule{0.750pt}{0.400pt}}
\put(839,649.17){\rule{1.500pt}{0.400pt}}
\multiput(839.00,650.17)(3.887,-2.000){2}{\rule{0.750pt}{0.400pt}}
\multiput(846.00,647.95)(1.355,-0.447){3}{\rule{1.033pt}{0.108pt}}
\multiput(846.00,648.17)(4.855,-3.000){2}{\rule{0.517pt}{0.400pt}}
\put(853,644.17){\rule{1.500pt}{0.400pt}}
\multiput(853.00,645.17)(3.887,-2.000){2}{\rule{0.750pt}{0.400pt}}
\multiput(860.00,642.95)(1.355,-0.447){3}{\rule{1.033pt}{0.108pt}}
\multiput(860.00,643.17)(4.855,-3.000){2}{\rule{0.517pt}{0.400pt}}
\multiput(867.00,639.95)(1.355,-0.447){3}{\rule{1.033pt}{0.108pt}}
\multiput(867.00,640.17)(4.855,-3.000){2}{\rule{0.517pt}{0.400pt}}
\multiput(874.00,636.95)(1.355,-0.447){3}{\rule{1.033pt}{0.108pt}}
\multiput(874.00,637.17)(4.855,-3.000){2}{\rule{0.517pt}{0.400pt}}
\multiput(881.00,633.95)(1.355,-0.447){3}{\rule{1.033pt}{0.108pt}}
\multiput(881.00,634.17)(4.855,-3.000){2}{\rule{0.517pt}{0.400pt}}
\multiput(888.00,630.95)(1.132,-0.447){3}{\rule{0.900pt}{0.108pt}}
\multiput(888.00,631.17)(4.132,-3.000){2}{\rule{0.450pt}{0.400pt}}
\multiput(894.00,627.95)(1.355,-0.447){3}{\rule{1.033pt}{0.108pt}}
\multiput(894.00,628.17)(4.855,-3.000){2}{\rule{0.517pt}{0.400pt}}
\multiput(901.00,624.94)(0.920,-0.468){5}{\rule{0.800pt}{0.113pt}}
\multiput(901.00,625.17)(5.340,-4.000){2}{\rule{0.400pt}{0.400pt}}
\multiput(908.00,620.94)(0.920,-0.468){5}{\rule{0.800pt}{0.113pt}}
\multiput(908.00,621.17)(5.340,-4.000){2}{\rule{0.400pt}{0.400pt}}
\multiput(915.00,616.94)(0.920,-0.468){5}{\rule{0.800pt}{0.113pt}}
\multiput(915.00,617.17)(5.340,-4.000){2}{\rule{0.400pt}{0.400pt}}
\multiput(922.00,612.94)(0.920,-0.468){5}{\rule{0.800pt}{0.113pt}}
\multiput(922.00,613.17)(5.340,-4.000){2}{\rule{0.400pt}{0.400pt}}
\multiput(929.00,608.93)(0.710,-0.477){7}{\rule{0.660pt}{0.115pt}}
\multiput(929.00,609.17)(5.630,-5.000){2}{\rule{0.330pt}{0.400pt}}
\multiput(936.00,603.94)(0.920,-0.468){5}{\rule{0.800pt}{0.113pt}}
\multiput(936.00,604.17)(5.340,-4.000){2}{\rule{0.400pt}{0.400pt}}
\multiput(943.00,599.93)(0.710,-0.477){7}{\rule{0.660pt}{0.115pt}}
\multiput(943.00,600.17)(5.630,-5.000){2}{\rule{0.330pt}{0.400pt}}
\multiput(950.00,594.93)(0.599,-0.477){7}{\rule{0.580pt}{0.115pt}}
\multiput(950.00,595.17)(4.796,-5.000){2}{\rule{0.290pt}{0.400pt}}
\multiput(956.00,589.93)(0.710,-0.477){7}{\rule{0.660pt}{0.115pt}}
\multiput(956.00,590.17)(5.630,-5.000){2}{\rule{0.330pt}{0.400pt}}
\multiput(963.00,584.93)(0.710,-0.477){7}{\rule{0.660pt}{0.115pt}}
\multiput(963.00,585.17)(5.630,-5.000){2}{\rule{0.330pt}{0.400pt}}
\multiput(970.00,579.93)(0.581,-0.482){9}{\rule{0.567pt}{0.116pt}}
\multiput(970.00,580.17)(5.824,-6.000){2}{\rule{0.283pt}{0.400pt}}
\multiput(977.00,573.93)(0.581,-0.482){9}{\rule{0.567pt}{0.116pt}}
\multiput(977.00,574.17)(5.824,-6.000){2}{\rule{0.283pt}{0.400pt}}
\multiput(984.00,567.93)(0.581,-0.482){9}{\rule{0.567pt}{0.116pt}}
\multiput(984.00,568.17)(5.824,-6.000){2}{\rule{0.283pt}{0.400pt}}
\multiput(991.00,561.93)(0.581,-0.482){9}{\rule{0.567pt}{0.116pt}}
\multiput(991.00,562.17)(5.824,-6.000){2}{\rule{0.283pt}{0.400pt}}
\multiput(998.00,555.93)(0.581,-0.482){9}{\rule{0.567pt}{0.116pt}}
\multiput(998.00,556.17)(5.824,-6.000){2}{\rule{0.283pt}{0.400pt}}
\multiput(1005.00,549.93)(0.581,-0.482){9}{\rule{0.567pt}{0.116pt}}
\multiput(1005.00,550.17)(5.824,-6.000){2}{\rule{0.283pt}{0.400pt}}
\multiput(1012.59,542.65)(0.482,-0.581){9}{\rule{0.116pt}{0.567pt}}
\multiput(1011.17,543.82)(6.000,-5.824){2}{\rule{0.400pt}{0.283pt}}
\multiput(1018.00,536.93)(0.492,-0.485){11}{\rule{0.500pt}{0.117pt}}
\multiput(1018.00,537.17)(5.962,-7.000){2}{\rule{0.250pt}{0.400pt}}
\multiput(1025.00,529.93)(0.492,-0.485){11}{\rule{0.500pt}{0.117pt}}
\multiput(1025.00,530.17)(5.962,-7.000){2}{\rule{0.250pt}{0.400pt}}
\multiput(1032.00,522.93)(0.492,-0.485){11}{\rule{0.500pt}{0.117pt}}
\multiput(1032.00,523.17)(5.962,-7.000){2}{\rule{0.250pt}{0.400pt}}
\multiput(1039.59,514.69)(0.485,-0.569){11}{\rule{0.117pt}{0.557pt}}
\multiput(1038.17,515.84)(7.000,-6.844){2}{\rule{0.400pt}{0.279pt}}
\multiput(1046.59,506.69)(0.485,-0.569){11}{\rule{0.117pt}{0.557pt}}
\multiput(1045.17,507.84)(7.000,-6.844){2}{\rule{0.400pt}{0.279pt}}
\multiput(1053.59,498.69)(0.485,-0.569){11}{\rule{0.117pt}{0.557pt}}
\multiput(1052.17,499.84)(7.000,-6.844){2}{\rule{0.400pt}{0.279pt}}
\multiput(1060.59,490.69)(0.485,-0.569){11}{\rule{0.117pt}{0.557pt}}
\multiput(1059.17,491.84)(7.000,-6.844){2}{\rule{0.400pt}{0.279pt}}
\multiput(1067.59,482.69)(0.485,-0.569){11}{\rule{0.117pt}{0.557pt}}
\multiput(1066.17,483.84)(7.000,-6.844){2}{\rule{0.400pt}{0.279pt}}
\multiput(1074.59,474.45)(0.485,-0.645){11}{\rule{0.117pt}{0.614pt}}
\multiput(1073.17,475.73)(7.000,-7.725){2}{\rule{0.400pt}{0.307pt}}
\multiput(1081.59,465.37)(0.482,-0.671){9}{\rule{0.116pt}{0.633pt}}
\multiput(1080.17,466.69)(6.000,-6.685){2}{\rule{0.400pt}{0.317pt}}
\multiput(1087.59,457.21)(0.485,-0.721){11}{\rule{0.117pt}{0.671pt}}
\multiput(1086.17,458.61)(7.000,-8.606){2}{\rule{0.400pt}{0.336pt}}
\multiput(1094.59,447.45)(0.485,-0.645){11}{\rule{0.117pt}{0.614pt}}
\multiput(1093.17,448.73)(7.000,-7.725){2}{\rule{0.400pt}{0.307pt}}
\multiput(1101.59,438.45)(0.485,-0.645){11}{\rule{0.117pt}{0.614pt}}
\multiput(1100.17,439.73)(7.000,-7.725){2}{\rule{0.400pt}{0.307pt}}
\multiput(1108.59,429.21)(0.485,-0.721){11}{\rule{0.117pt}{0.671pt}}
\multiput(1107.17,430.61)(7.000,-8.606){2}{\rule{0.400pt}{0.336pt}}
\multiput(1115.59,419.21)(0.485,-0.721){11}{\rule{0.117pt}{0.671pt}}
\multiput(1114.17,420.61)(7.000,-8.606){2}{\rule{0.400pt}{0.336pt}}
\multiput(1122.59,409.21)(0.485,-0.721){11}{\rule{0.117pt}{0.671pt}}
\multiput(1121.17,410.61)(7.000,-8.606){2}{\rule{0.400pt}{0.336pt}}
\multiput(1129.59,398.98)(0.485,-0.798){11}{\rule{0.117pt}{0.729pt}}
\multiput(1128.17,400.49)(7.000,-9.488){2}{\rule{0.400pt}{0.364pt}}
\multiput(1136.59,388.21)(0.485,-0.721){11}{\rule{0.117pt}{0.671pt}}
\multiput(1135.17,389.61)(7.000,-8.606){2}{\rule{0.400pt}{0.336pt}}
\multiput(481.59,495.00)(0.485,0.569){11}{\rule{0.117pt}{0.557pt}}
\multiput(480.17,495.00)(7.000,6.844){2}{\rule{0.400pt}{0.279pt}}
\multiput(488.00,503.59)(0.492,0.485){11}{\rule{0.500pt}{0.117pt}}
\multiput(488.00,502.17)(5.962,7.000){2}{\rule{0.250pt}{0.400pt}}
\multiput(495.59,510.00)(0.485,0.569){11}{\rule{0.117pt}{0.557pt}}
\multiput(494.17,510.00)(7.000,6.844){2}{\rule{0.400pt}{0.279pt}}
\multiput(502.59,518.00)(0.482,0.581){9}{\rule{0.116pt}{0.567pt}}
\multiput(501.17,518.00)(6.000,5.824){2}{\rule{0.400pt}{0.283pt}}
\multiput(508.00,525.59)(0.492,0.485){11}{\rule{0.500pt}{0.117pt}}
\multiput(508.00,524.17)(5.962,7.000){2}{\rule{0.250pt}{0.400pt}}
\multiput(515.00,532.59)(0.492,0.485){11}{\rule{0.500pt}{0.117pt}}
\multiput(515.00,531.17)(5.962,7.000){2}{\rule{0.250pt}{0.400pt}}
\multiput(522.00,539.59)(0.492,0.485){11}{\rule{0.500pt}{0.117pt}}
\multiput(522.00,538.17)(5.962,7.000){2}{\rule{0.250pt}{0.400pt}}
\multiput(529.00,546.59)(0.581,0.482){9}{\rule{0.567pt}{0.116pt}}
\multiput(529.00,545.17)(5.824,6.000){2}{\rule{0.283pt}{0.400pt}}
\multiput(536.00,552.59)(0.581,0.482){9}{\rule{0.567pt}{0.116pt}}
\multiput(536.00,551.17)(5.824,6.000){2}{\rule{0.283pt}{0.400pt}}
\multiput(543.00,558.59)(0.581,0.482){9}{\rule{0.567pt}{0.116pt}}
\multiput(543.00,557.17)(5.824,6.000){2}{\rule{0.283pt}{0.400pt}}
\multiput(550.00,564.59)(0.581,0.482){9}{\rule{0.567pt}{0.116pt}}
\multiput(550.00,563.17)(5.824,6.000){2}{\rule{0.283pt}{0.400pt}}
\multiput(557.00,570.59)(0.581,0.482){9}{\rule{0.567pt}{0.116pt}}
\multiput(557.00,569.17)(5.824,6.000){2}{\rule{0.283pt}{0.400pt}}
\multiput(564.00,576.59)(0.599,0.477){7}{\rule{0.580pt}{0.115pt}}
\multiput(564.00,575.17)(4.796,5.000){2}{\rule{0.290pt}{0.400pt}}
\multiput(570.00,581.59)(0.710,0.477){7}{\rule{0.660pt}{0.115pt}}
\multiput(570.00,580.17)(5.630,5.000){2}{\rule{0.330pt}{0.400pt}}
\multiput(577.00,586.59)(0.710,0.477){7}{\rule{0.660pt}{0.115pt}}
\multiput(577.00,585.17)(5.630,5.000){2}{\rule{0.330pt}{0.400pt}}
\multiput(584.00,591.59)(0.710,0.477){7}{\rule{0.660pt}{0.115pt}}
\multiput(584.00,590.17)(5.630,5.000){2}{\rule{0.330pt}{0.400pt}}
\multiput(591.00,596.60)(0.920,0.468){5}{\rule{0.800pt}{0.113pt}}
\multiput(591.00,595.17)(5.340,4.000){2}{\rule{0.400pt}{0.400pt}}
\multiput(598.00,600.59)(0.710,0.477){7}{\rule{0.660pt}{0.115pt}}
\multiput(598.00,599.17)(5.630,5.000){2}{\rule{0.330pt}{0.400pt}}
\multiput(605.00,605.60)(0.920,0.468){5}{\rule{0.800pt}{0.113pt}}
\multiput(605.00,604.17)(5.340,4.000){2}{\rule{0.400pt}{0.400pt}}
\multiput(612.00,609.60)(0.920,0.468){5}{\rule{0.800pt}{0.113pt}}
\multiput(612.00,608.17)(5.340,4.000){2}{\rule{0.400pt}{0.400pt}}
\multiput(619.00,613.60)(0.920,0.468){5}{\rule{0.800pt}{0.113pt}}
\multiput(619.00,612.17)(5.340,4.000){2}{\rule{0.400pt}{0.400pt}}
\multiput(626.00,617.61)(1.355,0.447){3}{\rule{1.033pt}{0.108pt}}
\multiput(626.00,616.17)(4.855,3.000){2}{\rule{0.517pt}{0.400pt}}
\multiput(633.00,620.60)(0.774,0.468){5}{\rule{0.700pt}{0.113pt}}
\multiput(633.00,619.17)(4.547,4.000){2}{\rule{0.350pt}{0.400pt}}
\multiput(639.00,624.61)(1.355,0.447){3}{\rule{1.033pt}{0.108pt}}
\multiput(639.00,623.17)(4.855,3.000){2}{\rule{0.517pt}{0.400pt}}
\multiput(646.00,627.61)(1.355,0.447){3}{\rule{1.033pt}{0.108pt}}
\multiput(646.00,626.17)(4.855,3.000){2}{\rule{0.517pt}{0.400pt}}
\multiput(653.00,630.61)(1.355,0.447){3}{\rule{1.033pt}{0.108pt}}
\multiput(653.00,629.17)(4.855,3.000){2}{\rule{0.517pt}{0.400pt}}
\multiput(660.00,633.61)(1.355,0.447){3}{\rule{1.033pt}{0.108pt}}
\multiput(660.00,632.17)(4.855,3.000){2}{\rule{0.517pt}{0.400pt}}
\multiput(667.00,636.61)(1.355,0.447){3}{\rule{1.033pt}{0.108pt}}
\multiput(667.00,635.17)(4.855,3.000){2}{\rule{0.517pt}{0.400pt}}
\put(674,639.17){\rule{1.500pt}{0.400pt}}
\multiput(674.00,638.17)(3.887,2.000){2}{\rule{0.750pt}{0.400pt}}
\put(681,641.17){\rule{1.500pt}{0.400pt}}
\multiput(681.00,640.17)(3.887,2.000){2}{\rule{0.750pt}{0.400pt}}
\put(688,643.17){\rule{1.500pt}{0.400pt}}
\multiput(688.00,642.17)(3.887,2.000){2}{\rule{0.750pt}{0.400pt}}
\put(695,645.17){\rule{1.300pt}{0.400pt}}
\multiput(695.00,644.17)(3.302,2.000){2}{\rule{0.650pt}{0.400pt}}
\put(701,647.17){\rule{1.500pt}{0.400pt}}
\multiput(701.00,646.17)(3.887,2.000){2}{\rule{0.750pt}{0.400pt}}
\put(708,648.67){\rule{1.686pt}{0.400pt}}
\multiput(708.00,648.17)(3.500,1.000){2}{\rule{0.843pt}{0.400pt}}
\put(715,650.17){\rule{1.500pt}{0.400pt}}
\multiput(715.00,649.17)(3.887,2.000){2}{\rule{0.750pt}{0.400pt}}
\put(722,651.67){\rule{1.686pt}{0.400pt}}
\multiput(722.00,651.17)(3.500,1.000){2}{\rule{0.843pt}{0.400pt}}
\put(729,652.67){\rule{1.686pt}{0.400pt}}
\multiput(729.00,652.17)(3.500,1.000){2}{\rule{0.843pt}{0.400pt}}
\put(736,653.67){\rule{1.686pt}{0.400pt}}
\multiput(736.00,653.17)(3.500,1.000){2}{\rule{0.843pt}{0.400pt}}
\put(743,654.67){\rule{1.686pt}{0.400pt}}
\multiput(743.00,654.17)(3.500,1.000){2}{\rule{0.843pt}{0.400pt}}
\put(791.0,660.0){\rule[-0.200pt]{1.686pt}{0.400pt}}
\put(757,655.67){\rule{1.445pt}{0.400pt}}
\multiput(757.00,655.17)(3.000,1.000){2}{\rule{0.723pt}{0.400pt}}
\put(750.0,656.0){\rule[-0.200pt]{1.686pt}{0.400pt}}
\put(763.0,657.0){\rule[-0.200pt]{1.686pt}{0.400pt}}
\put(770.0,657.0){\rule[-0.200pt]{1.686pt}{0.400pt}}
\put(777.0,657.0){\rule[-0.200pt]{1.686pt}{0.400pt}}
\put(791,655.67){\rule{1.686pt}{0.400pt}}
\multiput(791.00,656.17)(3.500,-1.000){2}{\rule{0.843pt}{0.400pt}}
\put(784.0,657.0){\rule[-0.200pt]{1.686pt}{0.400pt}}
\put(805,654.67){\rule{1.686pt}{0.400pt}}
\multiput(805.00,655.17)(3.500,-1.000){2}{\rule{0.843pt}{0.400pt}}
\put(812,653.67){\rule{1.686pt}{0.400pt}}
\multiput(812.00,654.17)(3.500,-1.000){2}{\rule{0.843pt}{0.400pt}}
\put(819,652.67){\rule{1.686pt}{0.400pt}}
\multiput(819.00,653.17)(3.500,-1.000){2}{\rule{0.843pt}{0.400pt}}
\put(826,651.67){\rule{1.445pt}{0.400pt}}
\multiput(826.00,652.17)(3.000,-1.000){2}{\rule{0.723pt}{0.400pt}}
\put(832,650.17){\rule{1.500pt}{0.400pt}}
\multiput(832.00,651.17)(3.887,-2.000){2}{\rule{0.750pt}{0.400pt}}
\put(839,648.67){\rule{1.686pt}{0.400pt}}
\multiput(839.00,649.17)(3.500,-1.000){2}{\rule{0.843pt}{0.400pt}}
\put(846,647.17){\rule{1.500pt}{0.400pt}}
\multiput(846.00,648.17)(3.887,-2.000){2}{\rule{0.750pt}{0.400pt}}
\put(853,645.17){\rule{1.500pt}{0.400pt}}
\multiput(853.00,646.17)(3.887,-2.000){2}{\rule{0.750pt}{0.400pt}}
\put(860,643.17){\rule{1.500pt}{0.400pt}}
\multiput(860.00,644.17)(3.887,-2.000){2}{\rule{0.750pt}{0.400pt}}
\put(867,641.17){\rule{1.500pt}{0.400pt}}
\multiput(867.00,642.17)(3.887,-2.000){2}{\rule{0.750pt}{0.400pt}}
\put(874,639.17){\rule{1.500pt}{0.400pt}}
\multiput(874.00,640.17)(3.887,-2.000){2}{\rule{0.750pt}{0.400pt}}
\multiput(881.00,637.95)(1.355,-0.447){3}{\rule{1.033pt}{0.108pt}}
\multiput(881.00,638.17)(4.855,-3.000){2}{\rule{0.517pt}{0.400pt}}
\multiput(888.00,634.95)(1.132,-0.447){3}{\rule{0.900pt}{0.108pt}}
\multiput(888.00,635.17)(4.132,-3.000){2}{\rule{0.450pt}{0.400pt}}
\put(894,631.17){\rule{1.500pt}{0.400pt}}
\multiput(894.00,632.17)(3.887,-2.000){2}{\rule{0.750pt}{0.400pt}}
\multiput(901.00,629.95)(1.355,-0.447){3}{\rule{1.033pt}{0.108pt}}
\multiput(901.00,630.17)(4.855,-3.000){2}{\rule{0.517pt}{0.400pt}}
\multiput(908.00,626.94)(0.920,-0.468){5}{\rule{0.800pt}{0.113pt}}
\multiput(908.00,627.17)(5.340,-4.000){2}{\rule{0.400pt}{0.400pt}}
\multiput(915.00,622.95)(1.355,-0.447){3}{\rule{1.033pt}{0.108pt}}
\multiput(915.00,623.17)(4.855,-3.000){2}{\rule{0.517pt}{0.400pt}}
\multiput(922.00,619.95)(1.355,-0.447){3}{\rule{1.033pt}{0.108pt}}
\multiput(922.00,620.17)(4.855,-3.000){2}{\rule{0.517pt}{0.400pt}}
\multiput(929.00,616.94)(0.920,-0.468){5}{\rule{0.800pt}{0.113pt}}
\multiput(929.00,617.17)(5.340,-4.000){2}{\rule{0.400pt}{0.400pt}}
\multiput(936.00,612.94)(0.920,-0.468){5}{\rule{0.800pt}{0.113pt}}
\multiput(936.00,613.17)(5.340,-4.000){2}{\rule{0.400pt}{0.400pt}}
\multiput(943.00,608.94)(0.920,-0.468){5}{\rule{0.800pt}{0.113pt}}
\multiput(943.00,609.17)(5.340,-4.000){2}{\rule{0.400pt}{0.400pt}}
\multiput(950.00,604.94)(0.774,-0.468){5}{\rule{0.700pt}{0.113pt}}
\multiput(950.00,605.17)(4.547,-4.000){2}{\rule{0.350pt}{0.400pt}}
\multiput(956.00,600.94)(0.920,-0.468){5}{\rule{0.800pt}{0.113pt}}
\multiput(956.00,601.17)(5.340,-4.000){2}{\rule{0.400pt}{0.400pt}}
\multiput(963.00,596.93)(0.710,-0.477){7}{\rule{0.660pt}{0.115pt}}
\multiput(963.00,597.17)(5.630,-5.000){2}{\rule{0.330pt}{0.400pt}}
\multiput(970.00,591.93)(0.710,-0.477){7}{\rule{0.660pt}{0.115pt}}
\multiput(970.00,592.17)(5.630,-5.000){2}{\rule{0.330pt}{0.400pt}}
\multiput(977.00,586.93)(0.710,-0.477){7}{\rule{0.660pt}{0.115pt}}
\multiput(977.00,587.17)(5.630,-5.000){2}{\rule{0.330pt}{0.400pt}}
\multiput(984.00,581.93)(0.710,-0.477){7}{\rule{0.660pt}{0.115pt}}
\multiput(984.00,582.17)(5.630,-5.000){2}{\rule{0.330pt}{0.400pt}}
\multiput(991.00,576.93)(0.710,-0.477){7}{\rule{0.660pt}{0.115pt}}
\multiput(991.00,577.17)(5.630,-5.000){2}{\rule{0.330pt}{0.400pt}}
\multiput(998.00,571.93)(0.710,-0.477){7}{\rule{0.660pt}{0.115pt}}
\multiput(998.00,572.17)(5.630,-5.000){2}{\rule{0.330pt}{0.400pt}}
\multiput(1005.00,566.93)(0.581,-0.482){9}{\rule{0.567pt}{0.116pt}}
\multiput(1005.00,567.17)(5.824,-6.000){2}{\rule{0.283pt}{0.400pt}}
\multiput(1012.00,560.93)(0.581,-0.482){9}{\rule{0.567pt}{0.116pt}}
\multiput(1012.00,561.17)(5.824,-6.000){2}{\rule{0.283pt}{0.400pt}}
\multiput(1019.00,554.93)(0.599,-0.477){7}{\rule{0.580pt}{0.115pt}}
\multiput(1019.00,555.17)(4.796,-5.000){2}{\rule{0.290pt}{0.400pt}}
\multiput(1025.00,549.93)(0.492,-0.485){11}{\rule{0.500pt}{0.117pt}}
\multiput(1025.00,550.17)(5.962,-7.000){2}{\rule{0.250pt}{0.400pt}}
\multiput(1032.00,542.93)(0.581,-0.482){9}{\rule{0.567pt}{0.116pt}}
\multiput(1032.00,543.17)(5.824,-6.000){2}{\rule{0.283pt}{0.400pt}}
\multiput(1039.00,536.93)(0.581,-0.482){9}{\rule{0.567pt}{0.116pt}}
\multiput(1039.00,537.17)(5.824,-6.000){2}{\rule{0.283pt}{0.400pt}}
\multiput(1046.00,530.93)(0.492,-0.485){11}{\rule{0.500pt}{0.117pt}}
\multiput(1046.00,531.17)(5.962,-7.000){2}{\rule{0.250pt}{0.400pt}}
\multiput(1053.00,523.93)(0.492,-0.485){11}{\rule{0.500pt}{0.117pt}}
\multiput(1053.00,524.17)(5.962,-7.000){2}{\rule{0.250pt}{0.400pt}}
\multiput(1060.00,516.93)(0.492,-0.485){11}{\rule{0.500pt}{0.117pt}}
\multiput(1060.00,517.17)(5.962,-7.000){2}{\rule{0.250pt}{0.400pt}}
\multiput(1067.00,509.93)(0.492,-0.485){11}{\rule{0.500pt}{0.117pt}}
\multiput(1067.00,510.17)(5.962,-7.000){2}{\rule{0.250pt}{0.400pt}}
\multiput(1074.59,501.69)(0.485,-0.569){11}{\rule{0.117pt}{0.557pt}}
\multiput(1073.17,502.84)(7.000,-6.844){2}{\rule{0.400pt}{0.279pt}}
\multiput(1081.59,493.65)(0.482,-0.581){9}{\rule{0.116pt}{0.567pt}}
\multiput(1080.17,494.82)(6.000,-5.824){2}{\rule{0.400pt}{0.283pt}}
\multiput(1087.59,486.69)(0.485,-0.569){11}{\rule{0.117pt}{0.557pt}}
\multiput(1086.17,487.84)(7.000,-6.844){2}{\rule{0.400pt}{0.279pt}}
\multiput(1094.59,478.69)(0.485,-0.569){11}{\rule{0.117pt}{0.557pt}}
\multiput(1093.17,479.84)(7.000,-6.844){2}{\rule{0.400pt}{0.279pt}}
\multiput(1101.59,470.45)(0.485,-0.645){11}{\rule{0.117pt}{0.614pt}}
\multiput(1100.17,471.73)(7.000,-7.725){2}{\rule{0.400pt}{0.307pt}}
\multiput(1108.59,461.69)(0.485,-0.569){11}{\rule{0.117pt}{0.557pt}}
\multiput(1107.17,462.84)(7.000,-6.844){2}{\rule{0.400pt}{0.279pt}}
\multiput(1115.59,453.45)(0.485,-0.645){11}{\rule{0.117pt}{0.614pt}}
\multiput(1114.17,454.73)(7.000,-7.725){2}{\rule{0.400pt}{0.307pt}}
\multiput(1122.59,444.45)(0.485,-0.645){11}{\rule{0.117pt}{0.614pt}}
\multiput(1121.17,445.73)(7.000,-7.725){2}{\rule{0.400pt}{0.307pt}}
\multiput(1129.59,435.45)(0.485,-0.645){11}{\rule{0.117pt}{0.614pt}}
\multiput(1128.17,436.73)(7.000,-7.725){2}{\rule{0.400pt}{0.307pt}}
\multiput(1136.59,426.45)(0.485,-0.645){11}{\rule{0.117pt}{0.614pt}}
\multiput(1135.17,427.73)(7.000,-7.725){2}{\rule{0.400pt}{0.307pt}}
\multiput(1143.59,417.21)(0.485,-0.721){11}{\rule{0.117pt}{0.671pt}}
\multiput(1142.17,418.61)(7.000,-8.606){2}{\rule{0.400pt}{0.336pt}}
\multiput(1150.59,406.82)(0.482,-0.852){9}{\rule{0.116pt}{0.767pt}}
\multiput(1149.17,408.41)(6.000,-8.409){2}{\rule{0.400pt}{0.383pt}}
\multiput(1156.59,397.21)(0.485,-0.721){11}{\rule{0.117pt}{0.671pt}}
\multiput(1155.17,398.61)(7.000,-8.606){2}{\rule{0.400pt}{0.336pt}}
\multiput(502.59,504.00)(0.482,0.581){9}{\rule{0.116pt}{0.567pt}}
\multiput(501.17,504.00)(6.000,5.824){2}{\rule{0.400pt}{0.283pt}}
\multiput(508.00,511.59)(0.492,0.485){11}{\rule{0.500pt}{0.117pt}}
\multiput(508.00,510.17)(5.962,7.000){2}{\rule{0.250pt}{0.400pt}}
\multiput(515.00,518.59)(0.492,0.485){11}{\rule{0.500pt}{0.117pt}}
\multiput(515.00,517.17)(5.962,7.000){2}{\rule{0.250pt}{0.400pt}}
\multiput(522.00,525.59)(0.581,0.482){9}{\rule{0.567pt}{0.116pt}}
\multiput(522.00,524.17)(5.824,6.000){2}{\rule{0.283pt}{0.400pt}}
\multiput(529.00,531.59)(0.492,0.485){11}{\rule{0.500pt}{0.117pt}}
\multiput(529.00,530.17)(5.962,7.000){2}{\rule{0.250pt}{0.400pt}}
\multiput(536.00,538.59)(0.581,0.482){9}{\rule{0.567pt}{0.116pt}}
\multiput(536.00,537.17)(5.824,6.000){2}{\rule{0.283pt}{0.400pt}}
\multiput(543.00,544.59)(0.581,0.482){9}{\rule{0.567pt}{0.116pt}}
\multiput(543.00,543.17)(5.824,6.000){2}{\rule{0.283pt}{0.400pt}}
\multiput(550.00,550.59)(0.710,0.477){7}{\rule{0.660pt}{0.115pt}}
\multiput(550.00,549.17)(5.630,5.000){2}{\rule{0.330pt}{0.400pt}}
\multiput(557.00,555.59)(0.581,0.482){9}{\rule{0.567pt}{0.116pt}}
\multiput(557.00,554.17)(5.824,6.000){2}{\rule{0.283pt}{0.400pt}}
\multiput(564.00,561.59)(0.710,0.477){7}{\rule{0.660pt}{0.115pt}}
\multiput(564.00,560.17)(5.630,5.000){2}{\rule{0.330pt}{0.400pt}}
\multiput(571.00,566.59)(0.599,0.477){7}{\rule{0.580pt}{0.115pt}}
\multiput(571.00,565.17)(4.796,5.000){2}{\rule{0.290pt}{0.400pt}}
\multiput(577.00,571.59)(0.710,0.477){7}{\rule{0.660pt}{0.115pt}}
\multiput(577.00,570.17)(5.630,5.000){2}{\rule{0.330pt}{0.400pt}}
\multiput(584.00,576.59)(0.710,0.477){7}{\rule{0.660pt}{0.115pt}}
\multiput(584.00,575.17)(5.630,5.000){2}{\rule{0.330pt}{0.400pt}}
\multiput(591.00,581.60)(0.920,0.468){5}{\rule{0.800pt}{0.113pt}}
\multiput(591.00,580.17)(5.340,4.000){2}{\rule{0.400pt}{0.400pt}}
\multiput(598.00,585.59)(0.710,0.477){7}{\rule{0.660pt}{0.115pt}}
\multiput(598.00,584.17)(5.630,5.000){2}{\rule{0.330pt}{0.400pt}}
\multiput(605.00,590.60)(0.920,0.468){5}{\rule{0.800pt}{0.113pt}}
\multiput(605.00,589.17)(5.340,4.000){2}{\rule{0.400pt}{0.400pt}}
\multiput(612.00,594.60)(0.920,0.468){5}{\rule{0.800pt}{0.113pt}}
\multiput(612.00,593.17)(5.340,4.000){2}{\rule{0.400pt}{0.400pt}}
\multiput(619.00,598.60)(0.920,0.468){5}{\rule{0.800pt}{0.113pt}}
\multiput(619.00,597.17)(5.340,4.000){2}{\rule{0.400pt}{0.400pt}}
\multiput(626.00,602.61)(1.355,0.447){3}{\rule{1.033pt}{0.108pt}}
\multiput(626.00,601.17)(4.855,3.000){2}{\rule{0.517pt}{0.400pt}}
\multiput(633.00,605.60)(0.774,0.468){5}{\rule{0.700pt}{0.113pt}}
\multiput(633.00,604.17)(4.547,4.000){2}{\rule{0.350pt}{0.400pt}}
\multiput(639.00,609.61)(1.355,0.447){3}{\rule{1.033pt}{0.108pt}}
\multiput(639.00,608.17)(4.855,3.000){2}{\rule{0.517pt}{0.400pt}}
\multiput(646.00,612.61)(1.355,0.447){3}{\rule{1.033pt}{0.108pt}}
\multiput(646.00,611.17)(4.855,3.000){2}{\rule{0.517pt}{0.400pt}}
\multiput(653.00,615.60)(0.920,0.468){5}{\rule{0.800pt}{0.113pt}}
\multiput(653.00,614.17)(5.340,4.000){2}{\rule{0.400pt}{0.400pt}}
\put(660,619.17){\rule{1.500pt}{0.400pt}}
\multiput(660.00,618.17)(3.887,2.000){2}{\rule{0.750pt}{0.400pt}}
\multiput(667.00,621.61)(1.355,0.447){3}{\rule{1.033pt}{0.108pt}}
\multiput(667.00,620.17)(4.855,3.000){2}{\rule{0.517pt}{0.400pt}}
\multiput(674.00,624.61)(1.355,0.447){3}{\rule{1.033pt}{0.108pt}}
\multiput(674.00,623.17)(4.855,3.000){2}{\rule{0.517pt}{0.400pt}}
\put(681,627.17){\rule{1.500pt}{0.400pt}}
\multiput(681.00,626.17)(3.887,2.000){2}{\rule{0.750pt}{0.400pt}}
\put(688,629.17){\rule{1.500pt}{0.400pt}}
\multiput(688.00,628.17)(3.887,2.000){2}{\rule{0.750pt}{0.400pt}}
\put(695,631.17){\rule{1.500pt}{0.400pt}}
\multiput(695.00,630.17)(3.887,2.000){2}{\rule{0.750pt}{0.400pt}}
\put(702,633.17){\rule{1.300pt}{0.400pt}}
\multiput(702.00,632.17)(3.302,2.000){2}{\rule{0.650pt}{0.400pt}}
\put(708,635.17){\rule{1.500pt}{0.400pt}}
\multiput(708.00,634.17)(3.887,2.000){2}{\rule{0.750pt}{0.400pt}}
\put(715,637.17){\rule{1.500pt}{0.400pt}}
\multiput(715.00,636.17)(3.887,2.000){2}{\rule{0.750pt}{0.400pt}}
\put(722,638.67){\rule{1.686pt}{0.400pt}}
\multiput(722.00,638.17)(3.500,1.000){2}{\rule{0.843pt}{0.400pt}}
\put(729,639.67){\rule{1.686pt}{0.400pt}}
\multiput(729.00,639.17)(3.500,1.000){2}{\rule{0.843pt}{0.400pt}}
\put(736,640.67){\rule{1.686pt}{0.400pt}}
\multiput(736.00,640.17)(3.500,1.000){2}{\rule{0.843pt}{0.400pt}}
\put(743,641.67){\rule{1.686pt}{0.400pt}}
\multiput(743.00,641.17)(3.500,1.000){2}{\rule{0.843pt}{0.400pt}}
\put(750,642.67){\rule{1.686pt}{0.400pt}}
\multiput(750.00,642.17)(3.500,1.000){2}{\rule{0.843pt}{0.400pt}}
\put(757,643.67){\rule{1.686pt}{0.400pt}}
\multiput(757.00,643.17)(3.500,1.000){2}{\rule{0.843pt}{0.400pt}}
\put(764,644.67){\rule{1.445pt}{0.400pt}}
\multiput(764.00,644.17)(3.000,1.000){2}{\rule{0.723pt}{0.400pt}}
\put(798.0,656.0){\rule[-0.200pt]{1.686pt}{0.400pt}}
\put(770.0,646.0){\rule[-0.200pt]{1.686pt}{0.400pt}}
\put(777.0,646.0){\rule[-0.200pt]{1.686pt}{0.400pt}}
\put(784.0,646.0){\rule[-0.200pt]{1.686pt}{0.400pt}}
\put(791.0,646.0){\rule[-0.200pt]{1.686pt}{0.400pt}}
\put(798.0,646.0){\rule[-0.200pt]{1.686pt}{0.400pt}}
\put(812,644.67){\rule{1.686pt}{0.400pt}}
\multiput(812.00,645.17)(3.500,-1.000){2}{\rule{0.843pt}{0.400pt}}
\put(805.0,646.0){\rule[-0.200pt]{1.686pt}{0.400pt}}
\put(826,643.67){\rule{1.445pt}{0.400pt}}
\multiput(826.00,644.17)(3.000,-1.000){2}{\rule{0.723pt}{0.400pt}}
\put(832,642.67){\rule{1.686pt}{0.400pt}}
\multiput(832.00,643.17)(3.500,-1.000){2}{\rule{0.843pt}{0.400pt}}
\put(839,641.67){\rule{1.686pt}{0.400pt}}
\multiput(839.00,642.17)(3.500,-1.000){2}{\rule{0.843pt}{0.400pt}}
\put(846,640.67){\rule{1.686pt}{0.400pt}}
\multiput(846.00,641.17)(3.500,-1.000){2}{\rule{0.843pt}{0.400pt}}
\put(853,639.17){\rule{1.500pt}{0.400pt}}
\multiput(853.00,640.17)(3.887,-2.000){2}{\rule{0.750pt}{0.400pt}}
\put(860,637.67){\rule{1.686pt}{0.400pt}}
\multiput(860.00,638.17)(3.500,-1.000){2}{\rule{0.843pt}{0.400pt}}
\put(867,636.17){\rule{1.500pt}{0.400pt}}
\multiput(867.00,637.17)(3.887,-2.000){2}{\rule{0.750pt}{0.400pt}}
\put(874,634.17){\rule{1.500pt}{0.400pt}}
\multiput(874.00,635.17)(3.887,-2.000){2}{\rule{0.750pt}{0.400pt}}
\put(881,632.67){\rule{1.686pt}{0.400pt}}
\multiput(881.00,633.17)(3.500,-1.000){2}{\rule{0.843pt}{0.400pt}}
\multiput(888.00,631.95)(1.355,-0.447){3}{\rule{1.033pt}{0.108pt}}
\multiput(888.00,632.17)(4.855,-3.000){2}{\rule{0.517pt}{0.400pt}}
\put(895,628.17){\rule{1.300pt}{0.400pt}}
\multiput(895.00,629.17)(3.302,-2.000){2}{\rule{0.650pt}{0.400pt}}
\put(901,626.17){\rule{1.500pt}{0.400pt}}
\multiput(901.00,627.17)(3.887,-2.000){2}{\rule{0.750pt}{0.400pt}}
\multiput(908.00,624.95)(1.355,-0.447){3}{\rule{1.033pt}{0.108pt}}
\multiput(908.00,625.17)(4.855,-3.000){2}{\rule{0.517pt}{0.400pt}}
\put(915,621.17){\rule{1.500pt}{0.400pt}}
\multiput(915.00,622.17)(3.887,-2.000){2}{\rule{0.750pt}{0.400pt}}
\multiput(922.00,619.95)(1.355,-0.447){3}{\rule{1.033pt}{0.108pt}}
\multiput(922.00,620.17)(4.855,-3.000){2}{\rule{0.517pt}{0.400pt}}
\multiput(929.00,616.95)(1.355,-0.447){3}{\rule{1.033pt}{0.108pt}}
\multiput(929.00,617.17)(4.855,-3.000){2}{\rule{0.517pt}{0.400pt}}
\multiput(936.00,613.95)(1.355,-0.447){3}{\rule{1.033pt}{0.108pt}}
\multiput(936.00,614.17)(4.855,-3.000){2}{\rule{0.517pt}{0.400pt}}
\multiput(943.00,610.95)(1.355,-0.447){3}{\rule{1.033pt}{0.108pt}}
\multiput(943.00,611.17)(4.855,-3.000){2}{\rule{0.517pt}{0.400pt}}
\multiput(950.00,607.94)(0.920,-0.468){5}{\rule{0.800pt}{0.113pt}}
\multiput(950.00,608.17)(5.340,-4.000){2}{\rule{0.400pt}{0.400pt}}
\multiput(957.00,603.95)(1.132,-0.447){3}{\rule{0.900pt}{0.108pt}}
\multiput(957.00,604.17)(4.132,-3.000){2}{\rule{0.450pt}{0.400pt}}
\multiput(963.00,600.94)(0.920,-0.468){5}{\rule{0.800pt}{0.113pt}}
\multiput(963.00,601.17)(5.340,-4.000){2}{\rule{0.400pt}{0.400pt}}
\multiput(970.00,596.94)(0.920,-0.468){5}{\rule{0.800pt}{0.113pt}}
\multiput(970.00,597.17)(5.340,-4.000){2}{\rule{0.400pt}{0.400pt}}
\multiput(977.00,592.94)(0.920,-0.468){5}{\rule{0.800pt}{0.113pt}}
\multiput(977.00,593.17)(5.340,-4.000){2}{\rule{0.400pt}{0.400pt}}
\multiput(984.00,588.94)(0.920,-0.468){5}{\rule{0.800pt}{0.113pt}}
\multiput(984.00,589.17)(5.340,-4.000){2}{\rule{0.400pt}{0.400pt}}
\multiput(991.00,584.93)(0.710,-0.477){7}{\rule{0.660pt}{0.115pt}}
\multiput(991.00,585.17)(5.630,-5.000){2}{\rule{0.330pt}{0.400pt}}
\multiput(998.00,579.94)(0.920,-0.468){5}{\rule{0.800pt}{0.113pt}}
\multiput(998.00,580.17)(5.340,-4.000){2}{\rule{0.400pt}{0.400pt}}
\multiput(1005.00,575.93)(0.710,-0.477){7}{\rule{0.660pt}{0.115pt}}
\multiput(1005.00,576.17)(5.630,-5.000){2}{\rule{0.330pt}{0.400pt}}
\multiput(1012.00,570.93)(0.710,-0.477){7}{\rule{0.660pt}{0.115pt}}
\multiput(1012.00,571.17)(5.630,-5.000){2}{\rule{0.330pt}{0.400pt}}
\multiput(1019.00,565.93)(0.599,-0.477){7}{\rule{0.580pt}{0.115pt}}
\multiput(1019.00,566.17)(4.796,-5.000){2}{\rule{0.290pt}{0.400pt}}
\multiput(1025.00,560.93)(0.710,-0.477){7}{\rule{0.660pt}{0.115pt}}
\multiput(1025.00,561.17)(5.630,-5.000){2}{\rule{0.330pt}{0.400pt}}
\multiput(1032.00,555.93)(0.710,-0.477){7}{\rule{0.660pt}{0.115pt}}
\multiput(1032.00,556.17)(5.630,-5.000){2}{\rule{0.330pt}{0.400pt}}
\multiput(1039.00,550.93)(0.710,-0.477){7}{\rule{0.660pt}{0.115pt}}
\multiput(1039.00,551.17)(5.630,-5.000){2}{\rule{0.330pt}{0.400pt}}
\multiput(1046.00,545.93)(0.581,-0.482){9}{\rule{0.567pt}{0.116pt}}
\multiput(1046.00,546.17)(5.824,-6.000){2}{\rule{0.283pt}{0.400pt}}
\multiput(1053.00,539.93)(0.581,-0.482){9}{\rule{0.567pt}{0.116pt}}
\multiput(1053.00,540.17)(5.824,-6.000){2}{\rule{0.283pt}{0.400pt}}
\multiput(1060.00,533.93)(0.581,-0.482){9}{\rule{0.567pt}{0.116pt}}
\multiput(1060.00,534.17)(5.824,-6.000){2}{\rule{0.283pt}{0.400pt}}
\multiput(1067.00,527.93)(0.581,-0.482){9}{\rule{0.567pt}{0.116pt}}
\multiput(1067.00,528.17)(5.824,-6.000){2}{\rule{0.283pt}{0.400pt}}
\multiput(1074.00,521.93)(0.581,-0.482){9}{\rule{0.567pt}{0.116pt}}
\multiput(1074.00,522.17)(5.824,-6.000){2}{\rule{0.283pt}{0.400pt}}
\multiput(1081.00,515.93)(0.492,-0.485){11}{\rule{0.500pt}{0.117pt}}
\multiput(1081.00,516.17)(5.962,-7.000){2}{\rule{0.250pt}{0.400pt}}
\multiput(1088.59,507.65)(0.482,-0.581){9}{\rule{0.116pt}{0.567pt}}
\multiput(1087.17,508.82)(6.000,-5.824){2}{\rule{0.400pt}{0.283pt}}
\multiput(1094.00,501.93)(0.581,-0.482){9}{\rule{0.567pt}{0.116pt}}
\multiput(1094.00,502.17)(5.824,-6.000){2}{\rule{0.283pt}{0.400pt}}
\multiput(1101.00,495.93)(0.492,-0.485){11}{\rule{0.500pt}{0.117pt}}
\multiput(1101.00,496.17)(5.962,-7.000){2}{\rule{0.250pt}{0.400pt}}
\multiput(1108.59,487.69)(0.485,-0.569){11}{\rule{0.117pt}{0.557pt}}
\multiput(1107.17,488.84)(7.000,-6.844){2}{\rule{0.400pt}{0.279pt}}
\multiput(1115.00,480.93)(0.492,-0.485){11}{\rule{0.500pt}{0.117pt}}
\multiput(1115.00,481.17)(5.962,-7.000){2}{\rule{0.250pt}{0.400pt}}
\multiput(1122.59,472.69)(0.485,-0.569){11}{\rule{0.117pt}{0.557pt}}
\multiput(1121.17,473.84)(7.000,-6.844){2}{\rule{0.400pt}{0.279pt}}
\multiput(1129.00,465.93)(0.492,-0.485){11}{\rule{0.500pt}{0.117pt}}
\multiput(1129.00,466.17)(5.962,-7.000){2}{\rule{0.250pt}{0.400pt}}
\multiput(1136.59,457.69)(0.485,-0.569){11}{\rule{0.117pt}{0.557pt}}
\multiput(1135.17,458.84)(7.000,-6.844){2}{\rule{0.400pt}{0.279pt}}
\multiput(1143.59,449.45)(0.485,-0.645){11}{\rule{0.117pt}{0.614pt}}
\multiput(1142.17,450.73)(7.000,-7.725){2}{\rule{0.400pt}{0.307pt}}
\multiput(1150.59,440.37)(0.482,-0.671){9}{\rule{0.116pt}{0.633pt}}
\multiput(1149.17,441.69)(6.000,-6.685){2}{\rule{0.400pt}{0.317pt}}
\multiput(1156.59,432.69)(0.485,-0.569){11}{\rule{0.117pt}{0.557pt}}
\multiput(1155.17,433.84)(7.000,-6.844){2}{\rule{0.400pt}{0.279pt}}
\multiput(1163.59,424.45)(0.485,-0.645){11}{\rule{0.117pt}{0.614pt}}
\multiput(1162.17,425.73)(7.000,-7.725){2}{\rule{0.400pt}{0.307pt}}
\multiput(1170.59,415.45)(0.485,-0.645){11}{\rule{0.117pt}{0.614pt}}
\multiput(1169.17,416.73)(7.000,-7.725){2}{\rule{0.400pt}{0.307pt}}
\multiput(1177.59,406.45)(0.485,-0.645){11}{\rule{0.117pt}{0.614pt}}
\multiput(1176.17,407.73)(7.000,-7.725){2}{\rule{0.400pt}{0.307pt}}
\multiput(522.00,514.59)(0.581,0.482){9}{\rule{0.567pt}{0.116pt}}
\multiput(522.00,513.17)(5.824,6.000){2}{\rule{0.283pt}{0.400pt}}
\multiput(529.00,520.59)(0.581,0.482){9}{\rule{0.567pt}{0.116pt}}
\multiput(529.00,519.17)(5.824,6.000){2}{\rule{0.283pt}{0.400pt}}
\multiput(536.00,526.59)(0.710,0.477){7}{\rule{0.660pt}{0.115pt}}
\multiput(536.00,525.17)(5.630,5.000){2}{\rule{0.330pt}{0.400pt}}
\multiput(543.00,531.59)(0.581,0.482){9}{\rule{0.567pt}{0.116pt}}
\multiput(543.00,530.17)(5.824,6.000){2}{\rule{0.283pt}{0.400pt}}
\multiput(550.00,537.59)(0.710,0.477){7}{\rule{0.660pt}{0.115pt}}
\multiput(550.00,536.17)(5.630,5.000){2}{\rule{0.330pt}{0.400pt}}
\multiput(557.00,542.59)(0.710,0.477){7}{\rule{0.660pt}{0.115pt}}
\multiput(557.00,541.17)(5.630,5.000){2}{\rule{0.330pt}{0.400pt}}
\multiput(564.00,547.59)(0.710,0.477){7}{\rule{0.660pt}{0.115pt}}
\multiput(564.00,546.17)(5.630,5.000){2}{\rule{0.330pt}{0.400pt}}
\multiput(571.00,552.59)(0.599,0.477){7}{\rule{0.580pt}{0.115pt}}
\multiput(571.00,551.17)(4.796,5.000){2}{\rule{0.290pt}{0.400pt}}
\multiput(577.00,557.60)(0.920,0.468){5}{\rule{0.800pt}{0.113pt}}
\multiput(577.00,556.17)(5.340,4.000){2}{\rule{0.400pt}{0.400pt}}
\multiput(584.00,561.59)(0.710,0.477){7}{\rule{0.660pt}{0.115pt}}
\multiput(584.00,560.17)(5.630,5.000){2}{\rule{0.330pt}{0.400pt}}
\multiput(591.00,566.60)(0.920,0.468){5}{\rule{0.800pt}{0.113pt}}
\multiput(591.00,565.17)(5.340,4.000){2}{\rule{0.400pt}{0.400pt}}
\multiput(598.00,570.60)(0.920,0.468){5}{\rule{0.800pt}{0.113pt}}
\multiput(598.00,569.17)(5.340,4.000){2}{\rule{0.400pt}{0.400pt}}
\multiput(605.00,574.60)(0.920,0.468){5}{\rule{0.800pt}{0.113pt}}
\multiput(605.00,573.17)(5.340,4.000){2}{\rule{0.400pt}{0.400pt}}
\multiput(612.00,578.60)(0.920,0.468){5}{\rule{0.800pt}{0.113pt}}
\multiput(612.00,577.17)(5.340,4.000){2}{\rule{0.400pt}{0.400pt}}
\multiput(619.00,582.61)(1.355,0.447){3}{\rule{1.033pt}{0.108pt}}
\multiput(619.00,581.17)(4.855,3.000){2}{\rule{0.517pt}{0.400pt}}
\multiput(626.00,585.60)(0.920,0.468){5}{\rule{0.800pt}{0.113pt}}
\multiput(626.00,584.17)(5.340,4.000){2}{\rule{0.400pt}{0.400pt}}
\multiput(633.00,589.61)(1.355,0.447){3}{\rule{1.033pt}{0.108pt}}
\multiput(633.00,588.17)(4.855,3.000){2}{\rule{0.517pt}{0.400pt}}
\multiput(640.00,592.61)(1.132,0.447){3}{\rule{0.900pt}{0.108pt}}
\multiput(640.00,591.17)(4.132,3.000){2}{\rule{0.450pt}{0.400pt}}
\multiput(646.00,595.61)(1.355,0.447){3}{\rule{1.033pt}{0.108pt}}
\multiput(646.00,594.17)(4.855,3.000){2}{\rule{0.517pt}{0.400pt}}
\multiput(653.00,598.61)(1.355,0.447){3}{\rule{1.033pt}{0.108pt}}
\multiput(653.00,597.17)(4.855,3.000){2}{\rule{0.517pt}{0.400pt}}
\multiput(660.00,601.61)(1.355,0.447){3}{\rule{1.033pt}{0.108pt}}
\multiput(660.00,600.17)(4.855,3.000){2}{\rule{0.517pt}{0.400pt}}
\multiput(667.00,604.61)(1.355,0.447){3}{\rule{1.033pt}{0.108pt}}
\multiput(667.00,603.17)(4.855,3.000){2}{\rule{0.517pt}{0.400pt}}
\put(674,607.17){\rule{1.500pt}{0.400pt}}
\multiput(674.00,606.17)(3.887,2.000){2}{\rule{0.750pt}{0.400pt}}
\put(681,609.17){\rule{1.500pt}{0.400pt}}
\multiput(681.00,608.17)(3.887,2.000){2}{\rule{0.750pt}{0.400pt}}
\multiput(688.00,611.61)(1.355,0.447){3}{\rule{1.033pt}{0.108pt}}
\multiput(688.00,610.17)(4.855,3.000){2}{\rule{0.517pt}{0.400pt}}
\put(695,614.17){\rule{1.500pt}{0.400pt}}
\multiput(695.00,613.17)(3.887,2.000){2}{\rule{0.750pt}{0.400pt}}
\put(702,616.17){\rule{1.300pt}{0.400pt}}
\multiput(702.00,615.17)(3.302,2.000){2}{\rule{0.650pt}{0.400pt}}
\put(708,617.67){\rule{1.686pt}{0.400pt}}
\multiput(708.00,617.17)(3.500,1.000){2}{\rule{0.843pt}{0.400pt}}
\put(715,619.17){\rule{1.500pt}{0.400pt}}
\multiput(715.00,618.17)(3.887,2.000){2}{\rule{0.750pt}{0.400pt}}
\put(722,621.17){\rule{1.500pt}{0.400pt}}
\multiput(722.00,620.17)(3.887,2.000){2}{\rule{0.750pt}{0.400pt}}
\put(729,622.67){\rule{1.686pt}{0.400pt}}
\multiput(729.00,622.17)(3.500,1.000){2}{\rule{0.843pt}{0.400pt}}
\put(736,623.67){\rule{1.686pt}{0.400pt}}
\multiput(736.00,623.17)(3.500,1.000){2}{\rule{0.843pt}{0.400pt}}
\put(743,624.67){\rule{1.686pt}{0.400pt}}
\multiput(743.00,624.17)(3.500,1.000){2}{\rule{0.843pt}{0.400pt}}
\put(750,625.67){\rule{1.686pt}{0.400pt}}
\multiput(750.00,625.17)(3.500,1.000){2}{\rule{0.843pt}{0.400pt}}
\put(757,626.67){\rule{1.686pt}{0.400pt}}
\multiput(757.00,626.17)(3.500,1.000){2}{\rule{0.843pt}{0.400pt}}
\put(764,627.67){\rule{1.445pt}{0.400pt}}
\multiput(764.00,627.17)(3.000,1.000){2}{\rule{0.723pt}{0.400pt}}
\put(819.0,645.0){\rule[-0.200pt]{1.686pt}{0.400pt}}
\put(777,628.67){\rule{1.686pt}{0.400pt}}
\multiput(777.00,628.17)(3.500,1.000){2}{\rule{0.843pt}{0.400pt}}
\put(770.0,629.0){\rule[-0.200pt]{1.686pt}{0.400pt}}
\put(784.0,630.0){\rule[-0.200pt]{1.686pt}{0.400pt}}
\put(798,629.67){\rule{1.686pt}{0.400pt}}
\multiput(798.00,629.17)(3.500,1.000){2}{\rule{0.843pt}{0.400pt}}
\put(805,629.67){\rule{1.686pt}{0.400pt}}
\multiput(805.00,630.17)(3.500,-1.000){2}{\rule{0.843pt}{0.400pt}}
\put(791.0,630.0){\rule[-0.200pt]{1.686pt}{0.400pt}}
\put(812.0,630.0){\rule[-0.200pt]{1.686pt}{0.400pt}}
\put(819.0,630.0){\rule[-0.200pt]{1.686pt}{0.400pt}}
\put(833,628.67){\rule{1.445pt}{0.400pt}}
\multiput(833.00,629.17)(3.000,-1.000){2}{\rule{0.723pt}{0.400pt}}
\put(839,627.67){\rule{1.686pt}{0.400pt}}
\multiput(839.00,628.17)(3.500,-1.000){2}{\rule{0.843pt}{0.400pt}}
\put(826.0,630.0){\rule[-0.200pt]{1.686pt}{0.400pt}}
\put(853,626.67){\rule{1.686pt}{0.400pt}}
\multiput(853.00,627.17)(3.500,-1.000){2}{\rule{0.843pt}{0.400pt}}
\put(860,625.67){\rule{1.686pt}{0.400pt}}
\multiput(860.00,626.17)(3.500,-1.000){2}{\rule{0.843pt}{0.400pt}}
\put(867,624.17){\rule{1.500pt}{0.400pt}}
\multiput(867.00,625.17)(3.887,-2.000){2}{\rule{0.750pt}{0.400pt}}
\put(874,622.67){\rule{1.686pt}{0.400pt}}
\multiput(874.00,623.17)(3.500,-1.000){2}{\rule{0.843pt}{0.400pt}}
\put(881,621.67){\rule{1.686pt}{0.400pt}}
\multiput(881.00,622.17)(3.500,-1.000){2}{\rule{0.843pt}{0.400pt}}
\put(888,620.17){\rule{1.500pt}{0.400pt}}
\multiput(888.00,621.17)(3.887,-2.000){2}{\rule{0.750pt}{0.400pt}}
\put(895,618.17){\rule{1.300pt}{0.400pt}}
\multiput(895.00,619.17)(3.302,-2.000){2}{\rule{0.650pt}{0.400pt}}
\put(901,616.67){\rule{1.686pt}{0.400pt}}
\multiput(901.00,617.17)(3.500,-1.000){2}{\rule{0.843pt}{0.400pt}}
\put(908,615.17){\rule{1.500pt}{0.400pt}}
\multiput(908.00,616.17)(3.887,-2.000){2}{\rule{0.750pt}{0.400pt}}
\put(915,613.17){\rule{1.500pt}{0.400pt}}
\multiput(915.00,614.17)(3.887,-2.000){2}{\rule{0.750pt}{0.400pt}}
\multiput(922.00,611.95)(1.355,-0.447){3}{\rule{1.033pt}{0.108pt}}
\multiput(922.00,612.17)(4.855,-3.000){2}{\rule{0.517pt}{0.400pt}}
\put(929,608.17){\rule{1.500pt}{0.400pt}}
\multiput(929.00,609.17)(3.887,-2.000){2}{\rule{0.750pt}{0.400pt}}
\put(936,606.17){\rule{1.500pt}{0.400pt}}
\multiput(936.00,607.17)(3.887,-2.000){2}{\rule{0.750pt}{0.400pt}}
\multiput(943.00,604.95)(1.355,-0.447){3}{\rule{1.033pt}{0.108pt}}
\multiput(943.00,605.17)(4.855,-3.000){2}{\rule{0.517pt}{0.400pt}}
\multiput(950.00,601.95)(1.355,-0.447){3}{\rule{1.033pt}{0.108pt}}
\multiput(950.00,602.17)(4.855,-3.000){2}{\rule{0.517pt}{0.400pt}}
\multiput(957.00,598.95)(1.355,-0.447){3}{\rule{1.033pt}{0.108pt}}
\multiput(957.00,599.17)(4.855,-3.000){2}{\rule{0.517pt}{0.400pt}}
\multiput(964.00,595.95)(1.132,-0.447){3}{\rule{0.900pt}{0.108pt}}
\multiput(964.00,596.17)(4.132,-3.000){2}{\rule{0.450pt}{0.400pt}}
\multiput(970.00,592.95)(1.355,-0.447){3}{\rule{1.033pt}{0.108pt}}
\multiput(970.00,593.17)(4.855,-3.000){2}{\rule{0.517pt}{0.400pt}}
\multiput(977.00,589.95)(1.355,-0.447){3}{\rule{1.033pt}{0.108pt}}
\multiput(977.00,590.17)(4.855,-3.000){2}{\rule{0.517pt}{0.400pt}}
\multiput(984.00,586.95)(1.355,-0.447){3}{\rule{1.033pt}{0.108pt}}
\multiput(984.00,587.17)(4.855,-3.000){2}{\rule{0.517pt}{0.400pt}}
\multiput(991.00,583.94)(0.920,-0.468){5}{\rule{0.800pt}{0.113pt}}
\multiput(991.00,584.17)(5.340,-4.000){2}{\rule{0.400pt}{0.400pt}}
\multiput(998.00,579.95)(1.355,-0.447){3}{\rule{1.033pt}{0.108pt}}
\multiput(998.00,580.17)(4.855,-3.000){2}{\rule{0.517pt}{0.400pt}}
\multiput(1005.00,576.94)(0.920,-0.468){5}{\rule{0.800pt}{0.113pt}}
\multiput(1005.00,577.17)(5.340,-4.000){2}{\rule{0.400pt}{0.400pt}}
\multiput(1012.00,572.94)(0.920,-0.468){5}{\rule{0.800pt}{0.113pt}}
\multiput(1012.00,573.17)(5.340,-4.000){2}{\rule{0.400pt}{0.400pt}}
\multiput(1019.00,568.94)(0.920,-0.468){5}{\rule{0.800pt}{0.113pt}}
\multiput(1019.00,569.17)(5.340,-4.000){2}{\rule{0.400pt}{0.400pt}}
\multiput(1026.00,564.94)(0.774,-0.468){5}{\rule{0.700pt}{0.113pt}}
\multiput(1026.00,565.17)(4.547,-4.000){2}{\rule{0.350pt}{0.400pt}}
\multiput(1032.00,560.94)(0.920,-0.468){5}{\rule{0.800pt}{0.113pt}}
\multiput(1032.00,561.17)(5.340,-4.000){2}{\rule{0.400pt}{0.400pt}}
\multiput(1039.00,556.93)(0.710,-0.477){7}{\rule{0.660pt}{0.115pt}}
\multiput(1039.00,557.17)(5.630,-5.000){2}{\rule{0.330pt}{0.400pt}}
\multiput(1046.00,551.93)(0.710,-0.477){7}{\rule{0.660pt}{0.115pt}}
\multiput(1046.00,552.17)(5.630,-5.000){2}{\rule{0.330pt}{0.400pt}}
\multiput(1053.00,546.94)(0.920,-0.468){5}{\rule{0.800pt}{0.113pt}}
\multiput(1053.00,547.17)(5.340,-4.000){2}{\rule{0.400pt}{0.400pt}}
\multiput(1060.00,542.93)(0.710,-0.477){7}{\rule{0.660pt}{0.115pt}}
\multiput(1060.00,543.17)(5.630,-5.000){2}{\rule{0.330pt}{0.400pt}}
\multiput(1067.00,537.93)(0.710,-0.477){7}{\rule{0.660pt}{0.115pt}}
\multiput(1067.00,538.17)(5.630,-5.000){2}{\rule{0.330pt}{0.400pt}}
\multiput(1074.00,532.93)(0.710,-0.477){7}{\rule{0.660pt}{0.115pt}}
\multiput(1074.00,533.17)(5.630,-5.000){2}{\rule{0.330pt}{0.400pt}}
\multiput(1081.00,527.93)(0.581,-0.482){9}{\rule{0.567pt}{0.116pt}}
\multiput(1081.00,528.17)(5.824,-6.000){2}{\rule{0.283pt}{0.400pt}}
\multiput(1088.00,521.93)(0.599,-0.477){7}{\rule{0.580pt}{0.115pt}}
\multiput(1088.00,522.17)(4.796,-5.000){2}{\rule{0.290pt}{0.400pt}}
\multiput(1094.00,516.93)(0.581,-0.482){9}{\rule{0.567pt}{0.116pt}}
\multiput(1094.00,517.17)(5.824,-6.000){2}{\rule{0.283pt}{0.400pt}}
\multiput(1101.00,510.93)(0.710,-0.477){7}{\rule{0.660pt}{0.115pt}}
\multiput(1101.00,511.17)(5.630,-5.000){2}{\rule{0.330pt}{0.400pt}}
\multiput(1108.00,505.93)(0.581,-0.482){9}{\rule{0.567pt}{0.116pt}}
\multiput(1108.00,506.17)(5.824,-6.000){2}{\rule{0.283pt}{0.400pt}}
\multiput(1115.00,499.93)(0.581,-0.482){9}{\rule{0.567pt}{0.116pt}}
\multiput(1115.00,500.17)(5.824,-6.000){2}{\rule{0.283pt}{0.400pt}}
\multiput(1122.00,493.93)(0.492,-0.485){11}{\rule{0.500pt}{0.117pt}}
\multiput(1122.00,494.17)(5.962,-7.000){2}{\rule{0.250pt}{0.400pt}}
\multiput(1129.00,486.93)(0.581,-0.482){9}{\rule{0.567pt}{0.116pt}}
\multiput(1129.00,487.17)(5.824,-6.000){2}{\rule{0.283pt}{0.400pt}}
\multiput(1136.00,480.93)(0.581,-0.482){9}{\rule{0.567pt}{0.116pt}}
\multiput(1136.00,481.17)(5.824,-6.000){2}{\rule{0.283pt}{0.400pt}}
\multiput(1143.00,474.93)(0.492,-0.485){11}{\rule{0.500pt}{0.117pt}}
\multiput(1143.00,475.17)(5.962,-7.000){2}{\rule{0.250pt}{0.400pt}}
\multiput(1150.00,467.93)(0.492,-0.485){11}{\rule{0.500pt}{0.117pt}}
\multiput(1150.00,468.17)(5.962,-7.000){2}{\rule{0.250pt}{0.400pt}}
\multiput(1157.59,459.65)(0.482,-0.581){9}{\rule{0.116pt}{0.567pt}}
\multiput(1156.17,460.82)(6.000,-5.824){2}{\rule{0.400pt}{0.283pt}}
\multiput(1163.00,453.93)(0.492,-0.485){11}{\rule{0.500pt}{0.117pt}}
\multiput(1163.00,454.17)(5.962,-7.000){2}{\rule{0.250pt}{0.400pt}}
\multiput(1170.59,445.69)(0.485,-0.569){11}{\rule{0.117pt}{0.557pt}}
\multiput(1169.17,446.84)(7.000,-6.844){2}{\rule{0.400pt}{0.279pt}}
\multiput(1177.00,438.93)(0.492,-0.485){11}{\rule{0.500pt}{0.117pt}}
\multiput(1177.00,439.17)(5.962,-7.000){2}{\rule{0.250pt}{0.400pt}}
\multiput(1184.59,430.69)(0.485,-0.569){11}{\rule{0.117pt}{0.557pt}}
\multiput(1183.17,431.84)(7.000,-6.844){2}{\rule{0.400pt}{0.279pt}}
\multiput(1191.59,422.69)(0.485,-0.569){11}{\rule{0.117pt}{0.557pt}}
\multiput(1190.17,423.84)(7.000,-6.844){2}{\rule{0.400pt}{0.279pt}}
\multiput(1198.59,414.69)(0.485,-0.569){11}{\rule{0.117pt}{0.557pt}}
\multiput(1197.17,415.84)(7.000,-6.844){2}{\rule{0.400pt}{0.279pt}}
\multiput(543.00,523.59)(0.710,0.477){7}{\rule{0.660pt}{0.115pt}}
\multiput(543.00,522.17)(5.630,5.000){2}{\rule{0.330pt}{0.400pt}}
\multiput(550.00,528.59)(0.710,0.477){7}{\rule{0.660pt}{0.115pt}}
\multiput(550.00,527.17)(5.630,5.000){2}{\rule{0.330pt}{0.400pt}}
\multiput(557.00,533.60)(0.920,0.468){5}{\rule{0.800pt}{0.113pt}}
\multiput(557.00,532.17)(5.340,4.000){2}{\rule{0.400pt}{0.400pt}}
\multiput(564.00,537.60)(0.920,0.468){5}{\rule{0.800pt}{0.113pt}}
\multiput(564.00,536.17)(5.340,4.000){2}{\rule{0.400pt}{0.400pt}}
\multiput(571.00,541.60)(0.920,0.468){5}{\rule{0.800pt}{0.113pt}}
\multiput(571.00,540.17)(5.340,4.000){2}{\rule{0.400pt}{0.400pt}}
\multiput(578.00,545.60)(0.774,0.468){5}{\rule{0.700pt}{0.113pt}}
\multiput(578.00,544.17)(4.547,4.000){2}{\rule{0.350pt}{0.400pt}}
\multiput(584.00,549.60)(0.920,0.468){5}{\rule{0.800pt}{0.113pt}}
\multiput(584.00,548.17)(5.340,4.000){2}{\rule{0.400pt}{0.400pt}}
\multiput(591.00,553.61)(1.355,0.447){3}{\rule{1.033pt}{0.108pt}}
\multiput(591.00,552.17)(4.855,3.000){2}{\rule{0.517pt}{0.400pt}}
\multiput(598.00,556.60)(0.920,0.468){5}{\rule{0.800pt}{0.113pt}}
\multiput(598.00,555.17)(5.340,4.000){2}{\rule{0.400pt}{0.400pt}}
\multiput(605.00,560.61)(1.355,0.447){3}{\rule{1.033pt}{0.108pt}}
\multiput(605.00,559.17)(4.855,3.000){2}{\rule{0.517pt}{0.400pt}}
\multiput(612.00,563.61)(1.355,0.447){3}{\rule{1.033pt}{0.108pt}}
\multiput(612.00,562.17)(4.855,3.000){2}{\rule{0.517pt}{0.400pt}}
\multiput(619.00,566.60)(0.920,0.468){5}{\rule{0.800pt}{0.113pt}}
\multiput(619.00,565.17)(5.340,4.000){2}{\rule{0.400pt}{0.400pt}}
\put(626,570.17){\rule{1.500pt}{0.400pt}}
\multiput(626.00,569.17)(3.887,2.000){2}{\rule{0.750pt}{0.400pt}}
\multiput(633.00,572.61)(1.355,0.447){3}{\rule{1.033pt}{0.108pt}}
\multiput(633.00,571.17)(4.855,3.000){2}{\rule{0.517pt}{0.400pt}}
\multiput(640.00,575.61)(1.132,0.447){3}{\rule{0.900pt}{0.108pt}}
\multiput(640.00,574.17)(4.132,3.000){2}{\rule{0.450pt}{0.400pt}}
\multiput(646.00,578.61)(1.355,0.447){3}{\rule{1.033pt}{0.108pt}}
\multiput(646.00,577.17)(4.855,3.000){2}{\rule{0.517pt}{0.400pt}}
\put(653,581.17){\rule{1.500pt}{0.400pt}}
\multiput(653.00,580.17)(3.887,2.000){2}{\rule{0.750pt}{0.400pt}}
\multiput(660.00,583.61)(1.355,0.447){3}{\rule{1.033pt}{0.108pt}}
\multiput(660.00,582.17)(4.855,3.000){2}{\rule{0.517pt}{0.400pt}}
\put(667,586.17){\rule{1.500pt}{0.400pt}}
\multiput(667.00,585.17)(3.887,2.000){2}{\rule{0.750pt}{0.400pt}}
\put(674,588.17){\rule{1.500pt}{0.400pt}}
\multiput(674.00,587.17)(3.887,2.000){2}{\rule{0.750pt}{0.400pt}}
\put(681,590.17){\rule{1.500pt}{0.400pt}}
\multiput(681.00,589.17)(3.887,2.000){2}{\rule{0.750pt}{0.400pt}}
\put(688,592.17){\rule{1.500pt}{0.400pt}}
\multiput(688.00,591.17)(3.887,2.000){2}{\rule{0.750pt}{0.400pt}}
\put(695,594.17){\rule{1.500pt}{0.400pt}}
\multiput(695.00,593.17)(3.887,2.000){2}{\rule{0.750pt}{0.400pt}}
\put(702,595.67){\rule{1.686pt}{0.400pt}}
\multiput(702.00,595.17)(3.500,1.000){2}{\rule{0.843pt}{0.400pt}}
\put(709,597.17){\rule{1.300pt}{0.400pt}}
\multiput(709.00,596.17)(3.302,2.000){2}{\rule{0.650pt}{0.400pt}}
\put(715,598.67){\rule{1.686pt}{0.400pt}}
\multiput(715.00,598.17)(3.500,1.000){2}{\rule{0.843pt}{0.400pt}}
\put(722,600.17){\rule{1.500pt}{0.400pt}}
\multiput(722.00,599.17)(3.887,2.000){2}{\rule{0.750pt}{0.400pt}}
\put(729,601.67){\rule{1.686pt}{0.400pt}}
\multiput(729.00,601.17)(3.500,1.000){2}{\rule{0.843pt}{0.400pt}}
\put(736,602.67){\rule{1.686pt}{0.400pt}}
\multiput(736.00,602.17)(3.500,1.000){2}{\rule{0.843pt}{0.400pt}}
\put(743,603.67){\rule{1.686pt}{0.400pt}}
\multiput(743.00,603.17)(3.500,1.000){2}{\rule{0.843pt}{0.400pt}}
\put(750,604.67){\rule{1.686pt}{0.400pt}}
\multiput(750.00,604.17)(3.500,1.000){2}{\rule{0.843pt}{0.400pt}}
\put(757,605.67){\rule{1.686pt}{0.400pt}}
\multiput(757.00,605.17)(3.500,1.000){2}{\rule{0.843pt}{0.400pt}}
\put(846.0,628.0){\rule[-0.200pt]{1.686pt}{0.400pt}}
\put(771,606.67){\rule{1.445pt}{0.400pt}}
\multiput(771.00,606.17)(3.000,1.000){2}{\rule{0.723pt}{0.400pt}}
\put(777,607.67){\rule{1.686pt}{0.400pt}}
\multiput(777.00,607.17)(3.500,1.000){2}{\rule{0.843pt}{0.400pt}}
\put(764.0,607.0){\rule[-0.200pt]{1.686pt}{0.400pt}}
\put(784.0,609.0){\rule[-0.200pt]{1.686pt}{0.400pt}}
\put(791.0,609.0){\rule[-0.200pt]{1.686pt}{0.400pt}}
\put(798.0,609.0){\rule[-0.200pt]{1.686pt}{0.400pt}}
\put(805.0,609.0){\rule[-0.200pt]{1.686pt}{0.400pt}}
\put(812.0,609.0){\rule[-0.200pt]{1.686pt}{0.400pt}}
\put(819.0,609.0){\rule[-0.200pt]{1.686pt}{0.400pt}}
\put(826.0,609.0){\rule[-0.200pt]{1.686pt}{0.400pt}}
\put(839,607.67){\rule{1.686pt}{0.400pt}}
\multiput(839.00,608.17)(3.500,-1.000){2}{\rule{0.843pt}{0.400pt}}
\put(833.0,609.0){\rule[-0.200pt]{1.445pt}{0.400pt}}
\put(853,606.67){\rule{1.686pt}{0.400pt}}
\multiput(853.00,607.17)(3.500,-1.000){2}{\rule{0.843pt}{0.400pt}}
\put(860,605.67){\rule{1.686pt}{0.400pt}}
\multiput(860.00,606.17)(3.500,-1.000){2}{\rule{0.843pt}{0.400pt}}
\put(867,604.67){\rule{1.686pt}{0.400pt}}
\multiput(867.00,605.17)(3.500,-1.000){2}{\rule{0.843pt}{0.400pt}}
\put(874,603.67){\rule{1.686pt}{0.400pt}}
\multiput(874.00,604.17)(3.500,-1.000){2}{\rule{0.843pt}{0.400pt}}
\put(881,602.67){\rule{1.686pt}{0.400pt}}
\multiput(881.00,603.17)(3.500,-1.000){2}{\rule{0.843pt}{0.400pt}}
\put(888,601.67){\rule{1.686pt}{0.400pt}}
\multiput(888.00,602.17)(3.500,-1.000){2}{\rule{0.843pt}{0.400pt}}
\put(895,600.67){\rule{1.686pt}{0.400pt}}
\multiput(895.00,601.17)(3.500,-1.000){2}{\rule{0.843pt}{0.400pt}}
\put(902,599.67){\rule{1.445pt}{0.400pt}}
\multiput(902.00,600.17)(3.000,-1.000){2}{\rule{0.723pt}{0.400pt}}
\put(908,598.17){\rule{1.500pt}{0.400pt}}
\multiput(908.00,599.17)(3.887,-2.000){2}{\rule{0.750pt}{0.400pt}}
\put(915,596.67){\rule{1.686pt}{0.400pt}}
\multiput(915.00,597.17)(3.500,-1.000){2}{\rule{0.843pt}{0.400pt}}
\put(922,595.17){\rule{1.500pt}{0.400pt}}
\multiput(922.00,596.17)(3.887,-2.000){2}{\rule{0.750pt}{0.400pt}}
\put(929,593.17){\rule{1.500pt}{0.400pt}}
\multiput(929.00,594.17)(3.887,-2.000){2}{\rule{0.750pt}{0.400pt}}
\put(936,591.17){\rule{1.500pt}{0.400pt}}
\multiput(936.00,592.17)(3.887,-2.000){2}{\rule{0.750pt}{0.400pt}}
\put(943,589.17){\rule{1.500pt}{0.400pt}}
\multiput(943.00,590.17)(3.887,-2.000){2}{\rule{0.750pt}{0.400pt}}
\put(950,587.17){\rule{1.500pt}{0.400pt}}
\multiput(950.00,588.17)(3.887,-2.000){2}{\rule{0.750pt}{0.400pt}}
\put(957,585.17){\rule{1.500pt}{0.400pt}}
\multiput(957.00,586.17)(3.887,-2.000){2}{\rule{0.750pt}{0.400pt}}
\put(964,583.17){\rule{1.300pt}{0.400pt}}
\multiput(964.00,584.17)(3.302,-2.000){2}{\rule{0.650pt}{0.400pt}}
\multiput(970.00,581.95)(1.355,-0.447){3}{\rule{1.033pt}{0.108pt}}
\multiput(970.00,582.17)(4.855,-3.000){2}{\rule{0.517pt}{0.400pt}}
\put(977,578.17){\rule{1.500pt}{0.400pt}}
\multiput(977.00,579.17)(3.887,-2.000){2}{\rule{0.750pt}{0.400pt}}
\multiput(984.00,576.95)(1.355,-0.447){3}{\rule{1.033pt}{0.108pt}}
\multiput(984.00,577.17)(4.855,-3.000){2}{\rule{0.517pt}{0.400pt}}
\put(991,573.17){\rule{1.500pt}{0.400pt}}
\multiput(991.00,574.17)(3.887,-2.000){2}{\rule{0.750pt}{0.400pt}}
\multiput(998.00,571.95)(1.355,-0.447){3}{\rule{1.033pt}{0.108pt}}
\multiput(998.00,572.17)(4.855,-3.000){2}{\rule{0.517pt}{0.400pt}}
\multiput(1005.00,568.95)(1.355,-0.447){3}{\rule{1.033pt}{0.108pt}}
\multiput(1005.00,569.17)(4.855,-3.000){2}{\rule{0.517pt}{0.400pt}}
\multiput(1012.00,565.95)(1.355,-0.447){3}{\rule{1.033pt}{0.108pt}}
\multiput(1012.00,566.17)(4.855,-3.000){2}{\rule{0.517pt}{0.400pt}}
\multiput(1019.00,562.95)(1.355,-0.447){3}{\rule{1.033pt}{0.108pt}}
\multiput(1019.00,563.17)(4.855,-3.000){2}{\rule{0.517pt}{0.400pt}}
\multiput(1026.00,559.94)(0.774,-0.468){5}{\rule{0.700pt}{0.113pt}}
\multiput(1026.00,560.17)(4.547,-4.000){2}{\rule{0.350pt}{0.400pt}}
\multiput(1032.00,555.95)(1.355,-0.447){3}{\rule{1.033pt}{0.108pt}}
\multiput(1032.00,556.17)(4.855,-3.000){2}{\rule{0.517pt}{0.400pt}}
\multiput(1039.00,552.95)(1.355,-0.447){3}{\rule{1.033pt}{0.108pt}}
\multiput(1039.00,553.17)(4.855,-3.000){2}{\rule{0.517pt}{0.400pt}}
\multiput(1046.00,549.94)(0.920,-0.468){5}{\rule{0.800pt}{0.113pt}}
\multiput(1046.00,550.17)(5.340,-4.000){2}{\rule{0.400pt}{0.400pt}}
\multiput(1053.00,545.94)(0.920,-0.468){5}{\rule{0.800pt}{0.113pt}}
\multiput(1053.00,546.17)(5.340,-4.000){2}{\rule{0.400pt}{0.400pt}}
\multiput(1060.00,541.95)(1.355,-0.447){3}{\rule{1.033pt}{0.108pt}}
\multiput(1060.00,542.17)(4.855,-3.000){2}{\rule{0.517pt}{0.400pt}}
\multiput(1067.00,538.94)(0.920,-0.468){5}{\rule{0.800pt}{0.113pt}}
\multiput(1067.00,539.17)(5.340,-4.000){2}{\rule{0.400pt}{0.400pt}}
\multiput(1074.00,534.94)(0.920,-0.468){5}{\rule{0.800pt}{0.113pt}}
\multiput(1074.00,535.17)(5.340,-4.000){2}{\rule{0.400pt}{0.400pt}}
\multiput(1081.00,530.93)(0.710,-0.477){7}{\rule{0.660pt}{0.115pt}}
\multiput(1081.00,531.17)(5.630,-5.000){2}{\rule{0.330pt}{0.400pt}}
\multiput(1088.00,525.94)(0.920,-0.468){5}{\rule{0.800pt}{0.113pt}}
\multiput(1088.00,526.17)(5.340,-4.000){2}{\rule{0.400pt}{0.400pt}}
\multiput(1095.00,521.94)(0.774,-0.468){5}{\rule{0.700pt}{0.113pt}}
\multiput(1095.00,522.17)(4.547,-4.000){2}{\rule{0.350pt}{0.400pt}}
\multiput(1101.00,517.93)(0.710,-0.477){7}{\rule{0.660pt}{0.115pt}}
\multiput(1101.00,518.17)(5.630,-5.000){2}{\rule{0.330pt}{0.400pt}}
\multiput(1108.00,512.94)(0.920,-0.468){5}{\rule{0.800pt}{0.113pt}}
\multiput(1108.00,513.17)(5.340,-4.000){2}{\rule{0.400pt}{0.400pt}}
\multiput(1115.00,508.93)(0.710,-0.477){7}{\rule{0.660pt}{0.115pt}}
\multiput(1115.00,509.17)(5.630,-5.000){2}{\rule{0.330pt}{0.400pt}}
\multiput(1122.00,503.93)(0.710,-0.477){7}{\rule{0.660pt}{0.115pt}}
\multiput(1122.00,504.17)(5.630,-5.000){2}{\rule{0.330pt}{0.400pt}}
\multiput(1129.00,498.93)(0.710,-0.477){7}{\rule{0.660pt}{0.115pt}}
\multiput(1129.00,499.17)(5.630,-5.000){2}{\rule{0.330pt}{0.400pt}}
\multiput(1136.00,493.93)(0.710,-0.477){7}{\rule{0.660pt}{0.115pt}}
\multiput(1136.00,494.17)(5.630,-5.000){2}{\rule{0.330pt}{0.400pt}}
\multiput(1143.00,488.93)(0.710,-0.477){7}{\rule{0.660pt}{0.115pt}}
\multiput(1143.00,489.17)(5.630,-5.000){2}{\rule{0.330pt}{0.400pt}}
\multiput(1150.00,483.93)(0.710,-0.477){7}{\rule{0.660pt}{0.115pt}}
\multiput(1150.00,484.17)(5.630,-5.000){2}{\rule{0.330pt}{0.400pt}}
\multiput(1157.00,478.93)(0.491,-0.482){9}{\rule{0.500pt}{0.116pt}}
\multiput(1157.00,479.17)(4.962,-6.000){2}{\rule{0.250pt}{0.400pt}}
\multiput(1163.00,472.93)(0.581,-0.482){9}{\rule{0.567pt}{0.116pt}}
\multiput(1163.00,473.17)(5.824,-6.000){2}{\rule{0.283pt}{0.400pt}}
\multiput(1170.00,466.93)(0.710,-0.477){7}{\rule{0.660pt}{0.115pt}}
\multiput(1170.00,467.17)(5.630,-5.000){2}{\rule{0.330pt}{0.400pt}}
\multiput(1177.00,461.93)(0.581,-0.482){9}{\rule{0.567pt}{0.116pt}}
\multiput(1177.00,462.17)(5.824,-6.000){2}{\rule{0.283pt}{0.400pt}}
\multiput(1184.00,455.93)(0.581,-0.482){9}{\rule{0.567pt}{0.116pt}}
\multiput(1184.00,456.17)(5.824,-6.000){2}{\rule{0.283pt}{0.400pt}}
\multiput(1191.00,449.93)(0.581,-0.482){9}{\rule{0.567pt}{0.116pt}}
\multiput(1191.00,450.17)(5.824,-6.000){2}{\rule{0.283pt}{0.400pt}}
\multiput(1198.00,443.93)(0.492,-0.485){11}{\rule{0.500pt}{0.117pt}}
\multiput(1198.00,444.17)(5.962,-7.000){2}{\rule{0.250pt}{0.400pt}}
\multiput(1205.00,436.93)(0.581,-0.482){9}{\rule{0.567pt}{0.116pt}}
\multiput(1205.00,437.17)(5.824,-6.000){2}{\rule{0.283pt}{0.400pt}}
\multiput(1212.00,430.93)(0.492,-0.485){11}{\rule{0.500pt}{0.117pt}}
\multiput(1212.00,431.17)(5.962,-7.000){2}{\rule{0.250pt}{0.400pt}}
\multiput(1219.00,423.93)(0.581,-0.482){9}{\rule{0.567pt}{0.116pt}}
\multiput(1219.00,424.17)(5.824,-6.000){2}{\rule{0.283pt}{0.400pt}}
\multiput(564.00,533.61)(1.355,0.447){3}{\rule{1.033pt}{0.108pt}}
\multiput(564.00,532.17)(4.855,3.000){2}{\rule{0.517pt}{0.400pt}}
\multiput(571.00,536.61)(1.355,0.447){3}{\rule{1.033pt}{0.108pt}}
\multiput(571.00,535.17)(4.855,3.000){2}{\rule{0.517pt}{0.400pt}}
\multiput(578.00,539.61)(1.132,0.447){3}{\rule{0.900pt}{0.108pt}}
\multiput(578.00,538.17)(4.132,3.000){2}{\rule{0.450pt}{0.400pt}}
\put(584,542.17){\rule{1.500pt}{0.400pt}}
\multiput(584.00,541.17)(3.887,2.000){2}{\rule{0.750pt}{0.400pt}}
\multiput(591.00,544.61)(1.355,0.447){3}{\rule{1.033pt}{0.108pt}}
\multiput(591.00,543.17)(4.855,3.000){2}{\rule{0.517pt}{0.400pt}}
\put(598,547.17){\rule{1.500pt}{0.400pt}}
\multiput(598.00,546.17)(3.887,2.000){2}{\rule{0.750pt}{0.400pt}}
\multiput(605.00,549.61)(1.355,0.447){3}{\rule{1.033pt}{0.108pt}}
\multiput(605.00,548.17)(4.855,3.000){2}{\rule{0.517pt}{0.400pt}}
\put(612,552.17){\rule{1.500pt}{0.400pt}}
\multiput(612.00,551.17)(3.887,2.000){2}{\rule{0.750pt}{0.400pt}}
\put(619,554.17){\rule{1.500pt}{0.400pt}}
\multiput(619.00,553.17)(3.887,2.000){2}{\rule{0.750pt}{0.400pt}}
\put(626,556.17){\rule{1.500pt}{0.400pt}}
\multiput(626.00,555.17)(3.887,2.000){2}{\rule{0.750pt}{0.400pt}}
\put(633,558.17){\rule{1.500pt}{0.400pt}}
\multiput(633.00,557.17)(3.887,2.000){2}{\rule{0.750pt}{0.400pt}}
\put(640,560.17){\rule{1.500pt}{0.400pt}}
\multiput(640.00,559.17)(3.887,2.000){2}{\rule{0.750pt}{0.400pt}}
\put(647,562.17){\rule{1.300pt}{0.400pt}}
\multiput(647.00,561.17)(3.302,2.000){2}{\rule{0.650pt}{0.400pt}}
\put(653,564.17){\rule{1.500pt}{0.400pt}}
\multiput(653.00,563.17)(3.887,2.000){2}{\rule{0.750pt}{0.400pt}}
\put(660,566.17){\rule{1.500pt}{0.400pt}}
\multiput(660.00,565.17)(3.887,2.000){2}{\rule{0.750pt}{0.400pt}}
\put(667,567.67){\rule{1.686pt}{0.400pt}}
\multiput(667.00,567.17)(3.500,1.000){2}{\rule{0.843pt}{0.400pt}}
\put(674,569.17){\rule{1.500pt}{0.400pt}}
\multiput(674.00,568.17)(3.887,2.000){2}{\rule{0.750pt}{0.400pt}}
\put(681,570.67){\rule{1.686pt}{0.400pt}}
\multiput(681.00,570.17)(3.500,1.000){2}{\rule{0.843pt}{0.400pt}}
\put(688,571.67){\rule{1.686pt}{0.400pt}}
\multiput(688.00,571.17)(3.500,1.000){2}{\rule{0.843pt}{0.400pt}}
\put(695,573.17){\rule{1.500pt}{0.400pt}}
\multiput(695.00,572.17)(3.887,2.000){2}{\rule{0.750pt}{0.400pt}}
\put(702,574.67){\rule{1.686pt}{0.400pt}}
\multiput(702.00,574.17)(3.500,1.000){2}{\rule{0.843pt}{0.400pt}}
\put(709,575.67){\rule{1.445pt}{0.400pt}}
\multiput(709.00,575.17)(3.000,1.000){2}{\rule{0.723pt}{0.400pt}}
\put(715,576.67){\rule{1.686pt}{0.400pt}}
\multiput(715.00,576.17)(3.500,1.000){2}{\rule{0.843pt}{0.400pt}}
\put(722,577.67){\rule{1.686pt}{0.400pt}}
\multiput(722.00,577.17)(3.500,1.000){2}{\rule{0.843pt}{0.400pt}}
\put(846.0,608.0){\rule[-0.200pt]{1.686pt}{0.400pt}}
\put(736,578.67){\rule{1.686pt}{0.400pt}}
\multiput(736.00,578.17)(3.500,1.000){2}{\rule{0.843pt}{0.400pt}}
\put(743,579.67){\rule{1.686pt}{0.400pt}}
\multiput(743.00,579.17)(3.500,1.000){2}{\rule{0.843pt}{0.400pt}}
\put(729.0,579.0){\rule[-0.200pt]{1.686pt}{0.400pt}}
\put(757,580.67){\rule{1.686pt}{0.400pt}}
\multiput(757.00,580.17)(3.500,1.000){2}{\rule{0.843pt}{0.400pt}}
\put(750.0,581.0){\rule[-0.200pt]{1.686pt}{0.400pt}}
\put(771,581.67){\rule{1.686pt}{0.400pt}}
\multiput(771.00,581.17)(3.500,1.000){2}{\rule{0.843pt}{0.400pt}}
\put(764.0,582.0){\rule[-0.200pt]{1.686pt}{0.400pt}}
\put(778.0,583.0){\rule[-0.200pt]{1.445pt}{0.400pt}}
\put(784.0,583.0){\rule[-0.200pt]{1.686pt}{0.400pt}}
\put(791.0,583.0){\rule[-0.200pt]{1.686pt}{0.400pt}}
\put(798.0,583.0){\rule[-0.200pt]{1.686pt}{0.400pt}}
\put(805.0,583.0){\rule[-0.200pt]{1.686pt}{0.400pt}}
\put(812.0,583.0){\rule[-0.200pt]{1.686pt}{0.400pt}}
\put(819.0,583.0){\rule[-0.200pt]{1.686pt}{0.400pt}}
\put(826.0,583.0){\rule[-0.200pt]{1.686pt}{0.400pt}}
\put(840,581.67){\rule{1.445pt}{0.400pt}}
\multiput(840.00,582.17)(3.000,-1.000){2}{\rule{0.723pt}{0.400pt}}
\put(833.0,583.0){\rule[-0.200pt]{1.686pt}{0.400pt}}
\put(853,580.67){\rule{1.686pt}{0.400pt}}
\multiput(853.00,581.17)(3.500,-1.000){2}{\rule{0.843pt}{0.400pt}}
\put(860,579.67){\rule{1.686pt}{0.400pt}}
\multiput(860.00,580.17)(3.500,-1.000){2}{\rule{0.843pt}{0.400pt}}
\put(846.0,582.0){\rule[-0.200pt]{1.686pt}{0.400pt}}
\put(874,578.67){\rule{1.686pt}{0.400pt}}
\multiput(874.00,579.17)(3.500,-1.000){2}{\rule{0.843pt}{0.400pt}}
\put(881,577.67){\rule{1.686pt}{0.400pt}}
\multiput(881.00,578.17)(3.500,-1.000){2}{\rule{0.843pt}{0.400pt}}
\put(888,576.67){\rule{1.686pt}{0.400pt}}
\multiput(888.00,577.17)(3.500,-1.000){2}{\rule{0.843pt}{0.400pt}}
\put(895,575.67){\rule{1.686pt}{0.400pt}}
\multiput(895.00,576.17)(3.500,-1.000){2}{\rule{0.843pt}{0.400pt}}
\put(902,574.67){\rule{1.445pt}{0.400pt}}
\multiput(902.00,575.17)(3.000,-1.000){2}{\rule{0.723pt}{0.400pt}}
\put(908,573.67){\rule{1.686pt}{0.400pt}}
\multiput(908.00,574.17)(3.500,-1.000){2}{\rule{0.843pt}{0.400pt}}
\put(915,572.67){\rule{1.686pt}{0.400pt}}
\multiput(915.00,573.17)(3.500,-1.000){2}{\rule{0.843pt}{0.400pt}}
\put(922,571.67){\rule{1.686pt}{0.400pt}}
\multiput(922.00,572.17)(3.500,-1.000){2}{\rule{0.843pt}{0.400pt}}
\put(929,570.17){\rule{1.500pt}{0.400pt}}
\multiput(929.00,571.17)(3.887,-2.000){2}{\rule{0.750pt}{0.400pt}}
\put(936,568.67){\rule{1.686pt}{0.400pt}}
\multiput(936.00,569.17)(3.500,-1.000){2}{\rule{0.843pt}{0.400pt}}
\put(943,567.17){\rule{1.500pt}{0.400pt}}
\multiput(943.00,568.17)(3.887,-2.000){2}{\rule{0.750pt}{0.400pt}}
\put(950,565.67){\rule{1.686pt}{0.400pt}}
\multiput(950.00,566.17)(3.500,-1.000){2}{\rule{0.843pt}{0.400pt}}
\put(957,564.17){\rule{1.500pt}{0.400pt}}
\multiput(957.00,565.17)(3.887,-2.000){2}{\rule{0.750pt}{0.400pt}}
\put(964,562.17){\rule{1.500pt}{0.400pt}}
\multiput(964.00,563.17)(3.887,-2.000){2}{\rule{0.750pt}{0.400pt}}
\put(971,560.67){\rule{1.445pt}{0.400pt}}
\multiput(971.00,561.17)(3.000,-1.000){2}{\rule{0.723pt}{0.400pt}}
\put(977,559.17){\rule{1.500pt}{0.400pt}}
\multiput(977.00,560.17)(3.887,-2.000){2}{\rule{0.750pt}{0.400pt}}
\put(984,557.17){\rule{1.500pt}{0.400pt}}
\multiput(984.00,558.17)(3.887,-2.000){2}{\rule{0.750pt}{0.400pt}}
\put(991,555.17){\rule{1.500pt}{0.400pt}}
\multiput(991.00,556.17)(3.887,-2.000){2}{\rule{0.750pt}{0.400pt}}
\put(998,553.17){\rule{1.500pt}{0.400pt}}
\multiput(998.00,554.17)(3.887,-2.000){2}{\rule{0.750pt}{0.400pt}}
\put(1005,551.17){\rule{1.500pt}{0.400pt}}
\multiput(1005.00,552.17)(3.887,-2.000){2}{\rule{0.750pt}{0.400pt}}
\multiput(1012.00,549.95)(1.355,-0.447){3}{\rule{1.033pt}{0.108pt}}
\multiput(1012.00,550.17)(4.855,-3.000){2}{\rule{0.517pt}{0.400pt}}
\put(1019,546.17){\rule{1.500pt}{0.400pt}}
\multiput(1019.00,547.17)(3.887,-2.000){2}{\rule{0.750pt}{0.400pt}}
\put(1026,544.17){\rule{1.500pt}{0.400pt}}
\multiput(1026.00,545.17)(3.887,-2.000){2}{\rule{0.750pt}{0.400pt}}
\multiput(1033.00,542.95)(1.132,-0.447){3}{\rule{0.900pt}{0.108pt}}
\multiput(1033.00,543.17)(4.132,-3.000){2}{\rule{0.450pt}{0.400pt}}
\put(1039,539.17){\rule{1.500pt}{0.400pt}}
\multiput(1039.00,540.17)(3.887,-2.000){2}{\rule{0.750pt}{0.400pt}}
\multiput(1046.00,537.95)(1.355,-0.447){3}{\rule{1.033pt}{0.108pt}}
\multiput(1046.00,538.17)(4.855,-3.000){2}{\rule{0.517pt}{0.400pt}}
\multiput(1053.00,534.95)(1.355,-0.447){3}{\rule{1.033pt}{0.108pt}}
\multiput(1053.00,535.17)(4.855,-3.000){2}{\rule{0.517pt}{0.400pt}}
\multiput(1060.00,531.95)(1.355,-0.447){3}{\rule{1.033pt}{0.108pt}}
\multiput(1060.00,532.17)(4.855,-3.000){2}{\rule{0.517pt}{0.400pt}}
\put(1067,528.17){\rule{1.500pt}{0.400pt}}
\multiput(1067.00,529.17)(3.887,-2.000){2}{\rule{0.750pt}{0.400pt}}
\multiput(1074.00,526.95)(1.355,-0.447){3}{\rule{1.033pt}{0.108pt}}
\multiput(1074.00,527.17)(4.855,-3.000){2}{\rule{0.517pt}{0.400pt}}
\multiput(1081.00,523.95)(1.355,-0.447){3}{\rule{1.033pt}{0.108pt}}
\multiput(1081.00,524.17)(4.855,-3.000){2}{\rule{0.517pt}{0.400pt}}
\multiput(1088.00,520.95)(1.355,-0.447){3}{\rule{1.033pt}{0.108pt}}
\multiput(1088.00,521.17)(4.855,-3.000){2}{\rule{0.517pt}{0.400pt}}
\multiput(1095.00,517.94)(0.774,-0.468){5}{\rule{0.700pt}{0.113pt}}
\multiput(1095.00,518.17)(4.547,-4.000){2}{\rule{0.350pt}{0.400pt}}
\multiput(1101.00,513.95)(1.355,-0.447){3}{\rule{1.033pt}{0.108pt}}
\multiput(1101.00,514.17)(4.855,-3.000){2}{\rule{0.517pt}{0.400pt}}
\multiput(1108.00,510.95)(1.355,-0.447){3}{\rule{1.033pt}{0.108pt}}
\multiput(1108.00,511.17)(4.855,-3.000){2}{\rule{0.517pt}{0.400pt}}
\multiput(1115.00,507.94)(0.920,-0.468){5}{\rule{0.800pt}{0.113pt}}
\multiput(1115.00,508.17)(5.340,-4.000){2}{\rule{0.400pt}{0.400pt}}
\multiput(1122.00,503.95)(1.355,-0.447){3}{\rule{1.033pt}{0.108pt}}
\multiput(1122.00,504.17)(4.855,-3.000){2}{\rule{0.517pt}{0.400pt}}
\multiput(1129.00,500.94)(0.920,-0.468){5}{\rule{0.800pt}{0.113pt}}
\multiput(1129.00,501.17)(5.340,-4.000){2}{\rule{0.400pt}{0.400pt}}
\multiput(1136.00,496.95)(1.355,-0.447){3}{\rule{1.033pt}{0.108pt}}
\multiput(1136.00,497.17)(4.855,-3.000){2}{\rule{0.517pt}{0.400pt}}
\multiput(1143.00,493.94)(0.920,-0.468){5}{\rule{0.800pt}{0.113pt}}
\multiput(1143.00,494.17)(5.340,-4.000){2}{\rule{0.400pt}{0.400pt}}
\multiput(1150.00,489.94)(0.920,-0.468){5}{\rule{0.800pt}{0.113pt}}
\multiput(1150.00,490.17)(5.340,-4.000){2}{\rule{0.400pt}{0.400pt}}
\multiput(1157.00,485.94)(0.920,-0.468){5}{\rule{0.800pt}{0.113pt}}
\multiput(1157.00,486.17)(5.340,-4.000){2}{\rule{0.400pt}{0.400pt}}
\multiput(1164.00,481.94)(0.774,-0.468){5}{\rule{0.700pt}{0.113pt}}
\multiput(1164.00,482.17)(4.547,-4.000){2}{\rule{0.350pt}{0.400pt}}
\multiput(1170.00,477.94)(0.920,-0.468){5}{\rule{0.800pt}{0.113pt}}
\multiput(1170.00,478.17)(5.340,-4.000){2}{\rule{0.400pt}{0.400pt}}
\multiput(1177.00,473.94)(0.920,-0.468){5}{\rule{0.800pt}{0.113pt}}
\multiput(1177.00,474.17)(5.340,-4.000){2}{\rule{0.400pt}{0.400pt}}
\multiput(1184.00,469.93)(0.710,-0.477){7}{\rule{0.660pt}{0.115pt}}
\multiput(1184.00,470.17)(5.630,-5.000){2}{\rule{0.330pt}{0.400pt}}
\multiput(1191.00,464.94)(0.920,-0.468){5}{\rule{0.800pt}{0.113pt}}
\multiput(1191.00,465.17)(5.340,-4.000){2}{\rule{0.400pt}{0.400pt}}
\multiput(1198.00,460.93)(0.710,-0.477){7}{\rule{0.660pt}{0.115pt}}
\multiput(1198.00,461.17)(5.630,-5.000){2}{\rule{0.330pt}{0.400pt}}
\multiput(1205.00,455.94)(0.920,-0.468){5}{\rule{0.800pt}{0.113pt}}
\multiput(1205.00,456.17)(5.340,-4.000){2}{\rule{0.400pt}{0.400pt}}
\multiput(1212.00,451.93)(0.710,-0.477){7}{\rule{0.660pt}{0.115pt}}
\multiput(1212.00,452.17)(5.630,-5.000){2}{\rule{0.330pt}{0.400pt}}
\multiput(1219.00,446.93)(0.710,-0.477){7}{\rule{0.660pt}{0.115pt}}
\multiput(1219.00,447.17)(5.630,-5.000){2}{\rule{0.330pt}{0.400pt}}
\multiput(1226.00,441.93)(0.599,-0.477){7}{\rule{0.580pt}{0.115pt}}
\multiput(1226.00,442.17)(4.796,-5.000){2}{\rule{0.290pt}{0.400pt}}
\multiput(1232.00,436.93)(0.710,-0.477){7}{\rule{0.660pt}{0.115pt}}
\multiput(1232.00,437.17)(5.630,-5.000){2}{\rule{0.330pt}{0.400pt}}
\multiput(1239.00,431.93)(0.710,-0.477){7}{\rule{0.660pt}{0.115pt}}
\multiput(1239.00,432.17)(5.630,-5.000){2}{\rule{0.330pt}{0.400pt}}
\put(585,542.67){\rule{1.445pt}{0.400pt}}
\multiput(585.00,542.17)(3.000,1.000){2}{\rule{0.723pt}{0.400pt}}
\put(591,543.67){\rule{1.686pt}{0.400pt}}
\multiput(591.00,543.17)(3.500,1.000){2}{\rule{0.843pt}{0.400pt}}
\put(598,544.67){\rule{1.686pt}{0.400pt}}
\multiput(598.00,544.17)(3.500,1.000){2}{\rule{0.843pt}{0.400pt}}
\put(605,545.67){\rule{1.686pt}{0.400pt}}
\multiput(605.00,545.17)(3.500,1.000){2}{\rule{0.843pt}{0.400pt}}
\put(612,546.67){\rule{1.686pt}{0.400pt}}
\multiput(612.00,546.17)(3.500,1.000){2}{\rule{0.843pt}{0.400pt}}
\put(867.0,580.0){\rule[-0.200pt]{1.686pt}{0.400pt}}
\put(626,547.67){\rule{1.686pt}{0.400pt}}
\multiput(626.00,547.17)(3.500,1.000){2}{\rule{0.843pt}{0.400pt}}
\put(633,548.67){\rule{1.686pt}{0.400pt}}
\multiput(633.00,548.17)(3.500,1.000){2}{\rule{0.843pt}{0.400pt}}
\put(640,549.67){\rule{1.686pt}{0.400pt}}
\multiput(640.00,549.17)(3.500,1.000){2}{\rule{0.843pt}{0.400pt}}
\put(619.0,548.0){\rule[-0.200pt]{1.686pt}{0.400pt}}
\put(653,550.67){\rule{1.686pt}{0.400pt}}
\multiput(653.00,550.17)(3.500,1.000){2}{\rule{0.843pt}{0.400pt}}
\put(647.0,551.0){\rule[-0.200pt]{1.445pt}{0.400pt}}
\put(667,551.67){\rule{1.686pt}{0.400pt}}
\multiput(667.00,551.17)(3.500,1.000){2}{\rule{0.843pt}{0.400pt}}
\put(660.0,552.0){\rule[-0.200pt]{1.686pt}{0.400pt}}
\put(681,552.67){\rule{1.686pt}{0.400pt}}
\multiput(681.00,552.17)(3.500,1.000){2}{\rule{0.843pt}{0.400pt}}
\put(674.0,553.0){\rule[-0.200pt]{1.686pt}{0.400pt}}
\put(688.0,554.0){\rule[-0.200pt]{1.686pt}{0.400pt}}
\put(702,553.67){\rule{1.686pt}{0.400pt}}
\multiput(702.00,553.17)(3.500,1.000){2}{\rule{0.843pt}{0.400pt}}
\put(695.0,554.0){\rule[-0.200pt]{1.686pt}{0.400pt}}
\put(709.0,555.0){\rule[-0.200pt]{1.686pt}{0.400pt}}
\put(716.0,555.0){\rule[-0.200pt]{1.445pt}{0.400pt}}
\put(722.0,555.0){\rule[-0.200pt]{1.686pt}{0.400pt}}
\put(729.0,555.0){\rule[-0.200pt]{1.686pt}{0.400pt}}
\put(736.0,555.0){\rule[-0.200pt]{1.686pt}{0.400pt}}
\put(743.0,555.0){\rule[-0.200pt]{1.686pt}{0.400pt}}
\put(750.0,555.0){\rule[-0.200pt]{1.686pt}{0.400pt}}
\put(757.0,555.0){\rule[-0.200pt]{1.686pt}{0.400pt}}
\put(764.0,555.0){\rule[-0.200pt]{1.686pt}{0.400pt}}
\put(771.0,555.0){\rule[-0.200pt]{1.686pt}{0.400pt}}
\put(784,553.67){\rule{1.686pt}{0.400pt}}
\multiput(784.00,554.17)(3.500,-1.000){2}{\rule{0.843pt}{0.400pt}}
\put(778.0,555.0){\rule[-0.200pt]{1.445pt}{0.400pt}}
\put(791.0,554.0){\rule[-0.200pt]{1.686pt}{0.400pt}}
\put(805,552.67){\rule{1.686pt}{0.400pt}}
\multiput(805.00,553.17)(3.500,-1.000){2}{\rule{0.843pt}{0.400pt}}
\put(798.0,554.0){\rule[-0.200pt]{1.686pt}{0.400pt}}
\put(819,551.67){\rule{1.686pt}{0.400pt}}
\multiput(819.00,552.17)(3.500,-1.000){2}{\rule{0.843pt}{0.400pt}}
\put(812.0,553.0){\rule[-0.200pt]{1.686pt}{0.400pt}}
\put(833,550.67){\rule{1.686pt}{0.400pt}}
\multiput(833.00,551.17)(3.500,-1.000){2}{\rule{0.843pt}{0.400pt}}
\put(826.0,552.0){\rule[-0.200pt]{1.686pt}{0.400pt}}
\put(846,549.67){\rule{1.686pt}{0.400pt}}
\multiput(846.00,550.17)(3.500,-1.000){2}{\rule{0.843pt}{0.400pt}}
\put(853,548.67){\rule{1.686pt}{0.400pt}}
\multiput(853.00,549.17)(3.500,-1.000){2}{\rule{0.843pt}{0.400pt}}
\put(840.0,551.0){\rule[-0.200pt]{1.445pt}{0.400pt}}
\put(867,547.67){\rule{1.686pt}{0.400pt}}
\multiput(867.00,548.17)(3.500,-1.000){2}{\rule{0.843pt}{0.400pt}}
\put(874,546.67){\rule{1.686pt}{0.400pt}}
\multiput(874.00,547.17)(3.500,-1.000){2}{\rule{0.843pt}{0.400pt}}
\put(881,545.67){\rule{1.686pt}{0.400pt}}
\multiput(881.00,546.17)(3.500,-1.000){2}{\rule{0.843pt}{0.400pt}}
\put(860.0,549.0){\rule[-0.200pt]{1.686pt}{0.400pt}}
\put(895,544.67){\rule{1.686pt}{0.400pt}}
\multiput(895.00,545.17)(3.500,-1.000){2}{\rule{0.843pt}{0.400pt}}
\put(902,543.67){\rule{1.686pt}{0.400pt}}
\multiput(902.00,544.17)(3.500,-1.000){2}{\rule{0.843pt}{0.400pt}}
\put(909,542.67){\rule{1.445pt}{0.400pt}}
\multiput(909.00,543.17)(3.000,-1.000){2}{\rule{0.723pt}{0.400pt}}
\put(915,541.67){\rule{1.686pt}{0.400pt}}
\multiput(915.00,542.17)(3.500,-1.000){2}{\rule{0.843pt}{0.400pt}}
\put(922,540.67){\rule{1.686pt}{0.400pt}}
\multiput(922.00,541.17)(3.500,-1.000){2}{\rule{0.843pt}{0.400pt}}
\put(929,539.67){\rule{1.686pt}{0.400pt}}
\multiput(929.00,540.17)(3.500,-1.000){2}{\rule{0.843pt}{0.400pt}}
\put(936,538.17){\rule{1.500pt}{0.400pt}}
\multiput(936.00,539.17)(3.887,-2.000){2}{\rule{0.750pt}{0.400pt}}
\put(943,536.67){\rule{1.686pt}{0.400pt}}
\multiput(943.00,537.17)(3.500,-1.000){2}{\rule{0.843pt}{0.400pt}}
\put(950,535.67){\rule{1.686pt}{0.400pt}}
\multiput(950.00,536.17)(3.500,-1.000){2}{\rule{0.843pt}{0.400pt}}
\put(957,534.67){\rule{1.686pt}{0.400pt}}
\multiput(957.00,535.17)(3.500,-1.000){2}{\rule{0.843pt}{0.400pt}}
\put(964,533.17){\rule{1.500pt}{0.400pt}}
\multiput(964.00,534.17)(3.887,-2.000){2}{\rule{0.750pt}{0.400pt}}
\put(971,531.67){\rule{1.445pt}{0.400pt}}
\multiput(971.00,532.17)(3.000,-1.000){2}{\rule{0.723pt}{0.400pt}}
\put(977,530.67){\rule{1.686pt}{0.400pt}}
\multiput(977.00,531.17)(3.500,-1.000){2}{\rule{0.843pt}{0.400pt}}
\put(984,529.17){\rule{1.500pt}{0.400pt}}
\multiput(984.00,530.17)(3.887,-2.000){2}{\rule{0.750pt}{0.400pt}}
\put(991,527.67){\rule{1.686pt}{0.400pt}}
\multiput(991.00,528.17)(3.500,-1.000){2}{\rule{0.843pt}{0.400pt}}
\put(998,526.17){\rule{1.500pt}{0.400pt}}
\multiput(998.00,527.17)(3.887,-2.000){2}{\rule{0.750pt}{0.400pt}}
\put(1005,524.67){\rule{1.686pt}{0.400pt}}
\multiput(1005.00,525.17)(3.500,-1.000){2}{\rule{0.843pt}{0.400pt}}
\put(1012,523.17){\rule{1.500pt}{0.400pt}}
\multiput(1012.00,524.17)(3.887,-2.000){2}{\rule{0.750pt}{0.400pt}}
\put(1019,521.67){\rule{1.686pt}{0.400pt}}
\multiput(1019.00,522.17)(3.500,-1.000){2}{\rule{0.843pt}{0.400pt}}
\put(1026,520.17){\rule{1.500pt}{0.400pt}}
\multiput(1026.00,521.17)(3.887,-2.000){2}{\rule{0.750pt}{0.400pt}}
\put(1033,518.17){\rule{1.500pt}{0.400pt}}
\multiput(1033.00,519.17)(3.887,-2.000){2}{\rule{0.750pt}{0.400pt}}
\put(1040,516.67){\rule{1.445pt}{0.400pt}}
\multiput(1040.00,517.17)(3.000,-1.000){2}{\rule{0.723pt}{0.400pt}}
\put(1046,515.17){\rule{1.500pt}{0.400pt}}
\multiput(1046.00,516.17)(3.887,-2.000){2}{\rule{0.750pt}{0.400pt}}
\put(1053,513.17){\rule{1.500pt}{0.400pt}}
\multiput(1053.00,514.17)(3.887,-2.000){2}{\rule{0.750pt}{0.400pt}}
\put(1060,511.17){\rule{1.500pt}{0.400pt}}
\multiput(1060.00,512.17)(3.887,-2.000){2}{\rule{0.750pt}{0.400pt}}
\put(1067,509.17){\rule{1.500pt}{0.400pt}}
\multiput(1067.00,510.17)(3.887,-2.000){2}{\rule{0.750pt}{0.400pt}}
\put(1074,507.17){\rule{1.500pt}{0.400pt}}
\multiput(1074.00,508.17)(3.887,-2.000){2}{\rule{0.750pt}{0.400pt}}
\put(1081,505.17){\rule{1.500pt}{0.400pt}}
\multiput(1081.00,506.17)(3.887,-2.000){2}{\rule{0.750pt}{0.400pt}}
\put(1088,503.17){\rule{1.500pt}{0.400pt}}
\multiput(1088.00,504.17)(3.887,-2.000){2}{\rule{0.750pt}{0.400pt}}
\put(1095,501.17){\rule{1.500pt}{0.400pt}}
\multiput(1095.00,502.17)(3.887,-2.000){2}{\rule{0.750pt}{0.400pt}}
\put(1102,499.17){\rule{1.300pt}{0.400pt}}
\multiput(1102.00,500.17)(3.302,-2.000){2}{\rule{0.650pt}{0.400pt}}
\put(1108,497.17){\rule{1.500pt}{0.400pt}}
\multiput(1108.00,498.17)(3.887,-2.000){2}{\rule{0.750pt}{0.400pt}}
\put(1115,495.17){\rule{1.500pt}{0.400pt}}
\multiput(1115.00,496.17)(3.887,-2.000){2}{\rule{0.750pt}{0.400pt}}
\put(1122,493.17){\rule{1.500pt}{0.400pt}}
\multiput(1122.00,494.17)(3.887,-2.000){2}{\rule{0.750pt}{0.400pt}}
\multiput(1129.00,491.95)(1.355,-0.447){3}{\rule{1.033pt}{0.108pt}}
\multiput(1129.00,492.17)(4.855,-3.000){2}{\rule{0.517pt}{0.400pt}}
\put(1136,488.17){\rule{1.500pt}{0.400pt}}
\multiput(1136.00,489.17)(3.887,-2.000){2}{\rule{0.750pt}{0.400pt}}
\put(1143,486.17){\rule{1.500pt}{0.400pt}}
\multiput(1143.00,487.17)(3.887,-2.000){2}{\rule{0.750pt}{0.400pt}}
\multiput(1150.00,484.95)(1.355,-0.447){3}{\rule{1.033pt}{0.108pt}}
\multiput(1150.00,485.17)(4.855,-3.000){2}{\rule{0.517pt}{0.400pt}}
\put(1157,481.17){\rule{1.500pt}{0.400pt}}
\multiput(1157.00,482.17)(3.887,-2.000){2}{\rule{0.750pt}{0.400pt}}
\multiput(1164.00,479.95)(1.132,-0.447){3}{\rule{0.900pt}{0.108pt}}
\multiput(1164.00,480.17)(4.132,-3.000){2}{\rule{0.450pt}{0.400pt}}
\put(1170,476.17){\rule{1.500pt}{0.400pt}}
\multiput(1170.00,477.17)(3.887,-2.000){2}{\rule{0.750pt}{0.400pt}}
\multiput(1177.00,474.95)(1.355,-0.447){3}{\rule{1.033pt}{0.108pt}}
\multiput(1177.00,475.17)(4.855,-3.000){2}{\rule{0.517pt}{0.400pt}}
\multiput(1184.00,471.95)(1.355,-0.447){3}{\rule{1.033pt}{0.108pt}}
\multiput(1184.00,472.17)(4.855,-3.000){2}{\rule{0.517pt}{0.400pt}}
\put(1191,468.17){\rule{1.500pt}{0.400pt}}
\multiput(1191.00,469.17)(3.887,-2.000){2}{\rule{0.750pt}{0.400pt}}
\multiput(1198.00,466.95)(1.355,-0.447){3}{\rule{1.033pt}{0.108pt}}
\multiput(1198.00,467.17)(4.855,-3.000){2}{\rule{0.517pt}{0.400pt}}
\multiput(1205.00,463.95)(1.355,-0.447){3}{\rule{1.033pt}{0.108pt}}
\multiput(1205.00,464.17)(4.855,-3.000){2}{\rule{0.517pt}{0.400pt}}
\multiput(1212.00,460.95)(1.355,-0.447){3}{\rule{1.033pt}{0.108pt}}
\multiput(1212.00,461.17)(4.855,-3.000){2}{\rule{0.517pt}{0.400pt}}
\multiput(1219.00,457.95)(1.355,-0.447){3}{\rule{1.033pt}{0.108pt}}
\multiput(1219.00,458.17)(4.855,-3.000){2}{\rule{0.517pt}{0.400pt}}
\multiput(1226.00,454.95)(1.355,-0.447){3}{\rule{1.033pt}{0.108pt}}
\multiput(1226.00,455.17)(4.855,-3.000){2}{\rule{0.517pt}{0.400pt}}
\multiput(1233.00,451.95)(1.132,-0.447){3}{\rule{0.900pt}{0.108pt}}
\multiput(1233.00,452.17)(4.132,-3.000){2}{\rule{0.450pt}{0.400pt}}
\multiput(1239.00,448.95)(1.355,-0.447){3}{\rule{1.033pt}{0.108pt}}
\multiput(1239.00,449.17)(4.855,-3.000){2}{\rule{0.517pt}{0.400pt}}
\multiput(1246.00,445.95)(1.355,-0.447){3}{\rule{1.033pt}{0.108pt}}
\multiput(1246.00,446.17)(4.855,-3.000){2}{\rule{0.517pt}{0.400pt}}
\multiput(1253.00,442.95)(1.355,-0.447){3}{\rule{1.033pt}{0.108pt}}
\multiput(1253.00,443.17)(4.855,-3.000){2}{\rule{0.517pt}{0.400pt}}
\multiput(1260.00,439.95)(1.355,-0.447){3}{\rule{1.033pt}{0.108pt}}
\multiput(1260.00,440.17)(4.855,-3.000){2}{\rule{0.517pt}{0.400pt}}
\put(605,550.67){\rule{1.686pt}{0.400pt}}
\multiput(605.00,551.17)(3.500,-1.000){2}{\rule{0.843pt}{0.400pt}}
\put(612,549.67){\rule{1.686pt}{0.400pt}}
\multiput(612.00,550.17)(3.500,-1.000){2}{\rule{0.843pt}{0.400pt}}
\put(619,548.67){\rule{1.686pt}{0.400pt}}
\multiput(619.00,549.17)(3.500,-1.000){2}{\rule{0.843pt}{0.400pt}}
\put(626,547.67){\rule{1.686pt}{0.400pt}}
\multiput(626.00,548.17)(3.500,-1.000){2}{\rule{0.843pt}{0.400pt}}
\put(633,546.67){\rule{1.686pt}{0.400pt}}
\multiput(633.00,547.17)(3.500,-1.000){2}{\rule{0.843pt}{0.400pt}}
\put(640,545.67){\rule{1.686pt}{0.400pt}}
\multiput(640.00,546.17)(3.500,-1.000){2}{\rule{0.843pt}{0.400pt}}
\put(647,544.67){\rule{1.686pt}{0.400pt}}
\multiput(647.00,545.17)(3.500,-1.000){2}{\rule{0.843pt}{0.400pt}}
\put(654,543.67){\rule{1.445pt}{0.400pt}}
\multiput(654.00,544.17)(3.000,-1.000){2}{\rule{0.723pt}{0.400pt}}
\put(660,542.67){\rule{1.686pt}{0.400pt}}
\multiput(660.00,543.17)(3.500,-1.000){2}{\rule{0.843pt}{0.400pt}}
\put(667,541.67){\rule{1.686pt}{0.400pt}}
\multiput(667.00,542.17)(3.500,-1.000){2}{\rule{0.843pt}{0.400pt}}
\put(674,540.17){\rule{1.500pt}{0.400pt}}
\multiput(674.00,541.17)(3.887,-2.000){2}{\rule{0.750pt}{0.400pt}}
\put(681,538.67){\rule{1.686pt}{0.400pt}}
\multiput(681.00,539.17)(3.500,-1.000){2}{\rule{0.843pt}{0.400pt}}
\put(688,537.67){\rule{1.686pt}{0.400pt}}
\multiput(688.00,538.17)(3.500,-1.000){2}{\rule{0.843pt}{0.400pt}}
\put(695,536.67){\rule{1.686pt}{0.400pt}}
\multiput(695.00,537.17)(3.500,-1.000){2}{\rule{0.843pt}{0.400pt}}
\put(702,535.67){\rule{1.686pt}{0.400pt}}
\multiput(702.00,536.17)(3.500,-1.000){2}{\rule{0.843pt}{0.400pt}}
\put(709,534.67){\rule{1.686pt}{0.400pt}}
\multiput(709.00,535.17)(3.500,-1.000){2}{\rule{0.843pt}{0.400pt}}
\put(716,533.67){\rule{1.445pt}{0.400pt}}
\multiput(716.00,534.17)(3.000,-1.000){2}{\rule{0.723pt}{0.400pt}}
\put(722,532.67){\rule{1.686pt}{0.400pt}}
\multiput(722.00,533.17)(3.500,-1.000){2}{\rule{0.843pt}{0.400pt}}
\put(729,531.67){\rule{1.686pt}{0.400pt}}
\multiput(729.00,532.17)(3.500,-1.000){2}{\rule{0.843pt}{0.400pt}}
\put(736,530.67){\rule{1.686pt}{0.400pt}}
\multiput(736.00,531.17)(3.500,-1.000){2}{\rule{0.843pt}{0.400pt}}
\put(743,529.67){\rule{1.686pt}{0.400pt}}
\multiput(743.00,530.17)(3.500,-1.000){2}{\rule{0.843pt}{0.400pt}}
\put(750,528.67){\rule{1.686pt}{0.400pt}}
\multiput(750.00,529.17)(3.500,-1.000){2}{\rule{0.843pt}{0.400pt}}
\put(757,527.67){\rule{1.686pt}{0.400pt}}
\multiput(757.00,528.17)(3.500,-1.000){2}{\rule{0.843pt}{0.400pt}}
\put(764,526.67){\rule{1.686pt}{0.400pt}}
\multiput(764.00,527.17)(3.500,-1.000){2}{\rule{0.843pt}{0.400pt}}
\put(771,525.67){\rule{1.686pt}{0.400pt}}
\multiput(771.00,526.17)(3.500,-1.000){2}{\rule{0.843pt}{0.400pt}}
\put(778,524.67){\rule{1.686pt}{0.400pt}}
\multiput(778.00,525.17)(3.500,-1.000){2}{\rule{0.843pt}{0.400pt}}
\put(785,523.67){\rule{1.445pt}{0.400pt}}
\multiput(785.00,524.17)(3.000,-1.000){2}{\rule{0.723pt}{0.400pt}}
\put(791,522.17){\rule{1.500pt}{0.400pt}}
\multiput(791.00,523.17)(3.887,-2.000){2}{\rule{0.750pt}{0.400pt}}
\put(798,520.67){\rule{1.686pt}{0.400pt}}
\multiput(798.00,521.17)(3.500,-1.000){2}{\rule{0.843pt}{0.400pt}}
\put(805,519.67){\rule{1.686pt}{0.400pt}}
\multiput(805.00,520.17)(3.500,-1.000){2}{\rule{0.843pt}{0.400pt}}
\put(812,518.67){\rule{1.686pt}{0.400pt}}
\multiput(812.00,519.17)(3.500,-1.000){2}{\rule{0.843pt}{0.400pt}}
\put(819,517.67){\rule{1.686pt}{0.400pt}}
\multiput(819.00,518.17)(3.500,-1.000){2}{\rule{0.843pt}{0.400pt}}
\put(826,516.67){\rule{1.686pt}{0.400pt}}
\multiput(826.00,517.17)(3.500,-1.000){2}{\rule{0.843pt}{0.400pt}}
\put(833,515.67){\rule{1.686pt}{0.400pt}}
\multiput(833.00,516.17)(3.500,-1.000){2}{\rule{0.843pt}{0.400pt}}
\put(840,514.67){\rule{1.686pt}{0.400pt}}
\multiput(840.00,515.17)(3.500,-1.000){2}{\rule{0.843pt}{0.400pt}}
\put(847,513.67){\rule{1.445pt}{0.400pt}}
\multiput(847.00,514.17)(3.000,-1.000){2}{\rule{0.723pt}{0.400pt}}
\put(853,512.67){\rule{1.686pt}{0.400pt}}
\multiput(853.00,513.17)(3.500,-1.000){2}{\rule{0.843pt}{0.400pt}}
\put(860,511.67){\rule{1.686pt}{0.400pt}}
\multiput(860.00,512.17)(3.500,-1.000){2}{\rule{0.843pt}{0.400pt}}
\put(867,510.67){\rule{1.686pt}{0.400pt}}
\multiput(867.00,511.17)(3.500,-1.000){2}{\rule{0.843pt}{0.400pt}}
\put(874,509.67){\rule{1.686pt}{0.400pt}}
\multiput(874.00,510.17)(3.500,-1.000){2}{\rule{0.843pt}{0.400pt}}
\put(881,508.67){\rule{1.686pt}{0.400pt}}
\multiput(881.00,509.17)(3.500,-1.000){2}{\rule{0.843pt}{0.400pt}}
\put(888,507.67){\rule{1.686pt}{0.400pt}}
\multiput(888.00,508.17)(3.500,-1.000){2}{\rule{0.843pt}{0.400pt}}
\put(895,506.67){\rule{1.686pt}{0.400pt}}
\multiput(895.00,507.17)(3.500,-1.000){2}{\rule{0.843pt}{0.400pt}}
\put(902,505.67){\rule{1.686pt}{0.400pt}}
\multiput(902.00,506.17)(3.500,-1.000){2}{\rule{0.843pt}{0.400pt}}
\put(909,504.67){\rule{1.445pt}{0.400pt}}
\multiput(909.00,505.17)(3.000,-1.000){2}{\rule{0.723pt}{0.400pt}}
\put(915,503.17){\rule{1.500pt}{0.400pt}}
\multiput(915.00,504.17)(3.887,-2.000){2}{\rule{0.750pt}{0.400pt}}
\put(922,501.67){\rule{1.686pt}{0.400pt}}
\multiput(922.00,502.17)(3.500,-1.000){2}{\rule{0.843pt}{0.400pt}}
\put(929,500.67){\rule{1.686pt}{0.400pt}}
\multiput(929.00,501.17)(3.500,-1.000){2}{\rule{0.843pt}{0.400pt}}
\put(936,499.67){\rule{1.686pt}{0.400pt}}
\multiput(936.00,500.17)(3.500,-1.000){2}{\rule{0.843pt}{0.400pt}}
\put(943,498.67){\rule{1.686pt}{0.400pt}}
\multiput(943.00,499.17)(3.500,-1.000){2}{\rule{0.843pt}{0.400pt}}
\put(950,497.67){\rule{1.686pt}{0.400pt}}
\multiput(950.00,498.17)(3.500,-1.000){2}{\rule{0.843pt}{0.400pt}}
\put(957,496.67){\rule{1.686pt}{0.400pt}}
\multiput(957.00,497.17)(3.500,-1.000){2}{\rule{0.843pt}{0.400pt}}
\put(964,495.67){\rule{1.686pt}{0.400pt}}
\multiput(964.00,496.17)(3.500,-1.000){2}{\rule{0.843pt}{0.400pt}}
\put(971,494.67){\rule{1.686pt}{0.400pt}}
\multiput(971.00,495.17)(3.500,-1.000){2}{\rule{0.843pt}{0.400pt}}
\put(978,493.67){\rule{1.445pt}{0.400pt}}
\multiput(978.00,494.17)(3.000,-1.000){2}{\rule{0.723pt}{0.400pt}}
\put(984,492.67){\rule{1.686pt}{0.400pt}}
\multiput(984.00,493.17)(3.500,-1.000){2}{\rule{0.843pt}{0.400pt}}
\put(991,491.67){\rule{1.686pt}{0.400pt}}
\multiput(991.00,492.17)(3.500,-1.000){2}{\rule{0.843pt}{0.400pt}}
\put(998,490.67){\rule{1.686pt}{0.400pt}}
\multiput(998.00,491.17)(3.500,-1.000){2}{\rule{0.843pt}{0.400pt}}
\put(1005,489.67){\rule{1.686pt}{0.400pt}}
\multiput(1005.00,490.17)(3.500,-1.000){2}{\rule{0.843pt}{0.400pt}}
\put(1012,488.67){\rule{1.686pt}{0.400pt}}
\multiput(1012.00,489.17)(3.500,-1.000){2}{\rule{0.843pt}{0.400pt}}
\put(1019,487.67){\rule{1.686pt}{0.400pt}}
\multiput(1019.00,488.17)(3.500,-1.000){2}{\rule{0.843pt}{0.400pt}}
\put(1026,486.67){\rule{1.686pt}{0.400pt}}
\multiput(1026.00,487.17)(3.500,-1.000){2}{\rule{0.843pt}{0.400pt}}
\put(1033,485.17){\rule{1.500pt}{0.400pt}}
\multiput(1033.00,486.17)(3.887,-2.000){2}{\rule{0.750pt}{0.400pt}}
\put(1040,483.67){\rule{1.445pt}{0.400pt}}
\multiput(1040.00,484.17)(3.000,-1.000){2}{\rule{0.723pt}{0.400pt}}
\put(1046,482.67){\rule{1.686pt}{0.400pt}}
\multiput(1046.00,483.17)(3.500,-1.000){2}{\rule{0.843pt}{0.400pt}}
\put(1053,481.67){\rule{1.686pt}{0.400pt}}
\multiput(1053.00,482.17)(3.500,-1.000){2}{\rule{0.843pt}{0.400pt}}
\put(1060,480.67){\rule{1.686pt}{0.400pt}}
\multiput(1060.00,481.17)(3.500,-1.000){2}{\rule{0.843pt}{0.400pt}}
\put(1067,479.67){\rule{1.686pt}{0.400pt}}
\multiput(1067.00,480.17)(3.500,-1.000){2}{\rule{0.843pt}{0.400pt}}
\put(1074,478.67){\rule{1.686pt}{0.400pt}}
\multiput(1074.00,479.17)(3.500,-1.000){2}{\rule{0.843pt}{0.400pt}}
\put(1081,477.67){\rule{1.686pt}{0.400pt}}
\multiput(1081.00,478.17)(3.500,-1.000){2}{\rule{0.843pt}{0.400pt}}
\put(1088,476.67){\rule{1.686pt}{0.400pt}}
\multiput(1088.00,477.17)(3.500,-1.000){2}{\rule{0.843pt}{0.400pt}}
\put(1095,475.67){\rule{1.686pt}{0.400pt}}
\multiput(1095.00,476.17)(3.500,-1.000){2}{\rule{0.843pt}{0.400pt}}
\put(1102,474.67){\rule{1.445pt}{0.400pt}}
\multiput(1102.00,475.17)(3.000,-1.000){2}{\rule{0.723pt}{0.400pt}}
\put(1108,473.67){\rule{1.686pt}{0.400pt}}
\multiput(1108.00,474.17)(3.500,-1.000){2}{\rule{0.843pt}{0.400pt}}
\put(1115,472.67){\rule{1.686pt}{0.400pt}}
\multiput(1115.00,473.17)(3.500,-1.000){2}{\rule{0.843pt}{0.400pt}}
\put(1122,471.67){\rule{1.686pt}{0.400pt}}
\multiput(1122.00,472.17)(3.500,-1.000){2}{\rule{0.843pt}{0.400pt}}
\put(1129,470.67){\rule{1.686pt}{0.400pt}}
\multiput(1129.00,471.17)(3.500,-1.000){2}{\rule{0.843pt}{0.400pt}}
\put(1136,469.67){\rule{1.686pt}{0.400pt}}
\multiput(1136.00,470.17)(3.500,-1.000){2}{\rule{0.843pt}{0.400pt}}
\put(1143,468.67){\rule{1.686pt}{0.400pt}}
\multiput(1143.00,469.17)(3.500,-1.000){2}{\rule{0.843pt}{0.400pt}}
\put(1150,467.17){\rule{1.500pt}{0.400pt}}
\multiput(1150.00,468.17)(3.887,-2.000){2}{\rule{0.750pt}{0.400pt}}
\put(1157,465.67){\rule{1.686pt}{0.400pt}}
\multiput(1157.00,466.17)(3.500,-1.000){2}{\rule{0.843pt}{0.400pt}}
\put(1164,464.67){\rule{1.686pt}{0.400pt}}
\multiput(1164.00,465.17)(3.500,-1.000){2}{\rule{0.843pt}{0.400pt}}
\put(1171,463.67){\rule{1.445pt}{0.400pt}}
\multiput(1171.00,464.17)(3.000,-1.000){2}{\rule{0.723pt}{0.400pt}}
\put(1177,462.67){\rule{1.686pt}{0.400pt}}
\multiput(1177.00,463.17)(3.500,-1.000){2}{\rule{0.843pt}{0.400pt}}
\put(1184,461.67){\rule{1.686pt}{0.400pt}}
\multiput(1184.00,462.17)(3.500,-1.000){2}{\rule{0.843pt}{0.400pt}}
\put(1191,460.67){\rule{1.686pt}{0.400pt}}
\multiput(1191.00,461.17)(3.500,-1.000){2}{\rule{0.843pt}{0.400pt}}
\put(1198,459.67){\rule{1.686pt}{0.400pt}}
\multiput(1198.00,460.17)(3.500,-1.000){2}{\rule{0.843pt}{0.400pt}}
\put(1205,458.67){\rule{1.686pt}{0.400pt}}
\multiput(1205.00,459.17)(3.500,-1.000){2}{\rule{0.843pt}{0.400pt}}
\put(1212,457.67){\rule{1.686pt}{0.400pt}}
\multiput(1212.00,458.17)(3.500,-1.000){2}{\rule{0.843pt}{0.400pt}}
\put(1219,456.67){\rule{1.686pt}{0.400pt}}
\multiput(1219.00,457.17)(3.500,-1.000){2}{\rule{0.843pt}{0.400pt}}
\put(1226,455.67){\rule{1.686pt}{0.400pt}}
\multiput(1226.00,456.17)(3.500,-1.000){2}{\rule{0.843pt}{0.400pt}}
\put(1233,454.67){\rule{1.445pt}{0.400pt}}
\multiput(1233.00,455.17)(3.000,-1.000){2}{\rule{0.723pt}{0.400pt}}
\put(1239,453.67){\rule{1.686pt}{0.400pt}}
\multiput(1239.00,454.17)(3.500,-1.000){2}{\rule{0.843pt}{0.400pt}}
\put(1246,452.67){\rule{1.686pt}{0.400pt}}
\multiput(1246.00,453.17)(3.500,-1.000){2}{\rule{0.843pt}{0.400pt}}
\put(1253,451.67){\rule{1.686pt}{0.400pt}}
\multiput(1253.00,452.17)(3.500,-1.000){2}{\rule{0.843pt}{0.400pt}}
\put(1260,450.67){\rule{1.686pt}{0.400pt}}
\multiput(1260.00,451.17)(3.500,-1.000){2}{\rule{0.843pt}{0.400pt}}
\put(1267,449.17){\rule{1.500pt}{0.400pt}}
\multiput(1267.00,450.17)(3.887,-2.000){2}{\rule{0.750pt}{0.400pt}}
\put(1274,447.67){\rule{1.686pt}{0.400pt}}
\multiput(1274.00,448.17)(3.500,-1.000){2}{\rule{0.843pt}{0.400pt}}
\put(1281,446.67){\rule{1.686pt}{0.400pt}}
\multiput(1281.00,447.17)(3.500,-1.000){2}{\rule{0.843pt}{0.400pt}}
\put(211,371.17){\rule{0.900pt}{0.400pt}}
\multiput(211.00,370.17)(2.132,2.000){2}{\rule{0.450pt}{0.400pt}}
\put(215,372.67){\rule{0.964pt}{0.400pt}}
\multiput(215.00,372.17)(2.000,1.000){2}{\rule{0.482pt}{0.400pt}}
\put(219,374.17){\rule{0.900pt}{0.400pt}}
\multiput(219.00,373.17)(2.132,2.000){2}{\rule{0.450pt}{0.400pt}}
\put(223,376.17){\rule{0.900pt}{0.400pt}}
\multiput(223.00,375.17)(2.132,2.000){2}{\rule{0.450pt}{0.400pt}}
\put(227,378.17){\rule{0.900pt}{0.400pt}}
\multiput(227.00,377.17)(2.132,2.000){2}{\rule{0.450pt}{0.400pt}}
\put(231,380.17){\rule{0.900pt}{0.400pt}}
\multiput(231.00,379.17)(2.132,2.000){2}{\rule{0.450pt}{0.400pt}}
\put(235,382.17){\rule{0.900pt}{0.400pt}}
\multiput(235.00,381.17)(2.132,2.000){2}{\rule{0.450pt}{0.400pt}}
\put(239,383.67){\rule{0.964pt}{0.400pt}}
\multiput(239.00,383.17)(2.000,1.000){2}{\rule{0.482pt}{0.400pt}}
\put(243,385.17){\rule{0.900pt}{0.400pt}}
\multiput(243.00,384.17)(2.132,2.000){2}{\rule{0.450pt}{0.400pt}}
\put(247,387.17){\rule{0.900pt}{0.400pt}}
\multiput(247.00,386.17)(2.132,2.000){2}{\rule{0.450pt}{0.400pt}}
\put(251,389.17){\rule{0.900pt}{0.400pt}}
\multiput(251.00,388.17)(2.132,2.000){2}{\rule{0.450pt}{0.400pt}}
\put(255,391.17){\rule{0.900pt}{0.400pt}}
\multiput(255.00,390.17)(2.132,2.000){2}{\rule{0.450pt}{0.400pt}}
\put(259,393.17){\rule{0.900pt}{0.400pt}}
\multiput(259.00,392.17)(2.132,2.000){2}{\rule{0.450pt}{0.400pt}}
\put(263,394.67){\rule{0.964pt}{0.400pt}}
\multiput(263.00,394.17)(2.000,1.000){2}{\rule{0.482pt}{0.400pt}}
\put(267,396.17){\rule{0.900pt}{0.400pt}}
\multiput(267.00,395.17)(2.132,2.000){2}{\rule{0.450pt}{0.400pt}}
\put(271,398.17){\rule{0.900pt}{0.400pt}}
\multiput(271.00,397.17)(2.132,2.000){2}{\rule{0.450pt}{0.400pt}}
\put(275,400.17){\rule{0.900pt}{0.400pt}}
\multiput(275.00,399.17)(2.132,2.000){2}{\rule{0.450pt}{0.400pt}}
\put(279,402.17){\rule{0.900pt}{0.400pt}}
\multiput(279.00,401.17)(2.132,2.000){2}{\rule{0.450pt}{0.400pt}}
\put(283,404.17){\rule{0.900pt}{0.400pt}}
\multiput(283.00,403.17)(2.132,2.000){2}{\rule{0.450pt}{0.400pt}}
\put(287,405.67){\rule{0.964pt}{0.400pt}}
\multiput(287.00,405.17)(2.000,1.000){2}{\rule{0.482pt}{0.400pt}}
\put(291,407.17){\rule{0.900pt}{0.400pt}}
\multiput(291.00,406.17)(2.132,2.000){2}{\rule{0.450pt}{0.400pt}}
\put(295,409.17){\rule{0.900pt}{0.400pt}}
\multiput(295.00,408.17)(2.132,2.000){2}{\rule{0.450pt}{0.400pt}}
\put(299,411.17){\rule{0.900pt}{0.400pt}}
\multiput(299.00,410.17)(2.132,2.000){2}{\rule{0.450pt}{0.400pt}}
\put(303,413.17){\rule{0.900pt}{0.400pt}}
\multiput(303.00,412.17)(2.132,2.000){2}{\rule{0.450pt}{0.400pt}}
\put(307,414.67){\rule{0.964pt}{0.400pt}}
\multiput(307.00,414.17)(2.000,1.000){2}{\rule{0.482pt}{0.400pt}}
\put(311,416.17){\rule{0.900pt}{0.400pt}}
\multiput(311.00,415.17)(2.132,2.000){2}{\rule{0.450pt}{0.400pt}}
\put(315,418.17){\rule{0.900pt}{0.400pt}}
\multiput(315.00,417.17)(2.132,2.000){2}{\rule{0.450pt}{0.400pt}}
\put(319,420.17){\rule{0.900pt}{0.400pt}}
\multiput(319.00,419.17)(2.132,2.000){2}{\rule{0.450pt}{0.400pt}}
\put(323,422.17){\rule{0.900pt}{0.400pt}}
\multiput(323.00,421.17)(2.132,2.000){2}{\rule{0.450pt}{0.400pt}}
\put(327,424.17){\rule{0.900pt}{0.400pt}}
\multiput(327.00,423.17)(2.132,2.000){2}{\rule{0.450pt}{0.400pt}}
\put(331,425.67){\rule{0.964pt}{0.400pt}}
\multiput(331.00,425.17)(2.000,1.000){2}{\rule{0.482pt}{0.400pt}}
\put(335,427.17){\rule{0.900pt}{0.400pt}}
\multiput(335.00,426.17)(2.132,2.000){2}{\rule{0.450pt}{0.400pt}}
\put(339,429.17){\rule{0.900pt}{0.400pt}}
\multiput(339.00,428.17)(2.132,2.000){2}{\rule{0.450pt}{0.400pt}}
\put(343,431.17){\rule{0.900pt}{0.400pt}}
\multiput(343.00,430.17)(2.132,2.000){2}{\rule{0.450pt}{0.400pt}}
\put(347,433.17){\rule{0.900pt}{0.400pt}}
\multiput(347.00,432.17)(2.132,2.000){2}{\rule{0.450pt}{0.400pt}}
\put(351,435.17){\rule{0.900pt}{0.400pt}}
\multiput(351.00,434.17)(2.132,2.000){2}{\rule{0.450pt}{0.400pt}}
\put(355,436.67){\rule{0.964pt}{0.400pt}}
\multiput(355.00,436.17)(2.000,1.000){2}{\rule{0.482pt}{0.400pt}}
\put(359,438.17){\rule{0.900pt}{0.400pt}}
\multiput(359.00,437.17)(2.132,2.000){2}{\rule{0.450pt}{0.400pt}}
\put(363,440.17){\rule{0.700pt}{0.400pt}}
\multiput(363.00,439.17)(1.547,2.000){2}{\rule{0.350pt}{0.400pt}}
\put(366,442.17){\rule{0.900pt}{0.400pt}}
\multiput(366.00,441.17)(2.132,2.000){2}{\rule{0.450pt}{0.400pt}}
\put(370,444.17){\rule{0.900pt}{0.400pt}}
\multiput(370.00,443.17)(2.132,2.000){2}{\rule{0.450pt}{0.400pt}}
\put(374,446.17){\rule{0.900pt}{0.400pt}}
\multiput(374.00,445.17)(2.132,2.000){2}{\rule{0.450pt}{0.400pt}}
\put(378,447.67){\rule{0.964pt}{0.400pt}}
\multiput(378.00,447.17)(2.000,1.000){2}{\rule{0.482pt}{0.400pt}}
\put(382,449.17){\rule{0.900pt}{0.400pt}}
\multiput(382.00,448.17)(2.132,2.000){2}{\rule{0.450pt}{0.400pt}}
\put(386,451.17){\rule{0.900pt}{0.400pt}}
\multiput(386.00,450.17)(2.132,2.000){2}{\rule{0.450pt}{0.400pt}}
\put(390,453.17){\rule{0.900pt}{0.400pt}}
\multiput(390.00,452.17)(2.132,2.000){2}{\rule{0.450pt}{0.400pt}}
\put(394,455.17){\rule{0.900pt}{0.400pt}}
\multiput(394.00,454.17)(2.132,2.000){2}{\rule{0.450pt}{0.400pt}}
\put(398,457.17){\rule{0.900pt}{0.400pt}}
\multiput(398.00,456.17)(2.132,2.000){2}{\rule{0.450pt}{0.400pt}}
\put(402,458.67){\rule{0.964pt}{0.400pt}}
\multiput(402.00,458.17)(2.000,1.000){2}{\rule{0.482pt}{0.400pt}}
\put(406,460.17){\rule{0.900pt}{0.400pt}}
\multiput(406.00,459.17)(2.132,2.000){2}{\rule{0.450pt}{0.400pt}}
\put(410,462.17){\rule{0.900pt}{0.400pt}}
\multiput(410.00,461.17)(2.132,2.000){2}{\rule{0.450pt}{0.400pt}}
\put(414,464.17){\rule{0.900pt}{0.400pt}}
\multiput(414.00,463.17)(2.132,2.000){2}{\rule{0.450pt}{0.400pt}}
\put(418,466.17){\rule{0.900pt}{0.400pt}}
\multiput(418.00,465.17)(2.132,2.000){2}{\rule{0.450pt}{0.400pt}}
\put(422,468.17){\rule{0.900pt}{0.400pt}}
\multiput(422.00,467.17)(2.132,2.000){2}{\rule{0.450pt}{0.400pt}}
\put(426,469.67){\rule{0.964pt}{0.400pt}}
\multiput(426.00,469.17)(2.000,1.000){2}{\rule{0.482pt}{0.400pt}}
\put(430,471.17){\rule{0.900pt}{0.400pt}}
\multiput(430.00,470.17)(2.132,2.000){2}{\rule{0.450pt}{0.400pt}}
\put(434,473.17){\rule{0.900pt}{0.400pt}}
\multiput(434.00,472.17)(2.132,2.000){2}{\rule{0.450pt}{0.400pt}}
\put(438,475.17){\rule{0.900pt}{0.400pt}}
\multiput(438.00,474.17)(2.132,2.000){2}{\rule{0.450pt}{0.400pt}}
\put(442,477.17){\rule{0.900pt}{0.400pt}}
\multiput(442.00,476.17)(2.132,2.000){2}{\rule{0.450pt}{0.400pt}}
\put(446,479.17){\rule{0.900pt}{0.400pt}}
\multiput(446.00,478.17)(2.132,2.000){2}{\rule{0.450pt}{0.400pt}}
\put(450,480.67){\rule{0.964pt}{0.400pt}}
\multiput(450.00,480.17)(2.000,1.000){2}{\rule{0.482pt}{0.400pt}}
\put(454,482.17){\rule{0.900pt}{0.400pt}}
\multiput(454.00,481.17)(2.132,2.000){2}{\rule{0.450pt}{0.400pt}}
\put(458,484.17){\rule{0.900pt}{0.400pt}}
\multiput(458.00,483.17)(2.132,2.000){2}{\rule{0.450pt}{0.400pt}}
\put(462,486.17){\rule{0.900pt}{0.400pt}}
\multiput(462.00,485.17)(2.132,2.000){2}{\rule{0.450pt}{0.400pt}}
\put(466,488.17){\rule{0.900pt}{0.400pt}}
\multiput(466.00,487.17)(2.132,2.000){2}{\rule{0.450pt}{0.400pt}}
\put(470,490.17){\rule{0.900pt}{0.400pt}}
\multiput(470.00,489.17)(2.132,2.000){2}{\rule{0.450pt}{0.400pt}}
\put(474,491.67){\rule{0.964pt}{0.400pt}}
\multiput(474.00,491.17)(2.000,1.000){2}{\rule{0.482pt}{0.400pt}}
\put(478,493.17){\rule{0.900pt}{0.400pt}}
\multiput(478.00,492.17)(2.132,2.000){2}{\rule{0.450pt}{0.400pt}}
\put(482,495.17){\rule{0.900pt}{0.400pt}}
\multiput(482.00,494.17)(2.132,2.000){2}{\rule{0.450pt}{0.400pt}}
\put(486,497.17){\rule{0.900pt}{0.400pt}}
\multiput(486.00,496.17)(2.132,2.000){2}{\rule{0.450pt}{0.400pt}}
\put(490,499.17){\rule{0.900pt}{0.400pt}}
\multiput(490.00,498.17)(2.132,2.000){2}{\rule{0.450pt}{0.400pt}}
\put(494,501.17){\rule{0.900pt}{0.400pt}}
\multiput(494.00,500.17)(2.132,2.000){2}{\rule{0.450pt}{0.400pt}}
\put(498,502.67){\rule{0.964pt}{0.400pt}}
\multiput(498.00,502.17)(2.000,1.000){2}{\rule{0.482pt}{0.400pt}}
\put(502,504.17){\rule{0.900pt}{0.400pt}}
\multiput(502.00,503.17)(2.132,2.000){2}{\rule{0.450pt}{0.400pt}}
\put(506,506.17){\rule{0.900pt}{0.400pt}}
\multiput(506.00,505.17)(2.132,2.000){2}{\rule{0.450pt}{0.400pt}}
\put(510,508.17){\rule{0.900pt}{0.400pt}}
\multiput(510.00,507.17)(2.132,2.000){2}{\rule{0.450pt}{0.400pt}}
\put(514,510.17){\rule{0.900pt}{0.400pt}}
\multiput(514.00,509.17)(2.132,2.000){2}{\rule{0.450pt}{0.400pt}}
\put(518,512.17){\rule{0.900pt}{0.400pt}}
\multiput(518.00,511.17)(2.132,2.000){2}{\rule{0.450pt}{0.400pt}}
\put(522,513.67){\rule{0.964pt}{0.400pt}}
\multiput(522.00,513.17)(2.000,1.000){2}{\rule{0.482pt}{0.400pt}}
\put(526,515.17){\rule{0.900pt}{0.400pt}}
\multiput(526.00,514.17)(2.132,2.000){2}{\rule{0.450pt}{0.400pt}}
\put(530,517.17){\rule{0.900pt}{0.400pt}}
\multiput(530.00,516.17)(2.132,2.000){2}{\rule{0.450pt}{0.400pt}}
\put(534,519.17){\rule{0.900pt}{0.400pt}}
\multiput(534.00,518.17)(2.132,2.000){2}{\rule{0.450pt}{0.400pt}}
\put(538,521.17){\rule{0.900pt}{0.400pt}}
\multiput(538.00,520.17)(2.132,2.000){2}{\rule{0.450pt}{0.400pt}}
\put(542,523.17){\rule{0.900pt}{0.400pt}}
\multiput(542.00,522.17)(2.132,2.000){2}{\rule{0.450pt}{0.400pt}}
\put(546,524.67){\rule{0.964pt}{0.400pt}}
\multiput(546.00,524.17)(2.000,1.000){2}{\rule{0.482pt}{0.400pt}}
\put(550,526.17){\rule{0.900pt}{0.400pt}}
\multiput(550.00,525.17)(2.132,2.000){2}{\rule{0.450pt}{0.400pt}}
\put(554,528.17){\rule{0.900pt}{0.400pt}}
\multiput(554.00,527.17)(2.132,2.000){2}{\rule{0.450pt}{0.400pt}}
\put(558,530.17){\rule{0.900pt}{0.400pt}}
\multiput(558.00,529.17)(2.132,2.000){2}{\rule{0.450pt}{0.400pt}}
\put(562,532.17){\rule{0.700pt}{0.400pt}}
\multiput(562.00,531.17)(1.547,2.000){2}{\rule{0.350pt}{0.400pt}}
\put(565,534.17){\rule{0.900pt}{0.400pt}}
\multiput(565.00,533.17)(2.132,2.000){2}{\rule{0.450pt}{0.400pt}}
\put(569,535.67){\rule{0.964pt}{0.400pt}}
\multiput(569.00,535.17)(2.000,1.000){2}{\rule{0.482pt}{0.400pt}}
\put(573,537.17){\rule{0.900pt}{0.400pt}}
\multiput(573.00,536.17)(2.132,2.000){2}{\rule{0.450pt}{0.400pt}}
\put(577,539.17){\rule{0.900pt}{0.400pt}}
\multiput(577.00,538.17)(2.132,2.000){2}{\rule{0.450pt}{0.400pt}}
\put(581,541.17){\rule{0.900pt}{0.400pt}}
\multiput(581.00,540.17)(2.132,2.000){2}{\rule{0.450pt}{0.400pt}}
\put(585,543.17){\rule{0.900pt}{0.400pt}}
\multiput(585.00,542.17)(2.132,2.000){2}{\rule{0.450pt}{0.400pt}}
\put(589,545.17){\rule{0.900pt}{0.400pt}}
\multiput(589.00,544.17)(2.132,2.000){2}{\rule{0.450pt}{0.400pt}}
\put(593,546.67){\rule{0.964pt}{0.400pt}}
\multiput(593.00,546.17)(2.000,1.000){2}{\rule{0.482pt}{0.400pt}}
\put(597,548.17){\rule{0.900pt}{0.400pt}}
\multiput(597.00,547.17)(2.132,2.000){2}{\rule{0.450pt}{0.400pt}}
\put(601,550.17){\rule{0.900pt}{0.400pt}}
\multiput(601.00,549.17)(2.132,2.000){2}{\rule{0.450pt}{0.400pt}}
\multiput(247.00,365.60)(0.481,0.468){5}{\rule{0.500pt}{0.113pt}}
\multiput(247.00,364.17)(2.962,4.000){2}{\rule{0.250pt}{0.400pt}}
\multiput(251.00,369.60)(0.481,0.468){5}{\rule{0.500pt}{0.113pt}}
\multiput(251.00,368.17)(2.962,4.000){2}{\rule{0.250pt}{0.400pt}}
\multiput(255.00,373.60)(0.481,0.468){5}{\rule{0.500pt}{0.113pt}}
\multiput(255.00,372.17)(2.962,4.000){2}{\rule{0.250pt}{0.400pt}}
\multiput(259.00,377.60)(0.481,0.468){5}{\rule{0.500pt}{0.113pt}}
\multiput(259.00,376.17)(2.962,4.000){2}{\rule{0.250pt}{0.400pt}}
\multiput(263.00,381.60)(0.481,0.468){5}{\rule{0.500pt}{0.113pt}}
\multiput(263.00,380.17)(2.962,4.000){2}{\rule{0.250pt}{0.400pt}}
\multiput(267.00,385.61)(0.685,0.447){3}{\rule{0.633pt}{0.108pt}}
\multiput(267.00,384.17)(2.685,3.000){2}{\rule{0.317pt}{0.400pt}}
\multiput(271.00,388.60)(0.481,0.468){5}{\rule{0.500pt}{0.113pt}}
\multiput(271.00,387.17)(2.962,4.000){2}{\rule{0.250pt}{0.400pt}}
\multiput(275.00,392.60)(0.481,0.468){5}{\rule{0.500pt}{0.113pt}}
\multiput(275.00,391.17)(2.962,4.000){2}{\rule{0.250pt}{0.400pt}}
\multiput(279.00,396.61)(0.685,0.447){3}{\rule{0.633pt}{0.108pt}}
\multiput(279.00,395.17)(2.685,3.000){2}{\rule{0.317pt}{0.400pt}}
\multiput(283.00,399.60)(0.481,0.468){5}{\rule{0.500pt}{0.113pt}}
\multiput(283.00,398.17)(2.962,4.000){2}{\rule{0.250pt}{0.400pt}}
\multiput(287.00,403.61)(0.685,0.447){3}{\rule{0.633pt}{0.108pt}}
\multiput(287.00,402.17)(2.685,3.000){2}{\rule{0.317pt}{0.400pt}}
\multiput(291.00,406.60)(0.481,0.468){5}{\rule{0.500pt}{0.113pt}}
\multiput(291.00,405.17)(2.962,4.000){2}{\rule{0.250pt}{0.400pt}}
\multiput(295.00,410.61)(0.685,0.447){3}{\rule{0.633pt}{0.108pt}}
\multiput(295.00,409.17)(2.685,3.000){2}{\rule{0.317pt}{0.400pt}}
\multiput(299.00,413.61)(0.685,0.447){3}{\rule{0.633pt}{0.108pt}}
\multiput(299.00,412.17)(2.685,3.000){2}{\rule{0.317pt}{0.400pt}}
\multiput(303.00,416.60)(0.481,0.468){5}{\rule{0.500pt}{0.113pt}}
\multiput(303.00,415.17)(2.962,4.000){2}{\rule{0.250pt}{0.400pt}}
\multiput(307.00,420.61)(0.685,0.447){3}{\rule{0.633pt}{0.108pt}}
\multiput(307.00,419.17)(2.685,3.000){2}{\rule{0.317pt}{0.400pt}}
\multiput(311.00,423.61)(0.685,0.447){3}{\rule{0.633pt}{0.108pt}}
\multiput(311.00,422.17)(2.685,3.000){2}{\rule{0.317pt}{0.400pt}}
\multiput(315.00,426.61)(0.685,0.447){3}{\rule{0.633pt}{0.108pt}}
\multiput(315.00,425.17)(2.685,3.000){2}{\rule{0.317pt}{0.400pt}}
\multiput(319.00,429.61)(0.685,0.447){3}{\rule{0.633pt}{0.108pt}}
\multiput(319.00,428.17)(2.685,3.000){2}{\rule{0.317pt}{0.400pt}}
\multiput(323.00,432.61)(0.685,0.447){3}{\rule{0.633pt}{0.108pt}}
\multiput(323.00,431.17)(2.685,3.000){2}{\rule{0.317pt}{0.400pt}}
\multiput(327.00,435.61)(0.685,0.447){3}{\rule{0.633pt}{0.108pt}}
\multiput(327.00,434.17)(2.685,3.000){2}{\rule{0.317pt}{0.400pt}}
\multiput(331.00,438.61)(0.685,0.447){3}{\rule{0.633pt}{0.108pt}}
\multiput(331.00,437.17)(2.685,3.000){2}{\rule{0.317pt}{0.400pt}}
\multiput(335.00,441.61)(0.685,0.447){3}{\rule{0.633pt}{0.108pt}}
\multiput(335.00,440.17)(2.685,3.000){2}{\rule{0.317pt}{0.400pt}}
\multiput(339.00,444.61)(0.685,0.447){3}{\rule{0.633pt}{0.108pt}}
\multiput(339.00,443.17)(2.685,3.000){2}{\rule{0.317pt}{0.400pt}}
\multiput(343.00,447.61)(0.685,0.447){3}{\rule{0.633pt}{0.108pt}}
\multiput(343.00,446.17)(2.685,3.000){2}{\rule{0.317pt}{0.400pt}}
\multiput(347.00,450.61)(0.685,0.447){3}{\rule{0.633pt}{0.108pt}}
\multiput(347.00,449.17)(2.685,3.000){2}{\rule{0.317pt}{0.400pt}}
\multiput(351.00,453.61)(0.685,0.447){3}{\rule{0.633pt}{0.108pt}}
\multiput(351.00,452.17)(2.685,3.000){2}{\rule{0.317pt}{0.400pt}}
\put(355,456.17){\rule{0.900pt}{0.400pt}}
\multiput(355.00,455.17)(2.132,2.000){2}{\rule{0.450pt}{0.400pt}}
\multiput(359.00,458.61)(0.685,0.447){3}{\rule{0.633pt}{0.108pt}}
\multiput(359.00,457.17)(2.685,3.000){2}{\rule{0.317pt}{0.400pt}}
\multiput(363.00,461.61)(0.685,0.447){3}{\rule{0.633pt}{0.108pt}}
\multiput(363.00,460.17)(2.685,3.000){2}{\rule{0.317pt}{0.400pt}}
\put(367,464.17){\rule{0.900pt}{0.400pt}}
\multiput(367.00,463.17)(2.132,2.000){2}{\rule{0.450pt}{0.400pt}}
\multiput(371.00,466.61)(0.685,0.447){3}{\rule{0.633pt}{0.108pt}}
\multiput(371.00,465.17)(2.685,3.000){2}{\rule{0.317pt}{0.400pt}}
\put(375,469.17){\rule{0.900pt}{0.400pt}}
\multiput(375.00,468.17)(2.132,2.000){2}{\rule{0.450pt}{0.400pt}}
\multiput(379.00,471.61)(0.685,0.447){3}{\rule{0.633pt}{0.108pt}}
\multiput(379.00,470.17)(2.685,3.000){2}{\rule{0.317pt}{0.400pt}}
\put(383,474.17){\rule{0.700pt}{0.400pt}}
\multiput(383.00,473.17)(1.547,2.000){2}{\rule{0.350pt}{0.400pt}}
\multiput(386.00,476.61)(0.685,0.447){3}{\rule{0.633pt}{0.108pt}}
\multiput(386.00,475.17)(2.685,3.000){2}{\rule{0.317pt}{0.400pt}}
\put(390,479.17){\rule{0.900pt}{0.400pt}}
\multiput(390.00,478.17)(2.132,2.000){2}{\rule{0.450pt}{0.400pt}}
\put(394,481.17){\rule{0.900pt}{0.400pt}}
\multiput(394.00,480.17)(2.132,2.000){2}{\rule{0.450pt}{0.400pt}}
\put(398,483.17){\rule{0.900pt}{0.400pt}}
\multiput(398.00,482.17)(2.132,2.000){2}{\rule{0.450pt}{0.400pt}}
\multiput(402.00,485.61)(0.685,0.447){3}{\rule{0.633pt}{0.108pt}}
\multiput(402.00,484.17)(2.685,3.000){2}{\rule{0.317pt}{0.400pt}}
\put(406,488.17){\rule{0.900pt}{0.400pt}}
\multiput(406.00,487.17)(2.132,2.000){2}{\rule{0.450pt}{0.400pt}}
\put(410,490.17){\rule{0.900pt}{0.400pt}}
\multiput(410.00,489.17)(2.132,2.000){2}{\rule{0.450pt}{0.400pt}}
\put(414,492.17){\rule{0.900pt}{0.400pt}}
\multiput(414.00,491.17)(2.132,2.000){2}{\rule{0.450pt}{0.400pt}}
\put(418,494.17){\rule{0.900pt}{0.400pt}}
\multiput(418.00,493.17)(2.132,2.000){2}{\rule{0.450pt}{0.400pt}}
\put(422,496.17){\rule{0.900pt}{0.400pt}}
\multiput(422.00,495.17)(2.132,2.000){2}{\rule{0.450pt}{0.400pt}}
\put(426,498.17){\rule{0.900pt}{0.400pt}}
\multiput(426.00,497.17)(2.132,2.000){2}{\rule{0.450pt}{0.400pt}}
\put(430,500.17){\rule{0.900pt}{0.400pt}}
\multiput(430.00,499.17)(2.132,2.000){2}{\rule{0.450pt}{0.400pt}}
\put(434,502.17){\rule{0.900pt}{0.400pt}}
\multiput(434.00,501.17)(2.132,2.000){2}{\rule{0.450pt}{0.400pt}}
\put(438,504.17){\rule{0.900pt}{0.400pt}}
\multiput(438.00,503.17)(2.132,2.000){2}{\rule{0.450pt}{0.400pt}}
\put(442,506.17){\rule{0.900pt}{0.400pt}}
\multiput(442.00,505.17)(2.132,2.000){2}{\rule{0.450pt}{0.400pt}}
\put(446,508.17){\rule{0.900pt}{0.400pt}}
\multiput(446.00,507.17)(2.132,2.000){2}{\rule{0.450pt}{0.400pt}}
\put(450,509.67){\rule{0.964pt}{0.400pt}}
\multiput(450.00,509.17)(2.000,1.000){2}{\rule{0.482pt}{0.400pt}}
\put(454,511.17){\rule{0.900pt}{0.400pt}}
\multiput(454.00,510.17)(2.132,2.000){2}{\rule{0.450pt}{0.400pt}}
\put(458,513.17){\rule{0.900pt}{0.400pt}}
\multiput(458.00,512.17)(2.132,2.000){2}{\rule{0.450pt}{0.400pt}}
\put(462,514.67){\rule{0.964pt}{0.400pt}}
\multiput(462.00,514.17)(2.000,1.000){2}{\rule{0.482pt}{0.400pt}}
\put(466,516.17){\rule{0.900pt}{0.400pt}}
\multiput(466.00,515.17)(2.132,2.000){2}{\rule{0.450pt}{0.400pt}}
\put(470,518.17){\rule{0.900pt}{0.400pt}}
\multiput(470.00,517.17)(2.132,2.000){2}{\rule{0.450pt}{0.400pt}}
\put(474,519.67){\rule{0.964pt}{0.400pt}}
\multiput(474.00,519.17)(2.000,1.000){2}{\rule{0.482pt}{0.400pt}}
\put(478,521.17){\rule{0.900pt}{0.400pt}}
\multiput(478.00,520.17)(2.132,2.000){2}{\rule{0.450pt}{0.400pt}}
\put(482,522.67){\rule{0.964pt}{0.400pt}}
\multiput(482.00,522.17)(2.000,1.000){2}{\rule{0.482pt}{0.400pt}}
\put(486,523.67){\rule{0.964pt}{0.400pt}}
\multiput(486.00,523.17)(2.000,1.000){2}{\rule{0.482pt}{0.400pt}}
\put(490,525.17){\rule{0.900pt}{0.400pt}}
\multiput(490.00,524.17)(2.132,2.000){2}{\rule{0.450pt}{0.400pt}}
\put(494,526.67){\rule{0.964pt}{0.400pt}}
\multiput(494.00,526.17)(2.000,1.000){2}{\rule{0.482pt}{0.400pt}}
\put(498,527.67){\rule{0.964pt}{0.400pt}}
\multiput(498.00,527.17)(2.000,1.000){2}{\rule{0.482pt}{0.400pt}}
\put(502,529.17){\rule{0.900pt}{0.400pt}}
\multiput(502.00,528.17)(2.132,2.000){2}{\rule{0.450pt}{0.400pt}}
\put(506,530.67){\rule{0.964pt}{0.400pt}}
\multiput(506.00,530.17)(2.000,1.000){2}{\rule{0.482pt}{0.400pt}}
\put(510,531.67){\rule{0.964pt}{0.400pt}}
\multiput(510.00,531.17)(2.000,1.000){2}{\rule{0.482pt}{0.400pt}}
\put(514,532.67){\rule{0.964pt}{0.400pt}}
\multiput(514.00,532.17)(2.000,1.000){2}{\rule{0.482pt}{0.400pt}}
\put(518,533.67){\rule{0.964pt}{0.400pt}}
\multiput(518.00,533.17)(2.000,1.000){2}{\rule{0.482pt}{0.400pt}}
\put(522,534.67){\rule{0.964pt}{0.400pt}}
\multiput(522.00,534.17)(2.000,1.000){2}{\rule{0.482pt}{0.400pt}}
\put(526,535.67){\rule{0.964pt}{0.400pt}}
\multiput(526.00,535.17)(2.000,1.000){2}{\rule{0.482pt}{0.400pt}}
\put(530,536.67){\rule{0.964pt}{0.400pt}}
\multiput(530.00,536.17)(2.000,1.000){2}{\rule{0.482pt}{0.400pt}}
\put(534,537.67){\rule{0.964pt}{0.400pt}}
\multiput(534.00,537.17)(2.000,1.000){2}{\rule{0.482pt}{0.400pt}}
\put(538,538.67){\rule{0.964pt}{0.400pt}}
\multiput(538.00,538.17)(2.000,1.000){2}{\rule{0.482pt}{0.400pt}}
\put(542,539.67){\rule{0.964pt}{0.400pt}}
\multiput(542.00,539.17)(2.000,1.000){2}{\rule{0.482pt}{0.400pt}}
\put(888.0,546.0){\rule[-0.200pt]{1.686pt}{0.400pt}}
\put(550,540.67){\rule{0.964pt}{0.400pt}}
\multiput(550.00,540.17)(2.000,1.000){2}{\rule{0.482pt}{0.400pt}}
\put(554,541.67){\rule{0.964pt}{0.400pt}}
\multiput(554.00,541.17)(2.000,1.000){2}{\rule{0.482pt}{0.400pt}}
\put(558,542.67){\rule{0.964pt}{0.400pt}}
\multiput(558.00,542.17)(2.000,1.000){2}{\rule{0.482pt}{0.400pt}}
\put(546.0,541.0){\rule[-0.200pt]{0.964pt}{0.400pt}}
\put(566,543.67){\rule{0.964pt}{0.400pt}}
\multiput(566.00,543.17)(2.000,1.000){2}{\rule{0.482pt}{0.400pt}}
\put(562.0,544.0){\rule[-0.200pt]{0.964pt}{0.400pt}}
\put(574,544.67){\rule{0.964pt}{0.400pt}}
\multiput(574.00,544.17)(2.000,1.000){2}{\rule{0.482pt}{0.400pt}}
\put(570.0,545.0){\rule[-0.200pt]{0.964pt}{0.400pt}}
\put(582,545.67){\rule{0.723pt}{0.400pt}}
\multiput(582.00,545.17)(1.500,1.000){2}{\rule{0.361pt}{0.400pt}}
\put(578.0,546.0){\rule[-0.200pt]{0.964pt}{0.400pt}}
\put(585.0,547.0){\rule[-0.200pt]{0.964pt}{0.400pt}}
\put(589.0,547.0){\rule[-0.200pt]{0.964pt}{0.400pt}}
\put(597,546.67){\rule{0.964pt}{0.400pt}}
\multiput(597.00,546.17)(2.000,1.000){2}{\rule{0.482pt}{0.400pt}}
\put(593.0,547.0){\rule[-0.200pt]{0.964pt}{0.400pt}}
\put(601.0,548.0){\rule[-0.200pt]{0.964pt}{0.400pt}}
\put(605.0,548.0){\rule[-0.200pt]{0.964pt}{0.400pt}}
\put(609.0,548.0){\rule[-0.200pt]{0.964pt}{0.400pt}}
\put(613.0,548.0){\rule[-0.200pt]{0.964pt}{0.400pt}}
\put(617.0,548.0){\rule[-0.200pt]{0.964pt}{0.400pt}}
\put(625,546.67){\rule{0.964pt}{0.400pt}}
\multiput(625.00,547.17)(2.000,-1.000){2}{\rule{0.482pt}{0.400pt}}
\put(621.0,548.0){\rule[-0.200pt]{0.964pt}{0.400pt}}
\put(629.0,547.0){\rule[-0.200pt]{0.964pt}{0.400pt}}
\put(633.0,547.0){\rule[-0.200pt]{0.964pt}{0.400pt}}
\multiput(283.60,360.00)(0.468,0.774){5}{\rule{0.113pt}{0.700pt}}
\multiput(282.17,360.00)(4.000,4.547){2}{\rule{0.400pt}{0.350pt}}
\multiput(287.60,366.00)(0.468,0.627){5}{\rule{0.113pt}{0.600pt}}
\multiput(286.17,366.00)(4.000,3.755){2}{\rule{0.400pt}{0.300pt}}
\multiput(291.60,371.00)(0.468,0.774){5}{\rule{0.113pt}{0.700pt}}
\multiput(290.17,371.00)(4.000,4.547){2}{\rule{0.400pt}{0.350pt}}
\multiput(295.60,377.00)(0.468,0.774){5}{\rule{0.113pt}{0.700pt}}
\multiput(294.17,377.00)(4.000,4.547){2}{\rule{0.400pt}{0.350pt}}
\multiput(299.60,383.00)(0.468,0.627){5}{\rule{0.113pt}{0.600pt}}
\multiput(298.17,383.00)(4.000,3.755){2}{\rule{0.400pt}{0.300pt}}
\multiput(303.60,388.00)(0.468,0.627){5}{\rule{0.113pt}{0.600pt}}
\multiput(302.17,388.00)(4.000,3.755){2}{\rule{0.400pt}{0.300pt}}
\multiput(307.60,393.00)(0.468,0.774){5}{\rule{0.113pt}{0.700pt}}
\multiput(306.17,393.00)(4.000,4.547){2}{\rule{0.400pt}{0.350pt}}
\multiput(311.60,399.00)(0.468,0.627){5}{\rule{0.113pt}{0.600pt}}
\multiput(310.17,399.00)(4.000,3.755){2}{\rule{0.400pt}{0.300pt}}
\multiput(315.60,404.00)(0.468,0.627){5}{\rule{0.113pt}{0.600pt}}
\multiput(314.17,404.00)(4.000,3.755){2}{\rule{0.400pt}{0.300pt}}
\multiput(319.60,409.00)(0.468,0.627){5}{\rule{0.113pt}{0.600pt}}
\multiput(318.17,409.00)(4.000,3.755){2}{\rule{0.400pt}{0.300pt}}
\multiput(323.60,414.00)(0.468,0.627){5}{\rule{0.113pt}{0.600pt}}
\multiput(322.17,414.00)(4.000,3.755){2}{\rule{0.400pt}{0.300pt}}
\multiput(327.60,419.00)(0.468,0.627){5}{\rule{0.113pt}{0.600pt}}
\multiput(326.17,419.00)(4.000,3.755){2}{\rule{0.400pt}{0.300pt}}
\multiput(331.00,424.60)(0.481,0.468){5}{\rule{0.500pt}{0.113pt}}
\multiput(331.00,423.17)(2.962,4.000){2}{\rule{0.250pt}{0.400pt}}
\multiput(335.60,428.00)(0.468,0.627){5}{\rule{0.113pt}{0.600pt}}
\multiput(334.17,428.00)(4.000,3.755){2}{\rule{0.400pt}{0.300pt}}
\multiput(339.60,433.00)(0.468,0.627){5}{\rule{0.113pt}{0.600pt}}
\multiput(338.17,433.00)(4.000,3.755){2}{\rule{0.400pt}{0.300pt}}
\multiput(343.00,438.60)(0.481,0.468){5}{\rule{0.500pt}{0.113pt}}
\multiput(343.00,437.17)(2.962,4.000){2}{\rule{0.250pt}{0.400pt}}
\multiput(347.00,442.60)(0.481,0.468){5}{\rule{0.500pt}{0.113pt}}
\multiput(347.00,441.17)(2.962,4.000){2}{\rule{0.250pt}{0.400pt}}
\multiput(351.60,446.00)(0.468,0.627){5}{\rule{0.113pt}{0.600pt}}
\multiput(350.17,446.00)(4.000,3.755){2}{\rule{0.400pt}{0.300pt}}
\multiput(355.00,451.60)(0.481,0.468){5}{\rule{0.500pt}{0.113pt}}
\multiput(355.00,450.17)(2.962,4.000){2}{\rule{0.250pt}{0.400pt}}
\multiput(359.00,455.60)(0.481,0.468){5}{\rule{0.500pt}{0.113pt}}
\multiput(359.00,454.17)(2.962,4.000){2}{\rule{0.250pt}{0.400pt}}
\multiput(363.00,459.60)(0.481,0.468){5}{\rule{0.500pt}{0.113pt}}
\multiput(363.00,458.17)(2.962,4.000){2}{\rule{0.250pt}{0.400pt}}
\multiput(367.00,463.60)(0.481,0.468){5}{\rule{0.500pt}{0.113pt}}
\multiput(367.00,462.17)(2.962,4.000){2}{\rule{0.250pt}{0.400pt}}
\multiput(371.00,467.60)(0.481,0.468){5}{\rule{0.500pt}{0.113pt}}
\multiput(371.00,466.17)(2.962,4.000){2}{\rule{0.250pt}{0.400pt}}
\multiput(375.00,471.60)(0.481,0.468){5}{\rule{0.500pt}{0.113pt}}
\multiput(375.00,470.17)(2.962,4.000){2}{\rule{0.250pt}{0.400pt}}
\multiput(379.00,475.61)(0.685,0.447){3}{\rule{0.633pt}{0.108pt}}
\multiput(379.00,474.17)(2.685,3.000){2}{\rule{0.317pt}{0.400pt}}
\multiput(383.00,478.60)(0.481,0.468){5}{\rule{0.500pt}{0.113pt}}
\multiput(383.00,477.17)(2.962,4.000){2}{\rule{0.250pt}{0.400pt}}
\multiput(387.00,482.60)(0.481,0.468){5}{\rule{0.500pt}{0.113pt}}
\multiput(387.00,481.17)(2.962,4.000){2}{\rule{0.250pt}{0.400pt}}
\multiput(391.00,486.61)(0.685,0.447){3}{\rule{0.633pt}{0.108pt}}
\multiput(391.00,485.17)(2.685,3.000){2}{\rule{0.317pt}{0.400pt}}
\multiput(395.00,489.61)(0.685,0.447){3}{\rule{0.633pt}{0.108pt}}
\multiput(395.00,488.17)(2.685,3.000){2}{\rule{0.317pt}{0.400pt}}
\multiput(399.00,492.60)(0.481,0.468){5}{\rule{0.500pt}{0.113pt}}
\multiput(399.00,491.17)(2.962,4.000){2}{\rule{0.250pt}{0.400pt}}
\multiput(403.00,496.61)(0.462,0.447){3}{\rule{0.500pt}{0.108pt}}
\multiput(403.00,495.17)(1.962,3.000){2}{\rule{0.250pt}{0.400pt}}
\multiput(406.00,499.61)(0.685,0.447){3}{\rule{0.633pt}{0.108pt}}
\multiput(406.00,498.17)(2.685,3.000){2}{\rule{0.317pt}{0.400pt}}
\multiput(410.00,502.61)(0.685,0.447){3}{\rule{0.633pt}{0.108pt}}
\multiput(410.00,501.17)(2.685,3.000){2}{\rule{0.317pt}{0.400pt}}
\multiput(414.00,505.61)(0.685,0.447){3}{\rule{0.633pt}{0.108pt}}
\multiput(414.00,504.17)(2.685,3.000){2}{\rule{0.317pt}{0.400pt}}
\multiput(418.00,508.61)(0.685,0.447){3}{\rule{0.633pt}{0.108pt}}
\multiput(418.00,507.17)(2.685,3.000){2}{\rule{0.317pt}{0.400pt}}
\multiput(422.00,511.61)(0.685,0.447){3}{\rule{0.633pt}{0.108pt}}
\multiput(422.00,510.17)(2.685,3.000){2}{\rule{0.317pt}{0.400pt}}
\multiput(426.00,514.61)(0.685,0.447){3}{\rule{0.633pt}{0.108pt}}
\multiput(426.00,513.17)(2.685,3.000){2}{\rule{0.317pt}{0.400pt}}
\multiput(430.00,517.61)(0.685,0.447){3}{\rule{0.633pt}{0.108pt}}
\multiput(430.00,516.17)(2.685,3.000){2}{\rule{0.317pt}{0.400pt}}
\put(434,520.17){\rule{0.900pt}{0.400pt}}
\multiput(434.00,519.17)(2.132,2.000){2}{\rule{0.450pt}{0.400pt}}
\multiput(438.00,522.61)(0.685,0.447){3}{\rule{0.633pt}{0.108pt}}
\multiput(438.00,521.17)(2.685,3.000){2}{\rule{0.317pt}{0.400pt}}
\put(442,525.17){\rule{0.900pt}{0.400pt}}
\multiput(442.00,524.17)(2.132,2.000){2}{\rule{0.450pt}{0.400pt}}
\multiput(446.00,527.61)(0.685,0.447){3}{\rule{0.633pt}{0.108pt}}
\multiput(446.00,526.17)(2.685,3.000){2}{\rule{0.317pt}{0.400pt}}
\put(450,530.17){\rule{0.900pt}{0.400pt}}
\multiput(450.00,529.17)(2.132,2.000){2}{\rule{0.450pt}{0.400pt}}
\put(454,532.17){\rule{0.900pt}{0.400pt}}
\multiput(454.00,531.17)(2.132,2.000){2}{\rule{0.450pt}{0.400pt}}
\multiput(458.00,534.61)(0.685,0.447){3}{\rule{0.633pt}{0.108pt}}
\multiput(458.00,533.17)(2.685,3.000){2}{\rule{0.317pt}{0.400pt}}
\put(462,537.17){\rule{0.900pt}{0.400pt}}
\multiput(462.00,536.17)(2.132,2.000){2}{\rule{0.450pt}{0.400pt}}
\put(466,539.17){\rule{0.900pt}{0.400pt}}
\multiput(466.00,538.17)(2.132,2.000){2}{\rule{0.450pt}{0.400pt}}
\put(470,541.17){\rule{0.900pt}{0.400pt}}
\multiput(470.00,540.17)(2.132,2.000){2}{\rule{0.450pt}{0.400pt}}
\put(474,543.17){\rule{0.900pt}{0.400pt}}
\multiput(474.00,542.17)(2.132,2.000){2}{\rule{0.450pt}{0.400pt}}
\put(478,544.67){\rule{0.964pt}{0.400pt}}
\multiput(478.00,544.17)(2.000,1.000){2}{\rule{0.482pt}{0.400pt}}
\put(482,546.17){\rule{0.900pt}{0.400pt}}
\multiput(482.00,545.17)(2.132,2.000){2}{\rule{0.450pt}{0.400pt}}
\put(486,548.17){\rule{0.900pt}{0.400pt}}
\multiput(486.00,547.17)(2.132,2.000){2}{\rule{0.450pt}{0.400pt}}
\put(490,549.67){\rule{0.964pt}{0.400pt}}
\multiput(490.00,549.17)(2.000,1.000){2}{\rule{0.482pt}{0.400pt}}
\put(494,551.17){\rule{0.900pt}{0.400pt}}
\multiput(494.00,550.17)(2.132,2.000){2}{\rule{0.450pt}{0.400pt}}
\put(498,552.67){\rule{0.964pt}{0.400pt}}
\multiput(498.00,552.17)(2.000,1.000){2}{\rule{0.482pt}{0.400pt}}
\put(502,554.17){\rule{0.900pt}{0.400pt}}
\multiput(502.00,553.17)(2.132,2.000){2}{\rule{0.450pt}{0.400pt}}
\put(506,555.67){\rule{0.964pt}{0.400pt}}
\multiput(506.00,555.17)(2.000,1.000){2}{\rule{0.482pt}{0.400pt}}
\put(510,556.67){\rule{0.964pt}{0.400pt}}
\multiput(510.00,556.17)(2.000,1.000){2}{\rule{0.482pt}{0.400pt}}
\put(514,558.17){\rule{0.900pt}{0.400pt}}
\multiput(514.00,557.17)(2.132,2.000){2}{\rule{0.450pt}{0.400pt}}
\put(518,559.67){\rule{0.964pt}{0.400pt}}
\multiput(518.00,559.17)(2.000,1.000){2}{\rule{0.482pt}{0.400pt}}
\put(522,560.67){\rule{0.964pt}{0.400pt}}
\multiput(522.00,560.17)(2.000,1.000){2}{\rule{0.482pt}{0.400pt}}
\put(526,561.67){\rule{0.964pt}{0.400pt}}
\multiput(526.00,561.17)(2.000,1.000){2}{\rule{0.482pt}{0.400pt}}
\put(530,562.67){\rule{0.964pt}{0.400pt}}
\multiput(530.00,562.17)(2.000,1.000){2}{\rule{0.482pt}{0.400pt}}
\put(637.0,547.0){\rule[-0.200pt]{0.964pt}{0.400pt}}
\put(538,563.67){\rule{0.964pt}{0.400pt}}
\multiput(538.00,563.17)(2.000,1.000){2}{\rule{0.482pt}{0.400pt}}
\put(542,564.67){\rule{0.964pt}{0.400pt}}
\multiput(542.00,564.17)(2.000,1.000){2}{\rule{0.482pt}{0.400pt}}
\put(534.0,564.0){\rule[-0.200pt]{0.964pt}{0.400pt}}
\put(550,565.67){\rule{0.964pt}{0.400pt}}
\multiput(550.00,565.17)(2.000,1.000){2}{\rule{0.482pt}{0.400pt}}
\put(546.0,566.0){\rule[-0.200pt]{0.964pt}{0.400pt}}
\put(558,566.67){\rule{0.964pt}{0.400pt}}
\multiput(558.00,566.17)(2.000,1.000){2}{\rule{0.482pt}{0.400pt}}
\put(554.0,567.0){\rule[-0.200pt]{0.964pt}{0.400pt}}
\put(562.0,568.0){\rule[-0.200pt]{0.964pt}{0.400pt}}
\put(566.0,568.0){\rule[-0.200pt]{0.964pt}{0.400pt}}
\put(570.0,568.0){\rule[-0.200pt]{0.964pt}{0.400pt}}
\put(574.0,568.0){\rule[-0.200pt]{0.964pt}{0.400pt}}
\put(578.0,568.0){\rule[-0.200pt]{0.964pt}{0.400pt}}
\put(582.0,568.0){\rule[-0.200pt]{0.964pt}{0.400pt}}
\put(586.0,568.0){\rule[-0.200pt]{0.964pt}{0.400pt}}
\put(594,566.67){\rule{0.964pt}{0.400pt}}
\multiput(594.00,567.17)(2.000,-1.000){2}{\rule{0.482pt}{0.400pt}}
\put(590.0,568.0){\rule[-0.200pt]{0.964pt}{0.400pt}}
\put(602,565.67){\rule{0.723pt}{0.400pt}}
\multiput(602.00,566.17)(1.500,-1.000){2}{\rule{0.361pt}{0.400pt}}
\put(598.0,567.0){\rule[-0.200pt]{0.964pt}{0.400pt}}
\put(609,564.67){\rule{0.964pt}{0.400pt}}
\multiput(609.00,565.17)(2.000,-1.000){2}{\rule{0.482pt}{0.400pt}}
\put(613,563.67){\rule{0.964pt}{0.400pt}}
\multiput(613.00,564.17)(2.000,-1.000){2}{\rule{0.482pt}{0.400pt}}
\put(617,562.67){\rule{0.964pt}{0.400pt}}
\multiput(617.00,563.17)(2.000,-1.000){2}{\rule{0.482pt}{0.400pt}}
\put(621,561.67){\rule{0.964pt}{0.400pt}}
\multiput(621.00,562.17)(2.000,-1.000){2}{\rule{0.482pt}{0.400pt}}
\put(625,560.67){\rule{0.964pt}{0.400pt}}
\multiput(625.00,561.17)(2.000,-1.000){2}{\rule{0.482pt}{0.400pt}}
\put(629,559.67){\rule{0.964pt}{0.400pt}}
\multiput(629.00,560.17)(2.000,-1.000){2}{\rule{0.482pt}{0.400pt}}
\put(633,558.67){\rule{0.964pt}{0.400pt}}
\multiput(633.00,559.17)(2.000,-1.000){2}{\rule{0.482pt}{0.400pt}}
\put(637,557.17){\rule{0.900pt}{0.400pt}}
\multiput(637.00,558.17)(2.132,-2.000){2}{\rule{0.450pt}{0.400pt}}
\put(641,555.67){\rule{0.964pt}{0.400pt}}
\multiput(641.00,556.17)(2.000,-1.000){2}{\rule{0.482pt}{0.400pt}}
\put(645,554.17){\rule{0.900pt}{0.400pt}}
\multiput(645.00,555.17)(2.132,-2.000){2}{\rule{0.450pt}{0.400pt}}
\put(649,552.67){\rule{0.964pt}{0.400pt}}
\multiput(649.00,553.17)(2.000,-1.000){2}{\rule{0.482pt}{0.400pt}}
\put(653,551.17){\rule{0.900pt}{0.400pt}}
\multiput(653.00,552.17)(2.132,-2.000){2}{\rule{0.450pt}{0.400pt}}
\put(657,549.17){\rule{0.900pt}{0.400pt}}
\multiput(657.00,550.17)(2.132,-2.000){2}{\rule{0.450pt}{0.400pt}}
\put(661,547.17){\rule{0.900pt}{0.400pt}}
\multiput(661.00,548.17)(2.132,-2.000){2}{\rule{0.450pt}{0.400pt}}
\put(665,545.17){\rule{0.900pt}{0.400pt}}
\multiput(665.00,546.17)(2.132,-2.000){2}{\rule{0.450pt}{0.400pt}}
\put(669,543.17){\rule{0.900pt}{0.400pt}}
\multiput(669.00,544.17)(2.132,-2.000){2}{\rule{0.450pt}{0.400pt}}
\put(673,541.17){\rule{0.900pt}{0.400pt}}
\multiput(673.00,542.17)(2.132,-2.000){2}{\rule{0.450pt}{0.400pt}}
\multiput(319.60,354.00)(0.468,1.066){5}{\rule{0.113pt}{0.900pt}}
\multiput(318.17,354.00)(4.000,6.132){2}{\rule{0.400pt}{0.450pt}}
\multiput(323.60,362.00)(0.468,0.920){5}{\rule{0.113pt}{0.800pt}}
\multiput(322.17,362.00)(4.000,5.340){2}{\rule{0.400pt}{0.400pt}}
\multiput(327.60,369.00)(0.468,0.920){5}{\rule{0.113pt}{0.800pt}}
\multiput(326.17,369.00)(4.000,5.340){2}{\rule{0.400pt}{0.400pt}}
\multiput(331.60,376.00)(0.468,0.920){5}{\rule{0.113pt}{0.800pt}}
\multiput(330.17,376.00)(4.000,5.340){2}{\rule{0.400pt}{0.400pt}}
\multiput(335.60,383.00)(0.468,0.920){5}{\rule{0.113pt}{0.800pt}}
\multiput(334.17,383.00)(4.000,5.340){2}{\rule{0.400pt}{0.400pt}}
\multiput(339.60,390.00)(0.468,0.920){5}{\rule{0.113pt}{0.800pt}}
\multiput(338.17,390.00)(4.000,5.340){2}{\rule{0.400pt}{0.400pt}}
\multiput(343.60,397.00)(0.468,0.920){5}{\rule{0.113pt}{0.800pt}}
\multiput(342.17,397.00)(4.000,5.340){2}{\rule{0.400pt}{0.400pt}}
\multiput(347.60,404.00)(0.468,0.774){5}{\rule{0.113pt}{0.700pt}}
\multiput(346.17,404.00)(4.000,4.547){2}{\rule{0.400pt}{0.350pt}}
\multiput(351.60,410.00)(0.468,0.774){5}{\rule{0.113pt}{0.700pt}}
\multiput(350.17,410.00)(4.000,4.547){2}{\rule{0.400pt}{0.350pt}}
\multiput(355.60,416.00)(0.468,0.920){5}{\rule{0.113pt}{0.800pt}}
\multiput(354.17,416.00)(4.000,5.340){2}{\rule{0.400pt}{0.400pt}}
\multiput(359.60,423.00)(0.468,0.774){5}{\rule{0.113pt}{0.700pt}}
\multiput(358.17,423.00)(4.000,4.547){2}{\rule{0.400pt}{0.350pt}}
\multiput(363.60,429.00)(0.468,0.774){5}{\rule{0.113pt}{0.700pt}}
\multiput(362.17,429.00)(4.000,4.547){2}{\rule{0.400pt}{0.350pt}}
\multiput(367.60,435.00)(0.468,0.774){5}{\rule{0.113pt}{0.700pt}}
\multiput(366.17,435.00)(4.000,4.547){2}{\rule{0.400pt}{0.350pt}}
\multiput(371.60,441.00)(0.468,0.627){5}{\rule{0.113pt}{0.600pt}}
\multiput(370.17,441.00)(4.000,3.755){2}{\rule{0.400pt}{0.300pt}}
\multiput(375.60,446.00)(0.468,0.774){5}{\rule{0.113pt}{0.700pt}}
\multiput(374.17,446.00)(4.000,4.547){2}{\rule{0.400pt}{0.350pt}}
\multiput(379.60,452.00)(0.468,0.774){5}{\rule{0.113pt}{0.700pt}}
\multiput(378.17,452.00)(4.000,4.547){2}{\rule{0.400pt}{0.350pt}}
\multiput(383.60,458.00)(0.468,0.627){5}{\rule{0.113pt}{0.600pt}}
\multiput(382.17,458.00)(4.000,3.755){2}{\rule{0.400pt}{0.300pt}}
\multiput(387.60,463.00)(0.468,0.627){5}{\rule{0.113pt}{0.600pt}}
\multiput(386.17,463.00)(4.000,3.755){2}{\rule{0.400pt}{0.300pt}}
\multiput(391.60,468.00)(0.468,0.627){5}{\rule{0.113pt}{0.600pt}}
\multiput(390.17,468.00)(4.000,3.755){2}{\rule{0.400pt}{0.300pt}}
\multiput(395.60,473.00)(0.468,0.627){5}{\rule{0.113pt}{0.600pt}}
\multiput(394.17,473.00)(4.000,3.755){2}{\rule{0.400pt}{0.300pt}}
\multiput(399.60,478.00)(0.468,0.627){5}{\rule{0.113pt}{0.600pt}}
\multiput(398.17,478.00)(4.000,3.755){2}{\rule{0.400pt}{0.300pt}}
\multiput(403.60,483.00)(0.468,0.627){5}{\rule{0.113pt}{0.600pt}}
\multiput(402.17,483.00)(4.000,3.755){2}{\rule{0.400pt}{0.300pt}}
\multiput(407.60,488.00)(0.468,0.627){5}{\rule{0.113pt}{0.600pt}}
\multiput(406.17,488.00)(4.000,3.755){2}{\rule{0.400pt}{0.300pt}}
\multiput(411.00,493.60)(0.481,0.468){5}{\rule{0.500pt}{0.113pt}}
\multiput(411.00,492.17)(2.962,4.000){2}{\rule{0.250pt}{0.400pt}}
\multiput(415.60,497.00)(0.468,0.627){5}{\rule{0.113pt}{0.600pt}}
\multiput(414.17,497.00)(4.000,3.755){2}{\rule{0.400pt}{0.300pt}}
\multiput(419.00,502.60)(0.481,0.468){5}{\rule{0.500pt}{0.113pt}}
\multiput(419.00,501.17)(2.962,4.000){2}{\rule{0.250pt}{0.400pt}}
\multiput(423.61,506.00)(0.447,0.685){3}{\rule{0.108pt}{0.633pt}}
\multiput(422.17,506.00)(3.000,2.685){2}{\rule{0.400pt}{0.317pt}}
\multiput(426.60,510.00)(0.468,0.627){5}{\rule{0.113pt}{0.600pt}}
\multiput(425.17,510.00)(4.000,3.755){2}{\rule{0.400pt}{0.300pt}}
\multiput(430.00,515.60)(0.481,0.468){5}{\rule{0.500pt}{0.113pt}}
\multiput(430.00,514.17)(2.962,4.000){2}{\rule{0.250pt}{0.400pt}}
\multiput(434.00,519.61)(0.685,0.447){3}{\rule{0.633pt}{0.108pt}}
\multiput(434.00,518.17)(2.685,3.000){2}{\rule{0.317pt}{0.400pt}}
\multiput(438.00,522.60)(0.481,0.468){5}{\rule{0.500pt}{0.113pt}}
\multiput(438.00,521.17)(2.962,4.000){2}{\rule{0.250pt}{0.400pt}}
\multiput(442.00,526.60)(0.481,0.468){5}{\rule{0.500pt}{0.113pt}}
\multiput(442.00,525.17)(2.962,4.000){2}{\rule{0.250pt}{0.400pt}}
\multiput(446.00,530.60)(0.481,0.468){5}{\rule{0.500pt}{0.113pt}}
\multiput(446.00,529.17)(2.962,4.000){2}{\rule{0.250pt}{0.400pt}}
\multiput(450.00,534.61)(0.685,0.447){3}{\rule{0.633pt}{0.108pt}}
\multiput(450.00,533.17)(2.685,3.000){2}{\rule{0.317pt}{0.400pt}}
\multiput(454.00,537.61)(0.685,0.447){3}{\rule{0.633pt}{0.108pt}}
\multiput(454.00,536.17)(2.685,3.000){2}{\rule{0.317pt}{0.400pt}}
\multiput(458.00,540.60)(0.481,0.468){5}{\rule{0.500pt}{0.113pt}}
\multiput(458.00,539.17)(2.962,4.000){2}{\rule{0.250pt}{0.400pt}}
\multiput(462.00,544.61)(0.685,0.447){3}{\rule{0.633pt}{0.108pt}}
\multiput(462.00,543.17)(2.685,3.000){2}{\rule{0.317pt}{0.400pt}}
\multiput(466.00,547.61)(0.685,0.447){3}{\rule{0.633pt}{0.108pt}}
\multiput(466.00,546.17)(2.685,3.000){2}{\rule{0.317pt}{0.400pt}}
\multiput(470.00,550.61)(0.685,0.447){3}{\rule{0.633pt}{0.108pt}}
\multiput(470.00,549.17)(2.685,3.000){2}{\rule{0.317pt}{0.400pt}}
\multiput(474.00,553.61)(0.685,0.447){3}{\rule{0.633pt}{0.108pt}}
\multiput(474.00,552.17)(2.685,3.000){2}{\rule{0.317pt}{0.400pt}}
\put(478,556.17){\rule{0.900pt}{0.400pt}}
\multiput(478.00,555.17)(2.132,2.000){2}{\rule{0.450pt}{0.400pt}}
\multiput(482.00,558.61)(0.685,0.447){3}{\rule{0.633pt}{0.108pt}}
\multiput(482.00,557.17)(2.685,3.000){2}{\rule{0.317pt}{0.400pt}}
\multiput(486.00,561.61)(0.685,0.447){3}{\rule{0.633pt}{0.108pt}}
\multiput(486.00,560.17)(2.685,3.000){2}{\rule{0.317pt}{0.400pt}}
\put(490,564.17){\rule{0.900pt}{0.400pt}}
\multiput(490.00,563.17)(2.132,2.000){2}{\rule{0.450pt}{0.400pt}}
\put(494,566.17){\rule{0.900pt}{0.400pt}}
\multiput(494.00,565.17)(2.132,2.000){2}{\rule{0.450pt}{0.400pt}}
\multiput(498.00,568.61)(0.685,0.447){3}{\rule{0.633pt}{0.108pt}}
\multiput(498.00,567.17)(2.685,3.000){2}{\rule{0.317pt}{0.400pt}}
\put(502,571.17){\rule{0.900pt}{0.400pt}}
\multiput(502.00,570.17)(2.132,2.000){2}{\rule{0.450pt}{0.400pt}}
\put(506,573.17){\rule{0.900pt}{0.400pt}}
\multiput(506.00,572.17)(2.132,2.000){2}{\rule{0.450pt}{0.400pt}}
\put(510,575.17){\rule{0.900pt}{0.400pt}}
\multiput(510.00,574.17)(2.132,2.000){2}{\rule{0.450pt}{0.400pt}}
\put(514,577.17){\rule{0.900pt}{0.400pt}}
\multiput(514.00,576.17)(2.132,2.000){2}{\rule{0.450pt}{0.400pt}}
\put(518,578.67){\rule{0.964pt}{0.400pt}}
\multiput(518.00,578.17)(2.000,1.000){2}{\rule{0.482pt}{0.400pt}}
\put(522,580.17){\rule{0.900pt}{0.400pt}}
\multiput(522.00,579.17)(2.132,2.000){2}{\rule{0.450pt}{0.400pt}}
\put(526,581.67){\rule{0.964pt}{0.400pt}}
\multiput(526.00,581.17)(2.000,1.000){2}{\rule{0.482pt}{0.400pt}}
\put(530,583.17){\rule{0.900pt}{0.400pt}}
\multiput(530.00,582.17)(2.132,2.000){2}{\rule{0.450pt}{0.400pt}}
\put(534,584.67){\rule{0.964pt}{0.400pt}}
\multiput(534.00,584.17)(2.000,1.000){2}{\rule{0.482pt}{0.400pt}}
\put(538,585.67){\rule{0.964pt}{0.400pt}}
\multiput(538.00,585.17)(2.000,1.000){2}{\rule{0.482pt}{0.400pt}}
\put(542,587.17){\rule{0.900pt}{0.400pt}}
\multiput(542.00,586.17)(2.132,2.000){2}{\rule{0.450pt}{0.400pt}}
\put(546,588.67){\rule{0.964pt}{0.400pt}}
\multiput(546.00,588.17)(2.000,1.000){2}{\rule{0.482pt}{0.400pt}}
\put(550,589.67){\rule{0.964pt}{0.400pt}}
\multiput(550.00,589.17)(2.000,1.000){2}{\rule{0.482pt}{0.400pt}}
\put(605.0,566.0){\rule[-0.200pt]{0.964pt}{0.400pt}}
\put(558,590.67){\rule{0.964pt}{0.400pt}}
\multiput(558.00,590.17)(2.000,1.000){2}{\rule{0.482pt}{0.400pt}}
\put(562,591.67){\rule{0.964pt}{0.400pt}}
\multiput(562.00,591.17)(2.000,1.000){2}{\rule{0.482pt}{0.400pt}}
\put(554.0,591.0){\rule[-0.200pt]{0.964pt}{0.400pt}}
\put(570,592.67){\rule{0.964pt}{0.400pt}}
\multiput(570.00,592.17)(2.000,1.000){2}{\rule{0.482pt}{0.400pt}}
\put(566.0,593.0){\rule[-0.200pt]{0.964pt}{0.400pt}}
\put(574.0,594.0){\rule[-0.200pt]{0.964pt}{0.400pt}}
\put(578.0,594.0){\rule[-0.200pt]{0.964pt}{0.400pt}}
\put(582.0,594.0){\rule[-0.200pt]{0.964pt}{0.400pt}}
\put(586.0,594.0){\rule[-0.200pt]{0.964pt}{0.400pt}}
\put(590.0,594.0){\rule[-0.200pt]{0.964pt}{0.400pt}}
\put(598,592.67){\rule{0.964pt}{0.400pt}}
\multiput(598.00,593.17)(2.000,-1.000){2}{\rule{0.482pt}{0.400pt}}
\put(594.0,594.0){\rule[-0.200pt]{0.964pt}{0.400pt}}
\put(606,591.67){\rule{0.964pt}{0.400pt}}
\multiput(606.00,592.17)(2.000,-1.000){2}{\rule{0.482pt}{0.400pt}}
\put(602.0,593.0){\rule[-0.200pt]{0.964pt}{0.400pt}}
\put(614,590.67){\rule{0.964pt}{0.400pt}}
\multiput(614.00,591.17)(2.000,-1.000){2}{\rule{0.482pt}{0.400pt}}
\put(618,589.67){\rule{0.964pt}{0.400pt}}
\multiput(618.00,590.17)(2.000,-1.000){2}{\rule{0.482pt}{0.400pt}}
\put(622,588.67){\rule{0.723pt}{0.400pt}}
\multiput(622.00,589.17)(1.500,-1.000){2}{\rule{0.361pt}{0.400pt}}
\put(625,587.67){\rule{0.964pt}{0.400pt}}
\multiput(625.00,588.17)(2.000,-1.000){2}{\rule{0.482pt}{0.400pt}}
\put(629,586.67){\rule{0.964pt}{0.400pt}}
\multiput(629.00,587.17)(2.000,-1.000){2}{\rule{0.482pt}{0.400pt}}
\put(633,585.17){\rule{0.900pt}{0.400pt}}
\multiput(633.00,586.17)(2.132,-2.000){2}{\rule{0.450pt}{0.400pt}}
\put(637,583.67){\rule{0.964pt}{0.400pt}}
\multiput(637.00,584.17)(2.000,-1.000){2}{\rule{0.482pt}{0.400pt}}
\put(641,582.17){\rule{0.900pt}{0.400pt}}
\multiput(641.00,583.17)(2.132,-2.000){2}{\rule{0.450pt}{0.400pt}}
\put(645,580.17){\rule{0.900pt}{0.400pt}}
\multiput(645.00,581.17)(2.132,-2.000){2}{\rule{0.450pt}{0.400pt}}
\put(649,578.67){\rule{0.964pt}{0.400pt}}
\multiput(649.00,579.17)(2.000,-1.000){2}{\rule{0.482pt}{0.400pt}}
\put(653,577.17){\rule{0.900pt}{0.400pt}}
\multiput(653.00,578.17)(2.132,-2.000){2}{\rule{0.450pt}{0.400pt}}
\put(657,575.17){\rule{0.900pt}{0.400pt}}
\multiput(657.00,576.17)(2.132,-2.000){2}{\rule{0.450pt}{0.400pt}}
\multiput(661.00,573.95)(0.685,-0.447){3}{\rule{0.633pt}{0.108pt}}
\multiput(661.00,574.17)(2.685,-3.000){2}{\rule{0.317pt}{0.400pt}}
\put(665,570.17){\rule{0.900pt}{0.400pt}}
\multiput(665.00,571.17)(2.132,-2.000){2}{\rule{0.450pt}{0.400pt}}
\put(669,568.17){\rule{0.900pt}{0.400pt}}
\multiput(669.00,569.17)(2.132,-2.000){2}{\rule{0.450pt}{0.400pt}}
\multiput(673.00,566.95)(0.685,-0.447){3}{\rule{0.633pt}{0.108pt}}
\multiput(673.00,567.17)(2.685,-3.000){2}{\rule{0.317pt}{0.400pt}}
\multiput(677.00,563.95)(0.685,-0.447){3}{\rule{0.633pt}{0.108pt}}
\multiput(677.00,564.17)(2.685,-3.000){2}{\rule{0.317pt}{0.400pt}}
\multiput(681.00,560.95)(0.685,-0.447){3}{\rule{0.633pt}{0.108pt}}
\multiput(681.00,561.17)(2.685,-3.000){2}{\rule{0.317pt}{0.400pt}}
\multiput(685.00,557.95)(0.685,-0.447){3}{\rule{0.633pt}{0.108pt}}
\multiput(685.00,558.17)(2.685,-3.000){2}{\rule{0.317pt}{0.400pt}}
\multiput(689.00,554.95)(0.685,-0.447){3}{\rule{0.633pt}{0.108pt}}
\multiput(689.00,555.17)(2.685,-3.000){2}{\rule{0.317pt}{0.400pt}}
\multiput(693.00,551.95)(0.685,-0.447){3}{\rule{0.633pt}{0.108pt}}
\multiput(693.00,552.17)(2.685,-3.000){2}{\rule{0.317pt}{0.400pt}}
\multiput(697.00,548.95)(0.685,-0.447){3}{\rule{0.633pt}{0.108pt}}
\multiput(697.00,549.17)(2.685,-3.000){2}{\rule{0.317pt}{0.400pt}}
\multiput(701.00,545.94)(0.481,-0.468){5}{\rule{0.500pt}{0.113pt}}
\multiput(701.00,546.17)(2.962,-4.000){2}{\rule{0.250pt}{0.400pt}}
\multiput(705.00,541.94)(0.481,-0.468){5}{\rule{0.500pt}{0.113pt}}
\multiput(705.00,542.17)(2.962,-4.000){2}{\rule{0.250pt}{0.400pt}}
\multiput(709.00,537.95)(0.685,-0.447){3}{\rule{0.633pt}{0.108pt}}
\multiput(709.00,538.17)(2.685,-3.000){2}{\rule{0.317pt}{0.400pt}}
\multiput(355.60,349.00)(0.468,1.066){5}{\rule{0.113pt}{0.900pt}}
\multiput(354.17,349.00)(4.000,6.132){2}{\rule{0.400pt}{0.450pt}}
\multiput(359.60,357.00)(0.468,1.212){5}{\rule{0.113pt}{1.000pt}}
\multiput(358.17,357.00)(4.000,6.924){2}{\rule{0.400pt}{0.500pt}}
\multiput(363.60,366.00)(0.468,1.212){5}{\rule{0.113pt}{1.000pt}}
\multiput(362.17,366.00)(4.000,6.924){2}{\rule{0.400pt}{0.500pt}}
\multiput(367.60,375.00)(0.468,1.066){5}{\rule{0.113pt}{0.900pt}}
\multiput(366.17,375.00)(4.000,6.132){2}{\rule{0.400pt}{0.450pt}}
\multiput(371.60,383.00)(0.468,1.066){5}{\rule{0.113pt}{0.900pt}}
\multiput(370.17,383.00)(4.000,6.132){2}{\rule{0.400pt}{0.450pt}}
\multiput(375.60,391.00)(0.468,1.066){5}{\rule{0.113pt}{0.900pt}}
\multiput(374.17,391.00)(4.000,6.132){2}{\rule{0.400pt}{0.450pt}}
\multiput(379.60,399.00)(0.468,1.066){5}{\rule{0.113pt}{0.900pt}}
\multiput(378.17,399.00)(4.000,6.132){2}{\rule{0.400pt}{0.450pt}}
\multiput(383.60,407.00)(0.468,0.920){5}{\rule{0.113pt}{0.800pt}}
\multiput(382.17,407.00)(4.000,5.340){2}{\rule{0.400pt}{0.400pt}}
\multiput(387.60,414.00)(0.468,1.066){5}{\rule{0.113pt}{0.900pt}}
\multiput(386.17,414.00)(4.000,6.132){2}{\rule{0.400pt}{0.450pt}}
\multiput(391.60,422.00)(0.468,0.920){5}{\rule{0.113pt}{0.800pt}}
\multiput(390.17,422.00)(4.000,5.340){2}{\rule{0.400pt}{0.400pt}}
\multiput(395.60,429.00)(0.468,0.920){5}{\rule{0.113pt}{0.800pt}}
\multiput(394.17,429.00)(4.000,5.340){2}{\rule{0.400pt}{0.400pt}}
\multiput(399.60,436.00)(0.468,0.920){5}{\rule{0.113pt}{0.800pt}}
\multiput(398.17,436.00)(4.000,5.340){2}{\rule{0.400pt}{0.400pt}}
\multiput(403.60,443.00)(0.468,0.920){5}{\rule{0.113pt}{0.800pt}}
\multiput(402.17,443.00)(4.000,5.340){2}{\rule{0.400pt}{0.400pt}}
\multiput(407.60,450.00)(0.468,0.920){5}{\rule{0.113pt}{0.800pt}}
\multiput(406.17,450.00)(4.000,5.340){2}{\rule{0.400pt}{0.400pt}}
\multiput(411.60,457.00)(0.468,0.774){5}{\rule{0.113pt}{0.700pt}}
\multiput(410.17,457.00)(4.000,4.547){2}{\rule{0.400pt}{0.350pt}}
\multiput(415.60,463.00)(0.468,0.920){5}{\rule{0.113pt}{0.800pt}}
\multiput(414.17,463.00)(4.000,5.340){2}{\rule{0.400pt}{0.400pt}}
\multiput(419.60,470.00)(0.468,0.774){5}{\rule{0.113pt}{0.700pt}}
\multiput(418.17,470.00)(4.000,4.547){2}{\rule{0.400pt}{0.350pt}}
\multiput(423.60,476.00)(0.468,0.774){5}{\rule{0.113pt}{0.700pt}}
\multiput(422.17,476.00)(4.000,4.547){2}{\rule{0.400pt}{0.350pt}}
\multiput(427.60,482.00)(0.468,0.774){5}{\rule{0.113pt}{0.700pt}}
\multiput(426.17,482.00)(4.000,4.547){2}{\rule{0.400pt}{0.350pt}}
\multiput(431.60,488.00)(0.468,0.774){5}{\rule{0.113pt}{0.700pt}}
\multiput(430.17,488.00)(4.000,4.547){2}{\rule{0.400pt}{0.350pt}}
\multiput(435.60,494.00)(0.468,0.627){5}{\rule{0.113pt}{0.600pt}}
\multiput(434.17,494.00)(4.000,3.755){2}{\rule{0.400pt}{0.300pt}}
\multiput(439.60,499.00)(0.468,0.774){5}{\rule{0.113pt}{0.700pt}}
\multiput(438.17,499.00)(4.000,4.547){2}{\rule{0.400pt}{0.350pt}}
\multiput(443.61,505.00)(0.447,0.909){3}{\rule{0.108pt}{0.767pt}}
\multiput(442.17,505.00)(3.000,3.409){2}{\rule{0.400pt}{0.383pt}}
\multiput(446.60,510.00)(0.468,0.774){5}{\rule{0.113pt}{0.700pt}}
\multiput(445.17,510.00)(4.000,4.547){2}{\rule{0.400pt}{0.350pt}}
\multiput(450.60,516.00)(0.468,0.627){5}{\rule{0.113pt}{0.600pt}}
\multiput(449.17,516.00)(4.000,3.755){2}{\rule{0.400pt}{0.300pt}}
\multiput(454.60,521.00)(0.468,0.627){5}{\rule{0.113pt}{0.600pt}}
\multiput(453.17,521.00)(4.000,3.755){2}{\rule{0.400pt}{0.300pt}}
\multiput(458.00,526.60)(0.481,0.468){5}{\rule{0.500pt}{0.113pt}}
\multiput(458.00,525.17)(2.962,4.000){2}{\rule{0.250pt}{0.400pt}}
\multiput(462.60,530.00)(0.468,0.627){5}{\rule{0.113pt}{0.600pt}}
\multiput(461.17,530.00)(4.000,3.755){2}{\rule{0.400pt}{0.300pt}}
\multiput(466.60,535.00)(0.468,0.627){5}{\rule{0.113pt}{0.600pt}}
\multiput(465.17,535.00)(4.000,3.755){2}{\rule{0.400pt}{0.300pt}}
\multiput(470.00,540.60)(0.481,0.468){5}{\rule{0.500pt}{0.113pt}}
\multiput(470.00,539.17)(2.962,4.000){2}{\rule{0.250pt}{0.400pt}}
\multiput(474.00,544.60)(0.481,0.468){5}{\rule{0.500pt}{0.113pt}}
\multiput(474.00,543.17)(2.962,4.000){2}{\rule{0.250pt}{0.400pt}}
\multiput(478.00,548.60)(0.481,0.468){5}{\rule{0.500pt}{0.113pt}}
\multiput(478.00,547.17)(2.962,4.000){2}{\rule{0.250pt}{0.400pt}}
\multiput(482.00,552.60)(0.481,0.468){5}{\rule{0.500pt}{0.113pt}}
\multiput(482.00,551.17)(2.962,4.000){2}{\rule{0.250pt}{0.400pt}}
\multiput(486.00,556.60)(0.481,0.468){5}{\rule{0.500pt}{0.113pt}}
\multiput(486.00,555.17)(2.962,4.000){2}{\rule{0.250pt}{0.400pt}}
\multiput(490.00,560.60)(0.481,0.468){5}{\rule{0.500pt}{0.113pt}}
\multiput(490.00,559.17)(2.962,4.000){2}{\rule{0.250pt}{0.400pt}}
\multiput(494.00,564.60)(0.481,0.468){5}{\rule{0.500pt}{0.113pt}}
\multiput(494.00,563.17)(2.962,4.000){2}{\rule{0.250pt}{0.400pt}}
\multiput(498.00,568.61)(0.685,0.447){3}{\rule{0.633pt}{0.108pt}}
\multiput(498.00,567.17)(2.685,3.000){2}{\rule{0.317pt}{0.400pt}}
\multiput(502.00,571.61)(0.685,0.447){3}{\rule{0.633pt}{0.108pt}}
\multiput(502.00,570.17)(2.685,3.000){2}{\rule{0.317pt}{0.400pt}}
\multiput(506.00,574.60)(0.481,0.468){5}{\rule{0.500pt}{0.113pt}}
\multiput(506.00,573.17)(2.962,4.000){2}{\rule{0.250pt}{0.400pt}}
\multiput(510.00,578.61)(0.685,0.447){3}{\rule{0.633pt}{0.108pt}}
\multiput(510.00,577.17)(2.685,3.000){2}{\rule{0.317pt}{0.400pt}}
\multiput(514.00,581.61)(0.685,0.447){3}{\rule{0.633pt}{0.108pt}}
\multiput(514.00,580.17)(2.685,3.000){2}{\rule{0.317pt}{0.400pt}}
\multiput(518.00,584.61)(0.685,0.447){3}{\rule{0.633pt}{0.108pt}}
\multiput(518.00,583.17)(2.685,3.000){2}{\rule{0.317pt}{0.400pt}}
\put(522,587.17){\rule{0.900pt}{0.400pt}}
\multiput(522.00,586.17)(2.132,2.000){2}{\rule{0.450pt}{0.400pt}}
\multiput(526.00,589.61)(0.685,0.447){3}{\rule{0.633pt}{0.108pt}}
\multiput(526.00,588.17)(2.685,3.000){2}{\rule{0.317pt}{0.400pt}}
\put(530,592.17){\rule{0.900pt}{0.400pt}}
\multiput(530.00,591.17)(2.132,2.000){2}{\rule{0.450pt}{0.400pt}}
\multiput(534.00,594.61)(0.685,0.447){3}{\rule{0.633pt}{0.108pt}}
\multiput(534.00,593.17)(2.685,3.000){2}{\rule{0.317pt}{0.400pt}}
\put(538,597.17){\rule{0.900pt}{0.400pt}}
\multiput(538.00,596.17)(2.132,2.000){2}{\rule{0.450pt}{0.400pt}}
\put(542,599.17){\rule{0.900pt}{0.400pt}}
\multiput(542.00,598.17)(2.132,2.000){2}{\rule{0.450pt}{0.400pt}}
\put(546,601.17){\rule{0.900pt}{0.400pt}}
\multiput(546.00,600.17)(2.132,2.000){2}{\rule{0.450pt}{0.400pt}}
\put(550,603.17){\rule{0.900pt}{0.400pt}}
\multiput(550.00,602.17)(2.132,2.000){2}{\rule{0.450pt}{0.400pt}}
\put(554,604.67){\rule{0.964pt}{0.400pt}}
\multiput(554.00,604.17)(2.000,1.000){2}{\rule{0.482pt}{0.400pt}}
\put(558,606.17){\rule{0.900pt}{0.400pt}}
\multiput(558.00,605.17)(2.132,2.000){2}{\rule{0.450pt}{0.400pt}}
\put(562,608.17){\rule{0.900pt}{0.400pt}}
\multiput(562.00,607.17)(2.132,2.000){2}{\rule{0.450pt}{0.400pt}}
\put(566,609.67){\rule{0.964pt}{0.400pt}}
\multiput(566.00,609.17)(2.000,1.000){2}{\rule{0.482pt}{0.400pt}}
\put(570,610.67){\rule{0.964pt}{0.400pt}}
\multiput(570.00,610.17)(2.000,1.000){2}{\rule{0.482pt}{0.400pt}}
\put(574,611.67){\rule{0.964pt}{0.400pt}}
\multiput(574.00,611.17)(2.000,1.000){2}{\rule{0.482pt}{0.400pt}}
\put(578,612.67){\rule{0.964pt}{0.400pt}}
\multiput(578.00,612.17)(2.000,1.000){2}{\rule{0.482pt}{0.400pt}}
\put(582,613.67){\rule{0.964pt}{0.400pt}}
\multiput(582.00,613.17)(2.000,1.000){2}{\rule{0.482pt}{0.400pt}}
\put(586,614.67){\rule{0.964pt}{0.400pt}}
\multiput(586.00,614.17)(2.000,1.000){2}{\rule{0.482pt}{0.400pt}}
\put(610.0,592.0){\rule[-0.200pt]{0.964pt}{0.400pt}}
\put(594,615.67){\rule{0.964pt}{0.400pt}}
\multiput(594.00,615.17)(2.000,1.000){2}{\rule{0.482pt}{0.400pt}}
\put(590.0,616.0){\rule[-0.200pt]{0.964pt}{0.400pt}}
\put(598.0,617.0){\rule[-0.200pt]{0.964pt}{0.400pt}}
\put(602.0,617.0){\rule[-0.200pt]{0.964pt}{0.400pt}}
\put(606.0,617.0){\rule[-0.200pt]{0.964pt}{0.400pt}}
\put(610.0,617.0){\rule[-0.200pt]{0.964pt}{0.400pt}}
\put(614.0,617.0){\rule[-0.200pt]{0.964pt}{0.400pt}}
\put(622,615.67){\rule{0.964pt}{0.400pt}}
\multiput(622.00,616.17)(2.000,-1.000){2}{\rule{0.482pt}{0.400pt}}
\put(626,614.67){\rule{0.964pt}{0.400pt}}
\multiput(626.00,615.17)(2.000,-1.000){2}{\rule{0.482pt}{0.400pt}}
\put(618.0,617.0){\rule[-0.200pt]{0.964pt}{0.400pt}}
\put(634,613.67){\rule{0.964pt}{0.400pt}}
\multiput(634.00,614.17)(2.000,-1.000){2}{\rule{0.482pt}{0.400pt}}
\put(638,612.67){\rule{0.964pt}{0.400pt}}
\multiput(638.00,613.17)(2.000,-1.000){2}{\rule{0.482pt}{0.400pt}}
\put(642,611.67){\rule{0.723pt}{0.400pt}}
\multiput(642.00,612.17)(1.500,-1.000){2}{\rule{0.361pt}{0.400pt}}
\put(645,610.17){\rule{0.900pt}{0.400pt}}
\multiput(645.00,611.17)(2.132,-2.000){2}{\rule{0.450pt}{0.400pt}}
\put(649,608.67){\rule{0.964pt}{0.400pt}}
\multiput(649.00,609.17)(2.000,-1.000){2}{\rule{0.482pt}{0.400pt}}
\put(653,607.17){\rule{0.900pt}{0.400pt}}
\multiput(653.00,608.17)(2.132,-2.000){2}{\rule{0.450pt}{0.400pt}}
\put(657,605.67){\rule{0.964pt}{0.400pt}}
\multiput(657.00,606.17)(2.000,-1.000){2}{\rule{0.482pt}{0.400pt}}
\put(661,604.17){\rule{0.900pt}{0.400pt}}
\multiput(661.00,605.17)(2.132,-2.000){2}{\rule{0.450pt}{0.400pt}}
\put(665,602.17){\rule{0.900pt}{0.400pt}}
\multiput(665.00,603.17)(2.132,-2.000){2}{\rule{0.450pt}{0.400pt}}
\put(669,600.17){\rule{0.900pt}{0.400pt}}
\multiput(669.00,601.17)(2.132,-2.000){2}{\rule{0.450pt}{0.400pt}}
\put(673,598.17){\rule{0.900pt}{0.400pt}}
\multiput(673.00,599.17)(2.132,-2.000){2}{\rule{0.450pt}{0.400pt}}
\multiput(677.00,596.95)(0.685,-0.447){3}{\rule{0.633pt}{0.108pt}}
\multiput(677.00,597.17)(2.685,-3.000){2}{\rule{0.317pt}{0.400pt}}
\put(681,593.17){\rule{0.900pt}{0.400pt}}
\multiput(681.00,594.17)(2.132,-2.000){2}{\rule{0.450pt}{0.400pt}}
\multiput(685.00,591.95)(0.685,-0.447){3}{\rule{0.633pt}{0.108pt}}
\multiput(685.00,592.17)(2.685,-3.000){2}{\rule{0.317pt}{0.400pt}}
\multiput(689.00,588.95)(0.685,-0.447){3}{\rule{0.633pt}{0.108pt}}
\multiput(689.00,589.17)(2.685,-3.000){2}{\rule{0.317pt}{0.400pt}}
\multiput(693.00,585.95)(0.685,-0.447){3}{\rule{0.633pt}{0.108pt}}
\multiput(693.00,586.17)(2.685,-3.000){2}{\rule{0.317pt}{0.400pt}}
\multiput(697.00,582.95)(0.685,-0.447){3}{\rule{0.633pt}{0.108pt}}
\multiput(697.00,583.17)(2.685,-3.000){2}{\rule{0.317pt}{0.400pt}}
\multiput(701.00,579.95)(0.685,-0.447){3}{\rule{0.633pt}{0.108pt}}
\multiput(701.00,580.17)(2.685,-3.000){2}{\rule{0.317pt}{0.400pt}}
\multiput(705.00,576.94)(0.481,-0.468){5}{\rule{0.500pt}{0.113pt}}
\multiput(705.00,577.17)(2.962,-4.000){2}{\rule{0.250pt}{0.400pt}}
\multiput(709.00,572.94)(0.481,-0.468){5}{\rule{0.500pt}{0.113pt}}
\multiput(709.00,573.17)(2.962,-4.000){2}{\rule{0.250pt}{0.400pt}}
\multiput(713.00,568.95)(0.685,-0.447){3}{\rule{0.633pt}{0.108pt}}
\multiput(713.00,569.17)(2.685,-3.000){2}{\rule{0.317pt}{0.400pt}}
\multiput(717.00,565.94)(0.481,-0.468){5}{\rule{0.500pt}{0.113pt}}
\multiput(717.00,566.17)(2.962,-4.000){2}{\rule{0.250pt}{0.400pt}}
\multiput(721.00,561.94)(0.481,-0.468){5}{\rule{0.500pt}{0.113pt}}
\multiput(721.00,562.17)(2.962,-4.000){2}{\rule{0.250pt}{0.400pt}}
\multiput(725.60,556.51)(0.468,-0.627){5}{\rule{0.113pt}{0.600pt}}
\multiput(724.17,557.75)(4.000,-3.755){2}{\rule{0.400pt}{0.300pt}}
\multiput(729.00,552.94)(0.481,-0.468){5}{\rule{0.500pt}{0.113pt}}
\multiput(729.00,553.17)(2.962,-4.000){2}{\rule{0.250pt}{0.400pt}}
\multiput(733.60,547.51)(0.468,-0.627){5}{\rule{0.113pt}{0.600pt}}
\multiput(732.17,548.75)(4.000,-3.755){2}{\rule{0.400pt}{0.300pt}}
\multiput(737.60,542.51)(0.468,-0.627){5}{\rule{0.113pt}{0.600pt}}
\multiput(736.17,543.75)(4.000,-3.755){2}{\rule{0.400pt}{0.300pt}}
\multiput(741.60,537.51)(0.468,-0.627){5}{\rule{0.113pt}{0.600pt}}
\multiput(740.17,538.75)(4.000,-3.755){2}{\rule{0.400pt}{0.300pt}}
\multiput(745.60,532.51)(0.468,-0.627){5}{\rule{0.113pt}{0.600pt}}
\multiput(744.17,533.75)(4.000,-3.755){2}{\rule{0.400pt}{0.300pt}}
\multiput(391.60,343.00)(0.468,1.358){5}{\rule{0.113pt}{1.100pt}}
\multiput(390.17,343.00)(4.000,7.717){2}{\rule{0.400pt}{0.550pt}}
\multiput(395.60,353.00)(0.468,1.358){5}{\rule{0.113pt}{1.100pt}}
\multiput(394.17,353.00)(4.000,7.717){2}{\rule{0.400pt}{0.550pt}}
\multiput(399.60,363.00)(0.468,1.212){5}{\rule{0.113pt}{1.000pt}}
\multiput(398.17,363.00)(4.000,6.924){2}{\rule{0.400pt}{0.500pt}}
\multiput(403.60,372.00)(0.468,1.358){5}{\rule{0.113pt}{1.100pt}}
\multiput(402.17,372.00)(4.000,7.717){2}{\rule{0.400pt}{0.550pt}}
\multiput(407.60,382.00)(0.468,1.212){5}{\rule{0.113pt}{1.000pt}}
\multiput(406.17,382.00)(4.000,6.924){2}{\rule{0.400pt}{0.500pt}}
\multiput(411.60,391.00)(0.468,1.212){5}{\rule{0.113pt}{1.000pt}}
\multiput(410.17,391.00)(4.000,6.924){2}{\rule{0.400pt}{0.500pt}}
\multiput(415.60,400.00)(0.468,1.212){5}{\rule{0.113pt}{1.000pt}}
\multiput(414.17,400.00)(4.000,6.924){2}{\rule{0.400pt}{0.500pt}}
\multiput(419.60,409.00)(0.468,1.066){5}{\rule{0.113pt}{0.900pt}}
\multiput(418.17,409.00)(4.000,6.132){2}{\rule{0.400pt}{0.450pt}}
\multiput(423.60,417.00)(0.468,1.066){5}{\rule{0.113pt}{0.900pt}}
\multiput(422.17,417.00)(4.000,6.132){2}{\rule{0.400pt}{0.450pt}}
\multiput(427.60,425.00)(0.468,1.212){5}{\rule{0.113pt}{1.000pt}}
\multiput(426.17,425.00)(4.000,6.924){2}{\rule{0.400pt}{0.500pt}}
\multiput(431.60,434.00)(0.468,1.066){5}{\rule{0.113pt}{0.900pt}}
\multiput(430.17,434.00)(4.000,6.132){2}{\rule{0.400pt}{0.450pt}}
\multiput(435.60,442.00)(0.468,0.920){5}{\rule{0.113pt}{0.800pt}}
\multiput(434.17,442.00)(4.000,5.340){2}{\rule{0.400pt}{0.400pt}}
\multiput(439.60,449.00)(0.468,1.066){5}{\rule{0.113pt}{0.900pt}}
\multiput(438.17,449.00)(4.000,6.132){2}{\rule{0.400pt}{0.450pt}}
\multiput(443.60,457.00)(0.468,0.920){5}{\rule{0.113pt}{0.800pt}}
\multiput(442.17,457.00)(4.000,5.340){2}{\rule{0.400pt}{0.400pt}}
\multiput(447.60,464.00)(0.468,1.066){5}{\rule{0.113pt}{0.900pt}}
\multiput(446.17,464.00)(4.000,6.132){2}{\rule{0.400pt}{0.450pt}}
\multiput(451.60,472.00)(0.468,0.920){5}{\rule{0.113pt}{0.800pt}}
\multiput(450.17,472.00)(4.000,5.340){2}{\rule{0.400pt}{0.400pt}}
\multiput(455.60,479.00)(0.468,0.920){5}{\rule{0.113pt}{0.800pt}}
\multiput(454.17,479.00)(4.000,5.340){2}{\rule{0.400pt}{0.400pt}}
\multiput(459.60,486.00)(0.468,0.920){5}{\rule{0.113pt}{0.800pt}}
\multiput(458.17,486.00)(4.000,5.340){2}{\rule{0.400pt}{0.400pt}}
\multiput(463.61,493.00)(0.447,1.132){3}{\rule{0.108pt}{0.900pt}}
\multiput(462.17,493.00)(3.000,4.132){2}{\rule{0.400pt}{0.450pt}}
\multiput(466.60,499.00)(0.468,0.920){5}{\rule{0.113pt}{0.800pt}}
\multiput(465.17,499.00)(4.000,5.340){2}{\rule{0.400pt}{0.400pt}}
\multiput(470.60,506.00)(0.468,0.774){5}{\rule{0.113pt}{0.700pt}}
\multiput(469.17,506.00)(4.000,4.547){2}{\rule{0.400pt}{0.350pt}}
\multiput(474.60,512.00)(0.468,0.774){5}{\rule{0.113pt}{0.700pt}}
\multiput(473.17,512.00)(4.000,4.547){2}{\rule{0.400pt}{0.350pt}}
\multiput(478.60,518.00)(0.468,0.774){5}{\rule{0.113pt}{0.700pt}}
\multiput(477.17,518.00)(4.000,4.547){2}{\rule{0.400pt}{0.350pt}}
\multiput(482.60,524.00)(0.468,0.627){5}{\rule{0.113pt}{0.600pt}}
\multiput(481.17,524.00)(4.000,3.755){2}{\rule{0.400pt}{0.300pt}}
\multiput(486.60,529.00)(0.468,0.774){5}{\rule{0.113pt}{0.700pt}}
\multiput(485.17,529.00)(4.000,4.547){2}{\rule{0.400pt}{0.350pt}}
\multiput(490.60,535.00)(0.468,0.774){5}{\rule{0.113pt}{0.700pt}}
\multiput(489.17,535.00)(4.000,4.547){2}{\rule{0.400pt}{0.350pt}}
\multiput(494.60,541.00)(0.468,0.627){5}{\rule{0.113pt}{0.600pt}}
\multiput(493.17,541.00)(4.000,3.755){2}{\rule{0.400pt}{0.300pt}}
\multiput(498.60,546.00)(0.468,0.627){5}{\rule{0.113pt}{0.600pt}}
\multiput(497.17,546.00)(4.000,3.755){2}{\rule{0.400pt}{0.300pt}}
\multiput(502.60,551.00)(0.468,0.627){5}{\rule{0.113pt}{0.600pt}}
\multiput(501.17,551.00)(4.000,3.755){2}{\rule{0.400pt}{0.300pt}}
\multiput(506.60,556.00)(0.468,0.627){5}{\rule{0.113pt}{0.600pt}}
\multiput(505.17,556.00)(4.000,3.755){2}{\rule{0.400pt}{0.300pt}}
\multiput(510.00,561.60)(0.481,0.468){5}{\rule{0.500pt}{0.113pt}}
\multiput(510.00,560.17)(2.962,4.000){2}{\rule{0.250pt}{0.400pt}}
\multiput(514.60,565.00)(0.468,0.627){5}{\rule{0.113pt}{0.600pt}}
\multiput(513.17,565.00)(4.000,3.755){2}{\rule{0.400pt}{0.300pt}}
\multiput(518.00,570.60)(0.481,0.468){5}{\rule{0.500pt}{0.113pt}}
\multiput(518.00,569.17)(2.962,4.000){2}{\rule{0.250pt}{0.400pt}}
\multiput(522.00,574.60)(0.481,0.468){5}{\rule{0.500pt}{0.113pt}}
\multiput(522.00,573.17)(2.962,4.000){2}{\rule{0.250pt}{0.400pt}}
\multiput(526.00,578.60)(0.481,0.468){5}{\rule{0.500pt}{0.113pt}}
\multiput(526.00,577.17)(2.962,4.000){2}{\rule{0.250pt}{0.400pt}}
\multiput(530.00,582.60)(0.481,0.468){5}{\rule{0.500pt}{0.113pt}}
\multiput(530.00,581.17)(2.962,4.000){2}{\rule{0.250pt}{0.400pt}}
\multiput(534.00,586.60)(0.481,0.468){5}{\rule{0.500pt}{0.113pt}}
\multiput(534.00,585.17)(2.962,4.000){2}{\rule{0.250pt}{0.400pt}}
\multiput(538.00,590.60)(0.481,0.468){5}{\rule{0.500pt}{0.113pt}}
\multiput(538.00,589.17)(2.962,4.000){2}{\rule{0.250pt}{0.400pt}}
\multiput(542.00,594.61)(0.685,0.447){3}{\rule{0.633pt}{0.108pt}}
\multiput(542.00,593.17)(2.685,3.000){2}{\rule{0.317pt}{0.400pt}}
\multiput(546.00,597.61)(0.685,0.447){3}{\rule{0.633pt}{0.108pt}}
\multiput(546.00,596.17)(2.685,3.000){2}{\rule{0.317pt}{0.400pt}}
\multiput(550.00,600.61)(0.685,0.447){3}{\rule{0.633pt}{0.108pt}}
\multiput(550.00,599.17)(2.685,3.000){2}{\rule{0.317pt}{0.400pt}}
\multiput(554.00,603.61)(0.685,0.447){3}{\rule{0.633pt}{0.108pt}}
\multiput(554.00,602.17)(2.685,3.000){2}{\rule{0.317pt}{0.400pt}}
\multiput(558.00,606.61)(0.685,0.447){3}{\rule{0.633pt}{0.108pt}}
\multiput(558.00,605.17)(2.685,3.000){2}{\rule{0.317pt}{0.400pt}}
\multiput(562.00,609.61)(0.685,0.447){3}{\rule{0.633pt}{0.108pt}}
\multiput(562.00,608.17)(2.685,3.000){2}{\rule{0.317pt}{0.400pt}}
\multiput(566.00,612.61)(0.685,0.447){3}{\rule{0.633pt}{0.108pt}}
\multiput(566.00,611.17)(2.685,3.000){2}{\rule{0.317pt}{0.400pt}}
\put(570,615.17){\rule{0.900pt}{0.400pt}}
\multiput(570.00,614.17)(2.132,2.000){2}{\rule{0.450pt}{0.400pt}}
\put(574,617.17){\rule{0.900pt}{0.400pt}}
\multiput(574.00,616.17)(2.132,2.000){2}{\rule{0.450pt}{0.400pt}}
\put(578,619.17){\rule{0.900pt}{0.400pt}}
\multiput(578.00,618.17)(2.132,2.000){2}{\rule{0.450pt}{0.400pt}}
\put(582,621.17){\rule{0.900pt}{0.400pt}}
\multiput(582.00,620.17)(2.132,2.000){2}{\rule{0.450pt}{0.400pt}}
\put(586,623.17){\rule{0.900pt}{0.400pt}}
\multiput(586.00,622.17)(2.132,2.000){2}{\rule{0.450pt}{0.400pt}}
\put(590,625.17){\rule{0.900pt}{0.400pt}}
\multiput(590.00,624.17)(2.132,2.000){2}{\rule{0.450pt}{0.400pt}}
\put(594,626.67){\rule{0.964pt}{0.400pt}}
\multiput(594.00,626.17)(2.000,1.000){2}{\rule{0.482pt}{0.400pt}}
\put(598,628.17){\rule{0.900pt}{0.400pt}}
\multiput(598.00,627.17)(2.132,2.000){2}{\rule{0.450pt}{0.400pt}}
\put(602,629.67){\rule{0.964pt}{0.400pt}}
\multiput(602.00,629.17)(2.000,1.000){2}{\rule{0.482pt}{0.400pt}}
\put(606,630.67){\rule{0.964pt}{0.400pt}}
\multiput(606.00,630.17)(2.000,1.000){2}{\rule{0.482pt}{0.400pt}}
\put(610,631.67){\rule{0.964pt}{0.400pt}}
\multiput(610.00,631.17)(2.000,1.000){2}{\rule{0.482pt}{0.400pt}}
\put(614,632.67){\rule{0.964pt}{0.400pt}}
\multiput(614.00,632.17)(2.000,1.000){2}{\rule{0.482pt}{0.400pt}}
\put(618,633.67){\rule{0.964pt}{0.400pt}}
\multiput(618.00,633.17)(2.000,1.000){2}{\rule{0.482pt}{0.400pt}}
\put(630.0,615.0){\rule[-0.200pt]{0.964pt}{0.400pt}}
\put(626,634.67){\rule{0.964pt}{0.400pt}}
\multiput(626.00,634.17)(2.000,1.000){2}{\rule{0.482pt}{0.400pt}}
\put(622.0,635.0){\rule[-0.200pt]{0.964pt}{0.400pt}}
\put(630.0,636.0){\rule[-0.200pt]{0.964pt}{0.400pt}}
\put(634.0,636.0){\rule[-0.200pt]{0.964pt}{0.400pt}}
\put(638.0,636.0){\rule[-0.200pt]{0.964pt}{0.400pt}}
\put(646,634.67){\rule{0.964pt}{0.400pt}}
\multiput(646.00,635.17)(2.000,-1.000){2}{\rule{0.482pt}{0.400pt}}
\put(642.0,636.0){\rule[-0.200pt]{0.964pt}{0.400pt}}
\put(654,633.67){\rule{0.964pt}{0.400pt}}
\multiput(654.00,634.17)(2.000,-1.000){2}{\rule{0.482pt}{0.400pt}}
\put(658,632.67){\rule{0.723pt}{0.400pt}}
\multiput(658.00,633.17)(1.500,-1.000){2}{\rule{0.361pt}{0.400pt}}
\put(661,631.67){\rule{0.964pt}{0.400pt}}
\multiput(661.00,632.17)(2.000,-1.000){2}{\rule{0.482pt}{0.400pt}}
\put(665,630.67){\rule{0.964pt}{0.400pt}}
\multiput(665.00,631.17)(2.000,-1.000){2}{\rule{0.482pt}{0.400pt}}
\put(669,629.67){\rule{0.964pt}{0.400pt}}
\multiput(669.00,630.17)(2.000,-1.000){2}{\rule{0.482pt}{0.400pt}}
\put(673,628.17){\rule{0.900pt}{0.400pt}}
\multiput(673.00,629.17)(2.132,-2.000){2}{\rule{0.450pt}{0.400pt}}
\put(677,626.67){\rule{0.964pt}{0.400pt}}
\multiput(677.00,627.17)(2.000,-1.000){2}{\rule{0.482pt}{0.400pt}}
\put(681,625.17){\rule{0.900pt}{0.400pt}}
\multiput(681.00,626.17)(2.132,-2.000){2}{\rule{0.450pt}{0.400pt}}
\put(685,623.17){\rule{0.900pt}{0.400pt}}
\multiput(685.00,624.17)(2.132,-2.000){2}{\rule{0.450pt}{0.400pt}}
\put(689,621.17){\rule{0.900pt}{0.400pt}}
\multiput(689.00,622.17)(2.132,-2.000){2}{\rule{0.450pt}{0.400pt}}
\put(693,619.17){\rule{0.900pt}{0.400pt}}
\multiput(693.00,620.17)(2.132,-2.000){2}{\rule{0.450pt}{0.400pt}}
\multiput(697.00,617.95)(0.685,-0.447){3}{\rule{0.633pt}{0.108pt}}
\multiput(697.00,618.17)(2.685,-3.000){2}{\rule{0.317pt}{0.400pt}}
\put(701,614.17){\rule{0.900pt}{0.400pt}}
\multiput(701.00,615.17)(2.132,-2.000){2}{\rule{0.450pt}{0.400pt}}
\multiput(705.00,612.95)(0.685,-0.447){3}{\rule{0.633pt}{0.108pt}}
\multiput(705.00,613.17)(2.685,-3.000){2}{\rule{0.317pt}{0.400pt}}
\multiput(709.00,609.95)(0.685,-0.447){3}{\rule{0.633pt}{0.108pt}}
\multiput(709.00,610.17)(2.685,-3.000){2}{\rule{0.317pt}{0.400pt}}
\multiput(713.00,606.95)(0.685,-0.447){3}{\rule{0.633pt}{0.108pt}}
\multiput(713.00,607.17)(2.685,-3.000){2}{\rule{0.317pt}{0.400pt}}
\multiput(717.00,603.95)(0.685,-0.447){3}{\rule{0.633pt}{0.108pt}}
\multiput(717.00,604.17)(2.685,-3.000){2}{\rule{0.317pt}{0.400pt}}
\multiput(721.00,600.94)(0.481,-0.468){5}{\rule{0.500pt}{0.113pt}}
\multiput(721.00,601.17)(2.962,-4.000){2}{\rule{0.250pt}{0.400pt}}
\multiput(725.00,596.95)(0.685,-0.447){3}{\rule{0.633pt}{0.108pt}}
\multiput(725.00,597.17)(2.685,-3.000){2}{\rule{0.317pt}{0.400pt}}
\multiput(729.00,593.94)(0.481,-0.468){5}{\rule{0.500pt}{0.113pt}}
\multiput(729.00,594.17)(2.962,-4.000){2}{\rule{0.250pt}{0.400pt}}
\multiput(733.00,589.94)(0.481,-0.468){5}{\rule{0.500pt}{0.113pt}}
\multiput(733.00,590.17)(2.962,-4.000){2}{\rule{0.250pt}{0.400pt}}
\multiput(737.00,585.94)(0.481,-0.468){5}{\rule{0.500pt}{0.113pt}}
\multiput(737.00,586.17)(2.962,-4.000){2}{\rule{0.250pt}{0.400pt}}
\multiput(741.60,580.51)(0.468,-0.627){5}{\rule{0.113pt}{0.600pt}}
\multiput(740.17,581.75)(4.000,-3.755){2}{\rule{0.400pt}{0.300pt}}
\multiput(745.00,576.94)(0.481,-0.468){5}{\rule{0.500pt}{0.113pt}}
\multiput(745.00,577.17)(2.962,-4.000){2}{\rule{0.250pt}{0.400pt}}
\multiput(749.60,571.51)(0.468,-0.627){5}{\rule{0.113pt}{0.600pt}}
\multiput(748.17,572.75)(4.000,-3.755){2}{\rule{0.400pt}{0.300pt}}
\multiput(753.60,566.51)(0.468,-0.627){5}{\rule{0.113pt}{0.600pt}}
\multiput(752.17,567.75)(4.000,-3.755){2}{\rule{0.400pt}{0.300pt}}
\multiput(757.60,561.51)(0.468,-0.627){5}{\rule{0.113pt}{0.600pt}}
\multiput(756.17,562.75)(4.000,-3.755){2}{\rule{0.400pt}{0.300pt}}
\multiput(761.60,556.51)(0.468,-0.627){5}{\rule{0.113pt}{0.600pt}}
\multiput(760.17,557.75)(4.000,-3.755){2}{\rule{0.400pt}{0.300pt}}
\multiput(765.60,551.09)(0.468,-0.774){5}{\rule{0.113pt}{0.700pt}}
\multiput(764.17,552.55)(4.000,-4.547){2}{\rule{0.400pt}{0.350pt}}
\multiput(769.60,545.51)(0.468,-0.627){5}{\rule{0.113pt}{0.600pt}}
\multiput(768.17,546.75)(4.000,-3.755){2}{\rule{0.400pt}{0.300pt}}
\multiput(773.60,540.09)(0.468,-0.774){5}{\rule{0.113pt}{0.700pt}}
\multiput(772.17,541.55)(4.000,-4.547){2}{\rule{0.400pt}{0.350pt}}
\multiput(777.60,534.09)(0.468,-0.774){5}{\rule{0.113pt}{0.700pt}}
\multiput(776.17,535.55)(4.000,-4.547){2}{\rule{0.400pt}{0.350pt}}
\multiput(781.60,528.09)(0.468,-0.774){5}{\rule{0.113pt}{0.700pt}}
\multiput(780.17,529.55)(4.000,-4.547){2}{\rule{0.400pt}{0.350pt}}
\multiput(427.60,338.00)(0.468,1.358){5}{\rule{0.113pt}{1.100pt}}
\multiput(426.17,338.00)(4.000,7.717){2}{\rule{0.400pt}{0.550pt}}
\multiput(431.60,348.00)(0.468,1.505){5}{\rule{0.113pt}{1.200pt}}
\multiput(430.17,348.00)(4.000,8.509){2}{\rule{0.400pt}{0.600pt}}
\multiput(435.60,359.00)(0.468,1.358){5}{\rule{0.113pt}{1.100pt}}
\multiput(434.17,359.00)(4.000,7.717){2}{\rule{0.400pt}{0.550pt}}
\multiput(439.60,369.00)(0.468,1.505){5}{\rule{0.113pt}{1.200pt}}
\multiput(438.17,369.00)(4.000,8.509){2}{\rule{0.400pt}{0.600pt}}
\multiput(443.60,380.00)(0.468,1.358){5}{\rule{0.113pt}{1.100pt}}
\multiput(442.17,380.00)(4.000,7.717){2}{\rule{0.400pt}{0.550pt}}
\multiput(447.60,390.00)(0.468,1.212){5}{\rule{0.113pt}{1.000pt}}
\multiput(446.17,390.00)(4.000,6.924){2}{\rule{0.400pt}{0.500pt}}
\multiput(451.60,399.00)(0.468,1.358){5}{\rule{0.113pt}{1.100pt}}
\multiput(450.17,399.00)(4.000,7.717){2}{\rule{0.400pt}{0.550pt}}
\multiput(455.60,409.00)(0.468,1.212){5}{\rule{0.113pt}{1.000pt}}
\multiput(454.17,409.00)(4.000,6.924){2}{\rule{0.400pt}{0.500pt}}
\multiput(459.60,418.00)(0.468,1.212){5}{\rule{0.113pt}{1.000pt}}
\multiput(458.17,418.00)(4.000,6.924){2}{\rule{0.400pt}{0.500pt}}
\multiput(463.60,427.00)(0.468,1.212){5}{\rule{0.113pt}{1.000pt}}
\multiput(462.17,427.00)(4.000,6.924){2}{\rule{0.400pt}{0.500pt}}
\multiput(467.60,436.00)(0.468,1.212){5}{\rule{0.113pt}{1.000pt}}
\multiput(466.17,436.00)(4.000,6.924){2}{\rule{0.400pt}{0.500pt}}
\multiput(471.60,445.00)(0.468,1.066){5}{\rule{0.113pt}{0.900pt}}
\multiput(470.17,445.00)(4.000,6.132){2}{\rule{0.400pt}{0.450pt}}
\multiput(475.60,453.00)(0.468,1.066){5}{\rule{0.113pt}{0.900pt}}
\multiput(474.17,453.00)(4.000,6.132){2}{\rule{0.400pt}{0.450pt}}
\multiput(479.60,461.00)(0.468,1.066){5}{\rule{0.113pt}{0.900pt}}
\multiput(478.17,461.00)(4.000,6.132){2}{\rule{0.400pt}{0.450pt}}
\multiput(483.61,469.00)(0.447,1.579){3}{\rule{0.108pt}{1.167pt}}
\multiput(482.17,469.00)(3.000,5.579){2}{\rule{0.400pt}{0.583pt}}
\multiput(486.60,477.00)(0.468,1.066){5}{\rule{0.113pt}{0.900pt}}
\multiput(485.17,477.00)(4.000,6.132){2}{\rule{0.400pt}{0.450pt}}
\multiput(490.60,485.00)(0.468,0.920){5}{\rule{0.113pt}{0.800pt}}
\multiput(489.17,485.00)(4.000,5.340){2}{\rule{0.400pt}{0.400pt}}
\multiput(494.60,492.00)(0.468,1.066){5}{\rule{0.113pt}{0.900pt}}
\multiput(493.17,492.00)(4.000,6.132){2}{\rule{0.400pt}{0.450pt}}
\multiput(498.60,500.00)(0.468,0.920){5}{\rule{0.113pt}{0.800pt}}
\multiput(497.17,500.00)(4.000,5.340){2}{\rule{0.400pt}{0.400pt}}
\multiput(502.60,507.00)(0.468,0.920){5}{\rule{0.113pt}{0.800pt}}
\multiput(501.17,507.00)(4.000,5.340){2}{\rule{0.400pt}{0.400pt}}
\multiput(506.60,514.00)(0.468,0.774){5}{\rule{0.113pt}{0.700pt}}
\multiput(505.17,514.00)(4.000,4.547){2}{\rule{0.400pt}{0.350pt}}
\multiput(510.60,520.00)(0.468,0.920){5}{\rule{0.113pt}{0.800pt}}
\multiput(509.17,520.00)(4.000,5.340){2}{\rule{0.400pt}{0.400pt}}
\multiput(514.60,527.00)(0.468,0.774){5}{\rule{0.113pt}{0.700pt}}
\multiput(513.17,527.00)(4.000,4.547){2}{\rule{0.400pt}{0.350pt}}
\multiput(518.60,533.00)(0.468,0.774){5}{\rule{0.113pt}{0.700pt}}
\multiput(517.17,533.00)(4.000,4.547){2}{\rule{0.400pt}{0.350pt}}
\multiput(522.60,539.00)(0.468,0.774){5}{\rule{0.113pt}{0.700pt}}
\multiput(521.17,539.00)(4.000,4.547){2}{\rule{0.400pt}{0.350pt}}
\multiput(526.60,545.00)(0.468,0.774){5}{\rule{0.113pt}{0.700pt}}
\multiput(525.17,545.00)(4.000,4.547){2}{\rule{0.400pt}{0.350pt}}
\multiput(530.60,551.00)(0.468,0.774){5}{\rule{0.113pt}{0.700pt}}
\multiput(529.17,551.00)(4.000,4.547){2}{\rule{0.400pt}{0.350pt}}
\multiput(534.60,557.00)(0.468,0.627){5}{\rule{0.113pt}{0.600pt}}
\multiput(533.17,557.00)(4.000,3.755){2}{\rule{0.400pt}{0.300pt}}
\multiput(538.60,562.00)(0.468,0.774){5}{\rule{0.113pt}{0.700pt}}
\multiput(537.17,562.00)(4.000,4.547){2}{\rule{0.400pt}{0.350pt}}
\multiput(542.60,568.00)(0.468,0.627){5}{\rule{0.113pt}{0.600pt}}
\multiput(541.17,568.00)(4.000,3.755){2}{\rule{0.400pt}{0.300pt}}
\multiput(546.60,573.00)(0.468,0.627){5}{\rule{0.113pt}{0.600pt}}
\multiput(545.17,573.00)(4.000,3.755){2}{\rule{0.400pt}{0.300pt}}
\multiput(550.00,578.60)(0.481,0.468){5}{\rule{0.500pt}{0.113pt}}
\multiput(550.00,577.17)(2.962,4.000){2}{\rule{0.250pt}{0.400pt}}
\multiput(554.60,582.00)(0.468,0.627){5}{\rule{0.113pt}{0.600pt}}
\multiput(553.17,582.00)(4.000,3.755){2}{\rule{0.400pt}{0.300pt}}
\multiput(558.00,587.60)(0.481,0.468){5}{\rule{0.500pt}{0.113pt}}
\multiput(558.00,586.17)(2.962,4.000){2}{\rule{0.250pt}{0.400pt}}
\multiput(562.60,591.00)(0.468,0.627){5}{\rule{0.113pt}{0.600pt}}
\multiput(561.17,591.00)(4.000,3.755){2}{\rule{0.400pt}{0.300pt}}
\multiput(566.00,596.60)(0.481,0.468){5}{\rule{0.500pt}{0.113pt}}
\multiput(566.00,595.17)(2.962,4.000){2}{\rule{0.250pt}{0.400pt}}
\multiput(570.00,600.60)(0.481,0.468){5}{\rule{0.500pt}{0.113pt}}
\multiput(570.00,599.17)(2.962,4.000){2}{\rule{0.250pt}{0.400pt}}
\multiput(574.00,604.60)(0.481,0.468){5}{\rule{0.500pt}{0.113pt}}
\multiput(574.00,603.17)(2.962,4.000){2}{\rule{0.250pt}{0.400pt}}
\multiput(578.00,608.61)(0.685,0.447){3}{\rule{0.633pt}{0.108pt}}
\multiput(578.00,607.17)(2.685,3.000){2}{\rule{0.317pt}{0.400pt}}
\multiput(582.00,611.60)(0.481,0.468){5}{\rule{0.500pt}{0.113pt}}
\multiput(582.00,610.17)(2.962,4.000){2}{\rule{0.250pt}{0.400pt}}
\multiput(586.00,615.61)(0.685,0.447){3}{\rule{0.633pt}{0.108pt}}
\multiput(586.00,614.17)(2.685,3.000){2}{\rule{0.317pt}{0.400pt}}
\multiput(590.00,618.61)(0.685,0.447){3}{\rule{0.633pt}{0.108pt}}
\multiput(590.00,617.17)(2.685,3.000){2}{\rule{0.317pt}{0.400pt}}
\multiput(594.00,621.61)(0.685,0.447){3}{\rule{0.633pt}{0.108pt}}
\multiput(594.00,620.17)(2.685,3.000){2}{\rule{0.317pt}{0.400pt}}
\multiput(598.00,624.61)(0.685,0.447){3}{\rule{0.633pt}{0.108pt}}
\multiput(598.00,623.17)(2.685,3.000){2}{\rule{0.317pt}{0.400pt}}
\put(602,627.17){\rule{0.900pt}{0.400pt}}
\multiput(602.00,626.17)(2.132,2.000){2}{\rule{0.450pt}{0.400pt}}
\multiput(606.00,629.61)(0.685,0.447){3}{\rule{0.633pt}{0.108pt}}
\multiput(606.00,628.17)(2.685,3.000){2}{\rule{0.317pt}{0.400pt}}
\put(610,632.17){\rule{0.900pt}{0.400pt}}
\multiput(610.00,631.17)(2.132,2.000){2}{\rule{0.450pt}{0.400pt}}
\put(614,634.17){\rule{0.900pt}{0.400pt}}
\multiput(614.00,633.17)(2.132,2.000){2}{\rule{0.450pt}{0.400pt}}
\put(618,636.17){\rule{0.900pt}{0.400pt}}
\multiput(618.00,635.17)(2.132,2.000){2}{\rule{0.450pt}{0.400pt}}
\put(622,638.17){\rule{0.900pt}{0.400pt}}
\multiput(622.00,637.17)(2.132,2.000){2}{\rule{0.450pt}{0.400pt}}
\put(626,640.17){\rule{0.900pt}{0.400pt}}
\multiput(626.00,639.17)(2.132,2.000){2}{\rule{0.450pt}{0.400pt}}
\put(630,641.67){\rule{0.964pt}{0.400pt}}
\multiput(630.00,641.17)(2.000,1.000){2}{\rule{0.482pt}{0.400pt}}
\put(634,643.17){\rule{0.900pt}{0.400pt}}
\multiput(634.00,642.17)(2.132,2.000){2}{\rule{0.450pt}{0.400pt}}
\put(638,644.67){\rule{0.964pt}{0.400pt}}
\multiput(638.00,644.17)(2.000,1.000){2}{\rule{0.482pt}{0.400pt}}
\put(642,645.67){\rule{0.964pt}{0.400pt}}
\multiput(642.00,645.17)(2.000,1.000){2}{\rule{0.482pt}{0.400pt}}
\put(646,646.67){\rule{0.964pt}{0.400pt}}
\multiput(646.00,646.17)(2.000,1.000){2}{\rule{0.482pt}{0.400pt}}
\put(650,647.67){\rule{0.964pt}{0.400pt}}
\multiput(650.00,647.17)(2.000,1.000){2}{\rule{0.482pt}{0.400pt}}
\put(650.0,635.0){\rule[-0.200pt]{0.964pt}{0.400pt}}
\put(654.0,649.0){\rule[-0.200pt]{0.964pt}{0.400pt}}
\put(662,648.67){\rule{0.964pt}{0.400pt}}
\multiput(662.00,648.17)(2.000,1.000){2}{\rule{0.482pt}{0.400pt}}
\put(658.0,649.0){\rule[-0.200pt]{0.964pt}{0.400pt}}
\put(666.0,650.0){\rule[-0.200pt]{0.964pt}{0.400pt}}
\put(674,648.67){\rule{0.964pt}{0.400pt}}
\multiput(674.00,649.17)(2.000,-1.000){2}{\rule{0.482pt}{0.400pt}}
\put(670.0,650.0){\rule[-0.200pt]{0.964pt}{0.400pt}}
\put(681,647.67){\rule{0.964pt}{0.400pt}}
\multiput(681.00,648.17)(2.000,-1.000){2}{\rule{0.482pt}{0.400pt}}
\put(685,646.67){\rule{0.964pt}{0.400pt}}
\multiput(685.00,647.17)(2.000,-1.000){2}{\rule{0.482pt}{0.400pt}}
\put(678.0,649.0){\rule[-0.200pt]{0.723pt}{0.400pt}}
\put(693,645.17){\rule{0.900pt}{0.400pt}}
\multiput(693.00,646.17)(2.132,-2.000){2}{\rule{0.450pt}{0.400pt}}
\put(697,643.67){\rule{0.964pt}{0.400pt}}
\multiput(697.00,644.17)(2.000,-1.000){2}{\rule{0.482pt}{0.400pt}}
\put(701,642.67){\rule{0.964pt}{0.400pt}}
\multiput(701.00,643.17)(2.000,-1.000){2}{\rule{0.482pt}{0.400pt}}
\put(705,641.17){\rule{0.900pt}{0.400pt}}
\multiput(705.00,642.17)(2.132,-2.000){2}{\rule{0.450pt}{0.400pt}}
\put(709,639.17){\rule{0.900pt}{0.400pt}}
\multiput(709.00,640.17)(2.132,-2.000){2}{\rule{0.450pt}{0.400pt}}
\put(713,637.17){\rule{0.900pt}{0.400pt}}
\multiput(713.00,638.17)(2.132,-2.000){2}{\rule{0.450pt}{0.400pt}}
\put(717,635.17){\rule{0.900pt}{0.400pt}}
\multiput(717.00,636.17)(2.132,-2.000){2}{\rule{0.450pt}{0.400pt}}
\put(721,633.17){\rule{0.900pt}{0.400pt}}
\multiput(721.00,634.17)(2.132,-2.000){2}{\rule{0.450pt}{0.400pt}}
\multiput(725.00,631.95)(0.685,-0.447){3}{\rule{0.633pt}{0.108pt}}
\multiput(725.00,632.17)(2.685,-3.000){2}{\rule{0.317pt}{0.400pt}}
\put(729,628.17){\rule{0.900pt}{0.400pt}}
\multiput(729.00,629.17)(2.132,-2.000){2}{\rule{0.450pt}{0.400pt}}
\multiput(733.00,626.95)(0.685,-0.447){3}{\rule{0.633pt}{0.108pt}}
\multiput(733.00,627.17)(2.685,-3.000){2}{\rule{0.317pt}{0.400pt}}
\multiput(737.00,623.95)(0.685,-0.447){3}{\rule{0.633pt}{0.108pt}}
\multiput(737.00,624.17)(2.685,-3.000){2}{\rule{0.317pt}{0.400pt}}
\multiput(741.00,620.94)(0.481,-0.468){5}{\rule{0.500pt}{0.113pt}}
\multiput(741.00,621.17)(2.962,-4.000){2}{\rule{0.250pt}{0.400pt}}
\multiput(745.00,616.95)(0.685,-0.447){3}{\rule{0.633pt}{0.108pt}}
\multiput(745.00,617.17)(2.685,-3.000){2}{\rule{0.317pt}{0.400pt}}
\multiput(749.00,613.94)(0.481,-0.468){5}{\rule{0.500pt}{0.113pt}}
\multiput(749.00,614.17)(2.962,-4.000){2}{\rule{0.250pt}{0.400pt}}
\multiput(753.00,609.95)(0.685,-0.447){3}{\rule{0.633pt}{0.108pt}}
\multiput(753.00,610.17)(2.685,-3.000){2}{\rule{0.317pt}{0.400pt}}
\multiput(757.00,606.94)(0.481,-0.468){5}{\rule{0.500pt}{0.113pt}}
\multiput(757.00,607.17)(2.962,-4.000){2}{\rule{0.250pt}{0.400pt}}
\multiput(761.60,601.51)(0.468,-0.627){5}{\rule{0.113pt}{0.600pt}}
\multiput(760.17,602.75)(4.000,-3.755){2}{\rule{0.400pt}{0.300pt}}
\multiput(765.00,597.94)(0.481,-0.468){5}{\rule{0.500pt}{0.113pt}}
\multiput(765.00,598.17)(2.962,-4.000){2}{\rule{0.250pt}{0.400pt}}
\multiput(769.60,592.51)(0.468,-0.627){5}{\rule{0.113pt}{0.600pt}}
\multiput(768.17,593.75)(4.000,-3.755){2}{\rule{0.400pt}{0.300pt}}
\multiput(773.00,588.94)(0.481,-0.468){5}{\rule{0.500pt}{0.113pt}}
\multiput(773.00,589.17)(2.962,-4.000){2}{\rule{0.250pt}{0.400pt}}
\multiput(777.60,583.51)(0.468,-0.627){5}{\rule{0.113pt}{0.600pt}}
\multiput(776.17,584.75)(4.000,-3.755){2}{\rule{0.400pt}{0.300pt}}
\multiput(781.60,578.51)(0.468,-0.627){5}{\rule{0.113pt}{0.600pt}}
\multiput(780.17,579.75)(4.000,-3.755){2}{\rule{0.400pt}{0.300pt}}
\multiput(785.60,573.09)(0.468,-0.774){5}{\rule{0.113pt}{0.700pt}}
\multiput(784.17,574.55)(4.000,-4.547){2}{\rule{0.400pt}{0.350pt}}
\multiput(789.60,567.51)(0.468,-0.627){5}{\rule{0.113pt}{0.600pt}}
\multiput(788.17,568.75)(4.000,-3.755){2}{\rule{0.400pt}{0.300pt}}
\multiput(793.60,562.09)(0.468,-0.774){5}{\rule{0.113pt}{0.700pt}}
\multiput(792.17,563.55)(4.000,-4.547){2}{\rule{0.400pt}{0.350pt}}
\multiput(797.60,556.09)(0.468,-0.774){5}{\rule{0.113pt}{0.700pt}}
\multiput(796.17,557.55)(4.000,-4.547){2}{\rule{0.400pt}{0.350pt}}
\multiput(801.60,549.68)(0.468,-0.920){5}{\rule{0.113pt}{0.800pt}}
\multiput(800.17,551.34)(4.000,-5.340){2}{\rule{0.400pt}{0.400pt}}
\multiput(805.60,543.09)(0.468,-0.774){5}{\rule{0.113pt}{0.700pt}}
\multiput(804.17,544.55)(4.000,-4.547){2}{\rule{0.400pt}{0.350pt}}
\multiput(809.60,536.68)(0.468,-0.920){5}{\rule{0.113pt}{0.800pt}}
\multiput(808.17,538.34)(4.000,-5.340){2}{\rule{0.400pt}{0.400pt}}
\multiput(813.60,529.68)(0.468,-0.920){5}{\rule{0.113pt}{0.800pt}}
\multiput(812.17,531.34)(4.000,-5.340){2}{\rule{0.400pt}{0.400pt}}
\multiput(817.60,522.68)(0.468,-0.920){5}{\rule{0.113pt}{0.800pt}}
\multiput(816.17,524.34)(4.000,-5.340){2}{\rule{0.400pt}{0.400pt}}
\multiput(463.60,332.00)(0.468,1.651){5}{\rule{0.113pt}{1.300pt}}
\multiput(462.17,332.00)(4.000,9.302){2}{\rule{0.400pt}{0.650pt}}
\multiput(467.60,344.00)(0.468,1.505){5}{\rule{0.113pt}{1.200pt}}
\multiput(466.17,344.00)(4.000,8.509){2}{\rule{0.400pt}{0.600pt}}
\multiput(471.60,355.00)(0.468,1.505){5}{\rule{0.113pt}{1.200pt}}
\multiput(470.17,355.00)(4.000,8.509){2}{\rule{0.400pt}{0.600pt}}
\multiput(475.60,366.00)(0.468,1.505){5}{\rule{0.113pt}{1.200pt}}
\multiput(474.17,366.00)(4.000,8.509){2}{\rule{0.400pt}{0.600pt}}
\multiput(479.60,377.00)(0.468,1.358){5}{\rule{0.113pt}{1.100pt}}
\multiput(478.17,377.00)(4.000,7.717){2}{\rule{0.400pt}{0.550pt}}
\multiput(483.60,387.00)(0.468,1.358){5}{\rule{0.113pt}{1.100pt}}
\multiput(482.17,387.00)(4.000,7.717){2}{\rule{0.400pt}{0.550pt}}
\multiput(487.60,397.00)(0.468,1.358){5}{\rule{0.113pt}{1.100pt}}
\multiput(486.17,397.00)(4.000,7.717){2}{\rule{0.400pt}{0.550pt}}
\multiput(491.60,407.00)(0.468,1.358){5}{\rule{0.113pt}{1.100pt}}
\multiput(490.17,407.00)(4.000,7.717){2}{\rule{0.400pt}{0.550pt}}
\multiput(495.60,417.00)(0.468,1.358){5}{\rule{0.113pt}{1.100pt}}
\multiput(494.17,417.00)(4.000,7.717){2}{\rule{0.400pt}{0.550pt}}
\multiput(499.60,427.00)(0.468,1.212){5}{\rule{0.113pt}{1.000pt}}
\multiput(498.17,427.00)(4.000,6.924){2}{\rule{0.400pt}{0.500pt}}
\multiput(503.61,436.00)(0.447,1.802){3}{\rule{0.108pt}{1.300pt}}
\multiput(502.17,436.00)(3.000,6.302){2}{\rule{0.400pt}{0.650pt}}
\multiput(506.60,445.00)(0.468,1.212){5}{\rule{0.113pt}{1.000pt}}
\multiput(505.17,445.00)(4.000,6.924){2}{\rule{0.400pt}{0.500pt}}
\multiput(510.60,454.00)(0.468,1.212){5}{\rule{0.113pt}{1.000pt}}
\multiput(509.17,454.00)(4.000,6.924){2}{\rule{0.400pt}{0.500pt}}
\multiput(514.60,463.00)(0.468,1.212){5}{\rule{0.113pt}{1.000pt}}
\multiput(513.17,463.00)(4.000,6.924){2}{\rule{0.400pt}{0.500pt}}
\multiput(518.60,472.00)(0.468,1.066){5}{\rule{0.113pt}{0.900pt}}
\multiput(517.17,472.00)(4.000,6.132){2}{\rule{0.400pt}{0.450pt}}
\multiput(522.60,480.00)(0.468,1.066){5}{\rule{0.113pt}{0.900pt}}
\multiput(521.17,480.00)(4.000,6.132){2}{\rule{0.400pt}{0.450pt}}
\multiput(526.60,488.00)(0.468,1.066){5}{\rule{0.113pt}{0.900pt}}
\multiput(525.17,488.00)(4.000,6.132){2}{\rule{0.400pt}{0.450pt}}
\multiput(530.60,496.00)(0.468,0.920){5}{\rule{0.113pt}{0.800pt}}
\multiput(529.17,496.00)(4.000,5.340){2}{\rule{0.400pt}{0.400pt}}
\multiput(534.60,503.00)(0.468,1.066){5}{\rule{0.113pt}{0.900pt}}
\multiput(533.17,503.00)(4.000,6.132){2}{\rule{0.400pt}{0.450pt}}
\multiput(538.60,511.00)(0.468,0.920){5}{\rule{0.113pt}{0.800pt}}
\multiput(537.17,511.00)(4.000,5.340){2}{\rule{0.400pt}{0.400pt}}
\multiput(542.60,518.00)(0.468,0.920){5}{\rule{0.113pt}{0.800pt}}
\multiput(541.17,518.00)(4.000,5.340){2}{\rule{0.400pt}{0.400pt}}
\multiput(546.60,525.00)(0.468,0.920){5}{\rule{0.113pt}{0.800pt}}
\multiput(545.17,525.00)(4.000,5.340){2}{\rule{0.400pt}{0.400pt}}
\multiput(550.60,532.00)(0.468,0.920){5}{\rule{0.113pt}{0.800pt}}
\multiput(549.17,532.00)(4.000,5.340){2}{\rule{0.400pt}{0.400pt}}
\multiput(554.60,539.00)(0.468,0.774){5}{\rule{0.113pt}{0.700pt}}
\multiput(553.17,539.00)(4.000,4.547){2}{\rule{0.400pt}{0.350pt}}
\multiput(558.60,545.00)(0.468,0.920){5}{\rule{0.113pt}{0.800pt}}
\multiput(557.17,545.00)(4.000,5.340){2}{\rule{0.400pt}{0.400pt}}
\multiput(562.60,552.00)(0.468,0.774){5}{\rule{0.113pt}{0.700pt}}
\multiput(561.17,552.00)(4.000,4.547){2}{\rule{0.400pt}{0.350pt}}
\multiput(566.60,558.00)(0.468,0.774){5}{\rule{0.113pt}{0.700pt}}
\multiput(565.17,558.00)(4.000,4.547){2}{\rule{0.400pt}{0.350pt}}
\multiput(570.60,564.00)(0.468,0.627){5}{\rule{0.113pt}{0.600pt}}
\multiput(569.17,564.00)(4.000,3.755){2}{\rule{0.400pt}{0.300pt}}
\multiput(574.60,569.00)(0.468,0.774){5}{\rule{0.113pt}{0.700pt}}
\multiput(573.17,569.00)(4.000,4.547){2}{\rule{0.400pt}{0.350pt}}
\multiput(578.60,575.00)(0.468,0.627){5}{\rule{0.113pt}{0.600pt}}
\multiput(577.17,575.00)(4.000,3.755){2}{\rule{0.400pt}{0.300pt}}
\multiput(582.60,580.00)(0.468,0.627){5}{\rule{0.113pt}{0.600pt}}
\multiput(581.17,580.00)(4.000,3.755){2}{\rule{0.400pt}{0.300pt}}
\multiput(586.60,585.00)(0.468,0.627){5}{\rule{0.113pt}{0.600pt}}
\multiput(585.17,585.00)(4.000,3.755){2}{\rule{0.400pt}{0.300pt}}
\multiput(590.60,590.00)(0.468,0.627){5}{\rule{0.113pt}{0.600pt}}
\multiput(589.17,590.00)(4.000,3.755){2}{\rule{0.400pt}{0.300pt}}
\multiput(594.60,595.00)(0.468,0.627){5}{\rule{0.113pt}{0.600pt}}
\multiput(593.17,595.00)(4.000,3.755){2}{\rule{0.400pt}{0.300pt}}
\multiput(598.00,600.60)(0.481,0.468){5}{\rule{0.500pt}{0.113pt}}
\multiput(598.00,599.17)(2.962,4.000){2}{\rule{0.250pt}{0.400pt}}
\multiput(602.00,604.60)(0.481,0.468){5}{\rule{0.500pt}{0.113pt}}
\multiput(602.00,603.17)(2.962,4.000){2}{\rule{0.250pt}{0.400pt}}
\multiput(606.60,608.00)(0.468,0.627){5}{\rule{0.113pt}{0.600pt}}
\multiput(605.17,608.00)(4.000,3.755){2}{\rule{0.400pt}{0.300pt}}
\multiput(610.00,613.61)(0.685,0.447){3}{\rule{0.633pt}{0.108pt}}
\multiput(610.00,612.17)(2.685,3.000){2}{\rule{0.317pt}{0.400pt}}
\multiput(614.00,616.60)(0.481,0.468){5}{\rule{0.500pt}{0.113pt}}
\multiput(614.00,615.17)(2.962,4.000){2}{\rule{0.250pt}{0.400pt}}
\multiput(618.00,620.60)(0.481,0.468){5}{\rule{0.500pt}{0.113pt}}
\multiput(618.00,619.17)(2.962,4.000){2}{\rule{0.250pt}{0.400pt}}
\multiput(622.00,624.61)(0.685,0.447){3}{\rule{0.633pt}{0.108pt}}
\multiput(622.00,623.17)(2.685,3.000){2}{\rule{0.317pt}{0.400pt}}
\multiput(626.00,627.61)(0.685,0.447){3}{\rule{0.633pt}{0.108pt}}
\multiput(626.00,626.17)(2.685,3.000){2}{\rule{0.317pt}{0.400pt}}
\multiput(630.00,630.61)(0.685,0.447){3}{\rule{0.633pt}{0.108pt}}
\multiput(630.00,629.17)(2.685,3.000){2}{\rule{0.317pt}{0.400pt}}
\multiput(634.00,633.61)(0.685,0.447){3}{\rule{0.633pt}{0.108pt}}
\multiput(634.00,632.17)(2.685,3.000){2}{\rule{0.317pt}{0.400pt}}
\multiput(638.00,636.61)(0.685,0.447){3}{\rule{0.633pt}{0.108pt}}
\multiput(638.00,635.17)(2.685,3.000){2}{\rule{0.317pt}{0.400pt}}
\multiput(642.00,639.61)(0.685,0.447){3}{\rule{0.633pt}{0.108pt}}
\multiput(642.00,638.17)(2.685,3.000){2}{\rule{0.317pt}{0.400pt}}
\put(646,642.17){\rule{0.900pt}{0.400pt}}
\multiput(646.00,641.17)(2.132,2.000){2}{\rule{0.450pt}{0.400pt}}
\put(650,644.17){\rule{0.900pt}{0.400pt}}
\multiput(650.00,643.17)(2.132,2.000){2}{\rule{0.450pt}{0.400pt}}
\put(654,646.17){\rule{0.900pt}{0.400pt}}
\multiput(654.00,645.17)(2.132,2.000){2}{\rule{0.450pt}{0.400pt}}
\put(658,648.17){\rule{0.900pt}{0.400pt}}
\multiput(658.00,647.17)(2.132,2.000){2}{\rule{0.450pt}{0.400pt}}
\put(662,650.17){\rule{0.900pt}{0.400pt}}
\multiput(662.00,649.17)(2.132,2.000){2}{\rule{0.450pt}{0.400pt}}
\put(666,651.67){\rule{0.964pt}{0.400pt}}
\multiput(666.00,651.17)(2.000,1.000){2}{\rule{0.482pt}{0.400pt}}
\put(670,652.67){\rule{0.964pt}{0.400pt}}
\multiput(670.00,652.17)(2.000,1.000){2}{\rule{0.482pt}{0.400pt}}
\put(674,654.17){\rule{0.900pt}{0.400pt}}
\multiput(674.00,653.17)(2.132,2.000){2}{\rule{0.450pt}{0.400pt}}
\put(689.0,647.0){\rule[-0.200pt]{0.964pt}{0.400pt}}
\put(682,655.67){\rule{0.964pt}{0.400pt}}
\multiput(682.00,655.17)(2.000,1.000){2}{\rule{0.482pt}{0.400pt}}
\put(686,656.67){\rule{0.964pt}{0.400pt}}
\multiput(686.00,656.17)(2.000,1.000){2}{\rule{0.482pt}{0.400pt}}
\put(678.0,656.0){\rule[-0.200pt]{0.964pt}{0.400pt}}
\put(694,657.67){\rule{0.964pt}{0.400pt}}
\multiput(694.00,657.17)(2.000,1.000){2}{\rule{0.482pt}{0.400pt}}
\put(690.0,658.0){\rule[-0.200pt]{0.964pt}{0.400pt}}
\put(698.0,659.0){\rule[-0.200pt]{0.723pt}{0.400pt}}
\put(705,657.67){\rule{0.964pt}{0.400pt}}
\multiput(705.00,658.17)(2.000,-1.000){2}{\rule{0.482pt}{0.400pt}}
\put(701.0,659.0){\rule[-0.200pt]{0.964pt}{0.400pt}}
\put(713,656.67){\rule{0.964pt}{0.400pt}}
\multiput(713.00,657.17)(2.000,-1.000){2}{\rule{0.482pt}{0.400pt}}
\put(709.0,658.0){\rule[-0.200pt]{0.964pt}{0.400pt}}
\put(721,655.67){\rule{0.964pt}{0.400pt}}
\multiput(721.00,656.17)(2.000,-1.000){2}{\rule{0.482pt}{0.400pt}}
\put(725,654.17){\rule{0.900pt}{0.400pt}}
\multiput(725.00,655.17)(2.132,-2.000){2}{\rule{0.450pt}{0.400pt}}
\put(729,652.67){\rule{0.964pt}{0.400pt}}
\multiput(729.00,653.17)(2.000,-1.000){2}{\rule{0.482pt}{0.400pt}}
\put(733,651.67){\rule{0.964pt}{0.400pt}}
\multiput(733.00,652.17)(2.000,-1.000){2}{\rule{0.482pt}{0.400pt}}
\put(737,650.17){\rule{0.900pt}{0.400pt}}
\multiput(737.00,651.17)(2.132,-2.000){2}{\rule{0.450pt}{0.400pt}}
\put(741,648.17){\rule{0.900pt}{0.400pt}}
\multiput(741.00,649.17)(2.132,-2.000){2}{\rule{0.450pt}{0.400pt}}
\put(745,646.17){\rule{0.900pt}{0.400pt}}
\multiput(745.00,647.17)(2.132,-2.000){2}{\rule{0.450pt}{0.400pt}}
\put(749,644.17){\rule{0.900pt}{0.400pt}}
\multiput(749.00,645.17)(2.132,-2.000){2}{\rule{0.450pt}{0.400pt}}
\multiput(753.00,642.95)(0.685,-0.447){3}{\rule{0.633pt}{0.108pt}}
\multiput(753.00,643.17)(2.685,-3.000){2}{\rule{0.317pt}{0.400pt}}
\put(757,639.17){\rule{0.900pt}{0.400pt}}
\multiput(757.00,640.17)(2.132,-2.000){2}{\rule{0.450pt}{0.400pt}}
\multiput(761.00,637.95)(0.685,-0.447){3}{\rule{0.633pt}{0.108pt}}
\multiput(761.00,638.17)(2.685,-3.000){2}{\rule{0.317pt}{0.400pt}}
\multiput(765.00,634.95)(0.685,-0.447){3}{\rule{0.633pt}{0.108pt}}
\multiput(765.00,635.17)(2.685,-3.000){2}{\rule{0.317pt}{0.400pt}}
\multiput(769.00,631.95)(0.685,-0.447){3}{\rule{0.633pt}{0.108pt}}
\multiput(769.00,632.17)(2.685,-3.000){2}{\rule{0.317pt}{0.400pt}}
\multiput(773.00,628.94)(0.481,-0.468){5}{\rule{0.500pt}{0.113pt}}
\multiput(773.00,629.17)(2.962,-4.000){2}{\rule{0.250pt}{0.400pt}}
\multiput(777.00,624.95)(0.685,-0.447){3}{\rule{0.633pt}{0.108pt}}
\multiput(777.00,625.17)(2.685,-3.000){2}{\rule{0.317pt}{0.400pt}}
\multiput(781.00,621.94)(0.481,-0.468){5}{\rule{0.500pt}{0.113pt}}
\multiput(781.00,622.17)(2.962,-4.000){2}{\rule{0.250pt}{0.400pt}}
\multiput(785.00,617.94)(0.481,-0.468){5}{\rule{0.500pt}{0.113pt}}
\multiput(785.00,618.17)(2.962,-4.000){2}{\rule{0.250pt}{0.400pt}}
\multiput(789.00,613.94)(0.481,-0.468){5}{\rule{0.500pt}{0.113pt}}
\multiput(789.00,614.17)(2.962,-4.000){2}{\rule{0.250pt}{0.400pt}}
\multiput(793.60,608.51)(0.468,-0.627){5}{\rule{0.113pt}{0.600pt}}
\multiput(792.17,609.75)(4.000,-3.755){2}{\rule{0.400pt}{0.300pt}}
\multiput(797.00,604.94)(0.481,-0.468){5}{\rule{0.500pt}{0.113pt}}
\multiput(797.00,605.17)(2.962,-4.000){2}{\rule{0.250pt}{0.400pt}}
\multiput(801.60,599.51)(0.468,-0.627){5}{\rule{0.113pt}{0.600pt}}
\multiput(800.17,600.75)(4.000,-3.755){2}{\rule{0.400pt}{0.300pt}}
\multiput(805.60,594.51)(0.468,-0.627){5}{\rule{0.113pt}{0.600pt}}
\multiput(804.17,595.75)(4.000,-3.755){2}{\rule{0.400pt}{0.300pt}}
\multiput(809.60,589.09)(0.468,-0.774){5}{\rule{0.113pt}{0.700pt}}
\multiput(808.17,590.55)(4.000,-4.547){2}{\rule{0.400pt}{0.350pt}}
\multiput(813.60,583.51)(0.468,-0.627){5}{\rule{0.113pt}{0.600pt}}
\multiput(812.17,584.75)(4.000,-3.755){2}{\rule{0.400pt}{0.300pt}}
\multiput(817.60,578.09)(0.468,-0.774){5}{\rule{0.113pt}{0.700pt}}
\multiput(816.17,579.55)(4.000,-4.547){2}{\rule{0.400pt}{0.350pt}}
\multiput(821.60,572.09)(0.468,-0.774){5}{\rule{0.113pt}{0.700pt}}
\multiput(820.17,573.55)(4.000,-4.547){2}{\rule{0.400pt}{0.350pt}}
\multiput(825.60,566.09)(0.468,-0.774){5}{\rule{0.113pt}{0.700pt}}
\multiput(824.17,567.55)(4.000,-4.547){2}{\rule{0.400pt}{0.350pt}}
\multiput(829.60,560.09)(0.468,-0.774){5}{\rule{0.113pt}{0.700pt}}
\multiput(828.17,561.55)(4.000,-4.547){2}{\rule{0.400pt}{0.350pt}}
\multiput(833.60,553.68)(0.468,-0.920){5}{\rule{0.113pt}{0.800pt}}
\multiput(832.17,555.34)(4.000,-5.340){2}{\rule{0.400pt}{0.400pt}}
\multiput(837.60,546.68)(0.468,-0.920){5}{\rule{0.113pt}{0.800pt}}
\multiput(836.17,548.34)(4.000,-5.340){2}{\rule{0.400pt}{0.400pt}}
\multiput(841.60,539.68)(0.468,-0.920){5}{\rule{0.113pt}{0.800pt}}
\multiput(840.17,541.34)(4.000,-5.340){2}{\rule{0.400pt}{0.400pt}}
\multiput(845.60,532.68)(0.468,-0.920){5}{\rule{0.113pt}{0.800pt}}
\multiput(844.17,534.34)(4.000,-5.340){2}{\rule{0.400pt}{0.400pt}}
\multiput(849.60,525.26)(0.468,-1.066){5}{\rule{0.113pt}{0.900pt}}
\multiput(848.17,527.13)(4.000,-6.132){2}{\rule{0.400pt}{0.450pt}}
\multiput(853.60,517.68)(0.468,-0.920){5}{\rule{0.113pt}{0.800pt}}
\multiput(852.17,519.34)(4.000,-5.340){2}{\rule{0.400pt}{0.400pt}}
\multiput(499.60,327.00)(0.468,1.651){5}{\rule{0.113pt}{1.300pt}}
\multiput(498.17,327.00)(4.000,9.302){2}{\rule{0.400pt}{0.650pt}}
\multiput(503.60,339.00)(0.468,1.505){5}{\rule{0.113pt}{1.200pt}}
\multiput(502.17,339.00)(4.000,8.509){2}{\rule{0.400pt}{0.600pt}}
\multiput(507.60,350.00)(0.468,1.651){5}{\rule{0.113pt}{1.300pt}}
\multiput(506.17,350.00)(4.000,9.302){2}{\rule{0.400pt}{0.650pt}}
\multiput(511.60,362.00)(0.468,1.505){5}{\rule{0.113pt}{1.200pt}}
\multiput(510.17,362.00)(4.000,8.509){2}{\rule{0.400pt}{0.600pt}}
\multiput(515.60,373.00)(0.468,1.505){5}{\rule{0.113pt}{1.200pt}}
\multiput(514.17,373.00)(4.000,8.509){2}{\rule{0.400pt}{0.600pt}}
\multiput(519.60,384.00)(0.468,1.358){5}{\rule{0.113pt}{1.100pt}}
\multiput(518.17,384.00)(4.000,7.717){2}{\rule{0.400pt}{0.550pt}}
\multiput(523.61,394.00)(0.447,2.248){3}{\rule{0.108pt}{1.567pt}}
\multiput(522.17,394.00)(3.000,7.748){2}{\rule{0.400pt}{0.783pt}}
\multiput(526.60,405.00)(0.468,1.358){5}{\rule{0.113pt}{1.100pt}}
\multiput(525.17,405.00)(4.000,7.717){2}{\rule{0.400pt}{0.550pt}}
\multiput(530.60,415.00)(0.468,1.358){5}{\rule{0.113pt}{1.100pt}}
\multiput(529.17,415.00)(4.000,7.717){2}{\rule{0.400pt}{0.550pt}}
\multiput(534.60,425.00)(0.468,1.358){5}{\rule{0.113pt}{1.100pt}}
\multiput(533.17,425.00)(4.000,7.717){2}{\rule{0.400pt}{0.550pt}}
\multiput(538.60,435.00)(0.468,1.212){5}{\rule{0.113pt}{1.000pt}}
\multiput(537.17,435.00)(4.000,6.924){2}{\rule{0.400pt}{0.500pt}}
\multiput(542.60,444.00)(0.468,1.212){5}{\rule{0.113pt}{1.000pt}}
\multiput(541.17,444.00)(4.000,6.924){2}{\rule{0.400pt}{0.500pt}}
\multiput(546.60,453.00)(0.468,1.212){5}{\rule{0.113pt}{1.000pt}}
\multiput(545.17,453.00)(4.000,6.924){2}{\rule{0.400pt}{0.500pt}}
\multiput(550.60,462.00)(0.468,1.212){5}{\rule{0.113pt}{1.000pt}}
\multiput(549.17,462.00)(4.000,6.924){2}{\rule{0.400pt}{0.500pt}}
\multiput(554.60,471.00)(0.468,1.212){5}{\rule{0.113pt}{1.000pt}}
\multiput(553.17,471.00)(4.000,6.924){2}{\rule{0.400pt}{0.500pt}}
\multiput(558.60,480.00)(0.468,1.066){5}{\rule{0.113pt}{0.900pt}}
\multiput(557.17,480.00)(4.000,6.132){2}{\rule{0.400pt}{0.450pt}}
\multiput(562.60,488.00)(0.468,1.066){5}{\rule{0.113pt}{0.900pt}}
\multiput(561.17,488.00)(4.000,6.132){2}{\rule{0.400pt}{0.450pt}}
\multiput(566.60,496.00)(0.468,1.066){5}{\rule{0.113pt}{0.900pt}}
\multiput(565.17,496.00)(4.000,6.132){2}{\rule{0.400pt}{0.450pt}}
\multiput(570.60,504.00)(0.468,1.066){5}{\rule{0.113pt}{0.900pt}}
\multiput(569.17,504.00)(4.000,6.132){2}{\rule{0.400pt}{0.450pt}}
\multiput(574.60,512.00)(0.468,0.920){5}{\rule{0.113pt}{0.800pt}}
\multiput(573.17,512.00)(4.000,5.340){2}{\rule{0.400pt}{0.400pt}}
\multiput(578.60,519.00)(0.468,1.066){5}{\rule{0.113pt}{0.900pt}}
\multiput(577.17,519.00)(4.000,6.132){2}{\rule{0.400pt}{0.450pt}}
\multiput(582.60,527.00)(0.468,0.920){5}{\rule{0.113pt}{0.800pt}}
\multiput(581.17,527.00)(4.000,5.340){2}{\rule{0.400pt}{0.400pt}}
\multiput(586.60,534.00)(0.468,0.920){5}{\rule{0.113pt}{0.800pt}}
\multiput(585.17,534.00)(4.000,5.340){2}{\rule{0.400pt}{0.400pt}}
\multiput(590.60,541.00)(0.468,0.774){5}{\rule{0.113pt}{0.700pt}}
\multiput(589.17,541.00)(4.000,4.547){2}{\rule{0.400pt}{0.350pt}}
\multiput(594.60,547.00)(0.468,0.920){5}{\rule{0.113pt}{0.800pt}}
\multiput(593.17,547.00)(4.000,5.340){2}{\rule{0.400pt}{0.400pt}}
\multiput(598.60,554.00)(0.468,0.774){5}{\rule{0.113pt}{0.700pt}}
\multiput(597.17,554.00)(4.000,4.547){2}{\rule{0.400pt}{0.350pt}}
\multiput(602.60,560.00)(0.468,0.774){5}{\rule{0.113pt}{0.700pt}}
\multiput(601.17,560.00)(4.000,4.547){2}{\rule{0.400pt}{0.350pt}}
\multiput(606.60,566.00)(0.468,0.774){5}{\rule{0.113pt}{0.700pt}}
\multiput(605.17,566.00)(4.000,4.547){2}{\rule{0.400pt}{0.350pt}}
\multiput(610.60,572.00)(0.468,0.774){5}{\rule{0.113pt}{0.700pt}}
\multiput(609.17,572.00)(4.000,4.547){2}{\rule{0.400pt}{0.350pt}}
\multiput(614.60,578.00)(0.468,0.627){5}{\rule{0.113pt}{0.600pt}}
\multiput(613.17,578.00)(4.000,3.755){2}{\rule{0.400pt}{0.300pt}}
\multiput(618.60,583.00)(0.468,0.774){5}{\rule{0.113pt}{0.700pt}}
\multiput(617.17,583.00)(4.000,4.547){2}{\rule{0.400pt}{0.350pt}}
\multiput(622.60,589.00)(0.468,0.627){5}{\rule{0.113pt}{0.600pt}}
\multiput(621.17,589.00)(4.000,3.755){2}{\rule{0.400pt}{0.300pt}}
\multiput(626.60,594.00)(0.468,0.627){5}{\rule{0.113pt}{0.600pt}}
\multiput(625.17,594.00)(4.000,3.755){2}{\rule{0.400pt}{0.300pt}}
\multiput(630.00,599.60)(0.481,0.468){5}{\rule{0.500pt}{0.113pt}}
\multiput(630.00,598.17)(2.962,4.000){2}{\rule{0.250pt}{0.400pt}}
\multiput(634.60,603.00)(0.468,0.627){5}{\rule{0.113pt}{0.600pt}}
\multiput(633.17,603.00)(4.000,3.755){2}{\rule{0.400pt}{0.300pt}}
\multiput(638.00,608.60)(0.481,0.468){5}{\rule{0.500pt}{0.113pt}}
\multiput(638.00,607.17)(2.962,4.000){2}{\rule{0.250pt}{0.400pt}}
\multiput(642.60,612.00)(0.468,0.627){5}{\rule{0.113pt}{0.600pt}}
\multiput(641.17,612.00)(4.000,3.755){2}{\rule{0.400pt}{0.300pt}}
\multiput(646.00,617.60)(0.481,0.468){5}{\rule{0.500pt}{0.113pt}}
\multiput(646.00,616.17)(2.962,4.000){2}{\rule{0.250pt}{0.400pt}}
\multiput(650.00,621.61)(0.685,0.447){3}{\rule{0.633pt}{0.108pt}}
\multiput(650.00,620.17)(2.685,3.000){2}{\rule{0.317pt}{0.400pt}}
\multiput(654.00,624.60)(0.481,0.468){5}{\rule{0.500pt}{0.113pt}}
\multiput(654.00,623.17)(2.962,4.000){2}{\rule{0.250pt}{0.400pt}}
\multiput(658.00,628.61)(0.685,0.447){3}{\rule{0.633pt}{0.108pt}}
\multiput(658.00,627.17)(2.685,3.000){2}{\rule{0.317pt}{0.400pt}}
\multiput(662.00,631.60)(0.481,0.468){5}{\rule{0.500pt}{0.113pt}}
\multiput(662.00,630.17)(2.962,4.000){2}{\rule{0.250pt}{0.400pt}}
\multiput(666.00,635.61)(0.685,0.447){3}{\rule{0.633pt}{0.108pt}}
\multiput(666.00,634.17)(2.685,3.000){2}{\rule{0.317pt}{0.400pt}}
\multiput(670.00,638.61)(0.685,0.447){3}{\rule{0.633pt}{0.108pt}}
\multiput(670.00,637.17)(2.685,3.000){2}{\rule{0.317pt}{0.400pt}}
\put(674,641.17){\rule{0.900pt}{0.400pt}}
\multiput(674.00,640.17)(2.132,2.000){2}{\rule{0.450pt}{0.400pt}}
\multiput(678.00,643.61)(0.685,0.447){3}{\rule{0.633pt}{0.108pt}}
\multiput(678.00,642.17)(2.685,3.000){2}{\rule{0.317pt}{0.400pt}}
\put(682,646.17){\rule{0.900pt}{0.400pt}}
\multiput(682.00,645.17)(2.132,2.000){2}{\rule{0.450pt}{0.400pt}}
\multiput(686.00,648.61)(0.685,0.447){3}{\rule{0.633pt}{0.108pt}}
\multiput(686.00,647.17)(2.685,3.000){2}{\rule{0.317pt}{0.400pt}}
\put(690,651.17){\rule{0.900pt}{0.400pt}}
\multiput(690.00,650.17)(2.132,2.000){2}{\rule{0.450pt}{0.400pt}}
\put(694,652.67){\rule{0.964pt}{0.400pt}}
\multiput(694.00,652.17)(2.000,1.000){2}{\rule{0.482pt}{0.400pt}}
\put(698,654.17){\rule{0.900pt}{0.400pt}}
\multiput(698.00,653.17)(2.132,2.000){2}{\rule{0.450pt}{0.400pt}}
\put(702,656.17){\rule{0.900pt}{0.400pt}}
\multiput(702.00,655.17)(2.132,2.000){2}{\rule{0.450pt}{0.400pt}}
\put(706,657.67){\rule{0.964pt}{0.400pt}}
\multiput(706.00,657.17)(2.000,1.000){2}{\rule{0.482pt}{0.400pt}}
\put(710,658.67){\rule{0.964pt}{0.400pt}}
\multiput(710.00,658.17)(2.000,1.000){2}{\rule{0.482pt}{0.400pt}}
\put(714,659.67){\rule{0.964pt}{0.400pt}}
\multiput(714.00,659.17)(2.000,1.000){2}{\rule{0.482pt}{0.400pt}}
\put(718,660.67){\rule{0.723pt}{0.400pt}}
\multiput(718.00,660.17)(1.500,1.000){2}{\rule{0.361pt}{0.400pt}}
\put(717.0,657.0){\rule[-0.200pt]{0.964pt}{0.400pt}}
\put(725,661.67){\rule{0.964pt}{0.400pt}}
\multiput(725.00,661.17)(2.000,1.000){2}{\rule{0.482pt}{0.400pt}}
\put(721.0,662.0){\rule[-0.200pt]{0.964pt}{0.400pt}}
\put(729.0,663.0){\rule[-0.200pt]{0.964pt}{0.400pt}}
\put(733.0,663.0){\rule[-0.200pt]{0.964pt}{0.400pt}}
\put(741,661.67){\rule{0.964pt}{0.400pt}}
\multiput(741.00,662.17)(2.000,-1.000){2}{\rule{0.482pt}{0.400pt}}
\put(737.0,663.0){\rule[-0.200pt]{0.964pt}{0.400pt}}
\put(749,660.67){\rule{0.964pt}{0.400pt}}
\multiput(749.00,661.17)(2.000,-1.000){2}{\rule{0.482pt}{0.400pt}}
\put(753,659.67){\rule{0.964pt}{0.400pt}}
\multiput(753.00,660.17)(2.000,-1.000){2}{\rule{0.482pt}{0.400pt}}
\put(757,658.67){\rule{0.964pt}{0.400pt}}
\multiput(757.00,659.17)(2.000,-1.000){2}{\rule{0.482pt}{0.400pt}}
\put(761,657.67){\rule{0.964pt}{0.400pt}}
\multiput(761.00,658.17)(2.000,-1.000){2}{\rule{0.482pt}{0.400pt}}
\put(765,656.17){\rule{0.900pt}{0.400pt}}
\multiput(765.00,657.17)(2.132,-2.000){2}{\rule{0.450pt}{0.400pt}}
\put(769,654.67){\rule{0.964pt}{0.400pt}}
\multiput(769.00,655.17)(2.000,-1.000){2}{\rule{0.482pt}{0.400pt}}
\put(773,653.17){\rule{0.900pt}{0.400pt}}
\multiput(773.00,654.17)(2.132,-2.000){2}{\rule{0.450pt}{0.400pt}}
\put(777,651.17){\rule{0.900pt}{0.400pt}}
\multiput(777.00,652.17)(2.132,-2.000){2}{\rule{0.450pt}{0.400pt}}
\put(781,649.17){\rule{0.900pt}{0.400pt}}
\multiput(781.00,650.17)(2.132,-2.000){2}{\rule{0.450pt}{0.400pt}}
\multiput(785.00,647.95)(0.685,-0.447){3}{\rule{0.633pt}{0.108pt}}
\multiput(785.00,648.17)(2.685,-3.000){2}{\rule{0.317pt}{0.400pt}}
\put(789,644.17){\rule{0.900pt}{0.400pt}}
\multiput(789.00,645.17)(2.132,-2.000){2}{\rule{0.450pt}{0.400pt}}
\multiput(793.00,642.95)(0.685,-0.447){3}{\rule{0.633pt}{0.108pt}}
\multiput(793.00,643.17)(2.685,-3.000){2}{\rule{0.317pt}{0.400pt}}
\multiput(797.00,639.95)(0.685,-0.447){3}{\rule{0.633pt}{0.108pt}}
\multiput(797.00,640.17)(2.685,-3.000){2}{\rule{0.317pt}{0.400pt}}
\multiput(801.00,636.94)(0.481,-0.468){5}{\rule{0.500pt}{0.113pt}}
\multiput(801.00,637.17)(2.962,-4.000){2}{\rule{0.250pt}{0.400pt}}
\multiput(805.00,632.95)(0.685,-0.447){3}{\rule{0.633pt}{0.108pt}}
\multiput(805.00,633.17)(2.685,-3.000){2}{\rule{0.317pt}{0.400pt}}
\multiput(809.00,629.94)(0.481,-0.468){5}{\rule{0.500pt}{0.113pt}}
\multiput(809.00,630.17)(2.962,-4.000){2}{\rule{0.250pt}{0.400pt}}
\multiput(813.00,625.95)(0.685,-0.447){3}{\rule{0.633pt}{0.108pt}}
\multiput(813.00,626.17)(2.685,-3.000){2}{\rule{0.317pt}{0.400pt}}
\multiput(817.00,622.94)(0.481,-0.468){5}{\rule{0.500pt}{0.113pt}}
\multiput(817.00,623.17)(2.962,-4.000){2}{\rule{0.250pt}{0.400pt}}
\multiput(821.60,617.51)(0.468,-0.627){5}{\rule{0.113pt}{0.600pt}}
\multiput(820.17,618.75)(4.000,-3.755){2}{\rule{0.400pt}{0.300pt}}
\multiput(825.00,613.94)(0.481,-0.468){5}{\rule{0.500pt}{0.113pt}}
\multiput(825.00,614.17)(2.962,-4.000){2}{\rule{0.250pt}{0.400pt}}
\multiput(829.60,608.51)(0.468,-0.627){5}{\rule{0.113pt}{0.600pt}}
\multiput(828.17,609.75)(4.000,-3.755){2}{\rule{0.400pt}{0.300pt}}
\multiput(833.60,603.51)(0.468,-0.627){5}{\rule{0.113pt}{0.600pt}}
\multiput(832.17,604.75)(4.000,-3.755){2}{\rule{0.400pt}{0.300pt}}
\multiput(837.60,598.51)(0.468,-0.627){5}{\rule{0.113pt}{0.600pt}}
\multiput(836.17,599.75)(4.000,-3.755){2}{\rule{0.400pt}{0.300pt}}
\multiput(841.60,593.51)(0.468,-0.627){5}{\rule{0.113pt}{0.600pt}}
\multiput(840.17,594.75)(4.000,-3.755){2}{\rule{0.400pt}{0.300pt}}
\multiput(845.60,588.09)(0.468,-0.774){5}{\rule{0.113pt}{0.700pt}}
\multiput(844.17,589.55)(4.000,-4.547){2}{\rule{0.400pt}{0.350pt}}
\multiput(849.60,582.09)(0.468,-0.774){5}{\rule{0.113pt}{0.700pt}}
\multiput(848.17,583.55)(4.000,-4.547){2}{\rule{0.400pt}{0.350pt}}
\multiput(853.60,576.09)(0.468,-0.774){5}{\rule{0.113pt}{0.700pt}}
\multiput(852.17,577.55)(4.000,-4.547){2}{\rule{0.400pt}{0.350pt}}
\multiput(857.60,570.09)(0.468,-0.774){5}{\rule{0.113pt}{0.700pt}}
\multiput(856.17,571.55)(4.000,-4.547){2}{\rule{0.400pt}{0.350pt}}
\multiput(861.60,564.09)(0.468,-0.774){5}{\rule{0.113pt}{0.700pt}}
\multiput(860.17,565.55)(4.000,-4.547){2}{\rule{0.400pt}{0.350pt}}
\multiput(865.60,557.68)(0.468,-0.920){5}{\rule{0.113pt}{0.800pt}}
\multiput(864.17,559.34)(4.000,-5.340){2}{\rule{0.400pt}{0.400pt}}
\multiput(869.60,550.68)(0.468,-0.920){5}{\rule{0.113pt}{0.800pt}}
\multiput(868.17,552.34)(4.000,-5.340){2}{\rule{0.400pt}{0.400pt}}
\multiput(873.60,543.68)(0.468,-0.920){5}{\rule{0.113pt}{0.800pt}}
\multiput(872.17,545.34)(4.000,-5.340){2}{\rule{0.400pt}{0.400pt}}
\multiput(877.60,536.26)(0.468,-1.066){5}{\rule{0.113pt}{0.900pt}}
\multiput(876.17,538.13)(4.000,-6.132){2}{\rule{0.400pt}{0.450pt}}
\multiput(881.60,528.26)(0.468,-1.066){5}{\rule{0.113pt}{0.900pt}}
\multiput(880.17,530.13)(4.000,-6.132){2}{\rule{0.400pt}{0.450pt}}
\multiput(885.60,520.26)(0.468,-1.066){5}{\rule{0.113pt}{0.900pt}}
\multiput(884.17,522.13)(4.000,-6.132){2}{\rule{0.400pt}{0.450pt}}
\multiput(889.60,512.26)(0.468,-1.066){5}{\rule{0.113pt}{0.900pt}}
\multiput(888.17,514.13)(4.000,-6.132){2}{\rule{0.400pt}{0.450pt}}
\multiput(535.60,321.00)(0.468,1.651){5}{\rule{0.113pt}{1.300pt}}
\multiput(534.17,321.00)(4.000,9.302){2}{\rule{0.400pt}{0.650pt}}
\multiput(539.60,333.00)(0.468,1.651){5}{\rule{0.113pt}{1.300pt}}
\multiput(538.17,333.00)(4.000,9.302){2}{\rule{0.400pt}{0.650pt}}
\multiput(543.61,345.00)(0.447,2.472){3}{\rule{0.108pt}{1.700pt}}
\multiput(542.17,345.00)(3.000,8.472){2}{\rule{0.400pt}{0.850pt}}
\multiput(546.60,357.00)(0.468,1.505){5}{\rule{0.113pt}{1.200pt}}
\multiput(545.17,357.00)(4.000,8.509){2}{\rule{0.400pt}{0.600pt}}
\multiput(550.60,368.00)(0.468,1.505){5}{\rule{0.113pt}{1.200pt}}
\multiput(549.17,368.00)(4.000,8.509){2}{\rule{0.400pt}{0.600pt}}
\multiput(554.60,379.00)(0.468,1.505){5}{\rule{0.113pt}{1.200pt}}
\multiput(553.17,379.00)(4.000,8.509){2}{\rule{0.400pt}{0.600pt}}
\multiput(558.60,390.00)(0.468,1.505){5}{\rule{0.113pt}{1.200pt}}
\multiput(557.17,390.00)(4.000,8.509){2}{\rule{0.400pt}{0.600pt}}
\multiput(562.60,401.00)(0.468,1.358){5}{\rule{0.113pt}{1.100pt}}
\multiput(561.17,401.00)(4.000,7.717){2}{\rule{0.400pt}{0.550pt}}
\multiput(566.60,411.00)(0.468,1.358){5}{\rule{0.113pt}{1.100pt}}
\multiput(565.17,411.00)(4.000,7.717){2}{\rule{0.400pt}{0.550pt}}
\multiput(570.60,421.00)(0.468,1.358){5}{\rule{0.113pt}{1.100pt}}
\multiput(569.17,421.00)(4.000,7.717){2}{\rule{0.400pt}{0.550pt}}
\multiput(574.60,431.00)(0.468,1.358){5}{\rule{0.113pt}{1.100pt}}
\multiput(573.17,431.00)(4.000,7.717){2}{\rule{0.400pt}{0.550pt}}
\multiput(578.60,441.00)(0.468,1.212){5}{\rule{0.113pt}{1.000pt}}
\multiput(577.17,441.00)(4.000,6.924){2}{\rule{0.400pt}{0.500pt}}
\multiput(582.60,450.00)(0.468,1.212){5}{\rule{0.113pt}{1.000pt}}
\multiput(581.17,450.00)(4.000,6.924){2}{\rule{0.400pt}{0.500pt}}
\multiput(586.60,459.00)(0.468,1.212){5}{\rule{0.113pt}{1.000pt}}
\multiput(585.17,459.00)(4.000,6.924){2}{\rule{0.400pt}{0.500pt}}
\multiput(590.60,468.00)(0.468,1.212){5}{\rule{0.113pt}{1.000pt}}
\multiput(589.17,468.00)(4.000,6.924){2}{\rule{0.400pt}{0.500pt}}
\multiput(594.60,477.00)(0.468,1.066){5}{\rule{0.113pt}{0.900pt}}
\multiput(593.17,477.00)(4.000,6.132){2}{\rule{0.400pt}{0.450pt}}
\multiput(598.60,485.00)(0.468,1.212){5}{\rule{0.113pt}{1.000pt}}
\multiput(597.17,485.00)(4.000,6.924){2}{\rule{0.400pt}{0.500pt}}
\multiput(602.60,494.00)(0.468,1.066){5}{\rule{0.113pt}{0.900pt}}
\multiput(601.17,494.00)(4.000,6.132){2}{\rule{0.400pt}{0.450pt}}
\multiput(606.60,502.00)(0.468,0.920){5}{\rule{0.113pt}{0.800pt}}
\multiput(605.17,502.00)(4.000,5.340){2}{\rule{0.400pt}{0.400pt}}
\multiput(610.60,509.00)(0.468,1.066){5}{\rule{0.113pt}{0.900pt}}
\multiput(609.17,509.00)(4.000,6.132){2}{\rule{0.400pt}{0.450pt}}
\multiput(614.60,517.00)(0.468,1.066){5}{\rule{0.113pt}{0.900pt}}
\multiput(613.17,517.00)(4.000,6.132){2}{\rule{0.400pt}{0.450pt}}
\multiput(618.60,525.00)(0.468,0.920){5}{\rule{0.113pt}{0.800pt}}
\multiput(617.17,525.00)(4.000,5.340){2}{\rule{0.400pt}{0.400pt}}
\multiput(622.60,532.00)(0.468,0.920){5}{\rule{0.113pt}{0.800pt}}
\multiput(621.17,532.00)(4.000,5.340){2}{\rule{0.400pt}{0.400pt}}
\multiput(626.60,539.00)(0.468,0.920){5}{\rule{0.113pt}{0.800pt}}
\multiput(625.17,539.00)(4.000,5.340){2}{\rule{0.400pt}{0.400pt}}
\multiput(630.60,546.00)(0.468,0.774){5}{\rule{0.113pt}{0.700pt}}
\multiput(629.17,546.00)(4.000,4.547){2}{\rule{0.400pt}{0.350pt}}
\multiput(634.60,552.00)(0.468,0.774){5}{\rule{0.113pt}{0.700pt}}
\multiput(633.17,552.00)(4.000,4.547){2}{\rule{0.400pt}{0.350pt}}
\multiput(638.60,558.00)(0.468,0.920){5}{\rule{0.113pt}{0.800pt}}
\multiput(637.17,558.00)(4.000,5.340){2}{\rule{0.400pt}{0.400pt}}
\multiput(642.60,565.00)(0.468,0.774){5}{\rule{0.113pt}{0.700pt}}
\multiput(641.17,565.00)(4.000,4.547){2}{\rule{0.400pt}{0.350pt}}
\multiput(646.60,571.00)(0.468,0.627){5}{\rule{0.113pt}{0.600pt}}
\multiput(645.17,571.00)(4.000,3.755){2}{\rule{0.400pt}{0.300pt}}
\multiput(650.60,576.00)(0.468,0.774){5}{\rule{0.113pt}{0.700pt}}
\multiput(649.17,576.00)(4.000,4.547){2}{\rule{0.400pt}{0.350pt}}
\multiput(654.60,582.00)(0.468,0.627){5}{\rule{0.113pt}{0.600pt}}
\multiput(653.17,582.00)(4.000,3.755){2}{\rule{0.400pt}{0.300pt}}
\multiput(658.60,587.00)(0.468,0.774){5}{\rule{0.113pt}{0.700pt}}
\multiput(657.17,587.00)(4.000,4.547){2}{\rule{0.400pt}{0.350pt}}
\multiput(662.60,593.00)(0.468,0.627){5}{\rule{0.113pt}{0.600pt}}
\multiput(661.17,593.00)(4.000,3.755){2}{\rule{0.400pt}{0.300pt}}
\multiput(666.00,598.60)(0.481,0.468){5}{\rule{0.500pt}{0.113pt}}
\multiput(666.00,597.17)(2.962,4.000){2}{\rule{0.250pt}{0.400pt}}
\multiput(670.60,602.00)(0.468,0.627){5}{\rule{0.113pt}{0.600pt}}
\multiput(669.17,602.00)(4.000,3.755){2}{\rule{0.400pt}{0.300pt}}
\multiput(674.00,607.60)(0.481,0.468){5}{\rule{0.500pt}{0.113pt}}
\multiput(674.00,606.17)(2.962,4.000){2}{\rule{0.250pt}{0.400pt}}
\multiput(678.60,611.00)(0.468,0.627){5}{\rule{0.113pt}{0.600pt}}
\multiput(677.17,611.00)(4.000,3.755){2}{\rule{0.400pt}{0.300pt}}
\multiput(682.00,616.60)(0.481,0.468){5}{\rule{0.500pt}{0.113pt}}
\multiput(682.00,615.17)(2.962,4.000){2}{\rule{0.250pt}{0.400pt}}
\multiput(686.00,620.60)(0.481,0.468){5}{\rule{0.500pt}{0.113pt}}
\multiput(686.00,619.17)(2.962,4.000){2}{\rule{0.250pt}{0.400pt}}
\multiput(690.00,624.61)(0.685,0.447){3}{\rule{0.633pt}{0.108pt}}
\multiput(690.00,623.17)(2.685,3.000){2}{\rule{0.317pt}{0.400pt}}
\multiput(694.00,627.60)(0.481,0.468){5}{\rule{0.500pt}{0.113pt}}
\multiput(694.00,626.17)(2.962,4.000){2}{\rule{0.250pt}{0.400pt}}
\multiput(698.00,631.61)(0.685,0.447){3}{\rule{0.633pt}{0.108pt}}
\multiput(698.00,630.17)(2.685,3.000){2}{\rule{0.317pt}{0.400pt}}
\multiput(702.00,634.61)(0.685,0.447){3}{\rule{0.633pt}{0.108pt}}
\multiput(702.00,633.17)(2.685,3.000){2}{\rule{0.317pt}{0.400pt}}
\multiput(706.00,637.61)(0.685,0.447){3}{\rule{0.633pt}{0.108pt}}
\multiput(706.00,636.17)(2.685,3.000){2}{\rule{0.317pt}{0.400pt}}
\multiput(710.00,640.61)(0.685,0.447){3}{\rule{0.633pt}{0.108pt}}
\multiput(710.00,639.17)(2.685,3.000){2}{\rule{0.317pt}{0.400pt}}
\multiput(714.00,643.61)(0.685,0.447){3}{\rule{0.633pt}{0.108pt}}
\multiput(714.00,642.17)(2.685,3.000){2}{\rule{0.317pt}{0.400pt}}
\put(718,646.17){\rule{0.900pt}{0.400pt}}
\multiput(718.00,645.17)(2.132,2.000){2}{\rule{0.450pt}{0.400pt}}
\put(722,648.17){\rule{0.900pt}{0.400pt}}
\multiput(722.00,647.17)(2.132,2.000){2}{\rule{0.450pt}{0.400pt}}
\put(726,650.17){\rule{0.900pt}{0.400pt}}
\multiput(726.00,649.17)(2.132,2.000){2}{\rule{0.450pt}{0.400pt}}
\put(730,652.17){\rule{0.900pt}{0.400pt}}
\multiput(730.00,651.17)(2.132,2.000){2}{\rule{0.450pt}{0.400pt}}
\put(734,654.17){\rule{0.900pt}{0.400pt}}
\multiput(734.00,653.17)(2.132,2.000){2}{\rule{0.450pt}{0.400pt}}
\put(738,655.67){\rule{0.723pt}{0.400pt}}
\multiput(738.00,655.17)(1.500,1.000){2}{\rule{0.361pt}{0.400pt}}
\put(741,656.67){\rule{0.964pt}{0.400pt}}
\multiput(741.00,656.17)(2.000,1.000){2}{\rule{0.482pt}{0.400pt}}
\put(745,657.67){\rule{0.964pt}{0.400pt}}
\multiput(745.00,657.17)(2.000,1.000){2}{\rule{0.482pt}{0.400pt}}
\put(749,658.67){\rule{0.964pt}{0.400pt}}
\multiput(749.00,658.17)(2.000,1.000){2}{\rule{0.482pt}{0.400pt}}
\put(753,659.67){\rule{0.964pt}{0.400pt}}
\multiput(753.00,659.17)(2.000,1.000){2}{\rule{0.482pt}{0.400pt}}
\put(757,660.67){\rule{0.964pt}{0.400pt}}
\multiput(757.00,660.17)(2.000,1.000){2}{\rule{0.482pt}{0.400pt}}
\put(745.0,662.0){\rule[-0.200pt]{0.964pt}{0.400pt}}
\put(761.0,662.0){\rule[-0.200pt]{0.964pt}{0.400pt}}
\put(765.0,662.0){\rule[-0.200pt]{0.964pt}{0.400pt}}
\put(769.0,662.0){\rule[-0.200pt]{0.964pt}{0.400pt}}
\put(773.0,662.0){\rule[-0.200pt]{0.964pt}{0.400pt}}
\put(781,660.67){\rule{0.964pt}{0.400pt}}
\multiput(781.00,661.17)(2.000,-1.000){2}{\rule{0.482pt}{0.400pt}}
\put(785,659.67){\rule{0.964pt}{0.400pt}}
\multiput(785.00,660.17)(2.000,-1.000){2}{\rule{0.482pt}{0.400pt}}
\put(789,658.67){\rule{0.964pt}{0.400pt}}
\multiput(789.00,659.17)(2.000,-1.000){2}{\rule{0.482pt}{0.400pt}}
\put(793,657.67){\rule{0.964pt}{0.400pt}}
\multiput(793.00,658.17)(2.000,-1.000){2}{\rule{0.482pt}{0.400pt}}
\put(797,656.67){\rule{0.964pt}{0.400pt}}
\multiput(797.00,657.17)(2.000,-1.000){2}{\rule{0.482pt}{0.400pt}}
\put(801,655.17){\rule{0.900pt}{0.400pt}}
\multiput(801.00,656.17)(2.132,-2.000){2}{\rule{0.450pt}{0.400pt}}
\put(805,653.17){\rule{0.900pt}{0.400pt}}
\multiput(805.00,654.17)(2.132,-2.000){2}{\rule{0.450pt}{0.400pt}}
\put(809,651.67){\rule{0.964pt}{0.400pt}}
\multiput(809.00,652.17)(2.000,-1.000){2}{\rule{0.482pt}{0.400pt}}
\multiput(813.00,650.95)(0.685,-0.447){3}{\rule{0.633pt}{0.108pt}}
\multiput(813.00,651.17)(2.685,-3.000){2}{\rule{0.317pt}{0.400pt}}
\put(817,647.17){\rule{0.900pt}{0.400pt}}
\multiput(817.00,648.17)(2.132,-2.000){2}{\rule{0.450pt}{0.400pt}}
\put(821,645.17){\rule{0.900pt}{0.400pt}}
\multiput(821.00,646.17)(2.132,-2.000){2}{\rule{0.450pt}{0.400pt}}
\multiput(825.00,643.95)(0.685,-0.447){3}{\rule{0.633pt}{0.108pt}}
\multiput(825.00,644.17)(2.685,-3.000){2}{\rule{0.317pt}{0.400pt}}
\multiput(829.00,640.95)(0.685,-0.447){3}{\rule{0.633pt}{0.108pt}}
\multiput(829.00,641.17)(2.685,-3.000){2}{\rule{0.317pt}{0.400pt}}
\multiput(833.00,637.95)(0.685,-0.447){3}{\rule{0.633pt}{0.108pt}}
\multiput(833.00,638.17)(2.685,-3.000){2}{\rule{0.317pt}{0.400pt}}
\multiput(837.00,634.95)(0.685,-0.447){3}{\rule{0.633pt}{0.108pt}}
\multiput(837.00,635.17)(2.685,-3.000){2}{\rule{0.317pt}{0.400pt}}
\multiput(841.00,631.94)(0.481,-0.468){5}{\rule{0.500pt}{0.113pt}}
\multiput(841.00,632.17)(2.962,-4.000){2}{\rule{0.250pt}{0.400pt}}
\multiput(845.00,627.94)(0.481,-0.468){5}{\rule{0.500pt}{0.113pt}}
\multiput(845.00,628.17)(2.962,-4.000){2}{\rule{0.250pt}{0.400pt}}
\multiput(849.00,623.94)(0.481,-0.468){5}{\rule{0.500pt}{0.113pt}}
\multiput(849.00,624.17)(2.962,-4.000){2}{\rule{0.250pt}{0.400pt}}
\multiput(853.00,619.94)(0.481,-0.468){5}{\rule{0.500pt}{0.113pt}}
\multiput(853.00,620.17)(2.962,-4.000){2}{\rule{0.250pt}{0.400pt}}
\multiput(857.00,615.94)(0.481,-0.468){5}{\rule{0.500pt}{0.113pt}}
\multiput(857.00,616.17)(2.962,-4.000){2}{\rule{0.250pt}{0.400pt}}
\multiput(861.60,610.51)(0.468,-0.627){5}{\rule{0.113pt}{0.600pt}}
\multiput(860.17,611.75)(4.000,-3.755){2}{\rule{0.400pt}{0.300pt}}
\multiput(865.60,605.51)(0.468,-0.627){5}{\rule{0.113pt}{0.600pt}}
\multiput(864.17,606.75)(4.000,-3.755){2}{\rule{0.400pt}{0.300pt}}
\multiput(869.60,600.51)(0.468,-0.627){5}{\rule{0.113pt}{0.600pt}}
\multiput(868.17,601.75)(4.000,-3.755){2}{\rule{0.400pt}{0.300pt}}
\multiput(873.60,595.51)(0.468,-0.627){5}{\rule{0.113pt}{0.600pt}}
\multiput(872.17,596.75)(4.000,-3.755){2}{\rule{0.400pt}{0.300pt}}
\multiput(877.60,590.09)(0.468,-0.774){5}{\rule{0.113pt}{0.700pt}}
\multiput(876.17,591.55)(4.000,-4.547){2}{\rule{0.400pt}{0.350pt}}
\multiput(881.60,584.51)(0.468,-0.627){5}{\rule{0.113pt}{0.600pt}}
\multiput(880.17,585.75)(4.000,-3.755){2}{\rule{0.400pt}{0.300pt}}
\multiput(885.60,579.09)(0.468,-0.774){5}{\rule{0.113pt}{0.700pt}}
\multiput(884.17,580.55)(4.000,-4.547){2}{\rule{0.400pt}{0.350pt}}
\multiput(889.60,573.09)(0.468,-0.774){5}{\rule{0.113pt}{0.700pt}}
\multiput(888.17,574.55)(4.000,-4.547){2}{\rule{0.400pt}{0.350pt}}
\multiput(893.60,566.68)(0.468,-0.920){5}{\rule{0.113pt}{0.800pt}}
\multiput(892.17,568.34)(4.000,-5.340){2}{\rule{0.400pt}{0.400pt}}
\multiput(897.60,559.68)(0.468,-0.920){5}{\rule{0.113pt}{0.800pt}}
\multiput(896.17,561.34)(4.000,-5.340){2}{\rule{0.400pt}{0.400pt}}
\multiput(901.60,552.68)(0.468,-0.920){5}{\rule{0.113pt}{0.800pt}}
\multiput(900.17,554.34)(4.000,-5.340){2}{\rule{0.400pt}{0.400pt}}
\multiput(905.60,545.68)(0.468,-0.920){5}{\rule{0.113pt}{0.800pt}}
\multiput(904.17,547.34)(4.000,-5.340){2}{\rule{0.400pt}{0.400pt}}
\multiput(909.60,538.68)(0.468,-0.920){5}{\rule{0.113pt}{0.800pt}}
\multiput(908.17,540.34)(4.000,-5.340){2}{\rule{0.400pt}{0.400pt}}
\multiput(913.60,531.26)(0.468,-1.066){5}{\rule{0.113pt}{0.900pt}}
\multiput(912.17,533.13)(4.000,-6.132){2}{\rule{0.400pt}{0.450pt}}
\multiput(917.60,523.26)(0.468,-1.066){5}{\rule{0.113pt}{0.900pt}}
\multiput(916.17,525.13)(4.000,-6.132){2}{\rule{0.400pt}{0.450pt}}
\multiput(921.60,515.26)(0.468,-1.066){5}{\rule{0.113pt}{0.900pt}}
\multiput(920.17,517.13)(4.000,-6.132){2}{\rule{0.400pt}{0.450pt}}
\multiput(925.60,506.85)(0.468,-1.212){5}{\rule{0.113pt}{1.000pt}}
\multiput(924.17,508.92)(4.000,-6.924){2}{\rule{0.400pt}{0.500pt}}
\multiput(570.60,316.00)(0.468,1.651){5}{\rule{0.113pt}{1.300pt}}
\multiput(569.17,316.00)(4.000,9.302){2}{\rule{0.400pt}{0.650pt}}
\multiput(574.60,328.00)(0.468,1.651){5}{\rule{0.113pt}{1.300pt}}
\multiput(573.17,328.00)(4.000,9.302){2}{\rule{0.400pt}{0.650pt}}
\multiput(578.60,340.00)(0.468,1.505){5}{\rule{0.113pt}{1.200pt}}
\multiput(577.17,340.00)(4.000,8.509){2}{\rule{0.400pt}{0.600pt}}
\multiput(582.60,351.00)(0.468,1.651){5}{\rule{0.113pt}{1.300pt}}
\multiput(581.17,351.00)(4.000,9.302){2}{\rule{0.400pt}{0.650pt}}
\multiput(586.60,363.00)(0.468,1.505){5}{\rule{0.113pt}{1.200pt}}
\multiput(585.17,363.00)(4.000,8.509){2}{\rule{0.400pt}{0.600pt}}
\multiput(590.60,374.00)(0.468,1.505){5}{\rule{0.113pt}{1.200pt}}
\multiput(589.17,374.00)(4.000,8.509){2}{\rule{0.400pt}{0.600pt}}
\multiput(594.60,385.00)(0.468,1.358){5}{\rule{0.113pt}{1.100pt}}
\multiput(593.17,385.00)(4.000,7.717){2}{\rule{0.400pt}{0.550pt}}
\multiput(598.60,395.00)(0.468,1.358){5}{\rule{0.113pt}{1.100pt}}
\multiput(597.17,395.00)(4.000,7.717){2}{\rule{0.400pt}{0.550pt}}
\multiput(602.60,405.00)(0.468,1.505){5}{\rule{0.113pt}{1.200pt}}
\multiput(601.17,405.00)(4.000,8.509){2}{\rule{0.400pt}{0.600pt}}
\multiput(606.60,416.00)(0.468,1.212){5}{\rule{0.113pt}{1.000pt}}
\multiput(605.17,416.00)(4.000,6.924){2}{\rule{0.400pt}{0.500pt}}
\multiput(610.60,425.00)(0.468,1.358){5}{\rule{0.113pt}{1.100pt}}
\multiput(609.17,425.00)(4.000,7.717){2}{\rule{0.400pt}{0.550pt}}
\multiput(614.60,435.00)(0.468,1.212){5}{\rule{0.113pt}{1.000pt}}
\multiput(613.17,435.00)(4.000,6.924){2}{\rule{0.400pt}{0.500pt}}
\multiput(618.60,444.00)(0.468,1.358){5}{\rule{0.113pt}{1.100pt}}
\multiput(617.17,444.00)(4.000,7.717){2}{\rule{0.400pt}{0.550pt}}
\multiput(622.60,454.00)(0.468,1.212){5}{\rule{0.113pt}{1.000pt}}
\multiput(621.17,454.00)(4.000,6.924){2}{\rule{0.400pt}{0.500pt}}
\multiput(626.60,463.00)(0.468,1.066){5}{\rule{0.113pt}{0.900pt}}
\multiput(625.17,463.00)(4.000,6.132){2}{\rule{0.400pt}{0.450pt}}
\multiput(630.60,471.00)(0.468,1.212){5}{\rule{0.113pt}{1.000pt}}
\multiput(629.17,471.00)(4.000,6.924){2}{\rule{0.400pt}{0.500pt}}
\multiput(634.60,480.00)(0.468,1.066){5}{\rule{0.113pt}{0.900pt}}
\multiput(633.17,480.00)(4.000,6.132){2}{\rule{0.400pt}{0.450pt}}
\multiput(638.60,488.00)(0.468,1.066){5}{\rule{0.113pt}{0.900pt}}
\multiput(637.17,488.00)(4.000,6.132){2}{\rule{0.400pt}{0.450pt}}
\multiput(642.60,496.00)(0.468,1.066){5}{\rule{0.113pt}{0.900pt}}
\multiput(641.17,496.00)(4.000,6.132){2}{\rule{0.400pt}{0.450pt}}
\multiput(646.60,504.00)(0.468,1.066){5}{\rule{0.113pt}{0.900pt}}
\multiput(645.17,504.00)(4.000,6.132){2}{\rule{0.400pt}{0.450pt}}
\multiput(650.60,512.00)(0.468,0.920){5}{\rule{0.113pt}{0.800pt}}
\multiput(649.17,512.00)(4.000,5.340){2}{\rule{0.400pt}{0.400pt}}
\multiput(654.60,519.00)(0.468,0.920){5}{\rule{0.113pt}{0.800pt}}
\multiput(653.17,519.00)(4.000,5.340){2}{\rule{0.400pt}{0.400pt}}
\multiput(658.60,526.00)(0.468,0.920){5}{\rule{0.113pt}{0.800pt}}
\multiput(657.17,526.00)(4.000,5.340){2}{\rule{0.400pt}{0.400pt}}
\multiput(662.60,533.00)(0.468,0.920){5}{\rule{0.113pt}{0.800pt}}
\multiput(661.17,533.00)(4.000,5.340){2}{\rule{0.400pt}{0.400pt}}
\multiput(666.60,540.00)(0.468,0.920){5}{\rule{0.113pt}{0.800pt}}
\multiput(665.17,540.00)(4.000,5.340){2}{\rule{0.400pt}{0.400pt}}
\multiput(670.60,547.00)(0.468,0.774){5}{\rule{0.113pt}{0.700pt}}
\multiput(669.17,547.00)(4.000,4.547){2}{\rule{0.400pt}{0.350pt}}
\multiput(674.60,553.00)(0.468,0.774){5}{\rule{0.113pt}{0.700pt}}
\multiput(673.17,553.00)(4.000,4.547){2}{\rule{0.400pt}{0.350pt}}
\multiput(678.60,559.00)(0.468,0.774){5}{\rule{0.113pt}{0.700pt}}
\multiput(677.17,559.00)(4.000,4.547){2}{\rule{0.400pt}{0.350pt}}
\multiput(682.60,565.00)(0.468,0.774){5}{\rule{0.113pt}{0.700pt}}
\multiput(681.17,565.00)(4.000,4.547){2}{\rule{0.400pt}{0.350pt}}
\multiput(686.60,571.00)(0.468,0.774){5}{\rule{0.113pt}{0.700pt}}
\multiput(685.17,571.00)(4.000,4.547){2}{\rule{0.400pt}{0.350pt}}
\multiput(690.60,577.00)(0.468,0.627){5}{\rule{0.113pt}{0.600pt}}
\multiput(689.17,577.00)(4.000,3.755){2}{\rule{0.400pt}{0.300pt}}
\multiput(694.60,582.00)(0.468,0.627){5}{\rule{0.113pt}{0.600pt}}
\multiput(693.17,582.00)(4.000,3.755){2}{\rule{0.400pt}{0.300pt}}
\multiput(698.60,587.00)(0.468,0.627){5}{\rule{0.113pt}{0.600pt}}
\multiput(697.17,587.00)(4.000,3.755){2}{\rule{0.400pt}{0.300pt}}
\multiput(702.60,592.00)(0.468,0.627){5}{\rule{0.113pt}{0.600pt}}
\multiput(701.17,592.00)(4.000,3.755){2}{\rule{0.400pt}{0.300pt}}
\multiput(706.60,597.00)(0.468,0.627){5}{\rule{0.113pt}{0.600pt}}
\multiput(705.17,597.00)(4.000,3.755){2}{\rule{0.400pt}{0.300pt}}
\multiput(710.00,602.60)(0.481,0.468){5}{\rule{0.500pt}{0.113pt}}
\multiput(710.00,601.17)(2.962,4.000){2}{\rule{0.250pt}{0.400pt}}
\multiput(714.00,606.60)(0.481,0.468){5}{\rule{0.500pt}{0.113pt}}
\multiput(714.00,605.17)(2.962,4.000){2}{\rule{0.250pt}{0.400pt}}
\multiput(718.00,610.60)(0.481,0.468){5}{\rule{0.500pt}{0.113pt}}
\multiput(718.00,609.17)(2.962,4.000){2}{\rule{0.250pt}{0.400pt}}
\multiput(722.00,614.60)(0.481,0.468){5}{\rule{0.500pt}{0.113pt}}
\multiput(722.00,613.17)(2.962,4.000){2}{\rule{0.250pt}{0.400pt}}
\multiput(726.00,618.60)(0.481,0.468){5}{\rule{0.500pt}{0.113pt}}
\multiput(726.00,617.17)(2.962,4.000){2}{\rule{0.250pt}{0.400pt}}
\multiput(730.00,622.61)(0.685,0.447){3}{\rule{0.633pt}{0.108pt}}
\multiput(730.00,621.17)(2.685,3.000){2}{\rule{0.317pt}{0.400pt}}
\multiput(734.00,625.60)(0.481,0.468){5}{\rule{0.500pt}{0.113pt}}
\multiput(734.00,624.17)(2.962,4.000){2}{\rule{0.250pt}{0.400pt}}
\multiput(738.00,629.61)(0.685,0.447){3}{\rule{0.633pt}{0.108pt}}
\multiput(738.00,628.17)(2.685,3.000){2}{\rule{0.317pt}{0.400pt}}
\multiput(742.00,632.61)(0.685,0.447){3}{\rule{0.633pt}{0.108pt}}
\multiput(742.00,631.17)(2.685,3.000){2}{\rule{0.317pt}{0.400pt}}
\put(746,635.17){\rule{0.900pt}{0.400pt}}
\multiput(746.00,634.17)(2.132,2.000){2}{\rule{0.450pt}{0.400pt}}
\multiput(750.00,637.61)(0.685,0.447){3}{\rule{0.633pt}{0.108pt}}
\multiput(750.00,636.17)(2.685,3.000){2}{\rule{0.317pt}{0.400pt}}
\put(754,640.17){\rule{0.900pt}{0.400pt}}
\multiput(754.00,639.17)(2.132,2.000){2}{\rule{0.450pt}{0.400pt}}
\multiput(758.00,642.61)(0.462,0.447){3}{\rule{0.500pt}{0.108pt}}
\multiput(758.00,641.17)(1.962,3.000){2}{\rule{0.250pt}{0.400pt}}
\put(761,645.17){\rule{0.900pt}{0.400pt}}
\multiput(761.00,644.17)(2.132,2.000){2}{\rule{0.450pt}{0.400pt}}
\put(765,646.67){\rule{0.964pt}{0.400pt}}
\multiput(765.00,646.17)(2.000,1.000){2}{\rule{0.482pt}{0.400pt}}
\put(769,648.17){\rule{0.900pt}{0.400pt}}
\multiput(769.00,647.17)(2.132,2.000){2}{\rule{0.450pt}{0.400pt}}
\put(773,650.17){\rule{0.900pt}{0.400pt}}
\multiput(773.00,649.17)(2.132,2.000){2}{\rule{0.450pt}{0.400pt}}
\put(777,651.67){\rule{0.964pt}{0.400pt}}
\multiput(777.00,651.17)(2.000,1.000){2}{\rule{0.482pt}{0.400pt}}
\put(781,652.67){\rule{0.964pt}{0.400pt}}
\multiput(781.00,652.17)(2.000,1.000){2}{\rule{0.482pt}{0.400pt}}
\put(785,653.67){\rule{0.964pt}{0.400pt}}
\multiput(785.00,653.17)(2.000,1.000){2}{\rule{0.482pt}{0.400pt}}
\put(789,654.67){\rule{0.964pt}{0.400pt}}
\multiput(789.00,654.17)(2.000,1.000){2}{\rule{0.482pt}{0.400pt}}
\put(777.0,662.0){\rule[-0.200pt]{0.964pt}{0.400pt}}
\put(793.0,656.0){\rule[-0.200pt]{0.964pt}{0.400pt}}
\put(801,655.67){\rule{0.964pt}{0.400pt}}
\multiput(801.00,655.17)(2.000,1.000){2}{\rule{0.482pt}{0.400pt}}
\put(797.0,656.0){\rule[-0.200pt]{0.964pt}{0.400pt}}
\put(809,655.67){\rule{0.964pt}{0.400pt}}
\multiput(809.00,656.17)(2.000,-1.000){2}{\rule{0.482pt}{0.400pt}}
\put(805.0,657.0){\rule[-0.200pt]{0.964pt}{0.400pt}}
\put(817,654.67){\rule{0.964pt}{0.400pt}}
\multiput(817.00,655.17)(2.000,-1.000){2}{\rule{0.482pt}{0.400pt}}
\put(813.0,656.0){\rule[-0.200pt]{0.964pt}{0.400pt}}
\put(825,653.67){\rule{0.964pt}{0.400pt}}
\multiput(825.00,654.17)(2.000,-1.000){2}{\rule{0.482pt}{0.400pt}}
\put(829,652.67){\rule{0.964pt}{0.400pt}}
\multiput(829.00,653.17)(2.000,-1.000){2}{\rule{0.482pt}{0.400pt}}
\put(833,651.17){\rule{0.900pt}{0.400pt}}
\multiput(833.00,652.17)(2.132,-2.000){2}{\rule{0.450pt}{0.400pt}}
\put(837,649.67){\rule{0.964pt}{0.400pt}}
\multiput(837.00,650.17)(2.000,-1.000){2}{\rule{0.482pt}{0.400pt}}
\put(841,648.17){\rule{0.900pt}{0.400pt}}
\multiput(841.00,649.17)(2.132,-2.000){2}{\rule{0.450pt}{0.400pt}}
\put(845,646.17){\rule{0.900pt}{0.400pt}}
\multiput(845.00,647.17)(2.132,-2.000){2}{\rule{0.450pt}{0.400pt}}
\put(849,644.17){\rule{0.900pt}{0.400pt}}
\multiput(849.00,645.17)(2.132,-2.000){2}{\rule{0.450pt}{0.400pt}}
\put(853,642.17){\rule{0.900pt}{0.400pt}}
\multiput(853.00,643.17)(2.132,-2.000){2}{\rule{0.450pt}{0.400pt}}
\multiput(857.00,640.95)(0.685,-0.447){3}{\rule{0.633pt}{0.108pt}}
\multiput(857.00,641.17)(2.685,-3.000){2}{\rule{0.317pt}{0.400pt}}
\multiput(861.00,637.95)(0.685,-0.447){3}{\rule{0.633pt}{0.108pt}}
\multiput(861.00,638.17)(2.685,-3.000){2}{\rule{0.317pt}{0.400pt}}
\multiput(865.00,634.95)(0.685,-0.447){3}{\rule{0.633pt}{0.108pt}}
\multiput(865.00,635.17)(2.685,-3.000){2}{\rule{0.317pt}{0.400pt}}
\multiput(869.00,631.95)(0.685,-0.447){3}{\rule{0.633pt}{0.108pt}}
\multiput(869.00,632.17)(2.685,-3.000){2}{\rule{0.317pt}{0.400pt}}
\multiput(873.00,628.95)(0.685,-0.447){3}{\rule{0.633pt}{0.108pt}}
\multiput(873.00,629.17)(2.685,-3.000){2}{\rule{0.317pt}{0.400pt}}
\multiput(877.00,625.94)(0.481,-0.468){5}{\rule{0.500pt}{0.113pt}}
\multiput(877.00,626.17)(2.962,-4.000){2}{\rule{0.250pt}{0.400pt}}
\multiput(881.00,621.95)(0.685,-0.447){3}{\rule{0.633pt}{0.108pt}}
\multiput(881.00,622.17)(2.685,-3.000){2}{\rule{0.317pt}{0.400pt}}
\multiput(885.00,618.94)(0.481,-0.468){5}{\rule{0.500pt}{0.113pt}}
\multiput(885.00,619.17)(2.962,-4.000){2}{\rule{0.250pt}{0.400pt}}
\multiput(889.00,614.94)(0.481,-0.468){5}{\rule{0.500pt}{0.113pt}}
\multiput(889.00,615.17)(2.962,-4.000){2}{\rule{0.250pt}{0.400pt}}
\multiput(893.60,609.51)(0.468,-0.627){5}{\rule{0.113pt}{0.600pt}}
\multiput(892.17,610.75)(4.000,-3.755){2}{\rule{0.400pt}{0.300pt}}
\multiput(897.00,605.94)(0.481,-0.468){5}{\rule{0.500pt}{0.113pt}}
\multiput(897.00,606.17)(2.962,-4.000){2}{\rule{0.250pt}{0.400pt}}
\multiput(901.60,600.51)(0.468,-0.627){5}{\rule{0.113pt}{0.600pt}}
\multiput(900.17,601.75)(4.000,-3.755){2}{\rule{0.400pt}{0.300pt}}
\multiput(905.60,595.51)(0.468,-0.627){5}{\rule{0.113pt}{0.600pt}}
\multiput(904.17,596.75)(4.000,-3.755){2}{\rule{0.400pt}{0.300pt}}
\multiput(909.60,590.09)(0.468,-0.774){5}{\rule{0.113pt}{0.700pt}}
\multiput(908.17,591.55)(4.000,-4.547){2}{\rule{0.400pt}{0.350pt}}
\multiput(913.60,584.51)(0.468,-0.627){5}{\rule{0.113pt}{0.600pt}}
\multiput(912.17,585.75)(4.000,-3.755){2}{\rule{0.400pt}{0.300pt}}
\multiput(917.60,579.09)(0.468,-0.774){5}{\rule{0.113pt}{0.700pt}}
\multiput(916.17,580.55)(4.000,-4.547){2}{\rule{0.400pt}{0.350pt}}
\multiput(921.60,573.09)(0.468,-0.774){5}{\rule{0.113pt}{0.700pt}}
\multiput(920.17,574.55)(4.000,-4.547){2}{\rule{0.400pt}{0.350pt}}
\multiput(925.60,567.09)(0.468,-0.774){5}{\rule{0.113pt}{0.700pt}}
\multiput(924.17,568.55)(4.000,-4.547){2}{\rule{0.400pt}{0.350pt}}
\multiput(929.60,561.09)(0.468,-0.774){5}{\rule{0.113pt}{0.700pt}}
\multiput(928.17,562.55)(4.000,-4.547){2}{\rule{0.400pt}{0.350pt}}
\multiput(933.60,554.68)(0.468,-0.920){5}{\rule{0.113pt}{0.800pt}}
\multiput(932.17,556.34)(4.000,-5.340){2}{\rule{0.400pt}{0.400pt}}
\multiput(937.60,547.68)(0.468,-0.920){5}{\rule{0.113pt}{0.800pt}}
\multiput(936.17,549.34)(4.000,-5.340){2}{\rule{0.400pt}{0.400pt}}
\multiput(941.60,540.68)(0.468,-0.920){5}{\rule{0.113pt}{0.800pt}}
\multiput(940.17,542.34)(4.000,-5.340){2}{\rule{0.400pt}{0.400pt}}
\multiput(945.60,533.26)(0.468,-1.066){5}{\rule{0.113pt}{0.900pt}}
\multiput(944.17,535.13)(4.000,-6.132){2}{\rule{0.400pt}{0.450pt}}
\multiput(949.60,525.68)(0.468,-0.920){5}{\rule{0.113pt}{0.800pt}}
\multiput(948.17,527.34)(4.000,-5.340){2}{\rule{0.400pt}{0.400pt}}
\multiput(953.61,517.16)(0.447,-1.579){3}{\rule{0.108pt}{1.167pt}}
\multiput(952.17,519.58)(3.000,-5.579){2}{\rule{0.400pt}{0.583pt}}
\multiput(956.60,509.85)(0.468,-1.212){5}{\rule{0.113pt}{1.000pt}}
\multiput(955.17,511.92)(4.000,-6.924){2}{\rule{0.400pt}{0.500pt}}
\multiput(960.60,501.26)(0.468,-1.066){5}{\rule{0.113pt}{0.900pt}}
\multiput(959.17,503.13)(4.000,-6.132){2}{\rule{0.400pt}{0.450pt}}
\multiput(606.60,310.00)(0.468,1.651){5}{\rule{0.113pt}{1.300pt}}
\multiput(605.17,310.00)(4.000,9.302){2}{\rule{0.400pt}{0.650pt}}
\multiput(610.60,322.00)(0.468,1.651){5}{\rule{0.113pt}{1.300pt}}
\multiput(609.17,322.00)(4.000,9.302){2}{\rule{0.400pt}{0.650pt}}
\multiput(614.60,334.00)(0.468,1.505){5}{\rule{0.113pt}{1.200pt}}
\multiput(613.17,334.00)(4.000,8.509){2}{\rule{0.400pt}{0.600pt}}
\multiput(618.60,345.00)(0.468,1.505){5}{\rule{0.113pt}{1.200pt}}
\multiput(617.17,345.00)(4.000,8.509){2}{\rule{0.400pt}{0.600pt}}
\multiput(622.60,356.00)(0.468,1.505){5}{\rule{0.113pt}{1.200pt}}
\multiput(621.17,356.00)(4.000,8.509){2}{\rule{0.400pt}{0.600pt}}
\multiput(626.60,367.00)(0.468,1.505){5}{\rule{0.113pt}{1.200pt}}
\multiput(625.17,367.00)(4.000,8.509){2}{\rule{0.400pt}{0.600pt}}
\multiput(630.60,378.00)(0.468,1.358){5}{\rule{0.113pt}{1.100pt}}
\multiput(629.17,378.00)(4.000,7.717){2}{\rule{0.400pt}{0.550pt}}
\multiput(634.60,388.00)(0.468,1.358){5}{\rule{0.113pt}{1.100pt}}
\multiput(633.17,388.00)(4.000,7.717){2}{\rule{0.400pt}{0.550pt}}
\multiput(638.60,398.00)(0.468,1.358){5}{\rule{0.113pt}{1.100pt}}
\multiput(637.17,398.00)(4.000,7.717){2}{\rule{0.400pt}{0.550pt}}
\multiput(642.60,408.00)(0.468,1.358){5}{\rule{0.113pt}{1.100pt}}
\multiput(641.17,408.00)(4.000,7.717){2}{\rule{0.400pt}{0.550pt}}
\multiput(646.60,418.00)(0.468,1.212){5}{\rule{0.113pt}{1.000pt}}
\multiput(645.17,418.00)(4.000,6.924){2}{\rule{0.400pt}{0.500pt}}
\multiput(650.60,427.00)(0.468,1.358){5}{\rule{0.113pt}{1.100pt}}
\multiput(649.17,427.00)(4.000,7.717){2}{\rule{0.400pt}{0.550pt}}
\multiput(654.60,437.00)(0.468,1.212){5}{\rule{0.113pt}{1.000pt}}
\multiput(653.17,437.00)(4.000,6.924){2}{\rule{0.400pt}{0.500pt}}
\multiput(658.60,446.00)(0.468,1.212){5}{\rule{0.113pt}{1.000pt}}
\multiput(657.17,446.00)(4.000,6.924){2}{\rule{0.400pt}{0.500pt}}
\multiput(662.60,455.00)(0.468,1.066){5}{\rule{0.113pt}{0.900pt}}
\multiput(661.17,455.00)(4.000,6.132){2}{\rule{0.400pt}{0.450pt}}
\multiput(666.60,463.00)(0.468,1.066){5}{\rule{0.113pt}{0.900pt}}
\multiput(665.17,463.00)(4.000,6.132){2}{\rule{0.400pt}{0.450pt}}
\multiput(670.60,471.00)(0.468,1.212){5}{\rule{0.113pt}{1.000pt}}
\multiput(669.17,471.00)(4.000,6.924){2}{\rule{0.400pt}{0.500pt}}
\multiput(674.60,480.00)(0.468,1.066){5}{\rule{0.113pt}{0.900pt}}
\multiput(673.17,480.00)(4.000,6.132){2}{\rule{0.400pt}{0.450pt}}
\multiput(678.60,488.00)(0.468,0.920){5}{\rule{0.113pt}{0.800pt}}
\multiput(677.17,488.00)(4.000,5.340){2}{\rule{0.400pt}{0.400pt}}
\multiput(682.60,495.00)(0.468,1.066){5}{\rule{0.113pt}{0.900pt}}
\multiput(681.17,495.00)(4.000,6.132){2}{\rule{0.400pt}{0.450pt}}
\multiput(686.60,503.00)(0.468,0.920){5}{\rule{0.113pt}{0.800pt}}
\multiput(685.17,503.00)(4.000,5.340){2}{\rule{0.400pt}{0.400pt}}
\multiput(690.60,510.00)(0.468,0.920){5}{\rule{0.113pt}{0.800pt}}
\multiput(689.17,510.00)(4.000,5.340){2}{\rule{0.400pt}{0.400pt}}
\multiput(694.60,517.00)(0.468,0.920){5}{\rule{0.113pt}{0.800pt}}
\multiput(693.17,517.00)(4.000,5.340){2}{\rule{0.400pt}{0.400pt}}
\multiput(698.60,524.00)(0.468,0.920){5}{\rule{0.113pt}{0.800pt}}
\multiput(697.17,524.00)(4.000,5.340){2}{\rule{0.400pt}{0.400pt}}
\multiput(702.60,531.00)(0.468,0.774){5}{\rule{0.113pt}{0.700pt}}
\multiput(701.17,531.00)(4.000,4.547){2}{\rule{0.400pt}{0.350pt}}
\multiput(706.60,537.00)(0.468,0.774){5}{\rule{0.113pt}{0.700pt}}
\multiput(705.17,537.00)(4.000,4.547){2}{\rule{0.400pt}{0.350pt}}
\multiput(710.60,543.00)(0.468,0.920){5}{\rule{0.113pt}{0.800pt}}
\multiput(709.17,543.00)(4.000,5.340){2}{\rule{0.400pt}{0.400pt}}
\multiput(714.60,550.00)(0.468,0.627){5}{\rule{0.113pt}{0.600pt}}
\multiput(713.17,550.00)(4.000,3.755){2}{\rule{0.400pt}{0.300pt}}
\multiput(718.60,555.00)(0.468,0.774){5}{\rule{0.113pt}{0.700pt}}
\multiput(717.17,555.00)(4.000,4.547){2}{\rule{0.400pt}{0.350pt}}
\multiput(722.60,561.00)(0.468,0.774){5}{\rule{0.113pt}{0.700pt}}
\multiput(721.17,561.00)(4.000,4.547){2}{\rule{0.400pt}{0.350pt}}
\multiput(726.60,567.00)(0.468,0.627){5}{\rule{0.113pt}{0.600pt}}
\multiput(725.17,567.00)(4.000,3.755){2}{\rule{0.400pt}{0.300pt}}
\multiput(730.60,572.00)(0.468,0.627){5}{\rule{0.113pt}{0.600pt}}
\multiput(729.17,572.00)(4.000,3.755){2}{\rule{0.400pt}{0.300pt}}
\multiput(734.60,577.00)(0.468,0.627){5}{\rule{0.113pt}{0.600pt}}
\multiput(733.17,577.00)(4.000,3.755){2}{\rule{0.400pt}{0.300pt}}
\multiput(738.60,582.00)(0.468,0.627){5}{\rule{0.113pt}{0.600pt}}
\multiput(737.17,582.00)(4.000,3.755){2}{\rule{0.400pt}{0.300pt}}
\multiput(742.00,587.60)(0.481,0.468){5}{\rule{0.500pt}{0.113pt}}
\multiput(742.00,586.17)(2.962,4.000){2}{\rule{0.250pt}{0.400pt}}
\multiput(746.60,591.00)(0.468,0.627){5}{\rule{0.113pt}{0.600pt}}
\multiput(745.17,591.00)(4.000,3.755){2}{\rule{0.400pt}{0.300pt}}
\multiput(750.00,596.60)(0.481,0.468){5}{\rule{0.500pt}{0.113pt}}
\multiput(750.00,595.17)(2.962,4.000){2}{\rule{0.250pt}{0.400pt}}
\multiput(754.00,600.60)(0.481,0.468){5}{\rule{0.500pt}{0.113pt}}
\multiput(754.00,599.17)(2.962,4.000){2}{\rule{0.250pt}{0.400pt}}
\multiput(758.00,604.60)(0.481,0.468){5}{\rule{0.500pt}{0.113pt}}
\multiput(758.00,603.17)(2.962,4.000){2}{\rule{0.250pt}{0.400pt}}
\multiput(762.00,608.61)(0.685,0.447){3}{\rule{0.633pt}{0.108pt}}
\multiput(762.00,607.17)(2.685,3.000){2}{\rule{0.317pt}{0.400pt}}
\multiput(766.00,611.60)(0.481,0.468){5}{\rule{0.500pt}{0.113pt}}
\multiput(766.00,610.17)(2.962,4.000){2}{\rule{0.250pt}{0.400pt}}
\multiput(770.00,615.61)(0.685,0.447){3}{\rule{0.633pt}{0.108pt}}
\multiput(770.00,614.17)(2.685,3.000){2}{\rule{0.317pt}{0.400pt}}
\multiput(774.00,618.61)(0.685,0.447){3}{\rule{0.633pt}{0.108pt}}
\multiput(774.00,617.17)(2.685,3.000){2}{\rule{0.317pt}{0.400pt}}
\multiput(778.00,621.61)(0.462,0.447){3}{\rule{0.500pt}{0.108pt}}
\multiput(778.00,620.17)(1.962,3.000){2}{\rule{0.250pt}{0.400pt}}
\multiput(781.00,624.61)(0.685,0.447){3}{\rule{0.633pt}{0.108pt}}
\multiput(781.00,623.17)(2.685,3.000){2}{\rule{0.317pt}{0.400pt}}
\multiput(785.00,627.61)(0.685,0.447){3}{\rule{0.633pt}{0.108pt}}
\multiput(785.00,626.17)(2.685,3.000){2}{\rule{0.317pt}{0.400pt}}
\put(789,630.17){\rule{0.900pt}{0.400pt}}
\multiput(789.00,629.17)(2.132,2.000){2}{\rule{0.450pt}{0.400pt}}
\put(793,632.17){\rule{0.900pt}{0.400pt}}
\multiput(793.00,631.17)(2.132,2.000){2}{\rule{0.450pt}{0.400pt}}
\put(797,634.17){\rule{0.900pt}{0.400pt}}
\multiput(797.00,633.17)(2.132,2.000){2}{\rule{0.450pt}{0.400pt}}
\put(801,636.17){\rule{0.900pt}{0.400pt}}
\multiput(801.00,635.17)(2.132,2.000){2}{\rule{0.450pt}{0.400pt}}
\put(805,638.17){\rule{0.900pt}{0.400pt}}
\multiput(805.00,637.17)(2.132,2.000){2}{\rule{0.450pt}{0.400pt}}
\put(809,639.67){\rule{0.964pt}{0.400pt}}
\multiput(809.00,639.17)(2.000,1.000){2}{\rule{0.482pt}{0.400pt}}
\put(813,640.67){\rule{0.964pt}{0.400pt}}
\multiput(813.00,640.17)(2.000,1.000){2}{\rule{0.482pt}{0.400pt}}
\put(817,641.67){\rule{0.964pt}{0.400pt}}
\multiput(817.00,641.17)(2.000,1.000){2}{\rule{0.482pt}{0.400pt}}
\put(821,642.67){\rule{0.964pt}{0.400pt}}
\multiput(821.00,642.17)(2.000,1.000){2}{\rule{0.482pt}{0.400pt}}
\put(825,643.67){\rule{0.964pt}{0.400pt}}
\multiput(825.00,643.17)(2.000,1.000){2}{\rule{0.482pt}{0.400pt}}
\put(829,644.67){\rule{0.964pt}{0.400pt}}
\multiput(829.00,644.17)(2.000,1.000){2}{\rule{0.482pt}{0.400pt}}
\put(821.0,655.0){\rule[-0.200pt]{0.964pt}{0.400pt}}
\put(833.0,646.0){\rule[-0.200pt]{0.964pt}{0.400pt}}
\put(837.0,646.0){\rule[-0.200pt]{0.964pt}{0.400pt}}
\put(841.0,646.0){\rule[-0.200pt]{0.964pt}{0.400pt}}
\put(845.0,646.0){\rule[-0.200pt]{0.964pt}{0.400pt}}
\put(853,644.67){\rule{0.964pt}{0.400pt}}
\multiput(853.00,645.17)(2.000,-1.000){2}{\rule{0.482pt}{0.400pt}}
\put(849.0,646.0){\rule[-0.200pt]{0.964pt}{0.400pt}}
\put(861,643.67){\rule{0.964pt}{0.400pt}}
\multiput(861.00,644.17)(2.000,-1.000){2}{\rule{0.482pt}{0.400pt}}
\put(865,642.67){\rule{0.964pt}{0.400pt}}
\multiput(865.00,643.17)(2.000,-1.000){2}{\rule{0.482pt}{0.400pt}}
\put(869,641.17){\rule{0.900pt}{0.400pt}}
\multiput(869.00,642.17)(2.132,-2.000){2}{\rule{0.450pt}{0.400pt}}
\put(873,639.67){\rule{0.964pt}{0.400pt}}
\multiput(873.00,640.17)(2.000,-1.000){2}{\rule{0.482pt}{0.400pt}}
\put(877,638.17){\rule{0.900pt}{0.400pt}}
\multiput(877.00,639.17)(2.132,-2.000){2}{\rule{0.450pt}{0.400pt}}
\put(881,636.17){\rule{0.900pt}{0.400pt}}
\multiput(881.00,637.17)(2.132,-2.000){2}{\rule{0.450pt}{0.400pt}}
\put(885,634.17){\rule{0.900pt}{0.400pt}}
\multiput(885.00,635.17)(2.132,-2.000){2}{\rule{0.450pt}{0.400pt}}
\put(889,632.17){\rule{0.900pt}{0.400pt}}
\multiput(889.00,633.17)(2.132,-2.000){2}{\rule{0.450pt}{0.400pt}}
\put(893,630.17){\rule{0.900pt}{0.400pt}}
\multiput(893.00,631.17)(2.132,-2.000){2}{\rule{0.450pt}{0.400pt}}
\multiput(897.00,628.95)(0.685,-0.447){3}{\rule{0.633pt}{0.108pt}}
\multiput(897.00,629.17)(2.685,-3.000){2}{\rule{0.317pt}{0.400pt}}
\multiput(901.00,625.95)(0.685,-0.447){3}{\rule{0.633pt}{0.108pt}}
\multiput(901.00,626.17)(2.685,-3.000){2}{\rule{0.317pt}{0.400pt}}
\multiput(905.00,622.95)(0.685,-0.447){3}{\rule{0.633pt}{0.108pt}}
\multiput(905.00,623.17)(2.685,-3.000){2}{\rule{0.317pt}{0.400pt}}
\multiput(909.00,619.95)(0.685,-0.447){3}{\rule{0.633pt}{0.108pt}}
\multiput(909.00,620.17)(2.685,-3.000){2}{\rule{0.317pt}{0.400pt}}
\multiput(913.00,616.95)(0.685,-0.447){3}{\rule{0.633pt}{0.108pt}}
\multiput(913.00,617.17)(2.685,-3.000){2}{\rule{0.317pt}{0.400pt}}
\multiput(917.00,613.94)(0.481,-0.468){5}{\rule{0.500pt}{0.113pt}}
\multiput(917.00,614.17)(2.962,-4.000){2}{\rule{0.250pt}{0.400pt}}
\multiput(921.00,609.94)(0.481,-0.468){5}{\rule{0.500pt}{0.113pt}}
\multiput(921.00,610.17)(2.962,-4.000){2}{\rule{0.250pt}{0.400pt}}
\multiput(925.00,605.94)(0.481,-0.468){5}{\rule{0.500pt}{0.113pt}}
\multiput(925.00,606.17)(2.962,-4.000){2}{\rule{0.250pt}{0.400pt}}
\multiput(929.00,601.94)(0.481,-0.468){5}{\rule{0.500pt}{0.113pt}}
\multiput(929.00,602.17)(2.962,-4.000){2}{\rule{0.250pt}{0.400pt}}
\multiput(933.60,596.51)(0.468,-0.627){5}{\rule{0.113pt}{0.600pt}}
\multiput(932.17,597.75)(4.000,-3.755){2}{\rule{0.400pt}{0.300pt}}
\multiput(937.00,592.94)(0.481,-0.468){5}{\rule{0.500pt}{0.113pt}}
\multiput(937.00,593.17)(2.962,-4.000){2}{\rule{0.250pt}{0.400pt}}
\multiput(941.60,587.51)(0.468,-0.627){5}{\rule{0.113pt}{0.600pt}}
\multiput(940.17,588.75)(4.000,-3.755){2}{\rule{0.400pt}{0.300pt}}
\multiput(945.60,582.51)(0.468,-0.627){5}{\rule{0.113pt}{0.600pt}}
\multiput(944.17,583.75)(4.000,-3.755){2}{\rule{0.400pt}{0.300pt}}
\multiput(949.60,577.09)(0.468,-0.774){5}{\rule{0.113pt}{0.700pt}}
\multiput(948.17,578.55)(4.000,-4.547){2}{\rule{0.400pt}{0.350pt}}
\multiput(953.60,571.51)(0.468,-0.627){5}{\rule{0.113pt}{0.600pt}}
\multiput(952.17,572.75)(4.000,-3.755){2}{\rule{0.400pt}{0.300pt}}
\multiput(957.60,566.09)(0.468,-0.774){5}{\rule{0.113pt}{0.700pt}}
\multiput(956.17,567.55)(4.000,-4.547){2}{\rule{0.400pt}{0.350pt}}
\multiput(961.60,560.09)(0.468,-0.774){5}{\rule{0.113pt}{0.700pt}}
\multiput(960.17,561.55)(4.000,-4.547){2}{\rule{0.400pt}{0.350pt}}
\multiput(965.60,553.68)(0.468,-0.920){5}{\rule{0.113pt}{0.800pt}}
\multiput(964.17,555.34)(4.000,-5.340){2}{\rule{0.400pt}{0.400pt}}
\multiput(969.60,547.09)(0.468,-0.774){5}{\rule{0.113pt}{0.700pt}}
\multiput(968.17,548.55)(4.000,-4.547){2}{\rule{0.400pt}{0.350pt}}
\multiput(973.61,539.71)(0.447,-1.355){3}{\rule{0.108pt}{1.033pt}}
\multiput(972.17,541.86)(3.000,-4.855){2}{\rule{0.400pt}{0.517pt}}
\multiput(976.60,533.68)(0.468,-0.920){5}{\rule{0.113pt}{0.800pt}}
\multiput(975.17,535.34)(4.000,-5.340){2}{\rule{0.400pt}{0.400pt}}
\multiput(980.60,526.68)(0.468,-0.920){5}{\rule{0.113pt}{0.800pt}}
\multiput(979.17,528.34)(4.000,-5.340){2}{\rule{0.400pt}{0.400pt}}
\multiput(984.60,519.26)(0.468,-1.066){5}{\rule{0.113pt}{0.900pt}}
\multiput(983.17,521.13)(4.000,-6.132){2}{\rule{0.400pt}{0.450pt}}
\multiput(988.60,511.68)(0.468,-0.920){5}{\rule{0.113pt}{0.800pt}}
\multiput(987.17,513.34)(4.000,-5.340){2}{\rule{0.400pt}{0.400pt}}
\multiput(992.60,504.26)(0.468,-1.066){5}{\rule{0.113pt}{0.900pt}}
\multiput(991.17,506.13)(4.000,-6.132){2}{\rule{0.400pt}{0.450pt}}
\multiput(996.60,495.85)(0.468,-1.212){5}{\rule{0.113pt}{1.000pt}}
\multiput(995.17,497.92)(4.000,-6.924){2}{\rule{0.400pt}{0.500pt}}
\multiput(642.60,305.00)(0.468,1.505){5}{\rule{0.113pt}{1.200pt}}
\multiput(641.17,305.00)(4.000,8.509){2}{\rule{0.400pt}{0.600pt}}
\multiput(646.60,316.00)(0.468,1.505){5}{\rule{0.113pt}{1.200pt}}
\multiput(645.17,316.00)(4.000,8.509){2}{\rule{0.400pt}{0.600pt}}
\multiput(650.60,327.00)(0.468,1.505){5}{\rule{0.113pt}{1.200pt}}
\multiput(649.17,327.00)(4.000,8.509){2}{\rule{0.400pt}{0.600pt}}
\multiput(654.60,338.00)(0.468,1.505){5}{\rule{0.113pt}{1.200pt}}
\multiput(653.17,338.00)(4.000,8.509){2}{\rule{0.400pt}{0.600pt}}
\multiput(658.60,349.00)(0.468,1.505){5}{\rule{0.113pt}{1.200pt}}
\multiput(657.17,349.00)(4.000,8.509){2}{\rule{0.400pt}{0.600pt}}
\multiput(662.60,360.00)(0.468,1.358){5}{\rule{0.113pt}{1.100pt}}
\multiput(661.17,360.00)(4.000,7.717){2}{\rule{0.400pt}{0.550pt}}
\multiput(666.60,370.00)(0.468,1.358){5}{\rule{0.113pt}{1.100pt}}
\multiput(665.17,370.00)(4.000,7.717){2}{\rule{0.400pt}{0.550pt}}
\multiput(670.60,380.00)(0.468,1.358){5}{\rule{0.113pt}{1.100pt}}
\multiput(669.17,380.00)(4.000,7.717){2}{\rule{0.400pt}{0.550pt}}
\multiput(674.60,390.00)(0.468,1.212){5}{\rule{0.113pt}{1.000pt}}
\multiput(673.17,390.00)(4.000,6.924){2}{\rule{0.400pt}{0.500pt}}
\multiput(678.60,399.00)(0.468,1.358){5}{\rule{0.113pt}{1.100pt}}
\multiput(677.17,399.00)(4.000,7.717){2}{\rule{0.400pt}{0.550pt}}
\multiput(682.60,409.00)(0.468,1.212){5}{\rule{0.113pt}{1.000pt}}
\multiput(681.17,409.00)(4.000,6.924){2}{\rule{0.400pt}{0.500pt}}
\multiput(686.60,418.00)(0.468,1.212){5}{\rule{0.113pt}{1.000pt}}
\multiput(685.17,418.00)(4.000,6.924){2}{\rule{0.400pt}{0.500pt}}
\multiput(690.60,427.00)(0.468,1.066){5}{\rule{0.113pt}{0.900pt}}
\multiput(689.17,427.00)(4.000,6.132){2}{\rule{0.400pt}{0.450pt}}
\multiput(694.60,435.00)(0.468,1.212){5}{\rule{0.113pt}{1.000pt}}
\multiput(693.17,435.00)(4.000,6.924){2}{\rule{0.400pt}{0.500pt}}
\multiput(698.60,444.00)(0.468,1.066){5}{\rule{0.113pt}{0.900pt}}
\multiput(697.17,444.00)(4.000,6.132){2}{\rule{0.400pt}{0.450pt}}
\multiput(702.60,452.00)(0.468,1.066){5}{\rule{0.113pt}{0.900pt}}
\multiput(701.17,452.00)(4.000,6.132){2}{\rule{0.400pt}{0.450pt}}
\multiput(706.60,460.00)(0.468,1.066){5}{\rule{0.113pt}{0.900pt}}
\multiput(705.17,460.00)(4.000,6.132){2}{\rule{0.400pt}{0.450pt}}
\multiput(710.60,468.00)(0.468,1.066){5}{\rule{0.113pt}{0.900pt}}
\multiput(709.17,468.00)(4.000,6.132){2}{\rule{0.400pt}{0.450pt}}
\multiput(714.60,476.00)(0.468,0.920){5}{\rule{0.113pt}{0.800pt}}
\multiput(713.17,476.00)(4.000,5.340){2}{\rule{0.400pt}{0.400pt}}
\multiput(718.60,483.00)(0.468,1.066){5}{\rule{0.113pt}{0.900pt}}
\multiput(717.17,483.00)(4.000,6.132){2}{\rule{0.400pt}{0.450pt}}
\multiput(722.60,491.00)(0.468,0.920){5}{\rule{0.113pt}{0.800pt}}
\multiput(721.17,491.00)(4.000,5.340){2}{\rule{0.400pt}{0.400pt}}
\multiput(726.60,498.00)(0.468,0.774){5}{\rule{0.113pt}{0.700pt}}
\multiput(725.17,498.00)(4.000,4.547){2}{\rule{0.400pt}{0.350pt}}
\multiput(730.60,504.00)(0.468,0.920){5}{\rule{0.113pt}{0.800pt}}
\multiput(729.17,504.00)(4.000,5.340){2}{\rule{0.400pt}{0.400pt}}
\multiput(734.60,511.00)(0.468,0.920){5}{\rule{0.113pt}{0.800pt}}
\multiput(733.17,511.00)(4.000,5.340){2}{\rule{0.400pt}{0.400pt}}
\multiput(738.60,518.00)(0.468,0.774){5}{\rule{0.113pt}{0.700pt}}
\multiput(737.17,518.00)(4.000,4.547){2}{\rule{0.400pt}{0.350pt}}
\multiput(742.60,524.00)(0.468,0.774){5}{\rule{0.113pt}{0.700pt}}
\multiput(741.17,524.00)(4.000,4.547){2}{\rule{0.400pt}{0.350pt}}
\multiput(746.60,530.00)(0.468,0.774){5}{\rule{0.113pt}{0.700pt}}
\multiput(745.17,530.00)(4.000,4.547){2}{\rule{0.400pt}{0.350pt}}
\multiput(750.60,536.00)(0.468,0.774){5}{\rule{0.113pt}{0.700pt}}
\multiput(749.17,536.00)(4.000,4.547){2}{\rule{0.400pt}{0.350pt}}
\multiput(754.60,542.00)(0.468,0.627){5}{\rule{0.113pt}{0.600pt}}
\multiput(753.17,542.00)(4.000,3.755){2}{\rule{0.400pt}{0.300pt}}
\multiput(758.60,547.00)(0.468,0.774){5}{\rule{0.113pt}{0.700pt}}
\multiput(757.17,547.00)(4.000,4.547){2}{\rule{0.400pt}{0.350pt}}
\multiput(762.60,553.00)(0.468,0.627){5}{\rule{0.113pt}{0.600pt}}
\multiput(761.17,553.00)(4.000,3.755){2}{\rule{0.400pt}{0.300pt}}
\multiput(766.60,558.00)(0.468,0.627){5}{\rule{0.113pt}{0.600pt}}
\multiput(765.17,558.00)(4.000,3.755){2}{\rule{0.400pt}{0.300pt}}
\multiput(770.60,563.00)(0.468,0.627){5}{\rule{0.113pt}{0.600pt}}
\multiput(769.17,563.00)(4.000,3.755){2}{\rule{0.400pt}{0.300pt}}
\multiput(774.00,568.60)(0.481,0.468){5}{\rule{0.500pt}{0.113pt}}
\multiput(774.00,567.17)(2.962,4.000){2}{\rule{0.250pt}{0.400pt}}
\multiput(778.60,572.00)(0.468,0.627){5}{\rule{0.113pt}{0.600pt}}
\multiput(777.17,572.00)(4.000,3.755){2}{\rule{0.400pt}{0.300pt}}
\multiput(782.00,577.60)(0.481,0.468){5}{\rule{0.500pt}{0.113pt}}
\multiput(782.00,576.17)(2.962,4.000){2}{\rule{0.250pt}{0.400pt}}
\multiput(786.00,581.60)(0.481,0.468){5}{\rule{0.500pt}{0.113pt}}
\multiput(786.00,580.17)(2.962,4.000){2}{\rule{0.250pt}{0.400pt}}
\multiput(790.00,585.60)(0.481,0.468){5}{\rule{0.500pt}{0.113pt}}
\multiput(790.00,584.17)(2.962,4.000){2}{\rule{0.250pt}{0.400pt}}
\multiput(794.00,589.60)(0.481,0.468){5}{\rule{0.500pt}{0.113pt}}
\multiput(794.00,588.17)(2.962,4.000){2}{\rule{0.250pt}{0.400pt}}
\multiput(798.00,593.61)(0.462,0.447){3}{\rule{0.500pt}{0.108pt}}
\multiput(798.00,592.17)(1.962,3.000){2}{\rule{0.250pt}{0.400pt}}
\multiput(801.00,596.60)(0.481,0.468){5}{\rule{0.500pt}{0.113pt}}
\multiput(801.00,595.17)(2.962,4.000){2}{\rule{0.250pt}{0.400pt}}
\multiput(805.00,600.61)(0.685,0.447){3}{\rule{0.633pt}{0.108pt}}
\multiput(805.00,599.17)(2.685,3.000){2}{\rule{0.317pt}{0.400pt}}
\multiput(809.00,603.61)(0.685,0.447){3}{\rule{0.633pt}{0.108pt}}
\multiput(809.00,602.17)(2.685,3.000){2}{\rule{0.317pt}{0.400pt}}
\multiput(813.00,606.61)(0.685,0.447){3}{\rule{0.633pt}{0.108pt}}
\multiput(813.00,605.17)(2.685,3.000){2}{\rule{0.317pt}{0.400pt}}
\put(817,609.17){\rule{0.900pt}{0.400pt}}
\multiput(817.00,608.17)(2.132,2.000){2}{\rule{0.450pt}{0.400pt}}
\multiput(821.00,611.61)(0.685,0.447){3}{\rule{0.633pt}{0.108pt}}
\multiput(821.00,610.17)(2.685,3.000){2}{\rule{0.317pt}{0.400pt}}
\put(825,614.17){\rule{0.900pt}{0.400pt}}
\multiput(825.00,613.17)(2.132,2.000){2}{\rule{0.450pt}{0.400pt}}
\put(829,616.17){\rule{0.900pt}{0.400pt}}
\multiput(829.00,615.17)(2.132,2.000){2}{\rule{0.450pt}{0.400pt}}
\multiput(833.00,618.61)(0.685,0.447){3}{\rule{0.633pt}{0.108pt}}
\multiput(833.00,617.17)(2.685,3.000){2}{\rule{0.317pt}{0.400pt}}
\put(837,620.67){\rule{0.964pt}{0.400pt}}
\multiput(837.00,620.17)(2.000,1.000){2}{\rule{0.482pt}{0.400pt}}
\put(841,622.17){\rule{0.900pt}{0.400pt}}
\multiput(841.00,621.17)(2.132,2.000){2}{\rule{0.450pt}{0.400pt}}
\put(845,623.67){\rule{0.964pt}{0.400pt}}
\multiput(845.00,623.17)(2.000,1.000){2}{\rule{0.482pt}{0.400pt}}
\put(849,625.17){\rule{0.900pt}{0.400pt}}
\multiput(849.00,624.17)(2.132,2.000){2}{\rule{0.450pt}{0.400pt}}
\put(853,626.67){\rule{0.964pt}{0.400pt}}
\multiput(853.00,626.17)(2.000,1.000){2}{\rule{0.482pt}{0.400pt}}
\put(857,627.67){\rule{0.964pt}{0.400pt}}
\multiput(857.00,627.17)(2.000,1.000){2}{\rule{0.482pt}{0.400pt}}
\put(861,628.67){\rule{0.964pt}{0.400pt}}
\multiput(861.00,628.17)(2.000,1.000){2}{\rule{0.482pt}{0.400pt}}
\put(857.0,645.0){\rule[-0.200pt]{0.964pt}{0.400pt}}
\put(869,629.67){\rule{0.964pt}{0.400pt}}
\multiput(869.00,629.17)(2.000,1.000){2}{\rule{0.482pt}{0.400pt}}
\put(865.0,630.0){\rule[-0.200pt]{0.964pt}{0.400pt}}
\put(873.0,631.0){\rule[-0.200pt]{0.964pt}{0.400pt}}
\put(877.0,631.0){\rule[-0.200pt]{0.964pt}{0.400pt}}
\put(881.0,631.0){\rule[-0.200pt]{0.964pt}{0.400pt}}
\put(889,629.67){\rule{0.964pt}{0.400pt}}
\multiput(889.00,630.17)(2.000,-1.000){2}{\rule{0.482pt}{0.400pt}}
\put(885.0,631.0){\rule[-0.200pt]{0.964pt}{0.400pt}}
\put(897,628.67){\rule{0.964pt}{0.400pt}}
\multiput(897.00,629.17)(2.000,-1.000){2}{\rule{0.482pt}{0.400pt}}
\put(901,627.67){\rule{0.964pt}{0.400pt}}
\multiput(901.00,628.17)(2.000,-1.000){2}{\rule{0.482pt}{0.400pt}}
\put(905,626.67){\rule{0.964pt}{0.400pt}}
\multiput(905.00,627.17)(2.000,-1.000){2}{\rule{0.482pt}{0.400pt}}
\put(909,625.67){\rule{0.964pt}{0.400pt}}
\multiput(909.00,626.17)(2.000,-1.000){2}{\rule{0.482pt}{0.400pt}}
\put(913,624.17){\rule{0.900pt}{0.400pt}}
\multiput(913.00,625.17)(2.132,-2.000){2}{\rule{0.450pt}{0.400pt}}
\put(917,622.17){\rule{0.900pt}{0.400pt}}
\multiput(917.00,623.17)(2.132,-2.000){2}{\rule{0.450pt}{0.400pt}}
\put(921,620.17){\rule{0.900pt}{0.400pt}}
\multiput(921.00,621.17)(2.132,-2.000){2}{\rule{0.450pt}{0.400pt}}
\put(925,618.17){\rule{0.900pt}{0.400pt}}
\multiput(925.00,619.17)(2.132,-2.000){2}{\rule{0.450pt}{0.400pt}}
\put(929,616.17){\rule{0.900pt}{0.400pt}}
\multiput(929.00,617.17)(2.132,-2.000){2}{\rule{0.450pt}{0.400pt}}
\put(933,614.17){\rule{0.900pt}{0.400pt}}
\multiput(933.00,615.17)(2.132,-2.000){2}{\rule{0.450pt}{0.400pt}}
\multiput(937.00,612.95)(0.685,-0.447){3}{\rule{0.633pt}{0.108pt}}
\multiput(937.00,613.17)(2.685,-3.000){2}{\rule{0.317pt}{0.400pt}}
\multiput(941.00,609.95)(0.685,-0.447){3}{\rule{0.633pt}{0.108pt}}
\multiput(941.00,610.17)(2.685,-3.000){2}{\rule{0.317pt}{0.400pt}}
\multiput(945.00,606.95)(0.685,-0.447){3}{\rule{0.633pt}{0.108pt}}
\multiput(945.00,607.17)(2.685,-3.000){2}{\rule{0.317pt}{0.400pt}}
\multiput(949.00,603.95)(0.685,-0.447){3}{\rule{0.633pt}{0.108pt}}
\multiput(949.00,604.17)(2.685,-3.000){2}{\rule{0.317pt}{0.400pt}}
\multiput(953.00,600.95)(0.685,-0.447){3}{\rule{0.633pt}{0.108pt}}
\multiput(953.00,601.17)(2.685,-3.000){2}{\rule{0.317pt}{0.400pt}}
\multiput(957.00,597.94)(0.481,-0.468){5}{\rule{0.500pt}{0.113pt}}
\multiput(957.00,598.17)(2.962,-4.000){2}{\rule{0.250pt}{0.400pt}}
\multiput(961.00,593.94)(0.481,-0.468){5}{\rule{0.500pt}{0.113pt}}
\multiput(961.00,594.17)(2.962,-4.000){2}{\rule{0.250pt}{0.400pt}}
\multiput(965.00,589.94)(0.481,-0.468){5}{\rule{0.500pt}{0.113pt}}
\multiput(965.00,590.17)(2.962,-4.000){2}{\rule{0.250pt}{0.400pt}}
\multiput(969.00,585.94)(0.481,-0.468){5}{\rule{0.500pt}{0.113pt}}
\multiput(969.00,586.17)(2.962,-4.000){2}{\rule{0.250pt}{0.400pt}}
\multiput(973.00,581.94)(0.481,-0.468){5}{\rule{0.500pt}{0.113pt}}
\multiput(973.00,582.17)(2.962,-4.000){2}{\rule{0.250pt}{0.400pt}}
\multiput(977.60,576.51)(0.468,-0.627){5}{\rule{0.113pt}{0.600pt}}
\multiput(976.17,577.75)(4.000,-3.755){2}{\rule{0.400pt}{0.300pt}}
\multiput(981.60,571.51)(0.468,-0.627){5}{\rule{0.113pt}{0.600pt}}
\multiput(980.17,572.75)(4.000,-3.755){2}{\rule{0.400pt}{0.300pt}}
\multiput(985.60,566.51)(0.468,-0.627){5}{\rule{0.113pt}{0.600pt}}
\multiput(984.17,567.75)(4.000,-3.755){2}{\rule{0.400pt}{0.300pt}}
\multiput(989.60,561.51)(0.468,-0.627){5}{\rule{0.113pt}{0.600pt}}
\multiput(988.17,562.75)(4.000,-3.755){2}{\rule{0.400pt}{0.300pt}}
\multiput(993.61,555.26)(0.447,-1.132){3}{\rule{0.108pt}{0.900pt}}
\multiput(992.17,557.13)(3.000,-4.132){2}{\rule{0.400pt}{0.450pt}}
\multiput(996.60,550.51)(0.468,-0.627){5}{\rule{0.113pt}{0.600pt}}
\multiput(995.17,551.75)(4.000,-3.755){2}{\rule{0.400pt}{0.300pt}}
\multiput(1000.60,545.09)(0.468,-0.774){5}{\rule{0.113pt}{0.700pt}}
\multiput(999.17,546.55)(4.000,-4.547){2}{\rule{0.400pt}{0.350pt}}
\multiput(1004.60,539.09)(0.468,-0.774){5}{\rule{0.113pt}{0.700pt}}
\multiput(1003.17,540.55)(4.000,-4.547){2}{\rule{0.400pt}{0.350pt}}
\multiput(1008.60,532.68)(0.468,-0.920){5}{\rule{0.113pt}{0.800pt}}
\multiput(1007.17,534.34)(4.000,-5.340){2}{\rule{0.400pt}{0.400pt}}
\multiput(1012.60,526.09)(0.468,-0.774){5}{\rule{0.113pt}{0.700pt}}
\multiput(1011.17,527.55)(4.000,-4.547){2}{\rule{0.400pt}{0.350pt}}
\multiput(1016.60,519.68)(0.468,-0.920){5}{\rule{0.113pt}{0.800pt}}
\multiput(1015.17,521.34)(4.000,-5.340){2}{\rule{0.400pt}{0.400pt}}
\multiput(1020.60,512.68)(0.468,-0.920){5}{\rule{0.113pt}{0.800pt}}
\multiput(1019.17,514.34)(4.000,-5.340){2}{\rule{0.400pt}{0.400pt}}
\multiput(1024.60,505.26)(0.468,-1.066){5}{\rule{0.113pt}{0.900pt}}
\multiput(1023.17,507.13)(4.000,-6.132){2}{\rule{0.400pt}{0.450pt}}
\multiput(1028.60,497.68)(0.468,-0.920){5}{\rule{0.113pt}{0.800pt}}
\multiput(1027.17,499.34)(4.000,-5.340){2}{\rule{0.400pt}{0.400pt}}
\multiput(1032.60,490.26)(0.468,-1.066){5}{\rule{0.113pt}{0.900pt}}
\multiput(1031.17,492.13)(4.000,-6.132){2}{\rule{0.400pt}{0.450pt}}
\multiput(678.60,299.00)(0.468,1.505){5}{\rule{0.113pt}{1.200pt}}
\multiput(677.17,299.00)(4.000,8.509){2}{\rule{0.400pt}{0.600pt}}
\multiput(682.60,310.00)(0.468,1.358){5}{\rule{0.113pt}{1.100pt}}
\multiput(681.17,310.00)(4.000,7.717){2}{\rule{0.400pt}{0.550pt}}
\multiput(686.60,320.00)(0.468,1.505){5}{\rule{0.113pt}{1.200pt}}
\multiput(685.17,320.00)(4.000,8.509){2}{\rule{0.400pt}{0.600pt}}
\multiput(690.60,331.00)(0.468,1.358){5}{\rule{0.113pt}{1.100pt}}
\multiput(689.17,331.00)(4.000,7.717){2}{\rule{0.400pt}{0.550pt}}
\multiput(694.60,341.00)(0.468,1.358){5}{\rule{0.113pt}{1.100pt}}
\multiput(693.17,341.00)(4.000,7.717){2}{\rule{0.400pt}{0.550pt}}
\multiput(698.60,351.00)(0.468,1.358){5}{\rule{0.113pt}{1.100pt}}
\multiput(697.17,351.00)(4.000,7.717){2}{\rule{0.400pt}{0.550pt}}
\multiput(702.60,361.00)(0.468,1.212){5}{\rule{0.113pt}{1.000pt}}
\multiput(701.17,361.00)(4.000,6.924){2}{\rule{0.400pt}{0.500pt}}
\multiput(706.60,370.00)(0.468,1.212){5}{\rule{0.113pt}{1.000pt}}
\multiput(705.17,370.00)(4.000,6.924){2}{\rule{0.400pt}{0.500pt}}
\multiput(710.60,379.00)(0.468,1.212){5}{\rule{0.113pt}{1.000pt}}
\multiput(709.17,379.00)(4.000,6.924){2}{\rule{0.400pt}{0.500pt}}
\multiput(714.60,388.00)(0.468,1.212){5}{\rule{0.113pt}{1.000pt}}
\multiput(713.17,388.00)(4.000,6.924){2}{\rule{0.400pt}{0.500pt}}
\multiput(718.60,397.00)(0.468,1.212){5}{\rule{0.113pt}{1.000pt}}
\multiput(717.17,397.00)(4.000,6.924){2}{\rule{0.400pt}{0.500pt}}
\multiput(722.60,406.00)(0.468,1.066){5}{\rule{0.113pt}{0.900pt}}
\multiput(721.17,406.00)(4.000,6.132){2}{\rule{0.400pt}{0.450pt}}
\multiput(726.60,414.00)(0.468,1.212){5}{\rule{0.113pt}{1.000pt}}
\multiput(725.17,414.00)(4.000,6.924){2}{\rule{0.400pt}{0.500pt}}
\multiput(730.60,423.00)(0.468,1.066){5}{\rule{0.113pt}{0.900pt}}
\multiput(729.17,423.00)(4.000,6.132){2}{\rule{0.400pt}{0.450pt}}
\multiput(734.60,431.00)(0.468,1.066){5}{\rule{0.113pt}{0.900pt}}
\multiput(733.17,431.00)(4.000,6.132){2}{\rule{0.400pt}{0.450pt}}
\multiput(738.60,439.00)(0.468,0.920){5}{\rule{0.113pt}{0.800pt}}
\multiput(737.17,439.00)(4.000,5.340){2}{\rule{0.400pt}{0.400pt}}
\multiput(742.60,446.00)(0.468,1.066){5}{\rule{0.113pt}{0.900pt}}
\multiput(741.17,446.00)(4.000,6.132){2}{\rule{0.400pt}{0.450pt}}
\multiput(746.60,454.00)(0.468,0.920){5}{\rule{0.113pt}{0.800pt}}
\multiput(745.17,454.00)(4.000,5.340){2}{\rule{0.400pt}{0.400pt}}
\multiput(750.60,461.00)(0.468,0.920){5}{\rule{0.113pt}{0.800pt}}
\multiput(749.17,461.00)(4.000,5.340){2}{\rule{0.400pt}{0.400pt}}
\multiput(754.60,468.00)(0.468,0.920){5}{\rule{0.113pt}{0.800pt}}
\multiput(753.17,468.00)(4.000,5.340){2}{\rule{0.400pt}{0.400pt}}
\multiput(758.60,475.00)(0.468,0.920){5}{\rule{0.113pt}{0.800pt}}
\multiput(757.17,475.00)(4.000,5.340){2}{\rule{0.400pt}{0.400pt}}
\multiput(762.60,482.00)(0.468,0.774){5}{\rule{0.113pt}{0.700pt}}
\multiput(761.17,482.00)(4.000,4.547){2}{\rule{0.400pt}{0.350pt}}
\multiput(766.60,488.00)(0.468,0.920){5}{\rule{0.113pt}{0.800pt}}
\multiput(765.17,488.00)(4.000,5.340){2}{\rule{0.400pt}{0.400pt}}
\multiput(770.60,495.00)(0.468,0.774){5}{\rule{0.113pt}{0.700pt}}
\multiput(769.17,495.00)(4.000,4.547){2}{\rule{0.400pt}{0.350pt}}
\multiput(774.60,501.00)(0.468,0.774){5}{\rule{0.113pt}{0.700pt}}
\multiput(773.17,501.00)(4.000,4.547){2}{\rule{0.400pt}{0.350pt}}
\multiput(778.60,507.00)(0.468,0.774){5}{\rule{0.113pt}{0.700pt}}
\multiput(777.17,507.00)(4.000,4.547){2}{\rule{0.400pt}{0.350pt}}
\multiput(782.60,513.00)(0.468,0.627){5}{\rule{0.113pt}{0.600pt}}
\multiput(781.17,513.00)(4.000,3.755){2}{\rule{0.400pt}{0.300pt}}
\multiput(786.60,518.00)(0.468,0.774){5}{\rule{0.113pt}{0.700pt}}
\multiput(785.17,518.00)(4.000,4.547){2}{\rule{0.400pt}{0.350pt}}
\multiput(790.60,524.00)(0.468,0.627){5}{\rule{0.113pt}{0.600pt}}
\multiput(789.17,524.00)(4.000,3.755){2}{\rule{0.400pt}{0.300pt}}
\multiput(794.60,529.00)(0.468,0.627){5}{\rule{0.113pt}{0.600pt}}
\multiput(793.17,529.00)(4.000,3.755){2}{\rule{0.400pt}{0.300pt}}
\multiput(798.60,534.00)(0.468,0.627){5}{\rule{0.113pt}{0.600pt}}
\multiput(797.17,534.00)(4.000,3.755){2}{\rule{0.400pt}{0.300pt}}
\multiput(802.60,539.00)(0.468,0.627){5}{\rule{0.113pt}{0.600pt}}
\multiput(801.17,539.00)(4.000,3.755){2}{\rule{0.400pt}{0.300pt}}
\multiput(806.00,544.60)(0.481,0.468){5}{\rule{0.500pt}{0.113pt}}
\multiput(806.00,543.17)(2.962,4.000){2}{\rule{0.250pt}{0.400pt}}
\multiput(810.60,548.00)(0.468,0.627){5}{\rule{0.113pt}{0.600pt}}
\multiput(809.17,548.00)(4.000,3.755){2}{\rule{0.400pt}{0.300pt}}
\multiput(814.00,553.60)(0.481,0.468){5}{\rule{0.500pt}{0.113pt}}
\multiput(814.00,552.17)(2.962,4.000){2}{\rule{0.250pt}{0.400pt}}
\multiput(818.61,557.00)(0.447,0.685){3}{\rule{0.108pt}{0.633pt}}
\multiput(817.17,557.00)(3.000,2.685){2}{\rule{0.400pt}{0.317pt}}
\multiput(821.00,561.60)(0.481,0.468){5}{\rule{0.500pt}{0.113pt}}
\multiput(821.00,560.17)(2.962,4.000){2}{\rule{0.250pt}{0.400pt}}
\multiput(825.00,565.60)(0.481,0.468){5}{\rule{0.500pt}{0.113pt}}
\multiput(825.00,564.17)(2.962,4.000){2}{\rule{0.250pt}{0.400pt}}
\multiput(829.00,569.60)(0.481,0.468){5}{\rule{0.500pt}{0.113pt}}
\multiput(829.00,568.17)(2.962,4.000){2}{\rule{0.250pt}{0.400pt}}
\multiput(833.00,573.61)(0.685,0.447){3}{\rule{0.633pt}{0.108pt}}
\multiput(833.00,572.17)(2.685,3.000){2}{\rule{0.317pt}{0.400pt}}
\multiput(837.00,576.61)(0.685,0.447){3}{\rule{0.633pt}{0.108pt}}
\multiput(837.00,575.17)(2.685,3.000){2}{\rule{0.317pt}{0.400pt}}
\multiput(841.00,579.61)(0.685,0.447){3}{\rule{0.633pt}{0.108pt}}
\multiput(841.00,578.17)(2.685,3.000){2}{\rule{0.317pt}{0.400pt}}
\multiput(845.00,582.61)(0.685,0.447){3}{\rule{0.633pt}{0.108pt}}
\multiput(845.00,581.17)(2.685,3.000){2}{\rule{0.317pt}{0.400pt}}
\multiput(849.00,585.61)(0.685,0.447){3}{\rule{0.633pt}{0.108pt}}
\multiput(849.00,584.17)(2.685,3.000){2}{\rule{0.317pt}{0.400pt}}
\multiput(853.00,588.61)(0.685,0.447){3}{\rule{0.633pt}{0.108pt}}
\multiput(853.00,587.17)(2.685,3.000){2}{\rule{0.317pt}{0.400pt}}
\put(857,591.17){\rule{0.900pt}{0.400pt}}
\multiput(857.00,590.17)(2.132,2.000){2}{\rule{0.450pt}{0.400pt}}
\multiput(861.00,593.61)(0.685,0.447){3}{\rule{0.633pt}{0.108pt}}
\multiput(861.00,592.17)(2.685,3.000){2}{\rule{0.317pt}{0.400pt}}
\put(865,596.17){\rule{0.900pt}{0.400pt}}
\multiput(865.00,595.17)(2.132,2.000){2}{\rule{0.450pt}{0.400pt}}
\put(869,598.17){\rule{0.900pt}{0.400pt}}
\multiput(869.00,597.17)(2.132,2.000){2}{\rule{0.450pt}{0.400pt}}
\put(873,600.17){\rule{0.900pt}{0.400pt}}
\multiput(873.00,599.17)(2.132,2.000){2}{\rule{0.450pt}{0.400pt}}
\put(877,601.67){\rule{0.964pt}{0.400pt}}
\multiput(877.00,601.17)(2.000,1.000){2}{\rule{0.482pt}{0.400pt}}
\put(881,603.17){\rule{0.900pt}{0.400pt}}
\multiput(881.00,602.17)(2.132,2.000){2}{\rule{0.450pt}{0.400pt}}
\put(885,604.67){\rule{0.964pt}{0.400pt}}
\multiput(885.00,604.17)(2.000,1.000){2}{\rule{0.482pt}{0.400pt}}
\put(889,605.67){\rule{0.964pt}{0.400pt}}
\multiput(889.00,605.17)(2.000,1.000){2}{\rule{0.482pt}{0.400pt}}
\put(893,606.67){\rule{0.964pt}{0.400pt}}
\multiput(893.00,606.17)(2.000,1.000){2}{\rule{0.482pt}{0.400pt}}
\put(897,607.67){\rule{0.964pt}{0.400pt}}
\multiput(897.00,607.17)(2.000,1.000){2}{\rule{0.482pt}{0.400pt}}
\put(901,608.67){\rule{0.964pt}{0.400pt}}
\multiput(901.00,608.17)(2.000,1.000){2}{\rule{0.482pt}{0.400pt}}
\put(893.0,630.0){\rule[-0.200pt]{0.964pt}{0.400pt}}
\put(909,609.67){\rule{0.964pt}{0.400pt}}
\multiput(909.00,609.17)(2.000,1.000){2}{\rule{0.482pt}{0.400pt}}
\put(905.0,610.0){\rule[-0.200pt]{0.964pt}{0.400pt}}
\put(913.0,611.0){\rule[-0.200pt]{0.964pt}{0.400pt}}
\put(917.0,611.0){\rule[-0.200pt]{0.964pt}{0.400pt}}
\put(921.0,611.0){\rule[-0.200pt]{0.964pt}{0.400pt}}
\put(929,609.67){\rule{0.964pt}{0.400pt}}
\multiput(929.00,610.17)(2.000,-1.000){2}{\rule{0.482pt}{0.400pt}}
\put(925.0,611.0){\rule[-0.200pt]{0.964pt}{0.400pt}}
\put(937,608.67){\rule{0.964pt}{0.400pt}}
\multiput(937.00,609.17)(2.000,-1.000){2}{\rule{0.482pt}{0.400pt}}
\put(941,607.67){\rule{0.964pt}{0.400pt}}
\multiput(941.00,608.17)(2.000,-1.000){2}{\rule{0.482pt}{0.400pt}}
\put(945,606.67){\rule{0.964pt}{0.400pt}}
\multiput(945.00,607.17)(2.000,-1.000){2}{\rule{0.482pt}{0.400pt}}
\put(949,605.67){\rule{0.964pt}{0.400pt}}
\multiput(949.00,606.17)(2.000,-1.000){2}{\rule{0.482pt}{0.400pt}}
\put(953,604.17){\rule{0.900pt}{0.400pt}}
\multiput(953.00,605.17)(2.132,-2.000){2}{\rule{0.450pt}{0.400pt}}
\put(957,602.17){\rule{0.900pt}{0.400pt}}
\multiput(957.00,603.17)(2.132,-2.000){2}{\rule{0.450pt}{0.400pt}}
\put(961,600.67){\rule{0.964pt}{0.400pt}}
\multiput(961.00,601.17)(2.000,-1.000){2}{\rule{0.482pt}{0.400pt}}
\put(965,599.17){\rule{0.900pt}{0.400pt}}
\multiput(965.00,600.17)(2.132,-2.000){2}{\rule{0.450pt}{0.400pt}}
\put(969,597.17){\rule{0.900pt}{0.400pt}}
\multiput(969.00,598.17)(2.132,-2.000){2}{\rule{0.450pt}{0.400pt}}
\multiput(973.00,595.95)(0.685,-0.447){3}{\rule{0.633pt}{0.108pt}}
\multiput(973.00,596.17)(2.685,-3.000){2}{\rule{0.317pt}{0.400pt}}
\put(977,592.17){\rule{0.900pt}{0.400pt}}
\multiput(977.00,593.17)(2.132,-2.000){2}{\rule{0.450pt}{0.400pt}}
\multiput(981.00,590.95)(0.685,-0.447){3}{\rule{0.633pt}{0.108pt}}
\multiput(981.00,591.17)(2.685,-3.000){2}{\rule{0.317pt}{0.400pt}}
\multiput(985.00,587.95)(0.685,-0.447){3}{\rule{0.633pt}{0.108pt}}
\multiput(985.00,588.17)(2.685,-3.000){2}{\rule{0.317pt}{0.400pt}}
\multiput(989.00,584.95)(0.685,-0.447){3}{\rule{0.633pt}{0.108pt}}
\multiput(989.00,585.17)(2.685,-3.000){2}{\rule{0.317pt}{0.400pt}}
\multiput(993.00,581.95)(0.685,-0.447){3}{\rule{0.633pt}{0.108pt}}
\multiput(993.00,582.17)(2.685,-3.000){2}{\rule{0.317pt}{0.400pt}}
\multiput(997.00,578.94)(0.481,-0.468){5}{\rule{0.500pt}{0.113pt}}
\multiput(997.00,579.17)(2.962,-4.000){2}{\rule{0.250pt}{0.400pt}}
\multiput(1001.00,574.95)(0.685,-0.447){3}{\rule{0.633pt}{0.108pt}}
\multiput(1001.00,575.17)(2.685,-3.000){2}{\rule{0.317pt}{0.400pt}}
\multiput(1005.00,571.94)(0.481,-0.468){5}{\rule{0.500pt}{0.113pt}}
\multiput(1005.00,572.17)(2.962,-4.000){2}{\rule{0.250pt}{0.400pt}}
\multiput(1009.00,567.94)(0.481,-0.468){5}{\rule{0.500pt}{0.113pt}}
\multiput(1009.00,568.17)(2.962,-4.000){2}{\rule{0.250pt}{0.400pt}}
\multiput(1013.61,562.37)(0.447,-0.685){3}{\rule{0.108pt}{0.633pt}}
\multiput(1012.17,563.69)(3.000,-2.685){2}{\rule{0.400pt}{0.317pt}}
\multiput(1016.60,558.51)(0.468,-0.627){5}{\rule{0.113pt}{0.600pt}}
\multiput(1015.17,559.75)(4.000,-3.755){2}{\rule{0.400pt}{0.300pt}}
\multiput(1020.00,554.94)(0.481,-0.468){5}{\rule{0.500pt}{0.113pt}}
\multiput(1020.00,555.17)(2.962,-4.000){2}{\rule{0.250pt}{0.400pt}}
\multiput(1024.60,549.51)(0.468,-0.627){5}{\rule{0.113pt}{0.600pt}}
\multiput(1023.17,550.75)(4.000,-3.755){2}{\rule{0.400pt}{0.300pt}}
\multiput(1028.60,544.51)(0.468,-0.627){5}{\rule{0.113pt}{0.600pt}}
\multiput(1027.17,545.75)(4.000,-3.755){2}{\rule{0.400pt}{0.300pt}}
\multiput(1032.60,539.51)(0.468,-0.627){5}{\rule{0.113pt}{0.600pt}}
\multiput(1031.17,540.75)(4.000,-3.755){2}{\rule{0.400pt}{0.300pt}}
\multiput(1036.60,534.51)(0.468,-0.627){5}{\rule{0.113pt}{0.600pt}}
\multiput(1035.17,535.75)(4.000,-3.755){2}{\rule{0.400pt}{0.300pt}}
\multiput(1040.60,529.09)(0.468,-0.774){5}{\rule{0.113pt}{0.700pt}}
\multiput(1039.17,530.55)(4.000,-4.547){2}{\rule{0.400pt}{0.350pt}}
\multiput(1044.60,523.09)(0.468,-0.774){5}{\rule{0.113pt}{0.700pt}}
\multiput(1043.17,524.55)(4.000,-4.547){2}{\rule{0.400pt}{0.350pt}}
\multiput(1048.60,517.09)(0.468,-0.774){5}{\rule{0.113pt}{0.700pt}}
\multiput(1047.17,518.55)(4.000,-4.547){2}{\rule{0.400pt}{0.350pt}}
\multiput(1052.60,511.09)(0.468,-0.774){5}{\rule{0.113pt}{0.700pt}}
\multiput(1051.17,512.55)(4.000,-4.547){2}{\rule{0.400pt}{0.350pt}}
\multiput(1056.60,504.68)(0.468,-0.920){5}{\rule{0.113pt}{0.800pt}}
\multiput(1055.17,506.34)(4.000,-5.340){2}{\rule{0.400pt}{0.400pt}}
\multiput(1060.60,498.09)(0.468,-0.774){5}{\rule{0.113pt}{0.700pt}}
\multiput(1059.17,499.55)(4.000,-4.547){2}{\rule{0.400pt}{0.350pt}}
\multiput(1064.60,491.68)(0.468,-0.920){5}{\rule{0.113pt}{0.800pt}}
\multiput(1063.17,493.34)(4.000,-5.340){2}{\rule{0.400pt}{0.400pt}}
\multiput(1068.60,484.26)(0.468,-1.066){5}{\rule{0.113pt}{0.900pt}}
\multiput(1067.17,486.13)(4.000,-6.132){2}{\rule{0.400pt}{0.450pt}}
\multiput(714.60,293.00)(0.468,1.358){5}{\rule{0.113pt}{1.100pt}}
\multiput(713.17,293.00)(4.000,7.717){2}{\rule{0.400pt}{0.550pt}}
\multiput(718.60,303.00)(0.468,1.358){5}{\rule{0.113pt}{1.100pt}}
\multiput(717.17,303.00)(4.000,7.717){2}{\rule{0.400pt}{0.550pt}}
\multiput(722.60,313.00)(0.468,1.358){5}{\rule{0.113pt}{1.100pt}}
\multiput(721.17,313.00)(4.000,7.717){2}{\rule{0.400pt}{0.550pt}}
\multiput(726.60,323.00)(0.468,1.212){5}{\rule{0.113pt}{1.000pt}}
\multiput(725.17,323.00)(4.000,6.924){2}{\rule{0.400pt}{0.500pt}}
\multiput(730.60,332.00)(0.468,1.212){5}{\rule{0.113pt}{1.000pt}}
\multiput(729.17,332.00)(4.000,6.924){2}{\rule{0.400pt}{0.500pt}}
\multiput(734.60,341.00)(0.468,1.212){5}{\rule{0.113pt}{1.000pt}}
\multiput(733.17,341.00)(4.000,6.924){2}{\rule{0.400pt}{0.500pt}}
\multiput(738.60,350.00)(0.468,1.212){5}{\rule{0.113pt}{1.000pt}}
\multiput(737.17,350.00)(4.000,6.924){2}{\rule{0.400pt}{0.500pt}}
\multiput(742.60,359.00)(0.468,1.066){5}{\rule{0.113pt}{0.900pt}}
\multiput(741.17,359.00)(4.000,6.132){2}{\rule{0.400pt}{0.450pt}}
\multiput(746.60,367.00)(0.468,1.212){5}{\rule{0.113pt}{1.000pt}}
\multiput(745.17,367.00)(4.000,6.924){2}{\rule{0.400pt}{0.500pt}}
\multiput(750.60,376.00)(0.468,1.066){5}{\rule{0.113pt}{0.900pt}}
\multiput(749.17,376.00)(4.000,6.132){2}{\rule{0.400pt}{0.450pt}}
\multiput(754.60,384.00)(0.468,1.066){5}{\rule{0.113pt}{0.900pt}}
\multiput(753.17,384.00)(4.000,6.132){2}{\rule{0.400pt}{0.450pt}}
\multiput(758.60,392.00)(0.468,1.066){5}{\rule{0.113pt}{0.900pt}}
\multiput(757.17,392.00)(4.000,6.132){2}{\rule{0.400pt}{0.450pt}}
\multiput(762.60,400.00)(0.468,0.920){5}{\rule{0.113pt}{0.800pt}}
\multiput(761.17,400.00)(4.000,5.340){2}{\rule{0.400pt}{0.400pt}}
\multiput(766.60,407.00)(0.468,1.066){5}{\rule{0.113pt}{0.900pt}}
\multiput(765.17,407.00)(4.000,6.132){2}{\rule{0.400pt}{0.450pt}}
\multiput(770.60,415.00)(0.468,0.920){5}{\rule{0.113pt}{0.800pt}}
\multiput(769.17,415.00)(4.000,5.340){2}{\rule{0.400pt}{0.400pt}}
\multiput(774.60,422.00)(0.468,0.920){5}{\rule{0.113pt}{0.800pt}}
\multiput(773.17,422.00)(4.000,5.340){2}{\rule{0.400pt}{0.400pt}}
\multiput(778.60,429.00)(0.468,0.920){5}{\rule{0.113pt}{0.800pt}}
\multiput(777.17,429.00)(4.000,5.340){2}{\rule{0.400pt}{0.400pt}}
\multiput(782.60,436.00)(0.468,0.920){5}{\rule{0.113pt}{0.800pt}}
\multiput(781.17,436.00)(4.000,5.340){2}{\rule{0.400pt}{0.400pt}}
\multiput(786.60,443.00)(0.468,0.774){5}{\rule{0.113pt}{0.700pt}}
\multiput(785.17,443.00)(4.000,4.547){2}{\rule{0.400pt}{0.350pt}}
\multiput(790.60,449.00)(0.468,0.920){5}{\rule{0.113pt}{0.800pt}}
\multiput(789.17,449.00)(4.000,5.340){2}{\rule{0.400pt}{0.400pt}}
\multiput(794.60,456.00)(0.468,0.774){5}{\rule{0.113pt}{0.700pt}}
\multiput(793.17,456.00)(4.000,4.547){2}{\rule{0.400pt}{0.350pt}}
\multiput(798.60,462.00)(0.468,0.774){5}{\rule{0.113pt}{0.700pt}}
\multiput(797.17,462.00)(4.000,4.547){2}{\rule{0.400pt}{0.350pt}}
\multiput(802.60,468.00)(0.468,0.774){5}{\rule{0.113pt}{0.700pt}}
\multiput(801.17,468.00)(4.000,4.547){2}{\rule{0.400pt}{0.350pt}}
\multiput(806.60,474.00)(0.468,0.774){5}{\rule{0.113pt}{0.700pt}}
\multiput(805.17,474.00)(4.000,4.547){2}{\rule{0.400pt}{0.350pt}}
\multiput(810.60,480.00)(0.468,0.627){5}{\rule{0.113pt}{0.600pt}}
\multiput(809.17,480.00)(4.000,3.755){2}{\rule{0.400pt}{0.300pt}}
\multiput(814.60,485.00)(0.468,0.774){5}{\rule{0.113pt}{0.700pt}}
\multiput(813.17,485.00)(4.000,4.547){2}{\rule{0.400pt}{0.350pt}}
\multiput(818.60,491.00)(0.468,0.627){5}{\rule{0.113pt}{0.600pt}}
\multiput(817.17,491.00)(4.000,3.755){2}{\rule{0.400pt}{0.300pt}}
\multiput(822.60,496.00)(0.468,0.627){5}{\rule{0.113pt}{0.600pt}}
\multiput(821.17,496.00)(4.000,3.755){2}{\rule{0.400pt}{0.300pt}}
\multiput(826.60,501.00)(0.468,0.627){5}{\rule{0.113pt}{0.600pt}}
\multiput(825.17,501.00)(4.000,3.755){2}{\rule{0.400pt}{0.300pt}}
\multiput(830.60,506.00)(0.468,0.627){5}{\rule{0.113pt}{0.600pt}}
\multiput(829.17,506.00)(4.000,3.755){2}{\rule{0.400pt}{0.300pt}}
\multiput(834.60,511.00)(0.468,0.627){5}{\rule{0.113pt}{0.600pt}}
\multiput(833.17,511.00)(4.000,3.755){2}{\rule{0.400pt}{0.300pt}}
\multiput(838.61,516.00)(0.447,0.685){3}{\rule{0.108pt}{0.633pt}}
\multiput(837.17,516.00)(3.000,2.685){2}{\rule{0.400pt}{0.317pt}}
\multiput(841.60,520.00)(0.468,0.627){5}{\rule{0.113pt}{0.600pt}}
\multiput(840.17,520.00)(4.000,3.755){2}{\rule{0.400pt}{0.300pt}}
\multiput(845.00,525.60)(0.481,0.468){5}{\rule{0.500pt}{0.113pt}}
\multiput(845.00,524.17)(2.962,4.000){2}{\rule{0.250pt}{0.400pt}}
\multiput(849.00,529.60)(0.481,0.468){5}{\rule{0.500pt}{0.113pt}}
\multiput(849.00,528.17)(2.962,4.000){2}{\rule{0.250pt}{0.400pt}}
\multiput(853.00,533.60)(0.481,0.468){5}{\rule{0.500pt}{0.113pt}}
\multiput(853.00,532.17)(2.962,4.000){2}{\rule{0.250pt}{0.400pt}}
\multiput(857.00,537.61)(0.685,0.447){3}{\rule{0.633pt}{0.108pt}}
\multiput(857.00,536.17)(2.685,3.000){2}{\rule{0.317pt}{0.400pt}}
\multiput(861.00,540.60)(0.481,0.468){5}{\rule{0.500pt}{0.113pt}}
\multiput(861.00,539.17)(2.962,4.000){2}{\rule{0.250pt}{0.400pt}}
\multiput(865.00,544.61)(0.685,0.447){3}{\rule{0.633pt}{0.108pt}}
\multiput(865.00,543.17)(2.685,3.000){2}{\rule{0.317pt}{0.400pt}}
\multiput(869.00,547.60)(0.481,0.468){5}{\rule{0.500pt}{0.113pt}}
\multiput(869.00,546.17)(2.962,4.000){2}{\rule{0.250pt}{0.400pt}}
\multiput(873.00,551.61)(0.685,0.447){3}{\rule{0.633pt}{0.108pt}}
\multiput(873.00,550.17)(2.685,3.000){2}{\rule{0.317pt}{0.400pt}}
\multiput(877.00,554.61)(0.685,0.447){3}{\rule{0.633pt}{0.108pt}}
\multiput(877.00,553.17)(2.685,3.000){2}{\rule{0.317pt}{0.400pt}}
\multiput(881.00,557.61)(0.685,0.447){3}{\rule{0.633pt}{0.108pt}}
\multiput(881.00,556.17)(2.685,3.000){2}{\rule{0.317pt}{0.400pt}}
\put(885,560.17){\rule{0.900pt}{0.400pt}}
\multiput(885.00,559.17)(2.132,2.000){2}{\rule{0.450pt}{0.400pt}}
\multiput(889.00,562.61)(0.685,0.447){3}{\rule{0.633pt}{0.108pt}}
\multiput(889.00,561.17)(2.685,3.000){2}{\rule{0.317pt}{0.400pt}}
\put(893,565.17){\rule{0.900pt}{0.400pt}}
\multiput(893.00,564.17)(2.132,2.000){2}{\rule{0.450pt}{0.400pt}}
\multiput(897.00,567.61)(0.685,0.447){3}{\rule{0.633pt}{0.108pt}}
\multiput(897.00,566.17)(2.685,3.000){2}{\rule{0.317pt}{0.400pt}}
\put(901,570.17){\rule{0.900pt}{0.400pt}}
\multiput(901.00,569.17)(2.132,2.000){2}{\rule{0.450pt}{0.400pt}}
\put(905,572.17){\rule{0.900pt}{0.400pt}}
\multiput(905.00,571.17)(2.132,2.000){2}{\rule{0.450pt}{0.400pt}}
\put(909,574.17){\rule{0.900pt}{0.400pt}}
\multiput(909.00,573.17)(2.132,2.000){2}{\rule{0.450pt}{0.400pt}}
\put(913,575.67){\rule{0.964pt}{0.400pt}}
\multiput(913.00,575.17)(2.000,1.000){2}{\rule{0.482pt}{0.400pt}}
\put(917,577.17){\rule{0.900pt}{0.400pt}}
\multiput(917.00,576.17)(2.132,2.000){2}{\rule{0.450pt}{0.400pt}}
\put(921,578.67){\rule{0.964pt}{0.400pt}}
\multiput(921.00,578.17)(2.000,1.000){2}{\rule{0.482pt}{0.400pt}}
\put(925,579.67){\rule{0.964pt}{0.400pt}}
\multiput(925.00,579.17)(2.000,1.000){2}{\rule{0.482pt}{0.400pt}}
\put(929,581.17){\rule{0.900pt}{0.400pt}}
\multiput(929.00,580.17)(2.132,2.000){2}{\rule{0.450pt}{0.400pt}}
\put(933,582.67){\rule{0.964pt}{0.400pt}}
\multiput(933.00,582.17)(2.000,1.000){2}{\rule{0.482pt}{0.400pt}}
\put(933.0,610.0){\rule[-0.200pt]{0.964pt}{0.400pt}}
\put(941,583.67){\rule{0.964pt}{0.400pt}}
\multiput(941.00,583.17)(2.000,1.000){2}{\rule{0.482pt}{0.400pt}}
\put(945,584.67){\rule{0.964pt}{0.400pt}}
\multiput(945.00,584.17)(2.000,1.000){2}{\rule{0.482pt}{0.400pt}}
\put(937.0,584.0){\rule[-0.200pt]{0.964pt}{0.400pt}}
\put(949.0,586.0){\rule[-0.200pt]{0.964pt}{0.400pt}}
\put(953.0,586.0){\rule[-0.200pt]{0.964pt}{0.400pt}}
\put(957.0,586.0){\rule[-0.200pt]{0.964pt}{0.400pt}}
\put(961.0,586.0){\rule[-0.200pt]{0.964pt}{0.400pt}}
\put(965.0,586.0){\rule[-0.200pt]{0.964pt}{0.400pt}}
\put(973,584.67){\rule{0.964pt}{0.400pt}}
\multiput(973.00,585.17)(2.000,-1.000){2}{\rule{0.482pt}{0.400pt}}
\put(977,583.67){\rule{0.964pt}{0.400pt}}
\multiput(977.00,584.17)(2.000,-1.000){2}{\rule{0.482pt}{0.400pt}}
\put(981,582.67){\rule{0.964pt}{0.400pt}}
\multiput(981.00,583.17)(2.000,-1.000){2}{\rule{0.482pt}{0.400pt}}
\put(969.0,586.0){\rule[-0.200pt]{0.964pt}{0.400pt}}
\put(989,581.17){\rule{0.900pt}{0.400pt}}
\multiput(989.00,582.17)(2.132,-2.000){2}{\rule{0.450pt}{0.400pt}}
\put(993,579.67){\rule{0.964pt}{0.400pt}}
\multiput(993.00,580.17)(2.000,-1.000){2}{\rule{0.482pt}{0.400pt}}
\put(997,578.67){\rule{0.964pt}{0.400pt}}
\multiput(997.00,579.17)(2.000,-1.000){2}{\rule{0.482pt}{0.400pt}}
\put(1001,577.17){\rule{0.900pt}{0.400pt}}
\multiput(1001.00,578.17)(2.132,-2.000){2}{\rule{0.450pt}{0.400pt}}
\put(1005,575.17){\rule{0.900pt}{0.400pt}}
\multiput(1005.00,576.17)(2.132,-2.000){2}{\rule{0.450pt}{0.400pt}}
\put(1009,573.17){\rule{0.900pt}{0.400pt}}
\multiput(1009.00,574.17)(2.132,-2.000){2}{\rule{0.450pt}{0.400pt}}
\put(1013,571.17){\rule{0.900pt}{0.400pt}}
\multiput(1013.00,572.17)(2.132,-2.000){2}{\rule{0.450pt}{0.400pt}}
\put(1017,569.17){\rule{0.900pt}{0.400pt}}
\multiput(1017.00,570.17)(2.132,-2.000){2}{\rule{0.450pt}{0.400pt}}
\put(1021,567.17){\rule{0.900pt}{0.400pt}}
\multiput(1021.00,568.17)(2.132,-2.000){2}{\rule{0.450pt}{0.400pt}}
\multiput(1025.00,565.95)(0.685,-0.447){3}{\rule{0.633pt}{0.108pt}}
\multiput(1025.00,566.17)(2.685,-3.000){2}{\rule{0.317pt}{0.400pt}}
\multiput(1029.00,562.95)(0.685,-0.447){3}{\rule{0.633pt}{0.108pt}}
\multiput(1029.00,563.17)(2.685,-3.000){2}{\rule{0.317pt}{0.400pt}}
\multiput(1033.00,559.95)(0.462,-0.447){3}{\rule{0.500pt}{0.108pt}}
\multiput(1033.00,560.17)(1.962,-3.000){2}{\rule{0.250pt}{0.400pt}}
\multiput(1036.00,556.95)(0.685,-0.447){3}{\rule{0.633pt}{0.108pt}}
\multiput(1036.00,557.17)(2.685,-3.000){2}{\rule{0.317pt}{0.400pt}}
\multiput(1040.00,553.95)(0.685,-0.447){3}{\rule{0.633pt}{0.108pt}}
\multiput(1040.00,554.17)(2.685,-3.000){2}{\rule{0.317pt}{0.400pt}}
\multiput(1044.00,550.95)(0.685,-0.447){3}{\rule{0.633pt}{0.108pt}}
\multiput(1044.00,551.17)(2.685,-3.000){2}{\rule{0.317pt}{0.400pt}}
\multiput(1048.00,547.94)(0.481,-0.468){5}{\rule{0.500pt}{0.113pt}}
\multiput(1048.00,548.17)(2.962,-4.000){2}{\rule{0.250pt}{0.400pt}}
\multiput(1052.00,543.94)(0.481,-0.468){5}{\rule{0.500pt}{0.113pt}}
\multiput(1052.00,544.17)(2.962,-4.000){2}{\rule{0.250pt}{0.400pt}}
\multiput(1056.00,539.94)(0.481,-0.468){5}{\rule{0.500pt}{0.113pt}}
\multiput(1056.00,540.17)(2.962,-4.000){2}{\rule{0.250pt}{0.400pt}}
\multiput(1060.00,535.94)(0.481,-0.468){5}{\rule{0.500pt}{0.113pt}}
\multiput(1060.00,536.17)(2.962,-4.000){2}{\rule{0.250pt}{0.400pt}}
\multiput(1064.00,531.94)(0.481,-0.468){5}{\rule{0.500pt}{0.113pt}}
\multiput(1064.00,532.17)(2.962,-4.000){2}{\rule{0.250pt}{0.400pt}}
\multiput(1068.60,526.51)(0.468,-0.627){5}{\rule{0.113pt}{0.600pt}}
\multiput(1067.17,527.75)(4.000,-3.755){2}{\rule{0.400pt}{0.300pt}}
\multiput(1072.00,522.94)(0.481,-0.468){5}{\rule{0.500pt}{0.113pt}}
\multiput(1072.00,523.17)(2.962,-4.000){2}{\rule{0.250pt}{0.400pt}}
\multiput(1076.60,517.51)(0.468,-0.627){5}{\rule{0.113pt}{0.600pt}}
\multiput(1075.17,518.75)(4.000,-3.755){2}{\rule{0.400pt}{0.300pt}}
\multiput(1080.60,512.51)(0.468,-0.627){5}{\rule{0.113pt}{0.600pt}}
\multiput(1079.17,513.75)(4.000,-3.755){2}{\rule{0.400pt}{0.300pt}}
\multiput(1084.60,507.09)(0.468,-0.774){5}{\rule{0.113pt}{0.700pt}}
\multiput(1083.17,508.55)(4.000,-4.547){2}{\rule{0.400pt}{0.350pt}}
\multiput(1088.60,501.51)(0.468,-0.627){5}{\rule{0.113pt}{0.600pt}}
\multiput(1087.17,502.75)(4.000,-3.755){2}{\rule{0.400pt}{0.300pt}}
\multiput(1092.60,496.09)(0.468,-0.774){5}{\rule{0.113pt}{0.700pt}}
\multiput(1091.17,497.55)(4.000,-4.547){2}{\rule{0.400pt}{0.350pt}}
\multiput(1096.60,490.09)(0.468,-0.774){5}{\rule{0.113pt}{0.700pt}}
\multiput(1095.17,491.55)(4.000,-4.547){2}{\rule{0.400pt}{0.350pt}}
\multiput(1100.60,484.09)(0.468,-0.774){5}{\rule{0.113pt}{0.700pt}}
\multiput(1099.17,485.55)(4.000,-4.547){2}{\rule{0.400pt}{0.350pt}}
\multiput(1104.60,478.09)(0.468,-0.774){5}{\rule{0.113pt}{0.700pt}}
\multiput(1103.17,479.55)(4.000,-4.547){2}{\rule{0.400pt}{0.350pt}}
\multiput(750.60,288.00)(0.468,1.212){5}{\rule{0.113pt}{1.000pt}}
\multiput(749.17,288.00)(4.000,6.924){2}{\rule{0.400pt}{0.500pt}}
\multiput(754.60,297.00)(0.468,1.212){5}{\rule{0.113pt}{1.000pt}}
\multiput(753.17,297.00)(4.000,6.924){2}{\rule{0.400pt}{0.500pt}}
\multiput(758.60,306.00)(0.468,1.066){5}{\rule{0.113pt}{0.900pt}}
\multiput(757.17,306.00)(4.000,6.132){2}{\rule{0.400pt}{0.450pt}}
\multiput(762.60,314.00)(0.468,1.066){5}{\rule{0.113pt}{0.900pt}}
\multiput(761.17,314.00)(4.000,6.132){2}{\rule{0.400pt}{0.450pt}}
\multiput(766.60,322.00)(0.468,1.066){5}{\rule{0.113pt}{0.900pt}}
\multiput(765.17,322.00)(4.000,6.132){2}{\rule{0.400pt}{0.450pt}}
\multiput(770.60,330.00)(0.468,1.066){5}{\rule{0.113pt}{0.900pt}}
\multiput(769.17,330.00)(4.000,6.132){2}{\rule{0.400pt}{0.450pt}}
\multiput(774.60,338.00)(0.468,1.066){5}{\rule{0.113pt}{0.900pt}}
\multiput(773.17,338.00)(4.000,6.132){2}{\rule{0.400pt}{0.450pt}}
\multiput(778.60,346.00)(0.468,1.066){5}{\rule{0.113pt}{0.900pt}}
\multiput(777.17,346.00)(4.000,6.132){2}{\rule{0.400pt}{0.450pt}}
\multiput(782.60,354.00)(0.468,0.920){5}{\rule{0.113pt}{0.800pt}}
\multiput(781.17,354.00)(4.000,5.340){2}{\rule{0.400pt}{0.400pt}}
\multiput(786.60,361.00)(0.468,1.066){5}{\rule{0.113pt}{0.900pt}}
\multiput(785.17,361.00)(4.000,6.132){2}{\rule{0.400pt}{0.450pt}}
\multiput(790.60,369.00)(0.468,0.920){5}{\rule{0.113pt}{0.800pt}}
\multiput(789.17,369.00)(4.000,5.340){2}{\rule{0.400pt}{0.400pt}}
\multiput(794.60,376.00)(0.468,0.920){5}{\rule{0.113pt}{0.800pt}}
\multiput(793.17,376.00)(4.000,5.340){2}{\rule{0.400pt}{0.400pt}}
\multiput(798.60,383.00)(0.468,0.920){5}{\rule{0.113pt}{0.800pt}}
\multiput(797.17,383.00)(4.000,5.340){2}{\rule{0.400pt}{0.400pt}}
\multiput(802.60,390.00)(0.468,0.774){5}{\rule{0.113pt}{0.700pt}}
\multiput(801.17,390.00)(4.000,4.547){2}{\rule{0.400pt}{0.350pt}}
\multiput(806.60,396.00)(0.468,0.920){5}{\rule{0.113pt}{0.800pt}}
\multiput(805.17,396.00)(4.000,5.340){2}{\rule{0.400pt}{0.400pt}}
\multiput(810.60,403.00)(0.468,0.774){5}{\rule{0.113pt}{0.700pt}}
\multiput(809.17,403.00)(4.000,4.547){2}{\rule{0.400pt}{0.350pt}}
\multiput(814.60,409.00)(0.468,0.774){5}{\rule{0.113pt}{0.700pt}}
\multiput(813.17,409.00)(4.000,4.547){2}{\rule{0.400pt}{0.350pt}}
\multiput(818.60,415.00)(0.468,0.774){5}{\rule{0.113pt}{0.700pt}}
\multiput(817.17,415.00)(4.000,4.547){2}{\rule{0.400pt}{0.350pt}}
\multiput(822.60,421.00)(0.468,0.774){5}{\rule{0.113pt}{0.700pt}}
\multiput(821.17,421.00)(4.000,4.547){2}{\rule{0.400pt}{0.350pt}}
\multiput(826.60,427.00)(0.468,0.774){5}{\rule{0.113pt}{0.700pt}}
\multiput(825.17,427.00)(4.000,4.547){2}{\rule{0.400pt}{0.350pt}}
\multiput(830.60,433.00)(0.468,0.774){5}{\rule{0.113pt}{0.700pt}}
\multiput(829.17,433.00)(4.000,4.547){2}{\rule{0.400pt}{0.350pt}}
\multiput(834.60,439.00)(0.468,0.627){5}{\rule{0.113pt}{0.600pt}}
\multiput(833.17,439.00)(4.000,3.755){2}{\rule{0.400pt}{0.300pt}}
\multiput(838.60,444.00)(0.468,0.774){5}{\rule{0.113pt}{0.700pt}}
\multiput(837.17,444.00)(4.000,4.547){2}{\rule{0.400pt}{0.350pt}}
\multiput(842.60,450.00)(0.468,0.627){5}{\rule{0.113pt}{0.600pt}}
\multiput(841.17,450.00)(4.000,3.755){2}{\rule{0.400pt}{0.300pt}}
\multiput(846.60,455.00)(0.468,0.627){5}{\rule{0.113pt}{0.600pt}}
\multiput(845.17,455.00)(4.000,3.755){2}{\rule{0.400pt}{0.300pt}}
\multiput(850.60,460.00)(0.468,0.627){5}{\rule{0.113pt}{0.600pt}}
\multiput(849.17,460.00)(4.000,3.755){2}{\rule{0.400pt}{0.300pt}}
\multiput(854.61,465.00)(0.447,0.909){3}{\rule{0.108pt}{0.767pt}}
\multiput(853.17,465.00)(3.000,3.409){2}{\rule{0.400pt}{0.383pt}}
\multiput(857.00,470.60)(0.481,0.468){5}{\rule{0.500pt}{0.113pt}}
\multiput(857.00,469.17)(2.962,4.000){2}{\rule{0.250pt}{0.400pt}}
\multiput(861.60,474.00)(0.468,0.627){5}{\rule{0.113pt}{0.600pt}}
\multiput(860.17,474.00)(4.000,3.755){2}{\rule{0.400pt}{0.300pt}}
\multiput(865.00,479.60)(0.481,0.468){5}{\rule{0.500pt}{0.113pt}}
\multiput(865.00,478.17)(2.962,4.000){2}{\rule{0.250pt}{0.400pt}}
\multiput(869.60,483.00)(0.468,0.627){5}{\rule{0.113pt}{0.600pt}}
\multiput(868.17,483.00)(4.000,3.755){2}{\rule{0.400pt}{0.300pt}}
\multiput(873.00,488.60)(0.481,0.468){5}{\rule{0.500pt}{0.113pt}}
\multiput(873.00,487.17)(2.962,4.000){2}{\rule{0.250pt}{0.400pt}}
\multiput(877.00,492.60)(0.481,0.468){5}{\rule{0.500pt}{0.113pt}}
\multiput(877.00,491.17)(2.962,4.000){2}{\rule{0.250pt}{0.400pt}}
\multiput(881.00,496.60)(0.481,0.468){5}{\rule{0.500pt}{0.113pt}}
\multiput(881.00,495.17)(2.962,4.000){2}{\rule{0.250pt}{0.400pt}}
\multiput(885.00,500.61)(0.685,0.447){3}{\rule{0.633pt}{0.108pt}}
\multiput(885.00,499.17)(2.685,3.000){2}{\rule{0.317pt}{0.400pt}}
\multiput(889.00,503.60)(0.481,0.468){5}{\rule{0.500pt}{0.113pt}}
\multiput(889.00,502.17)(2.962,4.000){2}{\rule{0.250pt}{0.400pt}}
\multiput(893.00,507.61)(0.685,0.447){3}{\rule{0.633pt}{0.108pt}}
\multiput(893.00,506.17)(2.685,3.000){2}{\rule{0.317pt}{0.400pt}}
\multiput(897.00,510.60)(0.481,0.468){5}{\rule{0.500pt}{0.113pt}}
\multiput(897.00,509.17)(2.962,4.000){2}{\rule{0.250pt}{0.400pt}}
\multiput(901.00,514.61)(0.685,0.447){3}{\rule{0.633pt}{0.108pt}}
\multiput(901.00,513.17)(2.685,3.000){2}{\rule{0.317pt}{0.400pt}}
\multiput(905.00,517.61)(0.685,0.447){3}{\rule{0.633pt}{0.108pt}}
\multiput(905.00,516.17)(2.685,3.000){2}{\rule{0.317pt}{0.400pt}}
\multiput(909.00,520.61)(0.685,0.447){3}{\rule{0.633pt}{0.108pt}}
\multiput(909.00,519.17)(2.685,3.000){2}{\rule{0.317pt}{0.400pt}}
\multiput(913.00,523.61)(0.685,0.447){3}{\rule{0.633pt}{0.108pt}}
\multiput(913.00,522.17)(2.685,3.000){2}{\rule{0.317pt}{0.400pt}}
\multiput(917.00,526.61)(0.685,0.447){3}{\rule{0.633pt}{0.108pt}}
\multiput(917.00,525.17)(2.685,3.000){2}{\rule{0.317pt}{0.400pt}}
\put(921,529.17){\rule{0.900pt}{0.400pt}}
\multiput(921.00,528.17)(2.132,2.000){2}{\rule{0.450pt}{0.400pt}}
\multiput(925.00,531.61)(0.685,0.447){3}{\rule{0.633pt}{0.108pt}}
\multiput(925.00,530.17)(2.685,3.000){2}{\rule{0.317pt}{0.400pt}}
\put(929,534.17){\rule{0.900pt}{0.400pt}}
\multiput(929.00,533.17)(2.132,2.000){2}{\rule{0.450pt}{0.400pt}}
\put(933,536.17){\rule{0.900pt}{0.400pt}}
\multiput(933.00,535.17)(2.132,2.000){2}{\rule{0.450pt}{0.400pt}}
\put(937,538.17){\rule{0.900pt}{0.400pt}}
\multiput(937.00,537.17)(2.132,2.000){2}{\rule{0.450pt}{0.400pt}}
\put(941,540.17){\rule{0.900pt}{0.400pt}}
\multiput(941.00,539.17)(2.132,2.000){2}{\rule{0.450pt}{0.400pt}}
\put(945,542.17){\rule{0.900pt}{0.400pt}}
\multiput(945.00,541.17)(2.132,2.000){2}{\rule{0.450pt}{0.400pt}}
\put(949,544.17){\rule{0.900pt}{0.400pt}}
\multiput(949.00,543.17)(2.132,2.000){2}{\rule{0.450pt}{0.400pt}}
\put(953,545.67){\rule{0.964pt}{0.400pt}}
\multiput(953.00,545.17)(2.000,1.000){2}{\rule{0.482pt}{0.400pt}}
\put(957,547.17){\rule{0.900pt}{0.400pt}}
\multiput(957.00,546.17)(2.132,2.000){2}{\rule{0.450pt}{0.400pt}}
\put(961,548.67){\rule{0.964pt}{0.400pt}}
\multiput(961.00,548.17)(2.000,1.000){2}{\rule{0.482pt}{0.400pt}}
\put(965,549.67){\rule{0.964pt}{0.400pt}}
\multiput(965.00,549.17)(2.000,1.000){2}{\rule{0.482pt}{0.400pt}}
\put(969,550.67){\rule{0.964pt}{0.400pt}}
\multiput(969.00,550.17)(2.000,1.000){2}{\rule{0.482pt}{0.400pt}}
\put(973,551.67){\rule{0.964pt}{0.400pt}}
\multiput(973.00,551.17)(2.000,1.000){2}{\rule{0.482pt}{0.400pt}}
\put(977,552.67){\rule{0.964pt}{0.400pt}}
\multiput(977.00,552.17)(2.000,1.000){2}{\rule{0.482pt}{0.400pt}}
\put(981,553.67){\rule{0.964pt}{0.400pt}}
\multiput(981.00,553.17)(2.000,1.000){2}{\rule{0.482pt}{0.400pt}}
\put(985,554.67){\rule{0.964pt}{0.400pt}}
\multiput(985.00,554.17)(2.000,1.000){2}{\rule{0.482pt}{0.400pt}}
\put(985.0,583.0){\rule[-0.200pt]{0.964pt}{0.400pt}}
\put(989.0,556.0){\rule[-0.200pt]{0.964pt}{0.400pt}}
\put(993.0,556.0){\rule[-0.200pt]{0.964pt}{0.400pt}}
\put(1001,555.67){\rule{0.964pt}{0.400pt}}
\multiput(1001.00,555.17)(2.000,1.000){2}{\rule{0.482pt}{0.400pt}}
\put(1005,555.67){\rule{0.964pt}{0.400pt}}
\multiput(1005.00,556.17)(2.000,-1.000){2}{\rule{0.482pt}{0.400pt}}
\put(997.0,556.0){\rule[-0.200pt]{0.964pt}{0.400pt}}
\put(1009.0,556.0){\rule[-0.200pt]{0.964pt}{0.400pt}}
\put(1017,554.67){\rule{0.964pt}{0.400pt}}
\multiput(1017.00,555.17)(2.000,-1.000){2}{\rule{0.482pt}{0.400pt}}
\put(1013.0,556.0){\rule[-0.200pt]{0.964pt}{0.400pt}}
\put(1025,553.67){\rule{0.964pt}{0.400pt}}
\multiput(1025.00,554.17)(2.000,-1.000){2}{\rule{0.482pt}{0.400pt}}
\put(1029,552.67){\rule{0.964pt}{0.400pt}}
\multiput(1029.00,553.17)(2.000,-1.000){2}{\rule{0.482pt}{0.400pt}}
\put(1033,551.67){\rule{0.964pt}{0.400pt}}
\multiput(1033.00,552.17)(2.000,-1.000){2}{\rule{0.482pt}{0.400pt}}
\put(1037,550.67){\rule{0.964pt}{0.400pt}}
\multiput(1037.00,551.17)(2.000,-1.000){2}{\rule{0.482pt}{0.400pt}}
\put(1041,549.67){\rule{0.964pt}{0.400pt}}
\multiput(1041.00,550.17)(2.000,-1.000){2}{\rule{0.482pt}{0.400pt}}
\put(1045,548.17){\rule{0.900pt}{0.400pt}}
\multiput(1045.00,549.17)(2.132,-2.000){2}{\rule{0.450pt}{0.400pt}}
\put(1049,546.67){\rule{0.964pt}{0.400pt}}
\multiput(1049.00,547.17)(2.000,-1.000){2}{\rule{0.482pt}{0.400pt}}
\put(1053,545.17){\rule{0.700pt}{0.400pt}}
\multiput(1053.00,546.17)(1.547,-2.000){2}{\rule{0.350pt}{0.400pt}}
\put(1056,543.17){\rule{0.900pt}{0.400pt}}
\multiput(1056.00,544.17)(2.132,-2.000){2}{\rule{0.450pt}{0.400pt}}
\put(1060,541.17){\rule{0.900pt}{0.400pt}}
\multiput(1060.00,542.17)(2.132,-2.000){2}{\rule{0.450pt}{0.400pt}}
\put(1064,539.17){\rule{0.900pt}{0.400pt}}
\multiput(1064.00,540.17)(2.132,-2.000){2}{\rule{0.450pt}{0.400pt}}
\put(1068,537.17){\rule{0.900pt}{0.400pt}}
\multiput(1068.00,538.17)(2.132,-2.000){2}{\rule{0.450pt}{0.400pt}}
\put(1072,535.17){\rule{0.900pt}{0.400pt}}
\multiput(1072.00,536.17)(2.132,-2.000){2}{\rule{0.450pt}{0.400pt}}
\multiput(1076.00,533.95)(0.685,-0.447){3}{\rule{0.633pt}{0.108pt}}
\multiput(1076.00,534.17)(2.685,-3.000){2}{\rule{0.317pt}{0.400pt}}
\multiput(1080.00,530.95)(0.685,-0.447){3}{\rule{0.633pt}{0.108pt}}
\multiput(1080.00,531.17)(2.685,-3.000){2}{\rule{0.317pt}{0.400pt}}
\multiput(1084.00,527.95)(0.685,-0.447){3}{\rule{0.633pt}{0.108pt}}
\multiput(1084.00,528.17)(2.685,-3.000){2}{\rule{0.317pt}{0.400pt}}
\multiput(1088.00,524.95)(0.685,-0.447){3}{\rule{0.633pt}{0.108pt}}
\multiput(1088.00,525.17)(2.685,-3.000){2}{\rule{0.317pt}{0.400pt}}
\multiput(1092.00,521.95)(0.685,-0.447){3}{\rule{0.633pt}{0.108pt}}
\multiput(1092.00,522.17)(2.685,-3.000){2}{\rule{0.317pt}{0.400pt}}
\multiput(1096.00,518.95)(0.685,-0.447){3}{\rule{0.633pt}{0.108pt}}
\multiput(1096.00,519.17)(2.685,-3.000){2}{\rule{0.317pt}{0.400pt}}
\multiput(1100.00,515.94)(0.481,-0.468){5}{\rule{0.500pt}{0.113pt}}
\multiput(1100.00,516.17)(2.962,-4.000){2}{\rule{0.250pt}{0.400pt}}
\multiput(1104.00,511.95)(0.685,-0.447){3}{\rule{0.633pt}{0.108pt}}
\multiput(1104.00,512.17)(2.685,-3.000){2}{\rule{0.317pt}{0.400pt}}
\multiput(1108.00,508.94)(0.481,-0.468){5}{\rule{0.500pt}{0.113pt}}
\multiput(1108.00,509.17)(2.962,-4.000){2}{\rule{0.250pt}{0.400pt}}
\multiput(1112.00,504.94)(0.481,-0.468){5}{\rule{0.500pt}{0.113pt}}
\multiput(1112.00,505.17)(2.962,-4.000){2}{\rule{0.250pt}{0.400pt}}
\multiput(1116.00,500.94)(0.481,-0.468){5}{\rule{0.500pt}{0.113pt}}
\multiput(1116.00,501.17)(2.962,-4.000){2}{\rule{0.250pt}{0.400pt}}
\multiput(1120.00,496.94)(0.481,-0.468){5}{\rule{0.500pt}{0.113pt}}
\multiput(1120.00,497.17)(2.962,-4.000){2}{\rule{0.250pt}{0.400pt}}
\multiput(1124.60,491.51)(0.468,-0.627){5}{\rule{0.113pt}{0.600pt}}
\multiput(1123.17,492.75)(4.000,-3.755){2}{\rule{0.400pt}{0.300pt}}
\multiput(1128.60,486.51)(0.468,-0.627){5}{\rule{0.113pt}{0.600pt}}
\multiput(1127.17,487.75)(4.000,-3.755){2}{\rule{0.400pt}{0.300pt}}
\multiput(1132.00,482.94)(0.481,-0.468){5}{\rule{0.500pt}{0.113pt}}
\multiput(1132.00,483.17)(2.962,-4.000){2}{\rule{0.250pt}{0.400pt}}
\multiput(1136.60,477.51)(0.468,-0.627){5}{\rule{0.113pt}{0.600pt}}
\multiput(1135.17,478.75)(4.000,-3.755){2}{\rule{0.400pt}{0.300pt}}
\multiput(1140.60,472.09)(0.468,-0.774){5}{\rule{0.113pt}{0.700pt}}
\multiput(1139.17,473.55)(4.000,-4.547){2}{\rule{0.400pt}{0.350pt}}
\multiput(786.60,282.00)(0.468,1.066){5}{\rule{0.113pt}{0.900pt}}
\multiput(785.17,282.00)(4.000,6.132){2}{\rule{0.400pt}{0.450pt}}
\multiput(790.60,290.00)(0.468,0.920){5}{\rule{0.113pt}{0.800pt}}
\multiput(789.17,290.00)(4.000,5.340){2}{\rule{0.400pt}{0.400pt}}
\multiput(794.60,297.00)(0.468,1.066){5}{\rule{0.113pt}{0.900pt}}
\multiput(793.17,297.00)(4.000,6.132){2}{\rule{0.400pt}{0.450pt}}
\multiput(798.60,305.00)(0.468,0.920){5}{\rule{0.113pt}{0.800pt}}
\multiput(797.17,305.00)(4.000,5.340){2}{\rule{0.400pt}{0.400pt}}
\multiput(802.60,312.00)(0.468,0.774){5}{\rule{0.113pt}{0.700pt}}
\multiput(801.17,312.00)(4.000,4.547){2}{\rule{0.400pt}{0.350pt}}
\multiput(806.60,318.00)(0.468,0.920){5}{\rule{0.113pt}{0.800pt}}
\multiput(805.17,318.00)(4.000,5.340){2}{\rule{0.400pt}{0.400pt}}
\multiput(810.60,325.00)(0.468,0.920){5}{\rule{0.113pt}{0.800pt}}
\multiput(809.17,325.00)(4.000,5.340){2}{\rule{0.400pt}{0.400pt}}
\multiput(814.60,332.00)(0.468,0.774){5}{\rule{0.113pt}{0.700pt}}
\multiput(813.17,332.00)(4.000,4.547){2}{\rule{0.400pt}{0.350pt}}
\multiput(818.60,338.00)(0.468,0.920){5}{\rule{0.113pt}{0.800pt}}
\multiput(817.17,338.00)(4.000,5.340){2}{\rule{0.400pt}{0.400pt}}
\multiput(822.60,345.00)(0.468,0.774){5}{\rule{0.113pt}{0.700pt}}
\multiput(821.17,345.00)(4.000,4.547){2}{\rule{0.400pt}{0.350pt}}
\multiput(826.60,351.00)(0.468,0.774){5}{\rule{0.113pt}{0.700pt}}
\multiput(825.17,351.00)(4.000,4.547){2}{\rule{0.400pt}{0.350pt}}
\multiput(830.60,357.00)(0.468,0.774){5}{\rule{0.113pt}{0.700pt}}
\multiput(829.17,357.00)(4.000,4.547){2}{\rule{0.400pt}{0.350pt}}
\multiput(834.60,363.00)(0.468,0.774){5}{\rule{0.113pt}{0.700pt}}
\multiput(833.17,363.00)(4.000,4.547){2}{\rule{0.400pt}{0.350pt}}
\multiput(838.60,369.00)(0.468,0.774){5}{\rule{0.113pt}{0.700pt}}
\multiput(837.17,369.00)(4.000,4.547){2}{\rule{0.400pt}{0.350pt}}
\multiput(842.60,375.00)(0.468,0.627){5}{\rule{0.113pt}{0.600pt}}
\multiput(841.17,375.00)(4.000,3.755){2}{\rule{0.400pt}{0.300pt}}
\multiput(846.60,380.00)(0.468,0.774){5}{\rule{0.113pt}{0.700pt}}
\multiput(845.17,380.00)(4.000,4.547){2}{\rule{0.400pt}{0.350pt}}
\multiput(850.60,386.00)(0.468,0.627){5}{\rule{0.113pt}{0.600pt}}
\multiput(849.17,386.00)(4.000,3.755){2}{\rule{0.400pt}{0.300pt}}
\multiput(854.60,391.00)(0.468,0.774){5}{\rule{0.113pt}{0.700pt}}
\multiput(853.17,391.00)(4.000,4.547){2}{\rule{0.400pt}{0.350pt}}
\multiput(858.60,397.00)(0.468,0.627){5}{\rule{0.113pt}{0.600pt}}
\multiput(857.17,397.00)(4.000,3.755){2}{\rule{0.400pt}{0.300pt}}
\multiput(862.60,402.00)(0.468,0.627){5}{\rule{0.113pt}{0.600pt}}
\multiput(861.17,402.00)(4.000,3.755){2}{\rule{0.400pt}{0.300pt}}
\multiput(866.60,407.00)(0.468,0.627){5}{\rule{0.113pt}{0.600pt}}
\multiput(865.17,407.00)(4.000,3.755){2}{\rule{0.400pt}{0.300pt}}
\multiput(870.00,412.60)(0.481,0.468){5}{\rule{0.500pt}{0.113pt}}
\multiput(870.00,411.17)(2.962,4.000){2}{\rule{0.250pt}{0.400pt}}
\multiput(874.61,416.00)(0.447,0.909){3}{\rule{0.108pt}{0.767pt}}
\multiput(873.17,416.00)(3.000,3.409){2}{\rule{0.400pt}{0.383pt}}
\multiput(877.60,421.00)(0.468,0.627){5}{\rule{0.113pt}{0.600pt}}
\multiput(876.17,421.00)(4.000,3.755){2}{\rule{0.400pt}{0.300pt}}
\multiput(881.00,426.60)(0.481,0.468){5}{\rule{0.500pt}{0.113pt}}
\multiput(881.00,425.17)(2.962,4.000){2}{\rule{0.250pt}{0.400pt}}
\multiput(885.00,430.60)(0.481,0.468){5}{\rule{0.500pt}{0.113pt}}
\multiput(885.00,429.17)(2.962,4.000){2}{\rule{0.250pt}{0.400pt}}
\multiput(889.60,434.00)(0.468,0.627){5}{\rule{0.113pt}{0.600pt}}
\multiput(888.17,434.00)(4.000,3.755){2}{\rule{0.400pt}{0.300pt}}
\multiput(893.00,439.60)(0.481,0.468){5}{\rule{0.500pt}{0.113pt}}
\multiput(893.00,438.17)(2.962,4.000){2}{\rule{0.250pt}{0.400pt}}
\multiput(897.00,443.60)(0.481,0.468){5}{\rule{0.500pt}{0.113pt}}
\multiput(897.00,442.17)(2.962,4.000){2}{\rule{0.250pt}{0.400pt}}
\multiput(901.00,447.60)(0.481,0.468){5}{\rule{0.500pt}{0.113pt}}
\multiput(901.00,446.17)(2.962,4.000){2}{\rule{0.250pt}{0.400pt}}
\multiput(905.00,451.60)(0.481,0.468){5}{\rule{0.500pt}{0.113pt}}
\multiput(905.00,450.17)(2.962,4.000){2}{\rule{0.250pt}{0.400pt}}
\multiput(909.00,455.61)(0.685,0.447){3}{\rule{0.633pt}{0.108pt}}
\multiput(909.00,454.17)(2.685,3.000){2}{\rule{0.317pt}{0.400pt}}
\multiput(913.00,458.60)(0.481,0.468){5}{\rule{0.500pt}{0.113pt}}
\multiput(913.00,457.17)(2.962,4.000){2}{\rule{0.250pt}{0.400pt}}
\multiput(917.00,462.61)(0.685,0.447){3}{\rule{0.633pt}{0.108pt}}
\multiput(917.00,461.17)(2.685,3.000){2}{\rule{0.317pt}{0.400pt}}
\multiput(921.00,465.60)(0.481,0.468){5}{\rule{0.500pt}{0.113pt}}
\multiput(921.00,464.17)(2.962,4.000){2}{\rule{0.250pt}{0.400pt}}
\multiput(925.00,469.61)(0.685,0.447){3}{\rule{0.633pt}{0.108pt}}
\multiput(925.00,468.17)(2.685,3.000){2}{\rule{0.317pt}{0.400pt}}
\multiput(929.00,472.61)(0.685,0.447){3}{\rule{0.633pt}{0.108pt}}
\multiput(929.00,471.17)(2.685,3.000){2}{\rule{0.317pt}{0.400pt}}
\multiput(933.00,475.61)(0.685,0.447){3}{\rule{0.633pt}{0.108pt}}
\multiput(933.00,474.17)(2.685,3.000){2}{\rule{0.317pt}{0.400pt}}
\multiput(937.00,478.61)(0.685,0.447){3}{\rule{0.633pt}{0.108pt}}
\multiput(937.00,477.17)(2.685,3.000){2}{\rule{0.317pt}{0.400pt}}
\multiput(941.00,481.61)(0.685,0.447){3}{\rule{0.633pt}{0.108pt}}
\multiput(941.00,480.17)(2.685,3.000){2}{\rule{0.317pt}{0.400pt}}
\multiput(945.00,484.61)(0.685,0.447){3}{\rule{0.633pt}{0.108pt}}
\multiput(945.00,483.17)(2.685,3.000){2}{\rule{0.317pt}{0.400pt}}
\put(949,487.17){\rule{0.900pt}{0.400pt}}
\multiput(949.00,486.17)(2.132,2.000){2}{\rule{0.450pt}{0.400pt}}
\multiput(953.00,489.61)(0.685,0.447){3}{\rule{0.633pt}{0.108pt}}
\multiput(953.00,488.17)(2.685,3.000){2}{\rule{0.317pt}{0.400pt}}
\put(957,492.17){\rule{0.900pt}{0.400pt}}
\multiput(957.00,491.17)(2.132,2.000){2}{\rule{0.450pt}{0.400pt}}
\multiput(961.00,494.61)(0.685,0.447){3}{\rule{0.633pt}{0.108pt}}
\multiput(961.00,493.17)(2.685,3.000){2}{\rule{0.317pt}{0.400pt}}
\put(965,497.17){\rule{0.900pt}{0.400pt}}
\multiput(965.00,496.17)(2.132,2.000){2}{\rule{0.450pt}{0.400pt}}
\put(969,499.17){\rule{0.900pt}{0.400pt}}
\multiput(969.00,498.17)(2.132,2.000){2}{\rule{0.450pt}{0.400pt}}
\put(973,501.17){\rule{0.900pt}{0.400pt}}
\multiput(973.00,500.17)(2.132,2.000){2}{\rule{0.450pt}{0.400pt}}
\put(977,503.17){\rule{0.900pt}{0.400pt}}
\multiput(977.00,502.17)(2.132,2.000){2}{\rule{0.450pt}{0.400pt}}
\put(981,505.17){\rule{0.900pt}{0.400pt}}
\multiput(981.00,504.17)(2.132,2.000){2}{\rule{0.450pt}{0.400pt}}
\put(985,507.17){\rule{0.900pt}{0.400pt}}
\multiput(985.00,506.17)(2.132,2.000){2}{\rule{0.450pt}{0.400pt}}
\put(989,508.67){\rule{0.964pt}{0.400pt}}
\multiput(989.00,508.17)(2.000,1.000){2}{\rule{0.482pt}{0.400pt}}
\put(993,510.17){\rule{0.900pt}{0.400pt}}
\multiput(993.00,509.17)(2.132,2.000){2}{\rule{0.450pt}{0.400pt}}
\put(997,511.67){\rule{0.964pt}{0.400pt}}
\multiput(997.00,511.17)(2.000,1.000){2}{\rule{0.482pt}{0.400pt}}
\put(1001,513.17){\rule{0.900pt}{0.400pt}}
\multiput(1001.00,512.17)(2.132,2.000){2}{\rule{0.450pt}{0.400pt}}
\put(1005,514.67){\rule{0.964pt}{0.400pt}}
\multiput(1005.00,514.17)(2.000,1.000){2}{\rule{0.482pt}{0.400pt}}
\put(1009,515.67){\rule{0.964pt}{0.400pt}}
\multiput(1009.00,515.17)(2.000,1.000){2}{\rule{0.482pt}{0.400pt}}
\put(1013,516.67){\rule{0.964pt}{0.400pt}}
\multiput(1013.00,516.17)(2.000,1.000){2}{\rule{0.482pt}{0.400pt}}
\put(1017,517.67){\rule{0.964pt}{0.400pt}}
\multiput(1017.00,517.17)(2.000,1.000){2}{\rule{0.482pt}{0.400pt}}
\put(1021,518.67){\rule{0.964pt}{0.400pt}}
\multiput(1021.00,518.17)(2.000,1.000){2}{\rule{0.482pt}{0.400pt}}
\put(1021.0,555.0){\rule[-0.200pt]{0.964pt}{0.400pt}}
\put(1029,519.67){\rule{0.964pt}{0.400pt}}
\multiput(1029.00,519.17)(2.000,1.000){2}{\rule{0.482pt}{0.400pt}}
\put(1025.0,520.0){\rule[-0.200pt]{0.964pt}{0.400pt}}
\put(1037,520.67){\rule{0.964pt}{0.400pt}}
\multiput(1037.00,520.17)(2.000,1.000){2}{\rule{0.482pt}{0.400pt}}
\put(1033.0,521.0){\rule[-0.200pt]{0.964pt}{0.400pt}}
\put(1041.0,522.0){\rule[-0.200pt]{0.964pt}{0.400pt}}
\put(1045.0,522.0){\rule[-0.200pt]{0.964pt}{0.400pt}}
\put(1049.0,522.0){\rule[-0.200pt]{0.964pt}{0.400pt}}
\put(1053.0,522.0){\rule[-0.200pt]{0.964pt}{0.400pt}}
\put(1057.0,522.0){\rule[-0.200pt]{0.964pt}{0.400pt}}
\put(1061.0,522.0){\rule[-0.200pt]{0.964pt}{0.400pt}}
\put(1069,520.67){\rule{0.964pt}{0.400pt}}
\multiput(1069.00,521.17)(2.000,-1.000){2}{\rule{0.482pt}{0.400pt}}
\put(1065.0,522.0){\rule[-0.200pt]{0.964pt}{0.400pt}}
\put(1076,519.67){\rule{0.964pt}{0.400pt}}
\multiput(1076.00,520.17)(2.000,-1.000){2}{\rule{0.482pt}{0.400pt}}
\put(1080,518.67){\rule{0.964pt}{0.400pt}}
\multiput(1080.00,519.17)(2.000,-1.000){2}{\rule{0.482pt}{0.400pt}}
\put(1084,517.67){\rule{0.964pt}{0.400pt}}
\multiput(1084.00,518.17)(2.000,-1.000){2}{\rule{0.482pt}{0.400pt}}
\put(1088,516.67){\rule{0.964pt}{0.400pt}}
\multiput(1088.00,517.17)(2.000,-1.000){2}{\rule{0.482pt}{0.400pt}}
\put(1092,515.67){\rule{0.964pt}{0.400pt}}
\multiput(1092.00,516.17)(2.000,-1.000){2}{\rule{0.482pt}{0.400pt}}
\put(1096,514.67){\rule{0.964pt}{0.400pt}}
\multiput(1096.00,515.17)(2.000,-1.000){2}{\rule{0.482pt}{0.400pt}}
\put(1100,513.67){\rule{0.964pt}{0.400pt}}
\multiput(1100.00,514.17)(2.000,-1.000){2}{\rule{0.482pt}{0.400pt}}
\put(1104,512.17){\rule{0.900pt}{0.400pt}}
\multiput(1104.00,513.17)(2.132,-2.000){2}{\rule{0.450pt}{0.400pt}}
\put(1108,510.17){\rule{0.900pt}{0.400pt}}
\multiput(1108.00,511.17)(2.132,-2.000){2}{\rule{0.450pt}{0.400pt}}
\put(1112,508.67){\rule{0.964pt}{0.400pt}}
\multiput(1112.00,509.17)(2.000,-1.000){2}{\rule{0.482pt}{0.400pt}}
\put(1116,507.17){\rule{0.900pt}{0.400pt}}
\multiput(1116.00,508.17)(2.132,-2.000){2}{\rule{0.450pt}{0.400pt}}
\put(1120,505.17){\rule{0.900pt}{0.400pt}}
\multiput(1120.00,506.17)(2.132,-2.000){2}{\rule{0.450pt}{0.400pt}}
\put(1124,503.17){\rule{0.900pt}{0.400pt}}
\multiput(1124.00,504.17)(2.132,-2.000){2}{\rule{0.450pt}{0.400pt}}
\put(1128,501.17){\rule{0.900pt}{0.400pt}}
\multiput(1128.00,502.17)(2.132,-2.000){2}{\rule{0.450pt}{0.400pt}}
\multiput(1132.00,499.95)(0.685,-0.447){3}{\rule{0.633pt}{0.108pt}}
\multiput(1132.00,500.17)(2.685,-3.000){2}{\rule{0.317pt}{0.400pt}}
\put(1136,496.17){\rule{0.900pt}{0.400pt}}
\multiput(1136.00,497.17)(2.132,-2.000){2}{\rule{0.450pt}{0.400pt}}
\multiput(1140.00,494.95)(0.685,-0.447){3}{\rule{0.633pt}{0.108pt}}
\multiput(1140.00,495.17)(2.685,-3.000){2}{\rule{0.317pt}{0.400pt}}
\put(1144,491.17){\rule{0.900pt}{0.400pt}}
\multiput(1144.00,492.17)(2.132,-2.000){2}{\rule{0.450pt}{0.400pt}}
\multiput(1148.00,489.95)(0.685,-0.447){3}{\rule{0.633pt}{0.108pt}}
\multiput(1148.00,490.17)(2.685,-3.000){2}{\rule{0.317pt}{0.400pt}}
\multiput(1152.00,486.95)(0.685,-0.447){3}{\rule{0.633pt}{0.108pt}}
\multiput(1152.00,487.17)(2.685,-3.000){2}{\rule{0.317pt}{0.400pt}}
\multiput(1156.00,483.95)(0.685,-0.447){3}{\rule{0.633pt}{0.108pt}}
\multiput(1156.00,484.17)(2.685,-3.000){2}{\rule{0.317pt}{0.400pt}}
\multiput(1160.00,480.94)(0.481,-0.468){5}{\rule{0.500pt}{0.113pt}}
\multiput(1160.00,481.17)(2.962,-4.000){2}{\rule{0.250pt}{0.400pt}}
\multiput(1164.00,476.95)(0.685,-0.447){3}{\rule{0.633pt}{0.108pt}}
\multiput(1164.00,477.17)(2.685,-3.000){2}{\rule{0.317pt}{0.400pt}}
\multiput(1168.00,473.94)(0.481,-0.468){5}{\rule{0.500pt}{0.113pt}}
\multiput(1168.00,474.17)(2.962,-4.000){2}{\rule{0.250pt}{0.400pt}}
\multiput(1172.00,469.95)(0.685,-0.447){3}{\rule{0.633pt}{0.108pt}}
\multiput(1172.00,470.17)(2.685,-3.000){2}{\rule{0.317pt}{0.400pt}}
\multiput(1176.00,466.94)(0.481,-0.468){5}{\rule{0.500pt}{0.113pt}}
\multiput(1176.00,467.17)(2.962,-4.000){2}{\rule{0.250pt}{0.400pt}}
\multiput(822.60,277.00)(0.468,0.774){5}{\rule{0.113pt}{0.700pt}}
\multiput(821.17,277.00)(4.000,4.547){2}{\rule{0.400pt}{0.350pt}}
\multiput(826.60,283.00)(0.468,0.774){5}{\rule{0.113pt}{0.700pt}}
\multiput(825.17,283.00)(4.000,4.547){2}{\rule{0.400pt}{0.350pt}}
\multiput(830.60,289.00)(0.468,0.627){5}{\rule{0.113pt}{0.600pt}}
\multiput(829.17,289.00)(4.000,3.755){2}{\rule{0.400pt}{0.300pt}}
\multiput(834.60,294.00)(0.468,0.774){5}{\rule{0.113pt}{0.700pt}}
\multiput(833.17,294.00)(4.000,4.547){2}{\rule{0.400pt}{0.350pt}}
\multiput(838.60,300.00)(0.468,0.627){5}{\rule{0.113pt}{0.600pt}}
\multiput(837.17,300.00)(4.000,3.755){2}{\rule{0.400pt}{0.300pt}}
\multiput(842.60,305.00)(0.468,0.774){5}{\rule{0.113pt}{0.700pt}}
\multiput(841.17,305.00)(4.000,4.547){2}{\rule{0.400pt}{0.350pt}}
\multiput(846.60,311.00)(0.468,0.627){5}{\rule{0.113pt}{0.600pt}}
\multiput(845.17,311.00)(4.000,3.755){2}{\rule{0.400pt}{0.300pt}}
\multiput(850.60,316.00)(0.468,0.627){5}{\rule{0.113pt}{0.600pt}}
\multiput(849.17,316.00)(4.000,3.755){2}{\rule{0.400pt}{0.300pt}}
\multiput(854.60,321.00)(0.468,0.627){5}{\rule{0.113pt}{0.600pt}}
\multiput(853.17,321.00)(4.000,3.755){2}{\rule{0.400pt}{0.300pt}}
\multiput(858.60,326.00)(0.468,0.627){5}{\rule{0.113pt}{0.600pt}}
\multiput(857.17,326.00)(4.000,3.755){2}{\rule{0.400pt}{0.300pt}}
\multiput(862.60,331.00)(0.468,0.627){5}{\rule{0.113pt}{0.600pt}}
\multiput(861.17,331.00)(4.000,3.755){2}{\rule{0.400pt}{0.300pt}}
\multiput(866.60,336.00)(0.468,0.627){5}{\rule{0.113pt}{0.600pt}}
\multiput(865.17,336.00)(4.000,3.755){2}{\rule{0.400pt}{0.300pt}}
\multiput(870.60,341.00)(0.468,0.627){5}{\rule{0.113pt}{0.600pt}}
\multiput(869.17,341.00)(4.000,3.755){2}{\rule{0.400pt}{0.300pt}}
\multiput(874.00,346.60)(0.481,0.468){5}{\rule{0.500pt}{0.113pt}}
\multiput(874.00,345.17)(2.962,4.000){2}{\rule{0.250pt}{0.400pt}}
\multiput(878.60,350.00)(0.468,0.627){5}{\rule{0.113pt}{0.600pt}}
\multiput(877.17,350.00)(4.000,3.755){2}{\rule{0.400pt}{0.300pt}}
\multiput(882.00,355.60)(0.481,0.468){5}{\rule{0.500pt}{0.113pt}}
\multiput(882.00,354.17)(2.962,4.000){2}{\rule{0.250pt}{0.400pt}}
\multiput(886.60,359.00)(0.468,0.627){5}{\rule{0.113pt}{0.600pt}}
\multiput(885.17,359.00)(4.000,3.755){2}{\rule{0.400pt}{0.300pt}}
\multiput(890.00,364.60)(0.481,0.468){5}{\rule{0.500pt}{0.113pt}}
\multiput(890.00,363.17)(2.962,4.000){2}{\rule{0.250pt}{0.400pt}}
\multiput(894.61,368.00)(0.447,0.685){3}{\rule{0.108pt}{0.633pt}}
\multiput(893.17,368.00)(3.000,2.685){2}{\rule{0.400pt}{0.317pt}}
\multiput(897.00,372.60)(0.481,0.468){5}{\rule{0.500pt}{0.113pt}}
\multiput(897.00,371.17)(2.962,4.000){2}{\rule{0.250pt}{0.400pt}}
\multiput(901.00,376.60)(0.481,0.468){5}{\rule{0.500pt}{0.113pt}}
\multiput(901.00,375.17)(2.962,4.000){2}{\rule{0.250pt}{0.400pt}}
\multiput(905.00,380.60)(0.481,0.468){5}{\rule{0.500pt}{0.113pt}}
\multiput(905.00,379.17)(2.962,4.000){2}{\rule{0.250pt}{0.400pt}}
\multiput(909.00,384.60)(0.481,0.468){5}{\rule{0.500pt}{0.113pt}}
\multiput(909.00,383.17)(2.962,4.000){2}{\rule{0.250pt}{0.400pt}}
\multiput(913.00,388.60)(0.481,0.468){5}{\rule{0.500pt}{0.113pt}}
\multiput(913.00,387.17)(2.962,4.000){2}{\rule{0.250pt}{0.400pt}}
\multiput(917.00,392.60)(0.481,0.468){5}{\rule{0.500pt}{0.113pt}}
\multiput(917.00,391.17)(2.962,4.000){2}{\rule{0.250pt}{0.400pt}}
\multiput(921.00,396.61)(0.685,0.447){3}{\rule{0.633pt}{0.108pt}}
\multiput(921.00,395.17)(2.685,3.000){2}{\rule{0.317pt}{0.400pt}}
\multiput(925.00,399.60)(0.481,0.468){5}{\rule{0.500pt}{0.113pt}}
\multiput(925.00,398.17)(2.962,4.000){2}{\rule{0.250pt}{0.400pt}}
\multiput(929.00,403.61)(0.685,0.447){3}{\rule{0.633pt}{0.108pt}}
\multiput(929.00,402.17)(2.685,3.000){2}{\rule{0.317pt}{0.400pt}}
\multiput(933.00,406.60)(0.481,0.468){5}{\rule{0.500pt}{0.113pt}}
\multiput(933.00,405.17)(2.962,4.000){2}{\rule{0.250pt}{0.400pt}}
\multiput(937.00,410.61)(0.685,0.447){3}{\rule{0.633pt}{0.108pt}}
\multiput(937.00,409.17)(2.685,3.000){2}{\rule{0.317pt}{0.400pt}}
\multiput(941.00,413.61)(0.685,0.447){3}{\rule{0.633pt}{0.108pt}}
\multiput(941.00,412.17)(2.685,3.000){2}{\rule{0.317pt}{0.400pt}}
\multiput(945.00,416.61)(0.685,0.447){3}{\rule{0.633pt}{0.108pt}}
\multiput(945.00,415.17)(2.685,3.000){2}{\rule{0.317pt}{0.400pt}}
\multiput(949.00,419.60)(0.481,0.468){5}{\rule{0.500pt}{0.113pt}}
\multiput(949.00,418.17)(2.962,4.000){2}{\rule{0.250pt}{0.400pt}}
\multiput(953.00,423.61)(0.685,0.447){3}{\rule{0.633pt}{0.108pt}}
\multiput(953.00,422.17)(2.685,3.000){2}{\rule{0.317pt}{0.400pt}}
\multiput(957.00,426.61)(0.685,0.447){3}{\rule{0.633pt}{0.108pt}}
\multiput(957.00,425.17)(2.685,3.000){2}{\rule{0.317pt}{0.400pt}}
\put(961,429.17){\rule{0.900pt}{0.400pt}}
\multiput(961.00,428.17)(2.132,2.000){2}{\rule{0.450pt}{0.400pt}}
\multiput(965.00,431.61)(0.685,0.447){3}{\rule{0.633pt}{0.108pt}}
\multiput(965.00,430.17)(2.685,3.000){2}{\rule{0.317pt}{0.400pt}}
\multiput(969.00,434.61)(0.685,0.447){3}{\rule{0.633pt}{0.108pt}}
\multiput(969.00,433.17)(2.685,3.000){2}{\rule{0.317pt}{0.400pt}}
\multiput(973.00,437.61)(0.685,0.447){3}{\rule{0.633pt}{0.108pt}}
\multiput(973.00,436.17)(2.685,3.000){2}{\rule{0.317pt}{0.400pt}}
\put(977,440.17){\rule{0.900pt}{0.400pt}}
\multiput(977.00,439.17)(2.132,2.000){2}{\rule{0.450pt}{0.400pt}}
\multiput(981.00,442.61)(0.685,0.447){3}{\rule{0.633pt}{0.108pt}}
\multiput(981.00,441.17)(2.685,3.000){2}{\rule{0.317pt}{0.400pt}}
\put(985,445.17){\rule{0.900pt}{0.400pt}}
\multiput(985.00,444.17)(2.132,2.000){2}{\rule{0.450pt}{0.400pt}}
\put(989,447.17){\rule{0.900pt}{0.400pt}}
\multiput(989.00,446.17)(2.132,2.000){2}{\rule{0.450pt}{0.400pt}}
\multiput(993.00,449.61)(0.685,0.447){3}{\rule{0.633pt}{0.108pt}}
\multiput(993.00,448.17)(2.685,3.000){2}{\rule{0.317pt}{0.400pt}}
\put(997,452.17){\rule{0.900pt}{0.400pt}}
\multiput(997.00,451.17)(2.132,2.000){2}{\rule{0.450pt}{0.400pt}}
\put(1001,454.17){\rule{0.900pt}{0.400pt}}
\multiput(1001.00,453.17)(2.132,2.000){2}{\rule{0.450pt}{0.400pt}}
\put(1005,456.17){\rule{0.900pt}{0.400pt}}
\multiput(1005.00,455.17)(2.132,2.000){2}{\rule{0.450pt}{0.400pt}}
\put(1009,458.17){\rule{0.900pt}{0.400pt}}
\multiput(1009.00,457.17)(2.132,2.000){2}{\rule{0.450pt}{0.400pt}}
\put(1013,460.17){\rule{0.900pt}{0.400pt}}
\multiput(1013.00,459.17)(2.132,2.000){2}{\rule{0.450pt}{0.400pt}}
\put(1017,462.17){\rule{0.900pt}{0.400pt}}
\multiput(1017.00,461.17)(2.132,2.000){2}{\rule{0.450pt}{0.400pt}}
\put(1021,463.67){\rule{0.964pt}{0.400pt}}
\multiput(1021.00,463.17)(2.000,1.000){2}{\rule{0.482pt}{0.400pt}}
\put(1025,465.17){\rule{0.900pt}{0.400pt}}
\multiput(1025.00,464.17)(2.132,2.000){2}{\rule{0.450pt}{0.400pt}}
\put(1029,467.17){\rule{0.900pt}{0.400pt}}
\multiput(1029.00,466.17)(2.132,2.000){2}{\rule{0.450pt}{0.400pt}}
\put(1033,468.67){\rule{0.964pt}{0.400pt}}
\multiput(1033.00,468.17)(2.000,1.000){2}{\rule{0.482pt}{0.400pt}}
\put(1037,470.17){\rule{0.900pt}{0.400pt}}
\multiput(1037.00,469.17)(2.132,2.000){2}{\rule{0.450pt}{0.400pt}}
\put(1041,471.67){\rule{0.964pt}{0.400pt}}
\multiput(1041.00,471.17)(2.000,1.000){2}{\rule{0.482pt}{0.400pt}}
\put(1045,472.67){\rule{0.964pt}{0.400pt}}
\multiput(1045.00,472.17)(2.000,1.000){2}{\rule{0.482pt}{0.400pt}}
\put(1049,474.17){\rule{0.900pt}{0.400pt}}
\multiput(1049.00,473.17)(2.132,2.000){2}{\rule{0.450pt}{0.400pt}}
\put(1053,475.67){\rule{0.964pt}{0.400pt}}
\multiput(1053.00,475.17)(2.000,1.000){2}{\rule{0.482pt}{0.400pt}}
\put(1057,476.67){\rule{0.964pt}{0.400pt}}
\multiput(1057.00,476.17)(2.000,1.000){2}{\rule{0.482pt}{0.400pt}}
\put(1061,477.67){\rule{0.964pt}{0.400pt}}
\multiput(1061.00,477.17)(2.000,1.000){2}{\rule{0.482pt}{0.400pt}}
\put(1065,478.67){\rule{0.964pt}{0.400pt}}
\multiput(1065.00,478.17)(2.000,1.000){2}{\rule{0.482pt}{0.400pt}}
\put(1069,479.67){\rule{0.964pt}{0.400pt}}
\multiput(1069.00,479.17)(2.000,1.000){2}{\rule{0.482pt}{0.400pt}}
\put(1073,480.67){\rule{0.964pt}{0.400pt}}
\multiput(1073.00,480.17)(2.000,1.000){2}{\rule{0.482pt}{0.400pt}}
\put(1073.0,521.0){\rule[-0.200pt]{0.723pt}{0.400pt}}
\put(1081,481.67){\rule{0.964pt}{0.400pt}}
\multiput(1081.00,481.17)(2.000,1.000){2}{\rule{0.482pt}{0.400pt}}
\put(1085,482.67){\rule{0.964pt}{0.400pt}}
\multiput(1085.00,482.17)(2.000,1.000){2}{\rule{0.482pt}{0.400pt}}
\put(1077.0,482.0){\rule[-0.200pt]{0.964pt}{0.400pt}}
\put(1093,483.67){\rule{0.723pt}{0.400pt}}
\multiput(1093.00,483.17)(1.500,1.000){2}{\rule{0.361pt}{0.400pt}}
\put(1089.0,484.0){\rule[-0.200pt]{0.964pt}{0.400pt}}
\put(1096.0,485.0){\rule[-0.200pt]{0.964pt}{0.400pt}}
\put(1100.0,485.0){\rule[-0.200pt]{0.964pt}{0.400pt}}
\put(1104.0,485.0){\rule[-0.200pt]{0.964pt}{0.400pt}}
\put(1108.0,485.0){\rule[-0.200pt]{0.964pt}{0.400pt}}
\put(1112.0,485.0){\rule[-0.200pt]{0.964pt}{0.400pt}}
\put(1116.0,485.0){\rule[-0.200pt]{0.964pt}{0.400pt}}
\put(1120.0,485.0){\rule[-0.200pt]{0.964pt}{0.400pt}}
\put(1124.0,485.0){\rule[-0.200pt]{0.964pt}{0.400pt}}
\put(1132,483.67){\rule{0.964pt}{0.400pt}}
\multiput(1132.00,484.17)(2.000,-1.000){2}{\rule{0.482pt}{0.400pt}}
\put(1128.0,485.0){\rule[-0.200pt]{0.964pt}{0.400pt}}
\put(1140,482.67){\rule{0.964pt}{0.400pt}}
\multiput(1140.00,483.17)(2.000,-1.000){2}{\rule{0.482pt}{0.400pt}}
\put(1136.0,484.0){\rule[-0.200pt]{0.964pt}{0.400pt}}
\put(1148,481.67){\rule{0.964pt}{0.400pt}}
\multiput(1148.00,482.17)(2.000,-1.000){2}{\rule{0.482pt}{0.400pt}}
\put(1152,480.67){\rule{0.964pt}{0.400pt}}
\multiput(1152.00,481.17)(2.000,-1.000){2}{\rule{0.482pt}{0.400pt}}
\put(1156,479.67){\rule{0.964pt}{0.400pt}}
\multiput(1156.00,480.17)(2.000,-1.000){2}{\rule{0.482pt}{0.400pt}}
\put(1160,478.67){\rule{0.964pt}{0.400pt}}
\multiput(1160.00,479.17)(2.000,-1.000){2}{\rule{0.482pt}{0.400pt}}
\put(1164,477.67){\rule{0.964pt}{0.400pt}}
\multiput(1164.00,478.17)(2.000,-1.000){2}{\rule{0.482pt}{0.400pt}}
\put(1168,476.67){\rule{0.964pt}{0.400pt}}
\multiput(1168.00,477.17)(2.000,-1.000){2}{\rule{0.482pt}{0.400pt}}
\put(1172,475.67){\rule{0.964pt}{0.400pt}}
\multiput(1172.00,476.17)(2.000,-1.000){2}{\rule{0.482pt}{0.400pt}}
\put(1176,474.67){\rule{0.964pt}{0.400pt}}
\multiput(1176.00,475.17)(2.000,-1.000){2}{\rule{0.482pt}{0.400pt}}
\put(1180,473.17){\rule{0.900pt}{0.400pt}}
\multiput(1180.00,474.17)(2.132,-2.000){2}{\rule{0.450pt}{0.400pt}}
\put(1184,471.67){\rule{0.964pt}{0.400pt}}
\multiput(1184.00,472.17)(2.000,-1.000){2}{\rule{0.482pt}{0.400pt}}
\put(1188,470.17){\rule{0.900pt}{0.400pt}}
\multiput(1188.00,471.17)(2.132,-2.000){2}{\rule{0.450pt}{0.400pt}}
\put(1192,468.17){\rule{0.900pt}{0.400pt}}
\multiput(1192.00,469.17)(2.132,-2.000){2}{\rule{0.450pt}{0.400pt}}
\put(1196,466.67){\rule{0.964pt}{0.400pt}}
\multiput(1196.00,467.17)(2.000,-1.000){2}{\rule{0.482pt}{0.400pt}}
\put(1200,465.17){\rule{0.900pt}{0.400pt}}
\multiput(1200.00,466.17)(2.132,-2.000){2}{\rule{0.450pt}{0.400pt}}
\put(1204,463.17){\rule{0.900pt}{0.400pt}}
\multiput(1204.00,464.17)(2.132,-2.000){2}{\rule{0.450pt}{0.400pt}}
\put(1208,461.17){\rule{0.900pt}{0.400pt}}
\multiput(1208.00,462.17)(2.132,-2.000){2}{\rule{0.450pt}{0.400pt}}
\multiput(1212.00,459.95)(0.685,-0.447){3}{\rule{0.633pt}{0.108pt}}
\multiput(1212.00,460.17)(2.685,-3.000){2}{\rule{0.317pt}{0.400pt}}
\multiput(858.00,271.60)(0.481,0.468){5}{\rule{0.500pt}{0.113pt}}
\multiput(858.00,270.17)(2.962,4.000){2}{\rule{0.250pt}{0.400pt}}
\multiput(862.00,275.60)(0.481,0.468){5}{\rule{0.500pt}{0.113pt}}
\multiput(862.00,274.17)(2.962,4.000){2}{\rule{0.250pt}{0.400pt}}
\multiput(866.00,279.60)(0.481,0.468){5}{\rule{0.500pt}{0.113pt}}
\multiput(866.00,278.17)(2.962,4.000){2}{\rule{0.250pt}{0.400pt}}
\multiput(870.00,283.60)(0.481,0.468){5}{\rule{0.500pt}{0.113pt}}
\multiput(870.00,282.17)(2.962,4.000){2}{\rule{0.250pt}{0.400pt}}
\multiput(874.00,287.60)(0.481,0.468){5}{\rule{0.500pt}{0.113pt}}
\multiput(874.00,286.17)(2.962,4.000){2}{\rule{0.250pt}{0.400pt}}
\multiput(878.00,291.60)(0.481,0.468){5}{\rule{0.500pt}{0.113pt}}
\multiput(878.00,290.17)(2.962,4.000){2}{\rule{0.250pt}{0.400pt}}
\multiput(882.00,295.61)(0.685,0.447){3}{\rule{0.633pt}{0.108pt}}
\multiput(882.00,294.17)(2.685,3.000){2}{\rule{0.317pt}{0.400pt}}
\multiput(886.00,298.60)(0.481,0.468){5}{\rule{0.500pt}{0.113pt}}
\multiput(886.00,297.17)(2.962,4.000){2}{\rule{0.250pt}{0.400pt}}
\multiput(890.00,302.60)(0.481,0.468){5}{\rule{0.500pt}{0.113pt}}
\multiput(890.00,301.17)(2.962,4.000){2}{\rule{0.250pt}{0.400pt}}
\multiput(894.00,306.61)(0.685,0.447){3}{\rule{0.633pt}{0.108pt}}
\multiput(894.00,305.17)(2.685,3.000){2}{\rule{0.317pt}{0.400pt}}
\multiput(898.00,309.60)(0.481,0.468){5}{\rule{0.500pt}{0.113pt}}
\multiput(898.00,308.17)(2.962,4.000){2}{\rule{0.250pt}{0.400pt}}
\multiput(902.00,313.61)(0.685,0.447){3}{\rule{0.633pt}{0.108pt}}
\multiput(902.00,312.17)(2.685,3.000){2}{\rule{0.317pt}{0.400pt}}
\multiput(906.00,316.61)(0.685,0.447){3}{\rule{0.633pt}{0.108pt}}
\multiput(906.00,315.17)(2.685,3.000){2}{\rule{0.317pt}{0.400pt}}
\multiput(910.00,319.60)(0.481,0.468){5}{\rule{0.500pt}{0.113pt}}
\multiput(910.00,318.17)(2.962,4.000){2}{\rule{0.250pt}{0.400pt}}
\multiput(914.00,323.61)(0.462,0.447){3}{\rule{0.500pt}{0.108pt}}
\multiput(914.00,322.17)(1.962,3.000){2}{\rule{0.250pt}{0.400pt}}
\multiput(917.00,326.61)(0.685,0.447){3}{\rule{0.633pt}{0.108pt}}
\multiput(917.00,325.17)(2.685,3.000){2}{\rule{0.317pt}{0.400pt}}
\multiput(921.00,329.61)(0.685,0.447){3}{\rule{0.633pt}{0.108pt}}
\multiput(921.00,328.17)(2.685,3.000){2}{\rule{0.317pt}{0.400pt}}
\multiput(925.00,332.60)(0.481,0.468){5}{\rule{0.500pt}{0.113pt}}
\multiput(925.00,331.17)(2.962,4.000){2}{\rule{0.250pt}{0.400pt}}
\multiput(929.00,336.61)(0.685,0.447){3}{\rule{0.633pt}{0.108pt}}
\multiput(929.00,335.17)(2.685,3.000){2}{\rule{0.317pt}{0.400pt}}
\multiput(933.00,339.61)(0.685,0.447){3}{\rule{0.633pt}{0.108pt}}
\multiput(933.00,338.17)(2.685,3.000){2}{\rule{0.317pt}{0.400pt}}
\multiput(937.00,342.61)(0.685,0.447){3}{\rule{0.633pt}{0.108pt}}
\multiput(937.00,341.17)(2.685,3.000){2}{\rule{0.317pt}{0.400pt}}
\multiput(941.00,345.61)(0.685,0.447){3}{\rule{0.633pt}{0.108pt}}
\multiput(941.00,344.17)(2.685,3.000){2}{\rule{0.317pt}{0.400pt}}
\multiput(945.00,348.61)(0.685,0.447){3}{\rule{0.633pt}{0.108pt}}
\multiput(945.00,347.17)(2.685,3.000){2}{\rule{0.317pt}{0.400pt}}
\multiput(949.00,351.61)(0.685,0.447){3}{\rule{0.633pt}{0.108pt}}
\multiput(949.00,350.17)(2.685,3.000){2}{\rule{0.317pt}{0.400pt}}
\put(953,354.17){\rule{0.900pt}{0.400pt}}
\multiput(953.00,353.17)(2.132,2.000){2}{\rule{0.450pt}{0.400pt}}
\multiput(957.00,356.61)(0.685,0.447){3}{\rule{0.633pt}{0.108pt}}
\multiput(957.00,355.17)(2.685,3.000){2}{\rule{0.317pt}{0.400pt}}
\multiput(961.00,359.61)(0.685,0.447){3}{\rule{0.633pt}{0.108pt}}
\multiput(961.00,358.17)(2.685,3.000){2}{\rule{0.317pt}{0.400pt}}
\multiput(965.00,362.61)(0.685,0.447){3}{\rule{0.633pt}{0.108pt}}
\multiput(965.00,361.17)(2.685,3.000){2}{\rule{0.317pt}{0.400pt}}
\put(969,365.17){\rule{0.900pt}{0.400pt}}
\multiput(969.00,364.17)(2.132,2.000){2}{\rule{0.450pt}{0.400pt}}
\multiput(973.00,367.61)(0.685,0.447){3}{\rule{0.633pt}{0.108pt}}
\multiput(973.00,366.17)(2.685,3.000){2}{\rule{0.317pt}{0.400pt}}
\multiput(977.00,370.61)(0.685,0.447){3}{\rule{0.633pt}{0.108pt}}
\multiput(977.00,369.17)(2.685,3.000){2}{\rule{0.317pt}{0.400pt}}
\put(981,373.17){\rule{0.900pt}{0.400pt}}
\multiput(981.00,372.17)(2.132,2.000){2}{\rule{0.450pt}{0.400pt}}
\multiput(985.00,375.61)(0.685,0.447){3}{\rule{0.633pt}{0.108pt}}
\multiput(985.00,374.17)(2.685,3.000){2}{\rule{0.317pt}{0.400pt}}
\put(989,378.17){\rule{0.900pt}{0.400pt}}
\multiput(989.00,377.17)(2.132,2.000){2}{\rule{0.450pt}{0.400pt}}
\put(993,380.17){\rule{0.900pt}{0.400pt}}
\multiput(993.00,379.17)(2.132,2.000){2}{\rule{0.450pt}{0.400pt}}
\multiput(997.00,382.61)(0.685,0.447){3}{\rule{0.633pt}{0.108pt}}
\multiput(997.00,381.17)(2.685,3.000){2}{\rule{0.317pt}{0.400pt}}
\put(1001,385.17){\rule{0.900pt}{0.400pt}}
\multiput(1001.00,384.17)(2.132,2.000){2}{\rule{0.450pt}{0.400pt}}
\put(1005,387.17){\rule{0.900pt}{0.400pt}}
\multiput(1005.00,386.17)(2.132,2.000){2}{\rule{0.450pt}{0.400pt}}
\multiput(1009.00,389.61)(0.685,0.447){3}{\rule{0.633pt}{0.108pt}}
\multiput(1009.00,388.17)(2.685,3.000){2}{\rule{0.317pt}{0.400pt}}
\put(1013,392.17){\rule{0.900pt}{0.400pt}}
\multiput(1013.00,391.17)(2.132,2.000){2}{\rule{0.450pt}{0.400pt}}
\put(1017,394.17){\rule{0.900pt}{0.400pt}}
\multiput(1017.00,393.17)(2.132,2.000){2}{\rule{0.450pt}{0.400pt}}
\put(1021,396.17){\rule{0.900pt}{0.400pt}}
\multiput(1021.00,395.17)(2.132,2.000){2}{\rule{0.450pt}{0.400pt}}
\put(1025,398.17){\rule{0.900pt}{0.400pt}}
\multiput(1025.00,397.17)(2.132,2.000){2}{\rule{0.450pt}{0.400pt}}
\put(1029,400.17){\rule{0.900pt}{0.400pt}}
\multiput(1029.00,399.17)(2.132,2.000){2}{\rule{0.450pt}{0.400pt}}
\multiput(1033.00,402.61)(0.685,0.447){3}{\rule{0.633pt}{0.108pt}}
\multiput(1033.00,401.17)(2.685,3.000){2}{\rule{0.317pt}{0.400pt}}
\put(1037,405.17){\rule{0.900pt}{0.400pt}}
\multiput(1037.00,404.17)(2.132,2.000){2}{\rule{0.450pt}{0.400pt}}
\put(1041,406.67){\rule{0.964pt}{0.400pt}}
\multiput(1041.00,406.17)(2.000,1.000){2}{\rule{0.482pt}{0.400pt}}
\put(1045,408.17){\rule{0.900pt}{0.400pt}}
\multiput(1045.00,407.17)(2.132,2.000){2}{\rule{0.450pt}{0.400pt}}
\put(1049,410.17){\rule{0.900pt}{0.400pt}}
\multiput(1049.00,409.17)(2.132,2.000){2}{\rule{0.450pt}{0.400pt}}
\put(1053,412.17){\rule{0.900pt}{0.400pt}}
\multiput(1053.00,411.17)(2.132,2.000){2}{\rule{0.450pt}{0.400pt}}
\put(1057,414.17){\rule{0.900pt}{0.400pt}}
\multiput(1057.00,413.17)(2.132,2.000){2}{\rule{0.450pt}{0.400pt}}
\put(1061,416.17){\rule{0.900pt}{0.400pt}}
\multiput(1061.00,415.17)(2.132,2.000){2}{\rule{0.450pt}{0.400pt}}
\put(1065,417.67){\rule{0.964pt}{0.400pt}}
\multiput(1065.00,417.17)(2.000,1.000){2}{\rule{0.482pt}{0.400pt}}
\put(1069,419.17){\rule{0.900pt}{0.400pt}}
\multiput(1069.00,418.17)(2.132,2.000){2}{\rule{0.450pt}{0.400pt}}
\put(1073,421.17){\rule{0.900pt}{0.400pt}}
\multiput(1073.00,420.17)(2.132,2.000){2}{\rule{0.450pt}{0.400pt}}
\put(1077,422.67){\rule{0.964pt}{0.400pt}}
\multiput(1077.00,422.17)(2.000,1.000){2}{\rule{0.482pt}{0.400pt}}
\put(1081,424.17){\rule{0.900pt}{0.400pt}}
\multiput(1081.00,423.17)(2.132,2.000){2}{\rule{0.450pt}{0.400pt}}
\put(1085,425.67){\rule{0.964pt}{0.400pt}}
\multiput(1085.00,425.17)(2.000,1.000){2}{\rule{0.482pt}{0.400pt}}
\put(1089,427.17){\rule{0.900pt}{0.400pt}}
\multiput(1089.00,426.17)(2.132,2.000){2}{\rule{0.450pt}{0.400pt}}
\put(1093,428.67){\rule{0.964pt}{0.400pt}}
\multiput(1093.00,428.17)(2.000,1.000){2}{\rule{0.482pt}{0.400pt}}
\put(1097,430.17){\rule{0.900pt}{0.400pt}}
\multiput(1097.00,429.17)(2.132,2.000){2}{\rule{0.450pt}{0.400pt}}
\put(1101,431.67){\rule{0.964pt}{0.400pt}}
\multiput(1101.00,431.17)(2.000,1.000){2}{\rule{0.482pt}{0.400pt}}
\put(1105,432.67){\rule{0.964pt}{0.400pt}}
\multiput(1105.00,432.17)(2.000,1.000){2}{\rule{0.482pt}{0.400pt}}
\put(1109,434.17){\rule{0.900pt}{0.400pt}}
\multiput(1109.00,433.17)(2.132,2.000){2}{\rule{0.450pt}{0.400pt}}
\put(1113,435.67){\rule{0.723pt}{0.400pt}}
\multiput(1113.00,435.17)(1.500,1.000){2}{\rule{0.361pt}{0.400pt}}
\put(1116,436.67){\rule{0.964pt}{0.400pt}}
\multiput(1116.00,436.17)(2.000,1.000){2}{\rule{0.482pt}{0.400pt}}
\put(1120,437.67){\rule{0.964pt}{0.400pt}}
\multiput(1120.00,437.17)(2.000,1.000){2}{\rule{0.482pt}{0.400pt}}
\put(1124,438.67){\rule{0.964pt}{0.400pt}}
\multiput(1124.00,438.17)(2.000,1.000){2}{\rule{0.482pt}{0.400pt}}
\put(1128,439.67){\rule{0.964pt}{0.400pt}}
\multiput(1128.00,439.17)(2.000,1.000){2}{\rule{0.482pt}{0.400pt}}
\put(1132,440.67){\rule{0.964pt}{0.400pt}}
\multiput(1132.00,440.17)(2.000,1.000){2}{\rule{0.482pt}{0.400pt}}
\put(1136,441.67){\rule{0.964pt}{0.400pt}}
\multiput(1136.00,441.17)(2.000,1.000){2}{\rule{0.482pt}{0.400pt}}
\put(1140,442.67){\rule{0.964pt}{0.400pt}}
\multiput(1140.00,442.17)(2.000,1.000){2}{\rule{0.482pt}{0.400pt}}
\put(1144,443.67){\rule{0.964pt}{0.400pt}}
\multiput(1144.00,443.17)(2.000,1.000){2}{\rule{0.482pt}{0.400pt}}
\put(1148,444.67){\rule{0.964pt}{0.400pt}}
\multiput(1148.00,444.17)(2.000,1.000){2}{\rule{0.482pt}{0.400pt}}
\put(1152,445.67){\rule{0.964pt}{0.400pt}}
\multiput(1152.00,445.17)(2.000,1.000){2}{\rule{0.482pt}{0.400pt}}
\put(1156,446.67){\rule{0.964pt}{0.400pt}}
\multiput(1156.00,446.17)(2.000,1.000){2}{\rule{0.482pt}{0.400pt}}
\put(1144.0,483.0){\rule[-0.200pt]{0.964pt}{0.400pt}}
\put(1164,447.67){\rule{0.964pt}{0.400pt}}
\multiput(1164.00,447.17)(2.000,1.000){2}{\rule{0.482pt}{0.400pt}}
\put(1168,448.67){\rule{0.964pt}{0.400pt}}
\multiput(1168.00,448.17)(2.000,1.000){2}{\rule{0.482pt}{0.400pt}}
\put(1160.0,448.0){\rule[-0.200pt]{0.964pt}{0.400pt}}
\put(1176,449.67){\rule{0.964pt}{0.400pt}}
\multiput(1176.00,449.17)(2.000,1.000){2}{\rule{0.482pt}{0.400pt}}
\put(1180,450.67){\rule{0.964pt}{0.400pt}}
\multiput(1180.00,450.17)(2.000,1.000){2}{\rule{0.482pt}{0.400pt}}
\put(1172.0,450.0){\rule[-0.200pt]{0.964pt}{0.400pt}}
\put(1184.0,452.0){\rule[-0.200pt]{0.964pt}{0.400pt}}
\put(1192,451.67){\rule{0.964pt}{0.400pt}}
\multiput(1192.00,451.17)(2.000,1.000){2}{\rule{0.482pt}{0.400pt}}
\put(1188.0,452.0){\rule[-0.200pt]{0.964pt}{0.400pt}}
\put(1196.0,453.0){\rule[-0.200pt]{0.964pt}{0.400pt}}
\put(1204,452.67){\rule{0.964pt}{0.400pt}}
\multiput(1204.00,452.17)(2.000,1.000){2}{\rule{0.482pt}{0.400pt}}
\put(1200.0,453.0){\rule[-0.200pt]{0.964pt}{0.400pt}}
\put(1208.0,454.0){\rule[-0.200pt]{0.964pt}{0.400pt}}
\put(1212.0,454.0){\rule[-0.200pt]{0.964pt}{0.400pt}}
\put(1216.0,454.0){\rule[-0.200pt]{0.964pt}{0.400pt}}
\put(1220.0,454.0){\rule[-0.200pt]{0.964pt}{0.400pt}}
\put(1224.0,454.0){\rule[-0.200pt]{0.964pt}{0.400pt}}
\put(1228.0,454.0){\rule[-0.200pt]{0.964pt}{0.400pt}}
\put(1232.0,454.0){\rule[-0.200pt]{0.964pt}{0.400pt}}
\put(1240,452.67){\rule{0.964pt}{0.400pt}}
\multiput(1240.00,453.17)(2.000,-1.000){2}{\rule{0.482pt}{0.400pt}}
\put(1236.0,454.0){\rule[-0.200pt]{0.964pt}{0.400pt}}
\put(1244.0,453.0){\rule[-0.200pt]{0.964pt}{0.400pt}}
\put(894,266.17){\rule{0.900pt}{0.400pt}}
\multiput(894.00,265.17)(2.132,2.000){2}{\rule{0.450pt}{0.400pt}}
\put(898,268.17){\rule{0.900pt}{0.400pt}}
\multiput(898.00,267.17)(2.132,2.000){2}{\rule{0.450pt}{0.400pt}}
\put(902,269.67){\rule{0.964pt}{0.400pt}}
\multiput(902.00,269.17)(2.000,1.000){2}{\rule{0.482pt}{0.400pt}}
\put(906,271.17){\rule{0.900pt}{0.400pt}}
\multiput(906.00,270.17)(2.132,2.000){2}{\rule{0.450pt}{0.400pt}}
\put(910,273.17){\rule{0.900pt}{0.400pt}}
\multiput(910.00,272.17)(2.132,2.000){2}{\rule{0.450pt}{0.400pt}}
\put(914,275.17){\rule{0.900pt}{0.400pt}}
\multiput(914.00,274.17)(2.132,2.000){2}{\rule{0.450pt}{0.400pt}}
\put(918,277.17){\rule{0.900pt}{0.400pt}}
\multiput(918.00,276.17)(2.132,2.000){2}{\rule{0.450pt}{0.400pt}}
\put(922,279.17){\rule{0.900pt}{0.400pt}}
\multiput(922.00,278.17)(2.132,2.000){2}{\rule{0.450pt}{0.400pt}}
\put(926,280.67){\rule{0.964pt}{0.400pt}}
\multiput(926.00,280.17)(2.000,1.000){2}{\rule{0.482pt}{0.400pt}}
\put(930,282.17){\rule{0.900pt}{0.400pt}}
\multiput(930.00,281.17)(2.132,2.000){2}{\rule{0.450pt}{0.400pt}}
\put(934,284.17){\rule{0.700pt}{0.400pt}}
\multiput(934.00,283.17)(1.547,2.000){2}{\rule{0.350pt}{0.400pt}}
\put(937,286.17){\rule{0.900pt}{0.400pt}}
\multiput(937.00,285.17)(2.132,2.000){2}{\rule{0.450pt}{0.400pt}}
\put(941,288.17){\rule{0.900pt}{0.400pt}}
\multiput(941.00,287.17)(2.132,2.000){2}{\rule{0.450pt}{0.400pt}}
\put(945,290.17){\rule{0.900pt}{0.400pt}}
\multiput(945.00,289.17)(2.132,2.000){2}{\rule{0.450pt}{0.400pt}}
\put(949,291.67){\rule{0.964pt}{0.400pt}}
\multiput(949.00,291.17)(2.000,1.000){2}{\rule{0.482pt}{0.400pt}}
\put(953,293.17){\rule{0.900pt}{0.400pt}}
\multiput(953.00,292.17)(2.132,2.000){2}{\rule{0.450pt}{0.400pt}}
\put(957,295.17){\rule{0.900pt}{0.400pt}}
\multiput(957.00,294.17)(2.132,2.000){2}{\rule{0.450pt}{0.400pt}}
\put(961,297.17){\rule{0.900pt}{0.400pt}}
\multiput(961.00,296.17)(2.132,2.000){2}{\rule{0.450pt}{0.400pt}}
\put(965,299.17){\rule{0.900pt}{0.400pt}}
\multiput(965.00,298.17)(2.132,2.000){2}{\rule{0.450pt}{0.400pt}}
\put(969,301.17){\rule{0.900pt}{0.400pt}}
\multiput(969.00,300.17)(2.132,2.000){2}{\rule{0.450pt}{0.400pt}}
\put(973,302.67){\rule{0.964pt}{0.400pt}}
\multiput(973.00,302.17)(2.000,1.000){2}{\rule{0.482pt}{0.400pt}}
\put(977,304.17){\rule{0.900pt}{0.400pt}}
\multiput(977.00,303.17)(2.132,2.000){2}{\rule{0.450pt}{0.400pt}}
\put(981,306.17){\rule{0.900pt}{0.400pt}}
\multiput(981.00,305.17)(2.132,2.000){2}{\rule{0.450pt}{0.400pt}}
\put(985,308.17){\rule{0.900pt}{0.400pt}}
\multiput(985.00,307.17)(2.132,2.000){2}{\rule{0.450pt}{0.400pt}}
\put(989,310.17){\rule{0.900pt}{0.400pt}}
\multiput(989.00,309.17)(2.132,2.000){2}{\rule{0.450pt}{0.400pt}}
\put(993,312.17){\rule{0.900pt}{0.400pt}}
\multiput(993.00,311.17)(2.132,2.000){2}{\rule{0.450pt}{0.400pt}}
\put(997,313.67){\rule{0.964pt}{0.400pt}}
\multiput(997.00,313.17)(2.000,1.000){2}{\rule{0.482pt}{0.400pt}}
\put(1001,315.17){\rule{0.900pt}{0.400pt}}
\multiput(1001.00,314.17)(2.132,2.000){2}{\rule{0.450pt}{0.400pt}}
\put(1005,317.17){\rule{0.900pt}{0.400pt}}
\multiput(1005.00,316.17)(2.132,2.000){2}{\rule{0.450pt}{0.400pt}}
\put(1009,319.17){\rule{0.900pt}{0.400pt}}
\multiput(1009.00,318.17)(2.132,2.000){2}{\rule{0.450pt}{0.400pt}}
\put(1013,321.17){\rule{0.900pt}{0.400pt}}
\multiput(1013.00,320.17)(2.132,2.000){2}{\rule{0.450pt}{0.400pt}}
\put(1017,323.17){\rule{0.900pt}{0.400pt}}
\multiput(1017.00,322.17)(2.132,2.000){2}{\rule{0.450pt}{0.400pt}}
\put(1021,324.67){\rule{0.964pt}{0.400pt}}
\multiput(1021.00,324.17)(2.000,1.000){2}{\rule{0.482pt}{0.400pt}}
\put(1025,326.17){\rule{0.900pt}{0.400pt}}
\multiput(1025.00,325.17)(2.132,2.000){2}{\rule{0.450pt}{0.400pt}}
\put(1029,328.17){\rule{0.900pt}{0.400pt}}
\multiput(1029.00,327.17)(2.132,2.000){2}{\rule{0.450pt}{0.400pt}}
\put(1033,330.17){\rule{0.900pt}{0.400pt}}
\multiput(1033.00,329.17)(2.132,2.000){2}{\rule{0.450pt}{0.400pt}}
\put(1037,332.17){\rule{0.900pt}{0.400pt}}
\multiput(1037.00,331.17)(2.132,2.000){2}{\rule{0.450pt}{0.400pt}}
\put(1041,334.17){\rule{0.900pt}{0.400pt}}
\multiput(1041.00,333.17)(2.132,2.000){2}{\rule{0.450pt}{0.400pt}}
\put(1045,335.67){\rule{0.964pt}{0.400pt}}
\multiput(1045.00,335.17)(2.000,1.000){2}{\rule{0.482pt}{0.400pt}}
\put(1049,337.17){\rule{0.900pt}{0.400pt}}
\multiput(1049.00,336.17)(2.132,2.000){2}{\rule{0.450pt}{0.400pt}}
\put(1053,339.17){\rule{0.900pt}{0.400pt}}
\multiput(1053.00,338.17)(2.132,2.000){2}{\rule{0.450pt}{0.400pt}}
\put(1057,341.17){\rule{0.900pt}{0.400pt}}
\multiput(1057.00,340.17)(2.132,2.000){2}{\rule{0.450pt}{0.400pt}}
\put(1061,343.17){\rule{0.900pt}{0.400pt}}
\multiput(1061.00,342.17)(2.132,2.000){2}{\rule{0.450pt}{0.400pt}}
\put(1065,345.17){\rule{0.900pt}{0.400pt}}
\multiput(1065.00,344.17)(2.132,2.000){2}{\rule{0.450pt}{0.400pt}}
\put(1069,346.67){\rule{0.964pt}{0.400pt}}
\multiput(1069.00,346.17)(2.000,1.000){2}{\rule{0.482pt}{0.400pt}}
\put(1073,348.17){\rule{0.900pt}{0.400pt}}
\multiput(1073.00,347.17)(2.132,2.000){2}{\rule{0.450pt}{0.400pt}}
\put(1077,350.17){\rule{0.900pt}{0.400pt}}
\multiput(1077.00,349.17)(2.132,2.000){2}{\rule{0.450pt}{0.400pt}}
\put(1081,352.17){\rule{0.900pt}{0.400pt}}
\multiput(1081.00,351.17)(2.132,2.000){2}{\rule{0.450pt}{0.400pt}}
\put(1085,354.17){\rule{0.900pt}{0.400pt}}
\multiput(1085.00,353.17)(2.132,2.000){2}{\rule{0.450pt}{0.400pt}}
\put(1089,356.17){\rule{0.900pt}{0.400pt}}
\multiput(1089.00,355.17)(2.132,2.000){2}{\rule{0.450pt}{0.400pt}}
\put(1093,357.67){\rule{0.964pt}{0.400pt}}
\multiput(1093.00,357.17)(2.000,1.000){2}{\rule{0.482pt}{0.400pt}}
\put(1097,359.17){\rule{0.900pt}{0.400pt}}
\multiput(1097.00,358.17)(2.132,2.000){2}{\rule{0.450pt}{0.400pt}}
\put(1101,361.17){\rule{0.900pt}{0.400pt}}
\multiput(1101.00,360.17)(2.132,2.000){2}{\rule{0.450pt}{0.400pt}}
\put(1105,363.17){\rule{0.900pt}{0.400pt}}
\multiput(1105.00,362.17)(2.132,2.000){2}{\rule{0.450pt}{0.400pt}}
\put(1109,365.17){\rule{0.900pt}{0.400pt}}
\multiput(1109.00,364.17)(2.132,2.000){2}{\rule{0.450pt}{0.400pt}}
\put(1113,367.17){\rule{0.900pt}{0.400pt}}
\multiput(1113.00,366.17)(2.132,2.000){2}{\rule{0.450pt}{0.400pt}}
\put(1117,368.67){\rule{0.964pt}{0.400pt}}
\multiput(1117.00,368.17)(2.000,1.000){2}{\rule{0.482pt}{0.400pt}}
\put(1121,370.17){\rule{0.900pt}{0.400pt}}
\multiput(1121.00,369.17)(2.132,2.000){2}{\rule{0.450pt}{0.400pt}}
\put(1125,372.17){\rule{0.900pt}{0.400pt}}
\multiput(1125.00,371.17)(2.132,2.000){2}{\rule{0.450pt}{0.400pt}}
\put(1129,374.17){\rule{0.900pt}{0.400pt}}
\multiput(1129.00,373.17)(2.132,2.000){2}{\rule{0.450pt}{0.400pt}}
\put(1133,376.17){\rule{0.700pt}{0.400pt}}
\multiput(1133.00,375.17)(1.547,2.000){2}{\rule{0.350pt}{0.400pt}}
\put(1136,378.17){\rule{0.900pt}{0.400pt}}
\multiput(1136.00,377.17)(2.132,2.000){2}{\rule{0.450pt}{0.400pt}}
\put(1140,379.67){\rule{0.964pt}{0.400pt}}
\multiput(1140.00,379.17)(2.000,1.000){2}{\rule{0.482pt}{0.400pt}}
\put(1144,381.17){\rule{0.900pt}{0.400pt}}
\multiput(1144.00,380.17)(2.132,2.000){2}{\rule{0.450pt}{0.400pt}}
\put(1148,383.17){\rule{0.900pt}{0.400pt}}
\multiput(1148.00,382.17)(2.132,2.000){2}{\rule{0.450pt}{0.400pt}}
\put(1152,385.17){\rule{0.900pt}{0.400pt}}
\multiput(1152.00,384.17)(2.132,2.000){2}{\rule{0.450pt}{0.400pt}}
\put(1156,387.17){\rule{0.900pt}{0.400pt}}
\multiput(1156.00,386.17)(2.132,2.000){2}{\rule{0.450pt}{0.400pt}}
\put(1160,389.17){\rule{0.900pt}{0.400pt}}
\multiput(1160.00,388.17)(2.132,2.000){2}{\rule{0.450pt}{0.400pt}}
\put(1164,390.67){\rule{0.964pt}{0.400pt}}
\multiput(1164.00,390.17)(2.000,1.000){2}{\rule{0.482pt}{0.400pt}}
\put(1168,392.17){\rule{0.900pt}{0.400pt}}
\multiput(1168.00,391.17)(2.132,2.000){2}{\rule{0.450pt}{0.400pt}}
\put(1172,394.17){\rule{0.900pt}{0.400pt}}
\multiput(1172.00,393.17)(2.132,2.000){2}{\rule{0.450pt}{0.400pt}}
\put(1176,396.17){\rule{0.900pt}{0.400pt}}
\multiput(1176.00,395.17)(2.132,2.000){2}{\rule{0.450pt}{0.400pt}}
\put(1180,398.17){\rule{0.900pt}{0.400pt}}
\multiput(1180.00,397.17)(2.132,2.000){2}{\rule{0.450pt}{0.400pt}}
\put(1184,400.17){\rule{0.900pt}{0.400pt}}
\multiput(1184.00,399.17)(2.132,2.000){2}{\rule{0.450pt}{0.400pt}}
\put(1188,401.67){\rule{0.964pt}{0.400pt}}
\multiput(1188.00,401.17)(2.000,1.000){2}{\rule{0.482pt}{0.400pt}}
\put(1192,403.17){\rule{0.900pt}{0.400pt}}
\multiput(1192.00,402.17)(2.132,2.000){2}{\rule{0.450pt}{0.400pt}}
\put(1196,405.17){\rule{0.900pt}{0.400pt}}
\multiput(1196.00,404.17)(2.132,2.000){2}{\rule{0.450pt}{0.400pt}}
\put(1200,407.17){\rule{0.900pt}{0.400pt}}
\multiput(1200.00,406.17)(2.132,2.000){2}{\rule{0.450pt}{0.400pt}}
\put(1204,409.17){\rule{0.900pt}{0.400pt}}
\multiput(1204.00,408.17)(2.132,2.000){2}{\rule{0.450pt}{0.400pt}}
\put(1208,411.17){\rule{0.900pt}{0.400pt}}
\multiput(1208.00,410.17)(2.132,2.000){2}{\rule{0.450pt}{0.400pt}}
\put(1212,412.67){\rule{0.964pt}{0.400pt}}
\multiput(1212.00,412.17)(2.000,1.000){2}{\rule{0.482pt}{0.400pt}}
\put(1216,414.17){\rule{0.900pt}{0.400pt}}
\multiput(1216.00,413.17)(2.132,2.000){2}{\rule{0.450pt}{0.400pt}}
\put(1220,416.17){\rule{0.900pt}{0.400pt}}
\multiput(1220.00,415.17)(2.132,2.000){2}{\rule{0.450pt}{0.400pt}}
\put(1224,418.17){\rule{0.900pt}{0.400pt}}
\multiput(1224.00,417.17)(2.132,2.000){2}{\rule{0.450pt}{0.400pt}}
\put(1228,420.17){\rule{0.900pt}{0.400pt}}
\multiput(1228.00,419.17)(2.132,2.000){2}{\rule{0.450pt}{0.400pt}}
\put(1232,422.17){\rule{0.900pt}{0.400pt}}
\multiput(1232.00,421.17)(2.132,2.000){2}{\rule{0.450pt}{0.400pt}}
\put(1236,423.67){\rule{0.964pt}{0.400pt}}
\multiput(1236.00,423.17)(2.000,1.000){2}{\rule{0.482pt}{0.400pt}}
\put(1240,425.17){\rule{0.900pt}{0.400pt}}
\multiput(1240.00,424.17)(2.132,2.000){2}{\rule{0.450pt}{0.400pt}}
\put(1244,427.17){\rule{0.900pt}{0.400pt}}
\multiput(1244.00,426.17)(2.132,2.000){2}{\rule{0.450pt}{0.400pt}}
\put(1248,429.17){\rule{0.900pt}{0.400pt}}
\multiput(1248.00,428.17)(2.132,2.000){2}{\rule{0.450pt}{0.400pt}}
\put(1252,431.17){\rule{0.900pt}{0.400pt}}
\multiput(1252.00,430.17)(2.132,2.000){2}{\rule{0.450pt}{0.400pt}}
\put(1256,433.17){\rule{0.900pt}{0.400pt}}
\multiput(1256.00,432.17)(2.132,2.000){2}{\rule{0.450pt}{0.400pt}}
\put(1260,434.67){\rule{0.964pt}{0.400pt}}
\multiput(1260.00,434.17)(2.000,1.000){2}{\rule{0.482pt}{0.400pt}}
\put(1264,436.17){\rule{0.900pt}{0.400pt}}
\multiput(1264.00,435.17)(2.132,2.000){2}{\rule{0.450pt}{0.400pt}}
\put(1268,438.17){\rule{0.900pt}{0.400pt}}
\multiput(1268.00,437.17)(2.132,2.000){2}{\rule{0.450pt}{0.400pt}}
\put(1272,440.17){\rule{0.900pt}{0.400pt}}
\multiput(1272.00,439.17)(2.132,2.000){2}{\rule{0.450pt}{0.400pt}}
\put(1276,442.17){\rule{0.900pt}{0.400pt}}
\multiput(1276.00,441.17)(2.132,2.000){2}{\rule{0.450pt}{0.400pt}}
\put(1280,444.17){\rule{0.900pt}{0.400pt}}
\multiput(1280.00,443.17)(2.132,2.000){2}{\rule{0.450pt}{0.400pt}}
\put(1284,445.67){\rule{0.964pt}{0.400pt}}
\multiput(1284.00,445.17)(2.000,1.000){2}{\rule{0.482pt}{0.400pt}}
\put(1248.0,453.0){\rule[-0.200pt]{0.964pt}{0.400pt}}
\multiput(1283.97,325.92)(-1.089,-0.500){359}{\rule{0.971pt}{0.120pt}}
\multiput(1285.99,326.17)(-391.985,-181.000){2}{\rule{0.485pt}{0.400pt}}
\multiput(212.00,249.92)(3.256,-0.499){207}{\rule{2.698pt}{0.120pt}}
\multiput(212.00,250.17)(676.400,-105.000){2}{\rule{1.349pt}{0.400pt}}
\put(894.0,146.0){\rule[-0.200pt]{0.400pt}{29.149pt}}
\end{picture}

\caption{The 2D entropy function}
\label{Fig1}
\end{figure}

Note that in practice $\ln(d!)$ is typically implemented as $\ln(\Gamma(1+d))$. The
first order derivative of $\ln(\Gamma(x))$ is called the digamma function denoted as
$\psi_0(x)$. This means that Eqs.~\ref{Eq:S1} and \ref{Eq:S2} can be rephrased as
\begin{eqnarray}
  S_{i}^{(1)}  &=& \ln\left(\Gamma\left(1+d_i\right)\right)
                +  \ln\left(\Gamma\left(2-d_i\right)\right) \\
  S_{ij}^{(2)} &=& \ln\left(\Gamma\left(1+d_i d_j\right)\right)
                +  \ln\left(\Gamma\left(1+\left[1-d_i\right]d_j\right)\right)
                +  \ln\left(\Gamma\left(1+d_i\left[1-d_j\right]\right)\right)
                +  \ln\left(\Gamma\left(1+\left[1-d_i\right]\left[1-d_j\right]\right)\right)
\end{eqnarray}

In order to optimize the energy with the expressions above we need to differentiate
them. 

\begin{eqnarray}
  \frac{\partial S_{i}^{(1)}}{\partial C_{it}}
  &=& \psi_0(1+d_i)\frac{\partial d_i}{\partial C_{it}}
   -  \psi_0(2-d_i)\frac{\partial d_i}{\partial C_{it}} \\
  \frac{\partial S_{ij}^{(2)}}{\partial C_{it}}
  &=& \psi_0(1+d_id_j)d_j\frac{\partial d_i}{\partial C_{it}}
   -  \psi_0(1+[1-d_i]d_j)d_j\frac{\partial d_i}{\partial C_{it}} \nonumber \\
  &+& \psi_0(1+d_i[1-d_j])[1-d_j]\frac{\partial d_i}{\partial C_{it}}
   -  \psi_0(1+[1-d_i][1-d_j])[1-d_j]\frac{\partial d_i}{\partial C_{it}}
\end{eqnarray}

\section{Optimizing the wave function}

The wave function may be optimized by starting from an energy expression with appropriate
Lagrangians and minimizing it. An approach similar to that of our previous
paper~\cite{van_Dam_2016} is used. There is a difference because the energy expression in
this work cannot be represented entirely in terms of one-electron density matrices as
the two-electron term is orbital dependent. This means that the trick of multiplying
the Lagrangian for the natural orbitals with the occupation numbers to enable arriving
at the Kohn-Sham equations for those orbitals cannot be used here. Like in our previous
work we will derive the derivatives with respect to the $\alpha$-spin channel variables
only as the $\beta$-spin channel derivatives are the same. Thus we consider the
problem
\begin{eqnarray}
  L &=&
  E(N^\alpha,C^\alpha;N^\beta,C^\beta)
  + \sum_{\sigma=\{\alpha,\beta\}}\sum_{i,j=1}^{n_b}\lambda^{N^\sigma}_{ij}
    \left(I_{ij}-\sum_{a,b=1}^{n_b}N^{\sigma*}_{ai}S_{ab}N^\sigma_{bj}\right) 
    \nonumber \\
  &&+ \sum_{\sigma=\{\alpha,\beta\}}\sum_{r,s=1}^{n_b}\lambda^{C^\sigma}_{rs}
      \left(I_{rs}-\sum_{i,j=1}^{n_b}C^{\sigma*}_{ir}I_{ij}C^\sigma_{js}\right) \\
  && \min_{N^\alpha,C^\alpha,\lambda^{N^\alpha}_{ij},\lambda^{C^\alpha}_{rs}} L
\end{eqnarray}
Before continuing the energy expression $E(N^\alpha,C^\alpha;N^\beta,C^\beta)$
must be clarified. In general the energy expression may be written as
\begin{eqnarray}
  E(N^\alpha,C^\alpha;N^\beta,C^\beta)
  &=& \sum_{\sigma,\sigma'=\{\alpha,\beta\}}
      \left(\sum_{a,b=1}^{n_b}\frac{1}{2}h_{ab}D^\sigma_{ab}
   +  \sum_{r,s=1}^{n_b}W^{\sigma\sigma'}_{rs}
      \left(\frac{1}{2}-S^{\sigma\sigma'}_{rs}\right)\right)
  \label{Eq:ENC}
\end{eqnarray}

In the subsequent derivations the derivatives w.r.t. the $\alpha$-spin channel will be
given. The equations for the $\beta$-spin channel can be obtained by interchanging the
the $\alpha$ and $\beta$ spin labels.

For the orthogonality of the natural orbitals we obtain from
$\partial L/\partial\lambda^{C\alpha}_{rs} = 0$
\begin{eqnarray}
  \sum_{i,j=1}^{n_b}C^{\alpha*}_{ir}I_{ij}C^\alpha_{js} &=& I_{rs}
\end{eqnarray}
Likewise for the natural orbitals
$\partial L/\partial\lambda^{N\alpha}_{ij} = 0$ gives
\begin{eqnarray}
  \sum_{a,b=1}^{n_b}N^{\alpha*}_{ai}S_{ab}N^\alpha_{bj} &=& I_{ij}
\end{eqnarray}

Considering derivatives w.r.t. the natural orbitals there are only two relevant
quantities. Those are the one-electron energy terms of Eq.~\ref{Eq:ENC} and the 
electron temperatures. The entropy depends only on the correlation functions
and hence has no non-zero derivatives w.r.t. the natural orbitals.

Differentiating the one-electron energy terms we obtain
\begin{eqnarray}
  \frac{\partial}{\partial N^{*}_{ek}}\sum_{a,b=1}^{n_b}h_{ab}D_{ab}
  &=& \frac{\partial}{\partial N^{*}_{ek}}
      \sum_{a,b,i=1}^{n_b}h_{ab}N_{ai}d_iN^*_{bi} \\
  &=& \sum_{a,b,i=1}^{n_b}h_{ab}N_{ai}d_i\delta_{eb}\delta_{ki} \\
  &=& \sum_{a,i=1}^{n_b}h_{ae}N_{ai}d_i\delta_{ki} \\
  &=& \sum_{a=1}^{n_b}h_{ae}N_{ak}d_k 
\end{eqnarray}
This derivative may be transformed into the natural orbital basis to give
\begin{eqnarray}
  \sum_{a,e=1}^{n_b}h_{ae}N_{ak}d_k N^*_{el}
  &=& d_k h_{kl}
\end{eqnarray}
The result is a matrix where the rows are scaled by the occupation numbers
of the natural orbitals.

Next we consider the derivative of the temperature of an electron pair
occupying generalized orbitals $r$ and $s$. Here Eq.~\ref{Eq:Tr} is replaced by
the $W$ from Eq.~\ref{Eq:ENC} slightly as
\begin{eqnarray}
   W_{rs} &=& \sum_{a,b,c,d=1}^{n_b}(ab|cd)\Gamma_{abcd}^{(rs)}
\end{eqnarray}
The derivative of the temperature depends on the derivatives of $\Gamma$ given by
\begin{eqnarray}
  \frac{\partial}{\partial N^{*}_{ek}}W_{rs}
  &=& \sum_{a,b,c,d=1}^{n_b}(ab|cd)
      \frac{\partial}{\partial N^{*}_{ek}}\Gamma_{abcd}^{(rs)} \\
  \frac{\partial}{\partial N^{*}_{ek}}\Gamma_{abcd}^{(rs)}
  &=& \left(\sum_{i=1}^{n_b}N_{ai}C_{ir}C^*_{ir}\delta_{ce}\delta_{ik}\right)
      \left(\sum_{j=1}^{n_b}N_{bj}C_{js}C^*_{js}N^*_{dj}\right) \nonumber \\
  &&+ \left(\sum_{i=1}^{n_b}N_{ai}C_{ir}C^*_{ir}N^*_{ci}\right)
      \left(\sum_{j=1}^{n_b}N_{bj}C_{js}C^*_{js}\delta_{de}\delta_{jk}\right) 
      \nonumber \\
  &&- \left(\sum_{i=1}^{n_b}N_{ai}C_{ir}C^*_{is}\delta_{ce}\delta_{ik}\right)
      \left(\sum_{j=1}^{n_b}N_{bj}C_{js}C^*_{jr}N^*_{dj}\right) \nonumber \\
  &&- \left(\sum_{i=1}^{n_b}N_{ai}C_{ir}C^*_{is}N^*_{ci}\right)
      \left(\sum_{j=1}^{n_b}N_{bj}C_{js}C^*_{jr}\delta_{de}\delta_{jk}\right) \\
  &=& \left(N_{ak}C_{kr}C^*_{kr}\delta_{ce}\right)
      \left(\sum_{j=1}^{n_b}N_{bj}C_{js}C^*_{js}N^*_{dj}\right) \nonumber \\
  &&+ \left(\sum_{i=1}^{n_b}N_{ai}C_{ir}C^*_{ir}N^*_{ci}\right)
      \left(N_{bk}C_{ks}C^*_{ks}\delta_{de}\right) 
      \nonumber \\
  &&- \left(N_{ak}C_{kr}C^*_{ks}\delta_{ce}\right)
      \left(\sum_{j=1}^{n_b}N_{bj}C_{js}C^*_{jr}N^*_{dj}\right) \nonumber \\
  &&- \left(\sum_{i=1}^{n_b}N_{ai}C_{ir}C^*_{is}N^*_{ci}\right)
      \left(N_{bk}C_{ks}C^*_{kr}\delta_{de}\right) \\
  \frac{\partial}{\partial N^{*}_{ek}}\Gamma_{abcd}^{(rs')}
  &=& \left(N_{ak}C_{kr}C^*_{kr}\delta_{ce}\right)
      \left(\sum_{j'=1}^{n_b}N'_{bj'}C'_{j's'}C'^*_{j's'}N'^*_{dj'}\right)
\end{eqnarray}
Considering the contraction of these derivatives with the two-electron integrals
equations similar to those for the Fock matrix are obtained. The difference, like for 
the one-electron terms, comes from the fact that the rows are scaled with the
occupation numbers. Unlike in the previous paper~\cite{van_Dam_2016} where the energy
expression was given entirely in terms of the total one-electron density matrix, here
the exchange contributions are scaled products of correlation function coefficients 
corresponding to different generalized orbitals. This means that we cannot separate
the occupation numbers out like was done in~\cite{van_Dam_2016}.

Next considering the derivatives w.r.t. to the correlation functions there are three 
terms or factors to consider. They are the one-electron energy terms, the two-electron
terms, and the entropy. Considering the one-electron energy terms we have
\begin{eqnarray}
  \frac{\partial}{\partial C^{*}_{kt}}\sum_{a,b=1}^{n_b}h_{ab}D_{ab}
  &=& \frac{\partial}{\partial C^{*}_{kt}}\sum_{a,b,i=1}^{n_b}\sum_{r=1}^{n_e}
      h_{ab}N_{ai}C_{ir}C^*_{ir}N^*_{bi} \\
  &=& \sum_{a,b,i=1}^{n_b}\sum_{r=1}^{n_e}
      h_{ab}N_{ai}C_{ir}N^*_{bi}\delta_{ik}\delta_{rt} \\
  &=& \sum_{a,b=1}^{n_b}
      h_{ab}N_{ak}C_{kt}N^*_{bk}
\end{eqnarray}
Regarding the temperature we get
\begin{eqnarray}
  \frac{\partial}{\partial C^{*}_{kt}}W_{rs}
  &=& \sum_{a,b,c,d=1}^{n_b}(ab|cd)
      \frac{\partial}{\partial C^{*}_{kt}}\Gamma_{abcd}^{(rs)} \\
  \frac{\partial}{\partial C^{*}_{kt}}\Gamma_{abcd}^{(rs)}
  &=& \frac{1}{2}\left[
      \sum_{r,s=1}^{n_e}
      \left(\sum_{i=1}^{n_b} N_{ai}C_{ir}N^*_{ci}\delta_{ik}\delta_{rt}\right)
      \left(\sum_{j=1}^{n_b} N_{bj}C_{js}C^*_{js}N^*_{dj}\right)
      \right.
      \nonumber \\
  &&+ \sum_{r,s=1}^{n_e}
      \left(\sum_{i=1}^{n_b} N_{ai}C_{ir}C^*_{ir}N^*_{ci}\right)
      \left(\sum_{j=1}^{n_b} N_{bj}C_{js}N^*_{dj}\delta_{jk}\delta_{st}\right)
      \nonumber \\
  &&- \sum_{r,s=1}^{n_e}
      \left(\sum_{i=1}^{n_b} N_{ai}C_{ir}N^*_{ci}\delta_{ik}\delta_{st}\right)
      \left(\sum_{j=1}^{n_b} N_{bj}C_{js}C^*_{jr}N^*_{dj}\right)
      \nonumber \\
  &&- \left.
      \sum_{r,s=1}^{n_e}
      \left(\sum_{i=1}^{n_b} N_{ai}C_{ir}C^*_{is}N^*_{ci}\right)
      \left(\sum_{j=1}^{n_b} N_{bj}C_{js}N^*_{dj}\delta_{jk}\delta_{rt}\right)
      \right] \\
  &=& \frac{1}{2}\left[
      \sum_{s=1}^{n_e}
      \left(                 N_{ak}C_{kt}N^*_{ck}\right)
      \left(\sum_{j=1}^{n_b} N_{bj}C_{js}C^*_{js}N^*_{dj}\right)
      \right.
      \nonumber \\
  &&+ \sum_{r=1}^{n_e}
      \left(\sum_{i=1}^{n_b} N_{ai}C_{ir}C^*_{ir}N^*_{ci}\right)
      \left(                 N_{bk}C_{kt}N^*_{dk}\right)
      \nonumber \\
  &&- \sum_{r=1}^{n_e}
      \left(                 N_{ak}C_{kr}N^*_{ck}\right)
      \left(\sum_{j=1}^{n_b} N_{bj}C_{jt}C^*_{jr}N^*_{dj}\right)
      \nonumber \\
  &&- \left.
      \sum_{s=1}^{n_e}
      \left(\sum_{i=1}^{n_b} N_{ai}C_{it}C^*_{is}N^*_{ci}\right)
      \left(                 N_{bk}C_{ks}N^*_{dk}\right)
      \right]
      \label{Eq:2densmat_aa_dC}
  \\
  \frac{\partial}{\partial C^{*}_{kt}}\Gamma_{abcd}^{(rs')}
  &=& \sum_{s'=1}^{n_e}\frac{1}{2}
      \left(                  N_{ak}C_{kt}N^*_{ck}\right)
      \left(\sum_{j'=1}^{n_b} N_{b'j'}C_{j's'}C^*_{j's'}N^*_{d'j'}\right)
      \label{Eq:2densmat_ab_dC}
\end{eqnarray}

The next aspect to consider is the entropy. Here two factors play an important
role: 1. The entropy expressions themselves; 2. The factors that partition the
entropy into contributions of individual electrons.
As the entropy is expressed in terms of the occupation numbers it would seem
sensible to start with differentiating those. Also note that the occupation
numbers depend only on the correlation coefficients and not on the natural
orbital coefficient. Hence a smaller set of equations will be needed.
\begin{eqnarray}
   \frac{\partial}{\partial C^*_{kt}}d^{(r)}_i
   &=& \frac{\partial}{\partial C^*_{kt}}C_{ir}C^*_{ir} \\
   &=& C_{ir}\delta_{ik}\delta_{rt} \\
   &=& C_{kt}\delta_{ik}\delta_{rt} \\
   \frac{\partial}{\partial C^*_{kt}}d_i
   &=& \frac{\partial}{\partial C^*_{kt}}\sum_{r=1}^{n_e}C_{ir}C^*_{ir} \\
   &=& \sum_{r=1}^{n_e}C_{ir}\delta_{ik}\delta_{rt} \\
   &=& \left\{\begin{array}{cc}
              C_{kt}\delta_{ik}, & t \le n_e \\
              0, & t > n_e \\
              \end{array}\right.\\
   \frac{\partial}{\partial C^*_{kt}}\frac{d^{(r)}_i}{d_i}
   &=& \frac{C_{kt}\delta_{ik}\delta_{rt}}{d_i}
      -\frac{d^{(r)}_iC_{kt}\delta_{ik}}{d_i^2} \\
   \frac{\partial}{\partial C^*_{kt}}S^{(1)}(d_i)
   &=& \frac{\partial S^{(1)}(d_i)}{\partial d_i}\frac{\partial d_i}{C^*_{kt}} \\
   \frac{\partial}{\partial d_i}S^{(1)}(d_i)
   &=& \ln(d_i) + 1 - \ln(1-d_i) - 1\\
   &=& \ln(d_i) - \ln(1-d_i)
\end{eqnarray}
A likewise scenario arises for the entropy related to the two electron 
density matrix.

For the two-electron density matrix occupation numbers we get
\begin{eqnarray}
   \frac{\partial}{\partial C^*_{kt}}d^{(rs)}_{ij}
   &=& \frac{\partial}{\partial C^*_{kt}}\left(
         C_{ir}C^*_{ir}C_{js}C^*_{js}
       - C_{ir}C^*_{is}C_{js}C^*_{jr}
       \right) \\
   &=& C_{ir}\delta_{ik}\delta_{rt}C_{js}C^*_{js}
    +  C_{ir}C^*_{ir}C_{js}\delta_{jk}\delta_{st} \nonumber \\
   &&- C_{ir}\delta_{ik}\delta_{st}C_{js}C^*_{jr}
    -  C_{ir}C^*_{is}C_{js}\delta_{jk}\delta_{rt} \\
   \frac{\partial}{\partial C^*_{kt}}d_{ij}
   &=  \frac{\partial}{\partial C^*_{kt}}&\sum_{r,s=1}^{n_e}\left(
         C_{ir}C^*_{ir}C_{js}C^*_{js}
       - C_{ir}C^*_{is}C_{js}C^*_{jr}
       \right) \\
   &=  \sum_{r,s=1}^{n_e}&\left(C_{ir}\delta_{ik}\delta_{rt}C_{js}C^*_{js}
    +  C_{ir}C^*_{ir}C_{js}\delta_{jk}\delta_{st}\right. \nonumber \\
   &&- \left.C_{ir}\delta_{ik}\delta_{st}C_{js}C^*_{jr}
    -  C_{ir}C^*_{is}C_{js}\delta_{jk}\delta_{rt}\right) \\
   &=  \sum_{r=1}^{n_e}&\left(C_{it}\delta_{ik}C_{jr}C^*_{jr}
    +  C_{ir}C^*_{ir}C_{jt}\delta_{jk}\right. \nonumber \\
   &&- \left.C_{ir}\delta_{ik}C_{jt}C^*_{jr}
    -  C_{it}C^*_{ir}C_{jr}\delta_{jk}\right) \\
   \frac{\partial}{\partial C^*_{kt}}S^{(2)}(d_{ij})
   &=& \frac{\partial S^{(2)}(d_{ij})}{\partial d_{ij}}\frac{\partial d_{ij}}{C^*_{kt}} \\
\end{eqnarray}

We need to differentiate the 2-electron density matrix wrt. the natural orbitals and the
correlation functions. We start from Eq.(\ref{Eq:2densmat_3}) and we differentiate wrt.
$N^*_{ek}$
\begin{eqnarray}
  \frac{\partial\Gamma_{abcd}}{\partial N^*_{ek}}
  &=& \frac{1}{4}
      \left[
      \left(\sum_{i,j=1}^{n_b} N_{ai}\delta_{ec}\delta_{ki}N_{bj}N^*_{dj}d_{ij}\right)
     +\left(\sum_{i,j=1}^{n_b} N_{ai}N^*_{ci}N_{bj}\delta_{ed}\delta_{kj}d_{ij}\right)
      \right.
      \nonumber \\
  &&+ \left(\sum_{i,j=1}^{n_b} N_{bi}\delta_{ed}\delta_{ki}N_{aj}N^*_{cj}d_{ij}\right)
    + \left(\sum_{i,j=1}^{n_b} N_{bi}N^*_{di}N_{aj}\delta_{ec}\delta_{kj}d_{ij}\right)
      \nonumber \\
  &&- \left(\sum_{i,j=1}^{n_b} N_{ai}\delta_{ed}\delta_{ki}N_{bj}N^*_{cj}d_{ij}\right)
    - \left(\sum_{i,j=1}^{n_b} N_{ai}N^*_{di}N_{bj}\delta_{ec}\delta_{kj}d_{ij}\right)
      \nonumber \\
  &&- \left.
      \left(\sum_{i,j=1}^{n_b} N_{bi}\delta_{ec}\delta_{ki}N_{aj}N^*_{dj}d_{ij}\right)
    - \left(\sum_{i,j=1}^{n_b} N_{bi}N^*_{ci}N_{aj}\delta_{ed}\delta_{kj}d_{ij}\right)
      \right] \\
  &= \frac{1}{4} &
      \left[
      \left(\sum_{j=1}^{n_b} N_{ak}\delta_{ec}N_{bj}N^*_{dj}d_{kj}\right)
     +\left(\sum_{i=1}^{n_b} N_{ai}N^*_{ci}N_{bk}\delta_{ed}d_{ik}\right)
      \right.
      \nonumber \\
  &&+ \left(\sum_{j=1}^{n_b} N_{bk}\delta_{ed}N_{aj}N^*_{cj}d_{kj}\right)
    + \left(\sum_{i=1}^{n_b} N_{bi}N^*_{di}N_{ak}\delta_{ec}d_{ik}\right)
      \nonumber \\
  &&- \left(\sum_{j=1}^{n_b} N_{ak}\delta_{ed}N_{bj}N^*_{cj}d_{kj}\right)
    - \left(\sum_{i=1}^{n_b} N_{ai}N^*_{di}N_{bk}\delta_{ec}d_{ik}\right)
      \nonumber \\
  &&- \left.
      \left(\sum_{j=1}^{n_b} N_{bk}\delta_{ec}N_{aj}N^*_{dj}d_{kj}\right)
    - \left(\sum_{i=1}^{n_b} N_{bi}N^*_{ci}N_{ak}\delta_{ed}d_{ik}\right)
      \right]
      \label{Eq:2densmat_dN}
\end{eqnarray}
Next we differentiate Eq.(\ref{Eq:2densmat_3}) wrt. $C^*_{kr}$. As the equation does not
depend on the correlation function coefficients explicitly the correlation function 
Fock matrix is generated mainly through the derivative of Eq.(\ref{Eq:2occupation_3}).
Hence we first consider the derivatives of the occupation numbers:
\begin{eqnarray}
  \frac{\partial d_{ij}}{\partial C^*_{kr}}
  &=& \sum_{s,t=1}^{n_e}
      C_{is}\delta_{ik}\delta_{sr}C_{jt}C^*_{jt}
     +C_{is}C^*_{is}C_{jt}\delta_{jk}\delta_{tr}
     -C_{is}\delta_{ik}\delta_{tr}C_{jt}C^*_{js}
     -C_{is}C^*_{it}C_{jt}\delta_{jk}\delta_{sr} \\
  &=& \sum_{s=1}^{n_e}
      C_{ir}\delta_{ik}C_{js}C^*_{js}
     +C_{is}C^*_{is}C_{jr}\delta_{jk}
     -C_{is}\delta_{ik}C_{jr}C^*_{js}
     -C_{ir}C^*_{is}C_{js}\delta_{jk}
     \label{Eq:deriv_2occupation}
\end{eqnarray}
Substituting Eq.(\ref{Eq:deriv_2occupation}) into Eq.(\ref{Eq:2densmat_3}) gives
\begin{eqnarray}
  \frac{\partial\Gamma_{abcd}}{\partial C^*_{kr}}
  &=& \frac{1}{4}\left[\sum_{i,j=1}^{n_b}\sum_{s=1}^{n_e}\left(
      N_{ai}N^*_{ci}N_{bj}N^*_{dj}+N_{bi}N^*_{di}N_{aj}N^*_{cj}
      -N_{ai}N^*_{di}N_{bj}N^*_{cj}-N_{bi}N^*_{ci}N_{aj}N^*_{dj}\right)\right.
      \nonumber \\
  &&  \left.\left(C_{ir}\delta_{ik}C_{js}C^*_{js}+C_{is}C^*_{is}C_{jr}\delta_{jk}
      -C_{is}\delta_{ik}C_{jr}C^*_{js}-C_{ir}C^*_{is}C_{js}\delta_{jk}\right)\right]
%     \\
% &=& \frac{1}{4}\left[\sum_{i,j=1}^{n_b}\sum_{s=1}^{n_e}\left(
%     N_{ai}N^*_{ci}N_{bj}N^*_{dj}+N_{bi}N^*_{di}N_{aj}N^*_{cj}
%     -N_{ai}N^*_{di}N_{bj}N^*_{cj}-N_{bi}N^*_{ci}N_{aj}N^*_{dj}\right)\right.
%     \nonumber \\
% &&  \left.\left(C_{ir}\delta_{ik}C_{js}C^*_{js}+C_{js}C^*_{js}C_{ir}\delta_{ik}
%     -C_{is}\delta_{ik}C_{jr}C^*_{js}-C_{jr}C^*_{js}C_{is}\delta_{ik}\right)\right] \\
% &=& \frac{1}{4}\left[\sum_{j=1}^{n_b}\sum_{s=1}^{n_e}\left(
%     N_{ak}N^*_{ck}N_{bj}N^*_{dj}+N_{bk}N^*_{dk}N_{aj}N^*_{cj}
%     -N_{ak}N^*_{dk}N_{bj}N^*_{cj}-N_{bk}N^*_{ck}N_{aj}N^*_{dj}\right)\right.
%     \nonumber \\
% &&  \left.\left(C_{kr}C_{js}C^*_{js}+C_{js}C^*_{js}C_{kr}
%     -C_{ks}C_{jr}C^*_{js}-C_{jr}C^*_{js}C_{ks}\right)\right] \\
% &=& \frac{1}{2}\left[\sum_{j=1}^{n_b}\sum_{s=1}^{n_e}\left(
%     N_{ak}N^*_{ck}N_{bj}N^*_{dj}+N_{bk}N^*_{dk}N_{aj}N^*_{cj}
%     -N_{ak}N^*_{dk}N_{bj}N^*_{cj}-N_{bk}N^*_{ck}N_{aj}N^*_{dj}\right)\right.
%     \nonumber \\
% &&  \left.\left(C_{kr}C_{js}C^*_{js}-C_{ks}C_{jr}C^*_{js}\right)\right] \\
\end{eqnarray}
Of course we want a sensible total derivative hence we need to sum over all $k$. 
\begin{eqnarray}
  \sum_{k=1}^{n_b}\frac{\partial\Gamma_{abcd}}{\partial C^*_{kr}}
  &=& \frac{1}{4}\left[\sum_{i,j,k=1}^{n_b}\sum_{s=1}^{n_e}\left(
      N_{ai}N^*_{ci}N_{bj}N^*_{dj}+N_{bi}N^*_{di}N_{aj}N^*_{cj}
      -N_{ai}N^*_{di}N_{bj}N^*_{cj}-N_{bi}N^*_{ci}N_{aj}N^*_{dj}\right)\right.
      \nonumber \\
  &&  \left.\left(C_{ir}\delta_{ik}C_{js}C^*_{js}+C_{is}C^*_{is}C_{jr}\delta_{jk}
      -C_{is}\delta_{ik}C_{jr}C^*_{js}-C_{ir}C^*_{is}C_{js}\delta_{jk}\right)\right] \\
  &=& \frac{1}{4}\left[\sum_{i,j,k=1}^{n_b}\sum_{s=1}^{n_e}
      \left(
      N_{ai}N^*_{ci}N_{bj}N^*_{dj}+N_{bi}N^*_{di}N_{aj}N^*_{cj}
      -N_{ai}N^*_{di}N_{bj}N^*_{cj}-N_{bi}N^*_{ci}N_{aj}N^*_{dj}\right)\right.
      \nonumber \\
  &&  \left(C_{ir}\delta_{ik}C_{js}C^*_{js}-C_{is}\delta_{ik}C_{jr}C^*_{js}\right)
      \nonumber \\
  &&  +\left(
      N_{ai}N^*_{ci}N_{bj}N^*_{dj}+N_{bi}N^*_{di}N_{aj}N^*_{cj}
      -N_{ai}N^*_{di}N_{bj}N^*_{cj}-N_{bi}N^*_{ci}N_{aj}N^*_{dj}\right)
      \nonumber \\
  &&  \left.\left(C_{is}C^*_{is}C_{jr}\delta_{jk}-C_{ir}C^*_{is}C_{js}\delta_{jk}\right)\right] \\
  &=& \frac{1}{4}\left[\sum_{i,j,k=1}^{n_b}\sum_{s=1}^{n_e}
      \left(
      N_{ai}N^*_{ci}N_{bj}N^*_{dj}+N_{bi}N^*_{di}N_{aj}N^*_{cj}
      -N_{ai}N^*_{di}N_{bj}N^*_{cj}-N_{bi}N^*_{ci}N_{aj}N^*_{dj}\right)\right.
      \nonumber \\
  &&  \left(C_{ir}\delta_{ik}C_{js}C^*_{js}-C_{is}\delta_{ik}C_{jr}C^*_{js}\right)
      \nonumber \\
  &&  +\left(
      N_{aj}N^*_{cj}N_{bi}N^*_{di}+N_{bj}N^*_{dj}N_{ai}N^*_{ci}
      -N_{aj}N^*_{dj}N_{bi}N^*_{ci}-N_{bj}N^*_{cj}N_{ai}N^*_{di}\right)
      \nonumber \\
  &&  \left.\left(C_{js}C^*_{js}C_{ir}\delta_{ik}-C_{jr}C^*_{js}C_{is}\delta_{ik}\right)\right] \\
  &=& \frac{1}{2}\left[\sum_{i,j,k=1}^{n_b}\sum_{s=1}^{n_e}
      \left(
      N_{ai}N^*_{ci}N_{bj}N^*_{dj}+N_{bi}N^*_{di}N_{aj}N^*_{cj}
      -N_{ai}N^*_{di}N_{bj}N^*_{cj}-N_{bi}N^*_{ci}N_{aj}N^*_{dj}\right)\right.
      \nonumber \\
  &&  \left.\left(C_{ir}\delta_{ik}C_{js}C^*_{js}-C_{is}\delta_{ik}C_{jr}C^*_{js}\right)\right] \\
  &=& \frac{1}{2}\left[\sum_{j,k=1}^{n_b}\sum_{s=1}^{n_e}
      \left(
      N_{ak}N^*_{ck}N_{bj}N^*_{dj}+N_{bk}N^*_{dk}N_{aj}N^*_{cj}
      -N_{ak}N^*_{dk}N_{bj}N^*_{cj}-N_{bk}N^*_{ck}N_{aj}N^*_{dj}\right)\right.
      \nonumber \\
  &&  \left.\left(C_{kr}C_{js}C^*_{js}-C_{ks}C_{jr}C^*_{js}\right)\right] \\
  &=& \frac{1}{2}\left[\sum_{j,k=1}^{n_b}\sum_{s=1}^{n_e}
      \left(
      N_{ak}N^*_{ck}N_{bj}N^*_{dj}+N_{bk}N^*_{dk}N_{aj}N^*_{cj}
      -N_{ak}N^*_{dk}N_{bj}N^*_{cj}-N_{bk}N^*_{ck}N_{aj}N^*_{dj}\right)
      C_{kr}C_{js}C^*_{js}\right.
      \nonumber \\
  && -\left.\left(
      N_{aj}N^*_{cj}N_{bk}N^*_{dk}+N_{bj}N^*_{dj}N_{ak}N^*_{ck}
      -N_{aj}N^*_{dj}N_{bk}N^*_{ck}-N_{bj}N^*_{cj}N_{ak}N^*_{dk}\right)
      C_{js}C_{kr}C^*_{ks}\right] \\
\end{eqnarray}



\section{Notes}

\subsection{Orbital updates}

In the implementation we developed previously we used a special expression that can be
expressed as follows. Let $U$ and $V$ be unitary matrices. Let $C$ be a matrix of 
correlation functions and $N$ be a matrix of natural orbitals. After an update new 
natural orbitals $N'$ are obtained as well as new correlation functions $C'$. 
Now we formulated the new matrices in terms of the old matrices as follows:
\begin{eqnarray}
   N' &=& NU \\
   C' &=& U^T C \\
   C' &=& CV
\end{eqnarray}
formulated more explicitly
\begin{eqnarray}
   N'_{aj} &=& \sum_i N_{ai}U_{ij} \\
   C'_{js} &=& \sum_i U_{ij}C_{is} \\
   C'_{js} &=& \sum_i C_{jt}V_{ts}
\end{eqnarray}
A bit of analysis shows that the coupled update of the natural orbitals and the
correlation functions is very confusing at best. Because $U$ is a unitary matrix
the two transformations essentially cancel eachother in the density matrix update.
This coupling might be different depending on the structure of a particular matrix
but even in that case it would be hard to say something about the transformation
in general. Hence let's drop the coupled natural orbital and correlation function
updates.

The one advantage of the coupled update is that it preserve the relationship
between the natural orbitals and the correlation functions. Remember that the
correlation functions are expressed in terms of the natural orbitals which are
in turn expressed in terms of the atomic orbitals. If the natural orbitals are
updated without an update to the correlation functions then the correlation
functions end up being expressed in terms of natural orbitals that no longer
exist. Clearly that situation is problematic.

It turns out that the this coupled update is a Bogoliubov transformation which is
what common implementations of Thermo Field Dynamics depend on. Given that attempts
to solve the equations without using this coupled update converge extremely 
slowly (if at all) it would seem that we need to investigate this type of
transformation. Investigating the solution for $V$ is straightforward as that is
the problem we have already solved but in a slightly different form. Solving 
for $U$ is different because we now need to differentiate an energy expression
given in terms of a matrix that simultaneously transforms both the natural orbitals
and the correlation functions. 

Given that $U$ is a unitary matrix we have that it must satisfy the following
conditions
\begin{eqnarray}
   UU^T-I &=& 0 \\
   U^TU-I &=& 0
\end{eqnarray}
differentiating these conditions we get
\begin{eqnarray}
  0 &=& \sum_{mn}\frac{\partial\sum_{j}U_{ij}U^T_{jk}-\delta_{ik}}{\partial U_{mn}} \\
  0 &=& \sum_{mn}\frac{\partial\sum_{j}U_{ij}U_{kj}-\delta_{ik}}{\partial U_{mn}} \\
  0 &=& \sum_{mn}\left(\sum_{j}\delta_{im}\delta_{jn}U_{kj}
                      +\sum_{j}U_{ij}\delta_{km}\delta_{jn}\right) \\
  0 &=& \sum_{mn}\left(\delta_{im}U_{kn}+U_{in}\delta_{km}\right) \\
  0 &=& \left(\sum_n U_{kn}+\sum_n U_{in}\right)_{ik} 
\end{eqnarray}

\subsection{Electron entropy}

In the model we have described above there is a term that depends on the entropy
of the 2-electron density matrix. Some basic experiments show that the usual from
of the entropy expression leads to some unexpected unsymmetric results. Hence we
may need different entropy expressions. In fact, as the $\alpha$-$\alpha$ and the 
$\beta$-$\beta$ blocks of the 2-electron density matrix cannot be expressed in
terms of the $\alpha$ or $\beta$ occupation numbers, we may actually find that 
the entropy expression for the $\alpha$-$\alpha$ and the $\beta$-$\beta$ blocks
has to be different from the $\alpha$-$\beta$ and $\beta$-$\alpha$ blocks.
Either a more detailed study of how entropies can be derived or some experiments
of which entropy expressions work best will be needed to come to something 
sensible. In the meantime we will have to derive equations that are general
enough that we can easily substitute alternative entropy expressions.

Instead of the entropy used above another entropy that is used is the entanglement
entropy:
\begin{eqnarray}
   S_1(\rho_i)    &=& \rho_i\ln(\rho_i) \\
   S_2(\rho_{ij}) &=& \rho_{ij}\ln(\rho_{ij}) \\
   S              &=& S_1(\rho_i)+S_1(\rho_j)-S_2(\rho_{ij})
\end{eqnarray}
In this expression $\rho_i$ and $\rho_j$ are the 1-electron orbital occupations
associated with orbitals $i$ and $j$, whereas $\rho_{ij}$ is the 2-electron
occupation number of orbital pair $ij$.

In the case of the $\alpha$-$\beta$ 2-electron density matrix block the
entanglement entropy can be simplified as:
\begin{eqnarray}
   S_1(\rho^\sigma_i)            &=& \rho^\sigma_i\ln(\rho^\sigma_i) \\
   S_2(\rho^{\alpha\beta}_{ij})  &=& \rho^\alpha_i\rho^\beta_j\ln(\rho^\alpha_i\rho^\beta_j) \\
   S(\rho^\alpha_i,\rho^\beta_j) &=& S_1(\rho^\alpha_i)+S_1(\rho^\beta_j)-S_2(\rho^\alpha_i\rho^\beta_j)
\end{eqnarray}
The derivatives of these expressions are:
\begin{eqnarray}
   \frac{\partial S_1(\rho^\sigma_i)}{\partial X}
   &=& \left[\ln(\rho^\sigma_i)+1\right]\frac{\partial\rho^\sigma_i}{\partial X} \\
   \frac{\partial S_2(\rho^\alpha_i,\rho^\beta_j)}{\partial X}
   &=& \left[\rho^\beta_j\ln(\rho^\alpha_i\rho^\beta_j)+\rho^\beta_j\right]
       \frac{\partial\rho^\alpha_i}{\partial X} \nonumber \\
   &+& \left[\rho^\alpha_j\ln(\rho^\alpha_i\rho^\beta_j)+\rho^\alpha_j\right]
       \frac{\partial\rho^\beta_i}{\partial X}
\end{eqnarray}

\subsection{Correlation model}

Overall the model for electron correlation for the $\alpha$-$\beta$ interactions
used here takes the form
\begin{eqnarray}
  E_2 &=& \sum_{ij} \langle ij | ij \rangle 
          \left(1-S(\rho^\alpha_i\rho^\beta_j)\right) \rho^\alpha_i\rho^\beta_j
\end{eqnarray}
where the factor $(1-S(\rho^\alpha,\rho^\beta))$ models the reduction in electron-electron repulsion
due to correlation, the last factor, $\rho^\alpha\rho^\beta$, is the 2-electron density matrix.

Differentiating the 2-electron energy with respect to the correlation coefficients we get
\begin{eqnarray}
  \frac{\partial E_2}{\partial C^{*\alpha}_{kr}}
  &=& \sum_{ij}\langle ij | ij \rangle \left[
          \left(1-S(\rho^\alpha_i\rho^\beta_j)\right)
          \left(\frac{\partial\rho^\alpha_i}{\partial C^{*\alpha}_{kr}}\right)\rho^\beta_j
        - \left(\frac{\partial S(\rho^\alpha_i\rho^\beta_j)}{\partial\rho^\alpha_i}\right)
          \left(\frac{\partial \rho^\alpha_i}{\partial C^{*\alpha}_{kr}}\right)
          \rho^\alpha_i\rho^\beta_j
      \right]
\end{eqnarray}

\subsection{Hydrogen molecule: comparison Full-CI and WFN1S}

\subsubsection{Scaling the 2-electron interaction}

If we take a hydrogen molecule and describe this system in a two orbital basis set then
we can the total energies. From such an comparison we should be able to work out what we need
to do to recover the correct energy. 

First of all in this simple representation there are only two orbitals which are fixed by
symmetry. These orbitals are the bonding and the anti-bonding orbitals, and assuming we 
have hydrogen atoms $A$ and $B$ these orbitals are
\begin{eqnarray}
   \phi_1(r) &=& (\chi_A(r)+\chi_B(r))/\sqrt{2} \\
   \phi_2(r) &=& (\chi_A(r)-\chi_B(r))/\sqrt{2} 
\end{eqnarray}
In terms of these orbitals we can write down the Full-CI wave function as a sum of
Slater determinants. Although, because we have only 1 $\alpha$-electron and 1 $\beta$-electron
it is actually sufficient to just list Hartree products.
\begin{eqnarray}
   \Psi &=& C_{11}\phi^\alpha_1(r_1)\phi^\beta_1(r_2)
         +  C_{12}\phi^\alpha_1(r_1)\phi^\beta_2(r_2)
         +  C_{21}\phi^\alpha_2(r_1)\phi^\beta_1(r_2)
         +  C_{22}\phi^\alpha_2(r_1)\phi^\beta_2(r_2)
\end{eqnarray}
By symmetry we have that $C_{12} = C_{21} = 0$. Furthermore because the wave function must be
normalized we have
\begin{eqnarray}
   1 &=& C_{11}^2+C_{22}^2
\end{eqnarray}
Hence $C_{22} = \pm \sqrt{1-C_{11}^2}$. Ultimately the 2-electron contribution to the total
energy can be written as $\tr(R,D)$ where $R$ is the matrix of 2-electron repulsion integrals,
$D$ is the 2-electron density matrix.
The electron repulsion matrix is
\begin{eqnarray}
   R &=& 
   \begin{pmatrix}
   \eria{1}{1}{1}{1} & \eria{1}{1}{1}{2} & \eria{1}{1}{2}{1} & \eria{1}{1}{2}{2} \\
   \eria{1}{2}{1}{1} & \eria{1}{2}{1}{2} & \eria{1}{2}{2}{1} & \eria{1}{2}{2}{2} \\
   \eria{2}{1}{1}{1} & \eria{2}{1}{1}{2} & \eria{2}{1}{2}{1} & \eria{2}{1}{2}{2} \\
   \eria{2}{2}{1}{1} & \eria{2}{2}{1}{2} & \eria{2}{2}{2}{1} & \eria{2}{2}{2}{2}
   \end{pmatrix} 
\end{eqnarray}
where $\eria{i}{j}{k}{l} = \reria{i}{j}{k}{l} = \int \phi_i^\alpha(r_1)\phi_j^\beta(r_2)\frac{1}{r_{12}}{\phi_k^\alpha}^*(r_1){\phi_l^\beta}^*(r_2)\mathrm{d}r_1\mathrm{d}r_2$.
Of course, because we are dealing with a 2-electron closed shell system electron 1 has $\alpha$-spin and
electron 2 has $\beta$-spin. I.e. $\langle i^\alpha j^\beta|k^\alpha l^\beta \rangle = 
(i^\alpha k^\alpha |j^\beta l^\beta)$.
The corresponding Full-CI 2-electron density matrix
\begin{eqnarray}
   D &=& 
   \begin{pmatrix}
   C_{11}^2     & 0 & 0 & C_{11}C_{22} \\
   0            & 0 & 0 & 0            \\
   0            & 0 & 0 & 0            \\
   C_{22}C_{11} & 0 & 0 & C_{22}^2
   \end{pmatrix}
   \label{eq:fci_h2}
\end{eqnarray}
By contrast, within the WFN1 approach the 2-electron density matrix in this case is constructed from the
1-electron density matrices. The 1-electron density matrix is constructed from the correlation functions.
The correlation functions are given by
\begin{eqnarray}
   \psi^\sigma(r_1)
        &=& c^\sigma_{1}\phi^\sigma_1(r_1)
         +  c^\sigma_{2}\phi^\sigma_2(r_1)
\end{eqnarray}
The corresponding 1-electron density matrix is
\begin{eqnarray}
   d &=& 
   \begin{pmatrix}
   \left.c_{1}^\sigma\right.^2 & 0                           \\
   0                           & \left.c_{2}^\sigma\right.^2
   \end{pmatrix} \\
   &=& 
   \begin{pmatrix}
   p_{1}^\sigma & 0            \\
   0            & p_{2}^\sigma
   \end{pmatrix}
\end{eqnarray}
and the 2-electron density matrix \textcolor{red}{This analysis is only true for closed shell systems where
the $\alpha$ and $\beta$ 1-electron density matrices are the same.}
\begin{eqnarray}
   D &=& 
   \begin{pmatrix}
   p_1^\alpha p_1^\beta  & 0                    & 0                    & 0                    \\
   0                     & p_1^\alpha p_2^\beta & 0                    & 0                    \\
   0                     & 0                    & p_2^\alpha p_1^\beta & 0                    \\
   0                     & 0                    & 0                    & p_2^\alpha p_2^\beta
   \end{pmatrix}
   \label{eq:wfn1_h2}
\end{eqnarray}
Clearly Eqs.~\ref{eq:fci_h2} and~\ref{eq:wfn1_h2} are very different except when 
$c_1^\alpha = c_1^\beta = 1$ and $C_{11} = 1$
(which implies that both $c_2^\alpha = c_2^\beta = 0$ and $C_{22} = 0$). To address this problem we 
will modify the 2-electron interaction in the WFN1 model to $\tr(R,SD)$ where the matrix $S$ is given by
\begin{eqnarray}
   S &=& 
   \begin{pmatrix}
   1            & 0                & 0                & 0            \\
   0            & 1-T(p_1^\alpha,p_2^\alpha,p_1^\beta,p_2^\beta) & 0                & 0            \\
   0            & 0                & 1-T(p_1^\alpha,p_2^\alpha,p_1^\beta,p_2^\beta) & 0            \\
   0            & 0                & 0                & 1          
   \end{pmatrix}
   \label{eq:wfn1_t}
\end{eqnarray}
and then we want to solve $\tr(R,D_{FCI})=\tr(R,SD_{WFN1})$ for $T$.
\begin{eqnarray}
  &&\langle 11|11 \rangle C_{11}^2 + 2 \langle 11|22 \rangle C_{11}C_{22} + \langle 22|22 \rangle C_{22}^2
    \nonumber \\
  &&= \langle 11|11 \rangle p_1^\alpha p_1^\beta + (1-T)\langle 12|12 \rangle p_1^\alpha p_2^\beta
    +  (1-T)\langle 21|21 \rangle p_2^\alpha p_1^\beta + \langle 22|22 \rangle p_2^\alpha p_2^\beta 
\end{eqnarray}
\begin{eqnarray}
  &&T\left(\langle 12|12 \rangle p_1^\alpha p_2^\beta+
           \langle 21|21 \rangle p_2^\alpha p_1^\beta\right) \nonumber \\
  &&=  \langle 11|11 \rangle \left(p_1^\alpha p_1^\beta-C_{11}^2\right)
    +  \langle 12|12 \rangle p_1^\alpha p_2^\beta
    +  \langle 21|21 \rangle p_2^\alpha p_1^\beta
    +  \langle 22|22 \rangle \left(p_2^\alpha p_2^\beta-C_{22}^2\right)
    - 2 \langle 11|22 \rangle C_{11}C_{22} 
\end{eqnarray}
\begin{eqnarray}
  T &=& \frac{
             \langle 11|11 \rangle \left(p_1^\alpha p_1^\beta-C_{11}^2\right)
         +   \langle 12|12 \rangle p_1^\alpha p_2^\beta
         +   \langle 21|21 \rangle p_2^\alpha p_1^\beta
         +   \langle 22|22 \rangle \left(p_2^\alpha p_2^\beta-C_{22}^2\right)
         - 2 \langle 11|22 \rangle C_{11}C_{22}
        }{
         \langle 12|12 \rangle p_1^\alpha p_2^\beta+\langle 21|21 \rangle p_2^\alpha p_1^\beta
        }
\end{eqnarray}
To use this expression in real calculations some approximations are needed
(for one, because we do not have the Full-CI wavefunction). Also note
that the denominator contains factors $p_1^\sigma p_2^{\sigma'}$ which can go to zero.
These factors do not pose a problem to the overall model as they cancel the
same factors from the 2-electron density matrix. 

For \ce{H2} in a minimal basis we have that 
\begin{eqnarray}
C_{11}^2 &=& p_{1}^\sigma \\
         &=& \sqrt{p_1^\alpha p_1^\beta} \\
         &=& \frac{1}{2}\left(p_1^\alpha + p_1^\beta\right)
\end{eqnarray}
all three options given above are equally valid for this system.
$\langle 12|12 \rangle = \langle 21|21 \rangle$, and
$C_{11}C_{22} = \pm \sqrt{c_1^2c_2^2}$. Using these relationships
we have
\begin{eqnarray}
  T &\approx& \frac{
             \langle 11|11 \rangle \left(c_1^2c_1^2-c_1^2\right)
         + 2 \langle 12|12 \rangle c_1^2c_2^2
         +   \langle 22|22 \rangle \left(c_2^2c_2^2-c_2^2\right)
         \pm 2 \langle 11|22 \rangle \sqrt{c_1^2c_2^2}
        }{
         2 \langle 12|12 \rangle c_1^2c_2^2
       } \\
  T &\approx&
         \left(
           \frac{1}{2}\frac{\langle 11|11 \rangle}{\langle 12|12 \rangle} \left(c_1^2c_1^2-c_1^2\right)
         + \frac{\langle 12|12 \rangle}{\langle 12|12 \rangle} c_1^2c_2^2
         + \frac{1}{2}\frac{\langle 22|22 \rangle}{\langle 12|12 \rangle} \left(c_2^2c_2^2-c_2^2\right)
         \pm \frac{\langle 11|22 \rangle}{\langle 12|12 \rangle} \sqrt{c_1^2c_2^2}
         \right)\left(\frac{1}{c_1^2c_2^2}\right)
\end{eqnarray}
Of course we also have that the occupation numbers $p_1$ and $p_2$ can be expressed as
$p_1 = c_1^2$ and $p_2 = c_2^2$. Substituting these relations gives
\begin{eqnarray}
  T &\approx&
         \left(
           \frac{1}{2}\frac{\langle 11|11 \rangle}{\langle 12|12 \rangle} \left(p_1^2-p_1\right)
         + p_1 p_2
         + \frac{1}{2}\frac{\langle 22|22 \rangle}{\langle 12|12 \rangle} \left(p_2^2-p_2\right)
         \pm \frac{\langle 11|22 \rangle}{\langle 12|12 \rangle} \sqrt{p_1 p_2}
         \right)\left(\frac{1}{p_1 p_2}\right)
         \label{Eq:T_raw}
\end{eqnarray}
Because we need to maximize $T$ to maximize the electron correlation to minimize the total energy,
and assuming that all 2-electron integrals are non-negative, we have that we need to select
the $+$ from the $\pm$ sign.

Note that Eq.~\ref{Eq:T_raw} is only valid for 2-electron systems with 1 $\alpha$- and 1
$\beta$-electron, represented in 2 orbitals. This means that the normalization condition
$p_1=1-p_2$ prevents both $p_1$ and $p_2$ being large (e.g. 1) at the same time. Both $p_1$ and
$p_2$ being $1$ would maximize Eq.~\ref{Eq:T_raw} whereas two fully occupied orbitals are
both uncorrelated and certainly uncorrelated with each other. Hence we need to find a way to
bake the constraint $p_1=1-p_2$ into expression Eq.~\ref{Eq:T_raw} to ensure it remains correct
for 2-electron systems, but also can be applied to systems with more electrons without 
becoming unphysical.

One trick we can play is
\begin{eqnarray}
 p_1 &=& \sqrt{p_1^2} \\
     &=& \sqrt{p_1(1-p_2)} 
\end{eqnarray}
In a scenario where we are interested in products of $p_1$ and $p_2$ this leads to
\begin{eqnarray}
  p_1p_2 &=& \sqrt{p_1^2p_2^2} \\
         &=& \sqrt{p_1(1-p_2)p_2(1-p_1)} \\
         &=& \sqrt{[p_1(1-p_1)][p_2(1-p_2)]}
\end{eqnarray}
without introducing any approximations for 2-electron systems in 2 orbitals ($p_1=1-p_2$
is an approximation for systems with more than 2 electrons and more than 2 orbitals). 
Likewise
\begin{eqnarray}
  p_1^2-p_1 &=& -(p_1-p_1^2) \\
            &=& -\sqrt{(p_1-p_1^2)^2} \\
            &=& -\sqrt{(p_1-p_1^2)(1-p_2-[1-p_2]^2)} \\
            &=& -\sqrt{(p_1-p_1^2)(1-p_2-[1-2p_2+p_2^2])} \\
            &=& -\sqrt{(p_1-p_1^2)(p_2-p_2^2)} 
\end{eqnarray}

\subsubsection{Subtracting the 2-electron interaction}

A different approach is to compare the 2-electron interaction as
\begin{eqnarray}
   T &=& E_{FullCI}-E_{WFN1S} \\
   E_{FullCI} &=& E_{WFN1S}+T
\end{eqnarray}
Then we have
\begin{eqnarray}
   E_{FullCI} &=& \eria{1}{1}{1}{1} C_{11}^2 + \eria{1}{1}{2}{2} C_{11}C_{22}^*
               +  \eria{2}{2}{1}{1} C_{22}C_{11}^* + \eria{2}{2}{2}{2} C_{22}^2 \\
   E_{WFN1S}  &=& \eria{1}{1}{1}{1} p_1^\alpha p_1^\beta + \eria{1}{2}{1}{2} p_1^\alpha p_2^\beta
               +  \eria{2}{1}{2}{1} p_2^\alpha p_1^\beta + \eria{2}{2}{2}{2} p_2^\alpha p_2^\beta \\
   T
   &=& \eria{1}{1}{1}{1} (C_{11}^2-p_1^\alpha p_1^\beta)
    +  \eria{2}{2}{2}{2} (C_{22}^2-p_2^\alpha p_2^\beta) \nonumber \\
   &&+ \eria{1}{1}{2}{2} C_{11}C_{22}^*
    +  \eria{2}{2}{1}{1} C_{22}C_{11}^*       \nonumber \\
   &&- \eria{1}{2}{1}{2} p_1^\alpha p_2^\beta
    -  \eria{2}{1}{2}{1} p_2^\alpha p_1^\beta 
   \label{Eq:T_subtract}
\end{eqnarray}
As before we can equate the Full-CI coefficients to
\begin{eqnarray}
C_{11}^2 &=& p_{1}^\sigma \\
         &=& \sqrt{p_1^\alpha p_1^\beta} \\
         &=& \frac{p_1^\alpha + p_1^\beta}{2}
\end{eqnarray}
Based on this we also have that
\begin{eqnarray}
C_{11} &=& \pm\sqrt{p_{1}^\sigma} \\
       &=& \pm\sqrt[4]{p_1^\alpha p_1^\beta} \\
       &=& \pm\sqrt{\frac{p_1^\alpha + p_1^\beta}{2}}
\end{eqnarray}
Obviously in Eq.~\ref{Eq:T_subtract} we have to choose the sign of $C_{11}C_{22}$ so as to minimize
the Full-CI energy. If we now insert the geometric mean approach into Eq.~\ref{Eq:T_subtract} we have
\begin{eqnarray}
   T
   &=& \eria{1}{1}{1}{1} (\sqrt{p_1^\alpha p_1^\beta}-p_1^\alpha p_1^\beta)
    +  \eria{2}{2}{2}{2} (\sqrt{p_2^\alpha p_2^\beta}-p_2^\alpha p_2^\beta) \nonumber \\
   &&- \eria{1}{1}{2}{2} \sqrt[4]{p_1^\alpha p_1^\beta p_2^\alpha p_2^\beta}
    -  \eria{2}{2}{1}{1} \sqrt[4]{p_2^\alpha p_2^\beta p_1^\alpha p_1^\beta} \nonumber \\
   &&- \eria{1}{2}{1}{2} p_1^\alpha p_2^\beta
    -  \eria{2}{1}{2}{1} p_2^\alpha p_1^\beta 
   \label{Eq:T_subtract_2}
\end{eqnarray}
Next we need to test what happens if one of the orbitals is not correlated. I.e. $T$ should be $0$ for all
cases listed in Table~\ref{Table:T_subtract_occ}.
\begin{table}
\begin{tabular}{cccc|c}
\hline
$p_1^\alpha$ & $p_1^\beta$ & $p_2^\alpha$ & $p_2^\beta$ & $T$ \\
\hline
0            & 0           & 0            & 0           &  0  \\
0            & 0           & 0            & 1           &  0  \\
0            & 0           & 1            & 0           &  0  \\
0            & 1           & 0            & 0           &  0  \\
1            & 0           & 0            & 0           &  0  \\
0            & 0           & 1            & 1           &  0  \\
0            & 1           & 0            & 1           &  0  \\
1            & 0           & 0            & 1           &  $T_5$   \\
0            & 1           & 1            & 0           &  $T_6$   \\
1            & 0           & 1            & 0           &  0  \\
1            & 1           & 0            & 0           &  0  \\
0            & 1           & 1            & 1           &  $T_6$   \\
1            & 0           & 1            & 1           &  $T_5$   \\
1            & 1           & 0            & 1           &  $T_5$   \\
1            & 1           & 1            & 0           &  $T_6$   \\
1            & 1           & 1            & 1           &  $T_{3-6}$  \\
\hline
\end{tabular}
\caption{Occupation numbers for which $T$ must be zero. The column $T$ lists
         the terms of Eq.~\ref{Eq:T_subtract_2} that remain non-zero.}
\label{Table:T_subtract_occ}
\end{table}
Note that the terms $T_1$ and $T_2$ are not problematic. Both these terms are actually non-negative 
and therefore raise the energy instead of lowering it. Hence minimizing the total energy will tend
to miminize these terms. Therefore the important terms to deal with are the terms $T_3$ to $T_6$.
To address these terms there are two important relationships that hold for the closed shell 
2-electron molecules in a two orbital basis:
\begin{eqnarray}
  p_i^\sigma &=& 1-p_j^\sigma \\
  p_i^\sigma &=& p_i^{\sigma'}
\end{eqnarray}
Using these relationships one can replace the following quantity as
\begin{eqnarray}
  p_1^\alpha p_2^\beta 
  &=& \sqrt{p_1^\alpha p_1^\alpha p_2^\beta p_2^\beta} \\
  &=& \sqrt{p_1^\alpha(1-p_2^\alpha)p_2^\beta(1-p_1^\beta)} \\
  &=& \sqrt{p_1^\alpha(1-p_2^\beta)p_2^\beta(1-p_1^\alpha)} \\
  &=& \sqrt{p_1^\alpha(1-p_1^\alpha)p_2^\beta(1-p_2^\beta)}
\end{eqnarray}
likewise
\begin{eqnarray}
  p_1^\beta p_2^\alpha 
  &=& \sqrt{p_1^\beta(1-p_1^\beta)p_2^\alpha(1-p_2^\alpha)}
\end{eqnarray}
and
\begin{eqnarray}
  \sqrt[4]{p_1^\alpha p_1^\beta p_2^\alpha p_2^\beta}
  &=& \sqrt[4]{p_1^\alpha (1-p_2^\beta)(1-p_1^\alpha)p_2^\beta} \\
  &=& \sqrt[4]{p_1^\alpha (1-p_1^\alpha) p_2^\beta (1-p_2^\beta)} \\
  &=& \sqrt[4]{p_1^\beta (1-p_1^\beta) p_2^\alpha (1-p_2^\alpha)}
\end{eqnarray}
Using the relationships above in the expression for $T$ we may obtain
\begin{eqnarray}
   T
   &=& \eria{1}{1}{1}{1} \left(\sqrt{p_1^\alpha p_1^\beta}-p_1^\alpha p_1^\beta\right)
    +  \eria{2}{2}{2}{2} \left(\sqrt{p_2^\alpha p_2^\beta}-p_2^\alpha p_2^\beta\right) \nonumber \\
   &&- \eria{1}{1}{2}{2} \sqrt[4]{p_1^\alpha (1-p_1^\alpha) p_2^\beta (1-p_2^\beta)}
    -  \eria{2}{2}{1}{1} \sqrt[4]{p_2^\alpha (1-p_2^\alpha) p_1^\beta (1-p_1^\beta)} \nonumber \\
   &&- \eria{1}{2}{1}{2} \sqrt{p_1^\alpha (1-p_1^\alpha) p_2^\beta (1-p_2^\beta)}
    -  \eria{2}{1}{2}{1} \sqrt{p_2^\alpha (1-p_2^\alpha) p_1^\beta (1-p_1^\beta)}
   \label{Eq:T_subtract_3}
\end{eqnarray}

An important question related to extending this approximation to system with more
electrons or more basis functions is how to do this. In the system studied thus
far we have assumed only 2 spatial orbitals. In any realistic system there will likely
be more orbitals. The fact that thus far we have only considered 2 orbital systems
imposes a balance between the different terms in $T$. For example, in a 2 orbital system
there are equally many $ii$ as $ij$ terms. For a system with more than two orbitals the
number of $ij$ terms goes as $N^2$ whereas the number of $ii$ terms goes as $N$. 

To deal with the situation sketched above we break $T$ apart into different terms.
First of all there are the diagonal terms $ii$:
\begin{eqnarray}
   T_{ii} &=&  \eria{i}{i}{i}{i} \left(\sqrt{p_i^\alpha p_i^\beta}-p_i^\alpha p_i^\beta\right)
\end{eqnarray}
then there are the $ij$ terms where $i \ne j$:
\begin{eqnarray}
   T_{ij} &=&  -\eria{i}{i}{j}{j} \sqrt[4]{p_i^\alpha(1-p_i^\alpha) p_j^\beta(1-p_j^\beta)} \nonumber \\
          &-& \eria{i}{j}{i}{j} \sqrt{p_i^\alpha(1-p_i^\alpha) p_j^\beta(1-p_j^\beta)}
\end{eqnarray}

Next we need to consider the derivatives of the expressions for $T_{ii}$ and $T_{ij}$. 
The occupation numbers featuring in these expressions depend on the correlation functions only,
whereas the 2-electron integrals appearing in these expressions depend on the natural orbitals.
For $T_{ii}$ we need to consider 2 $\alpha$-electron derivative expressions:
\begin{eqnarray}
   \frac{\partial T_{ii}}{\partial \left.N_{ek}^\alpha\right.^*}
   &=& \left(\sqrt{p_i^\alpha p_i^\beta}-p_i^\alpha p_i^\beta\right)
       \frac{\partial \eria{i}{i}{i}{i}}{\partial \left.N_{ek}^\alpha\right.^*} \\
   &=& \left(\sqrt{p_i^\alpha p_i^\beta}-p_i^\alpha p_i^\beta\right)
       \sum_{a,b,c,d}\eria{a}{b}{c}{d}
       \frac{\partial N_{ai}^\alpha N_{ci}^{\alpha*}N_{bi}^\beta N_{di}^{\beta*}}
            {\partial \left.N_{ek}^\alpha\right.^*} \\
   &=& \left(\sqrt{p_i^\alpha p_i^\beta}-p_i^\alpha p_i^\beta\right)
       \sum_{a,b,c,d}\eria{a}{b}{c}{d}
       N_{ai}^\alpha N_{bi}^\beta N_{di}^{\beta*}\delta_{ce}\delta_{ik} \\
   &=& \left(\sqrt{p_k^\alpha p_k^\beta}-p_k^\alpha p_k^\beta\right)
       \sum_{a,b,d}\eria{a}{b}{e}{d}
       N_{ak}^\alpha N_{bk}^\beta N_{dk}^{\beta*}\delta_{ik} \\
   \frac{\partial T_{ii}}{\partial \left.C_{kr}^\alpha\right.^*}
   &=& \eria{i}{i}{i}{i}
       \frac{\partial\sqrt{p_i^\alpha p_i^\beta}-p_i^\alpha p_i^\beta}{\partial p_i^\alpha}
       \frac{\partial p_i^\alpha}{\partial\left.C_{kr}^\alpha\right.^*} \\
   &=& \eria{i}{i}{i}{i}
       \frac{\partial\sqrt{p_i^\alpha p_i^\beta}-p_i^\alpha p_i^\beta}{\partial p_i^\alpha}
       \frac{\partial \sum_s C_{is}^{\alpha}C_{is}^{\alpha*}}
            {\partial\left.C_{kr}^\alpha\right.^*} \\
   &=& \eria{i}{i}{i}{i}
       \frac{\partial\sqrt{p_i^\alpha p_i^\beta}-p_i^\alpha p_i^\beta}{\partial p_i^\alpha}
       \frac{\partial \sum_s C_{is}^{\alpha}C_{is}^{\alpha*}}
            {\partial\left.C_{kr}^\alpha\right.^*} \\
   &=& \eria{i}{i}{i}{i}
       \left(\frac{1}{2}\sqrt{\frac{p_i^\beta}{p_i^\alpha}}-p_i^\beta\right)
       C_{ir}^{\alpha}\delta_{ik}
\end{eqnarray}
and the 2 $\beta$-electron counter parts.
For $T_{ij}$ we get
\begin{eqnarray}
   \frac{\partial T_{ij} }{\partial \left.N_{ek}^\alpha\right.^*}
   &=&  -\frac{\partial\eria{i}{i}{j}{j}}{\partial \left.N_{ek}^\alpha\right.^*}
         \sqrt[4]{p_i^\alpha(1-p_i^\alpha) p_j^\beta(1-p_j^\beta)} \nonumber \\
   &&   -\frac{\partial\eria{i}{j}{i}{j}}{\partial \left.N_{ek}^\alpha\right.^*}
         \sqrt{p_i^\alpha(1-p_i^\alpha) p_j^\beta(1-p_j^\beta)} \\
   &=&  -\sum_{a,b,c,d}\eria{a}{b}{c}{d}
         \frac{\partial N_{ai}^\alpha N_{cj}^{\alpha*}N_{bi}^\beta N_{dj}^{\beta*}}
              {\partial \left.N_{ek}^\alpha\right.^*}
         \sqrt[4]{p_i^\alpha(1-p_i^\alpha) p_j^\beta(1-p_j^\beta)} \nonumber \\
   &&   -\sum_{a,b,c,d}\eria{a}{b}{c}{d}
         \frac{\partial N_{ai}^\alpha N_{ci}^{\alpha*}N_{bj}^\beta N_{dj}^{\beta*}}
              {\partial \left.N_{ek}^\alpha\right.^*}
         \sqrt{p_i^\alpha(1-p_i^\alpha) p_j^\beta(1-p_j^\beta)} \\
   &=&  -\sum_{a,b,c,d}\eria{a}{b}{c}{d}
         N_{ai}^\alpha N_{bi}^\beta N_{dj}^{\beta*}\delta_{ce}\delta_{jk}
         \sqrt[4]{p_i^\alpha(1-p_i^\alpha) p_j^\beta(1-p_j^\beta)} \nonumber \\
   &&   -\sum_{a,b,c,d}\eria{a}{b}{c}{d}
         N_{ai}^\alpha N_{bj}^\beta N_{dj}^{\beta*}\delta_{ce}\delta_{ik}
         \sqrt{p_i^\alpha(1-p_i^\alpha) p_j^\beta(1-p_j^\beta)} \\
   &=&  -\sum_{a,b,d}\eria{a}{b}{e}{d}
         N_{ai}^\alpha N_{bi}^\beta N_{dj}^{\beta*}\delta_{jk}
         \sqrt[4]{p_i^\alpha(1-p_i^\alpha) p_j^\beta(1-p_j^\beta)} \nonumber \\
   &&   -\sum_{a,b,d}\eria{a}{b}{e}{d}
         N_{ai}^\alpha N_{bj}^\beta N_{dj}^{\beta*}\delta_{ik}
         \sqrt{p_i^\alpha(1-p_i^\alpha) p_j^\beta(1-p_j^\beta)} \\
   \frac{\partial T_{ij}}{\partial \left.C_{kr}^\alpha\right.^*}
   &=&  -\eria{i}{i}{j}{j} 
         \frac{\partial\sqrt[4]{p_i^\alpha(1-p_i^\alpha) p_j^\beta(1-p_j^\beta)}}
              {\partial p_i^\alpha}
         \frac{\partial p_i^\alpha}
              {\partial \left.C_{kr}^\alpha\right.^*}
         \nonumber \\
   &&   -\eria{i}{j}{i}{j}
         \frac{\sqrt{p_i^\alpha(1-p_i^\alpha) p_j^\beta(1-p_j^\beta)}}
              {\partial p_i^\alpha}
         \frac{\partial p_i^\alpha}
              {\partial \left.C_{kr}^\alpha\right.^*} \\
   &=&  -\frac{1}{4}\eria{i}{i}{j}{j} 
         \left[
             \sqrt[4]{\frac{1-p_i^\alpha}{\left(p_i^\alpha\right)^3}p_j^\beta(1-p_j^\beta)}
            -\sqrt[4]{\frac{p_i^\alpha}{\left(1-p_i^\alpha\right)^3}p_j^\beta(1-p_j^\beta)}
         \right]
         C_{ir}^{\alpha}\delta_{ik}
         \nonumber \\
   &&   -\frac{1}{2}\eria{i}{j}{i}{j}
         \left[
             \sqrt{\frac{1-p_i^\alpha}{p_i^\alpha}p_j^\beta(1-p_j^\beta)}
            -\sqrt{\frac{p_i^\alpha}{1-p_i^\alpha}p_j^\beta(1-p_j^\beta)}
         \right]
         C_{ir}^{\alpha}\delta_{ik} \\
\end{eqnarray}

{\bf WRONG:}
A problem with the equations given above is the cost of evaluating them. The $T_{ij}$ expressions
could be expressed in terms of $\alpha$ and $\beta$ 1-electron density matrix functions if the
$i = j$ were included. Trying to skip the $i = j$ terms requires an explicit loop over
both $i$ and $j$ driving the complexity of the algorithm to $O(N^6)$. We can reduce this 
complexity by including $i = j$ terms in $T_{ij}$ but then subtracting them again in the
$T'_{ii}$ terms. Here we have renamed $T_{ii}$ to $T'_{ii}$ to reflect the fact the total 
correlation energy no longer is $E=\sum_{ij}T_{ij}$ with the old definition of $T$ but now
$E=\sum_{ij}T_{ij}+\sum_i T'_{ii}$.
Then we can express the $T_{ij}$ terms as products of 1-electron density 
matrix functions. The $T'_{ii}$ contributions do require an explicit loop over $i$ (but not $j$)
and therefore have a complexity of $O(N^5)$.

With the above in mind the $T_{ij}$ expression becomes:
\begin{eqnarray}
   \forall_{i,j}\;
   T_{ij} &=&  -\eria{i}{i}{j}{j} \sqrt[4]{p_i^\alpha(1-p_i^\alpha) p_j^\beta(1-p_j^\beta)} \nonumber \\
          &&   -\eria{i}{j}{i}{j} \sqrt{p_i^\alpha(1-p_i^\alpha) p_j^\beta(1-p_j^\beta)}
\end{eqnarray}
whereas 
\begin{eqnarray}
  T'_{ii} &=&  \eria{i}{i}{i}{i} \left(\sqrt{p_i^\alpha p_i^\beta}-p_i^\alpha p_i^\beta\right)
               \nonumber \\
          &&  +\eria{i}{i}{i}{i} \sqrt[4]{p_i^\alpha(1-p_i^\alpha) p_i^\beta(1-p_i^\beta)}
               \nonumber \\
          &&  +\eria{i}{i}{i}{i} \sqrt{p_i^\alpha(1-p_i^\alpha) p_i^\beta(1-p_i^\beta)}
\end{eqnarray}

{\bf EXPLANATION:}
The idea outlined above is wrong because the first term of $T_{ij}$ contains integrals
that involve the electron "density" $\phi_i^\alpha\phi_j^{\alpha*}$ and 
$\phi_i^\beta\phi_j^{\beta*}$. These "densities" couple $i$ and $j$ factors meaning that
$\sqrt[4]{p_i^\alpha(1-p_i^\alpha)p_j^\beta(1-p_j^\beta)}$ cannot be split into 
factors $\sqrt[4]{p_i^\alpha(1-p_i^\alpha)}$ and $\sqrt[4]{p_j^\beta(1-p_j^\beta)}$.
As a result we always have to have an $O(N^6)$ loop unless we are willing to drop
the first term in $T_{ij}$ altogether. Whether we can afford to do this accuracy wise, is 
a big question.

{\bf WRONG AGAIN:}
We can still formulate this model at an $O(N^5)$ cost. Consider that we have two
types of terms. First there are the $T_{ii}$ terms and second are the $T_{ij}, i \ne j$
terms.

The $T_{ii}$ terms are given by 
\begin{eqnarray}
  T_{ii} &=& \eria{i}{i}{i}{i}\left(\sqrt{p_i^\alpha p_i^\beta}-p_i^\alpha p_i^\beta\right) 
\end{eqnarray}
To evaluate $T_{ii}$ at a cost of $O(N^5)$ we need to approach this like a 4-index 
transformation
\begin{eqnarray}
  T_{abci}^\beta         &=& \sum_d \eria{a}{b}{c}{d}N_{di}^{\beta *} \\
  T_{aici}^\beta         &=& \sum_b T_{abci}^\beta N_{bi}^{\beta} \\
  T_{aiii}^{\alpha\beta} &=& \sum_c T_{aici}^\beta N_{ci}^{\alpha *} \\
  T_{iiii}^{\alpha\beta} &=& \sum_a T_{aiii}^{\alpha\beta} N_{ai}^{\alpha} \\
  T_{ii}                 &=& T_{iiii}^{\alpha\beta}
                             \left(\sqrt{p_i^\alpha p_i^\beta}-p_i^\alpha p_i^\beta\right) 
\end{eqnarray}
The first transformation step is clearly an $O(N^5)$ step. After that the order of the
complexity goes down as there are increasingly fewer different indeces left.

The $T_{ij}$ terms need to be approached in a similar manner. However, because the
transformed integrals in the first and second term are different in nature we have
to evaluate these terms separately. 
The $T_{ij}$ expression is given by
\begin{eqnarray}
   T_{ij} &=&  -\eria{i}{i}{j}{j} \sqrt[4]{p_i^\alpha(1-p_i^\alpha) p_j^\beta(1-p_j^\beta)} \nonumber \\
          &&   -\eria{i}{j}{i}{j} \sqrt{p_i^\alpha(1-p_i^\alpha) p_j^\beta(1-p_j^\beta)}
\end{eqnarray}
To separate the terms we rewrite this expression as
\begin{eqnarray}
   T_{ij}       &=& -T_{ij}^{(4)} - T_{ij}^{(2)} \\
   T_{ij}^{(4)} &=& \eria{i}{i}{j}{j} \sqrt[4]{p_i^\alpha(1-p_i^\alpha) p_j^\beta(1-p_j^\beta)} \\
   T_{ij}^{(2)} &=& \eria{i}{j}{i}{j} \sqrt{p_i^\alpha(1-p_i^\alpha) p_j^\beta(1-p_j^\beta)}
\end{eqnarray}
where we have labelled the terms with the order of the roots involved. 
Analogous to the $T_{ii}$ terms we have for the first term of $T_{ij}$
\begin{eqnarray}
  T_{abcj}^{\beta(4)}       &=& \sum_d \eria{a}{b}{c}{d}   N_{dj}^{\beta *} \\
  T_{aicj}^{\beta(4)}       &=& \sum_b T_{abcj}^{\beta(4)} N_{bi}^{\beta} \\
  T_{aijj}^{\alpha\beta(4)} &=& \sum_c T_{aicj}^{\beta(4)} N_{cj}^{\alpha *} \\
  T_{iijj}^{\alpha\beta(4)} &=& \sum_a T_{aijj}^{\alpha\beta(4)} N_{ai}^{\alpha} \\
  T_{ij}^{(4)}              &=& T_{iijj}^{\alpha\beta(4)}
                                \sqrt[4]{p_i^\alpha(1-p_i^\alpha) p_j^\beta(1-p_j^\beta)}
\end{eqnarray}
for the second terms we have
\begin{eqnarray}
  T_{abcj}^{\beta(2)}       &=& \sum_d \eria{a}{b}{c}{d}   N_{dj}^{\beta *} \\
  T_{ajcj}^{\beta(2)}       &=& \sum_b T_{abcj}^{\beta(2)} N_{bj}^{\beta} \\
  T_{ajij}^{\alpha\beta(2)} &=& \sum_c T_{ajcj}^{\beta(2)} N_{ci}^{\alpha *} \\
  T_{ijij}^{\alpha\beta(2)} &=& \sum_a T_{ajij}^{\alpha\beta(2)} N_{ai}^{\alpha} \\
  T_{ij}^{(2)}              &=& T_{ijij}^{\alpha\beta(2)}
                                \sqrt{p_i^\alpha(1-p_i^\alpha) p_j^\beta(1-p_j^\beta)}
\end{eqnarray}

To optimize the energy we need derivative of the energy expression with respect to the
natural orbital coefficients and with respect to the correlation function coefficients.
Starting with the occupation number derivatives with respect to the correlation 
function coefficients we can look at the various factors.
First of all the occupation number of a particular natural orbital is given by
\begin{eqnarray}
   p_i^\alpha &=& \sum_{s=1}^{n_e}C_{is}^\alpha C_{is}^{*\alpha}
\end{eqnarray}
Differentiating this with respect to $C_{kr}^{*\alpha}$ we get
\begin{eqnarray}
   \frac{\partial p_i^\alpha}{\partial C_{kr}^{*\alpha}}
   &=& \sum_{s=1}^{n_e}C_{is}^\alpha \delta_{ki}\delta_{rs} \\
   &=& \left\{
       \begin{array}{l}
         C_{kr}^\alpha, r \le n_e \\
         0, r > n_e
       \end{array}
       \right.
\end{eqnarray}
For the square root of the occupation number we have
\begin{eqnarray}
   \frac{\partial \sqrt{p_i^\alpha}}{\partial C_{kr}^{*\alpha}}
   &=& \frac{\partial \sqrt{p_i^\alpha}}{\partial p_i^\alpha}
       \frac{\partial p_i^\alpha}{\partial C_{kr}^{*\alpha}} \\
   &=& \frac{1}{2\sqrt{p_i^\alpha}}\frac{\partial p_i^\alpha}{\partial C_{kr}^{*\alpha}}
\end{eqnarray}
For the factors in the $T_{ij}^{(2)}$ terms we have
\begin{eqnarray}
   \frac{\partial \sqrt{p_i^\alpha(1-p_i^\alpha)}}{\partial C_{kr}^{*\alpha}}
   &=& \frac{\partial \sqrt{p_i^\alpha(1-p_i^\alpha)}}{\partial p_i^\alpha}
       \frac{\partial p_i^\alpha}{\partial C_{kr}^{*\alpha}} \\
   &=& \frac{(1-p_i^\alpha)-p_i^\alpha}{2\sqrt{p_i^\alpha(1-p_i^\alpha)}}
       \frac{\partial p_i^\alpha}{\partial C_{kr}^{*\alpha}} \\
   &=& \frac{1-2p_i^\alpha}{2\sqrt{p_i^\alpha(1-p_i^\alpha)}}
       \frac{\partial p_i^\alpha}{\partial C_{kr}^{*\alpha}}
\end{eqnarray}
For the factors in the $T_{ij}^{(4)}$ terms we have
\begin{eqnarray}
   \frac{\partial \sqrt[4]{p_i^\alpha(1-p_i^\alpha)}}{\partial C_{kr}^{*\alpha}}
   &=& \frac{\partial \sqrt[4]{p_i^\alpha(1-p_i^\alpha)}}{\partial p_i^\alpha}
       \frac{\partial p_i^\alpha}{\partial C_{kr}^{*\alpha}} \\
   &=& \frac{(1-p_i^\alpha)-p_i^\alpha}{4\sqrt[4]{p_i^\alpha(1-p_i^\alpha)}^3}
       \frac{\partial p_i^\alpha}{\partial C_{kr}^{*\alpha}} \\
   &=& \frac{1-2p_i^\alpha}{4\sqrt[4]{p_i^\alpha(1-p_i^\alpha)}^3}
       \frac{\partial p_i^\alpha}{\partial C_{kr}^{*\alpha}}
\end{eqnarray}

Next we need to consider the derivatives with respect to the natural orbital
coefficients. In these derivative expressions the factors depending on the occupation
numbers are constants. However, the indeces that feature in them are relevant
because they are affected by differentiation through the delta-functions that are
introduced. Hence, we first name these constants as
\begin{eqnarray}
   P_{ii}
   &=& P_{i^\alpha i^\beta} \\
   &=& \sqrt{p_i^\alpha p_i^\beta} - p_i^\alpha p_i^\beta \\
   P^{(2)}_{ij}
   &=& P^{(2)}_{i^\alpha j^\beta} \\
   &=& \sqrt{p_i^\alpha(1-p_i^\alpha)p_j^\beta(1-p_j^\beta)} \\
   P^{(4)}_{ij}
   &=& P^{(4)}_{i^\alpha j^\beta} \\
   &=& \sqrt[4]{p_i^\alpha(1-p_i^\alpha)p_j^\beta(1-p_j^\beta)} \\
\end{eqnarray}
With the expressions above we can rephrase the energy terms as
\begin{eqnarray}
  T_{ii} 
  &=& \eria{i}{i}{i}{i}P_{ii} \\
  T_{ij}
  &=& -\eria{i}{i}{j}{j}P^{(4)}_{ij} \nonumber \\
  &&  -\eria{i}{j}{i}{j}P^{(2)}_{ij} \\
  &=& -T^{(4)}_{ij}-T^{(2)}_{ij}
\end{eqnarray}
Next we differentiate the energy expressions with respect to the $\alpha$-electron
natural orbital coefficients. The derivatives with respect to the $\beta$-electron
natural orbital coefficients are the same just with $\alpha$ and $\beta$ interchanged.
Starting with $T_{ii}$ we have
\begin{eqnarray}
   \frac{\partial T_{ii}}{\partial N^{\alpha*}_{ek}}
   &=& \sum_c \eria{i}{i}{c}{i}P_{ii}
       \frac{\partial N^{\alpha*}_{ci}}{\partial N^{\alpha*}_{ek}} \\
   &=& \sum_c \eria{i}{i}{c}{i}P_{ii}\delta_{ce}\delta_{ik} \\
   &=& \eria{k}{k}{e}{k}P_{kk}
\end{eqnarray}
Likewise, differentiating $T^{(2)}_{ij}$ gives
\begin{eqnarray}
   \frac{\partial T^{(2)}_{ij}}{\partial N^{\alpha*}_{ek}}
   &=& \sum_c\eria{i}{j}{c}{j}P^{(2)}_{ij}
       \frac{\partial N^{\alpha*}_{ci}}{\partial N^{\alpha*}_{ek}} \\
   &=& \sum_c\eria{i}{j}{c}{j}P^{(2)}_{ij}\delta_{ce}\delta_{ik} \\
   &=& \eria{k}{j}{e}{j}P^{(2)}_{kj}
\end{eqnarray}
Finally, differentiating $T^{(4)}_{ij}$ gives
\begin{eqnarray}
   \frac{\partial T^{(4)}_{ij}}{\partial N^{\alpha*}_{ek}}
   &=& \sum_c\eria{i}{i}{c}{j}P^{(4)}_{ij}
       \frac{\partial N^{\alpha*}_{cj}}{\partial N^{\alpha*}_{ek}} \\
   &=& \sum_c\eria{i}{i}{c}{j}P^{(4)}_{ij}\delta_{ce}\delta_{jk} \\
   &=& \eria{i}{i}{e}{k}P^{(4)}_{ik}
\end{eqnarray}
Note that ultimately we want Fock matrices in the natural orbitals basis
for the natural orbital optimization. For this reason we transform the remaining
atomic orbital index to natural orbital index $l$ as in
\begin{eqnarray}
   F^{(ii)}_{kl}
   &=& \sum_e\eria{k}{k}{e}{k}P_{kk}N^{\alpha*}_{el} \\
   F^{(2,ij)}_{kl}
   &=& \sum_e\eria{k}{j}{e}{j}P^{(2)}_{kj}N^{\alpha*}_{el}
       \label{eq:F2ijkl} \\
   F^{(4,ij)}_{kl}
   &=& \sum_e\eria{i}{i}{e}{k}P^{(4)}_{ik}N^{\alpha*}_{el}
       \label{eq:F4ijkl} \\
\end{eqnarray}
Note that for the full Fock matrices of Eq.~\ref{eq:F2ijkl} and~\ref{eq:F4ijkl}
we still need to sum over $j \ne k$ and $i \ne k$ respectively.






\subsection{Hydrogen 4 molecule: comparison Full-CI and WFN1S}

If we place 4 hydrogen atoms represented with a minimal basis set in a square
we have another system with potentially strong correlation effects. In this case
we have 4 electrons, 2 $\alpha$-electrons and 2 $\beta$-electrons, and we have 4
orbitals. This makes for $4!/(2!2!)=6$ determinants per spin channel, and therefore
36 determinants in total. The orbitals can be depicted as show in Table~\ref{table:h4}.
\begin{table}
\begin{tabular}{cccc}
$\phi_1$ & $\phi_2$ & $\phi_3$ & $\phi_4$ \\
$\begin{array}{cc}
 + & + \\
 + & + \\
\end{array}$
&
$\begin{array}{cc}
 - & - \\
 + & + \\
\end{array}$
&
$\begin{array}{cc}
 - & + \\
 - & + \\
\end{array}$
&
$\begin{array}{cc}
 + & - \\
 - & + \\
\end{array}$
\end{tabular}
\label{table:h4}
\end{table}
The molecule has 4 mirror planes (2 running along the diagonal through 2 hydrogen atoms, and 2 
that run between the hydrogen atoms). The latter 2 mirror planes most clearly show that all
orbitals have a different symmetry. However, we can easily construct some kind of multiplication
table of the orbital symmetries.
\begin{table}
\begin{tabular}{cc|c}
\hline
$\phi_i$ & $\phi_j$ & $\phi_i*\phi_j=\phi_k$ \\
\hline
$\phi_1$ & $\phi_1$ & $\phi_1$ \\
$\phi_1$ & $\phi_2$ & $\phi_2$ \\
$\phi_1$ & $\phi_3$ & $\phi_3$ \\
$\phi_1$ & $\phi_4$ & $\phi_4$ \\
$\phi_2$ & $\phi_2$ & $\phi_1$ \\
$\phi_2$ & $\phi_3$ & $\phi_4$ \\
$\phi_2$ & $\phi_4$ & $\phi_3$ \\
$\phi_3$ & $\phi_3$ & $\phi_1$ \\
$\phi_3$ & $\phi_4$ & $\phi_2$ \\
$\phi_4$ & $\phi_4$ & $\phi_1$ \\
\hline
\end{tabular}
\end{table}
Hence we have the following Slater determinants 
contributing to the overall wavefunction:
$|12||12|$, $|13||13|$, $|14||14|$, $|23||23|$, $|24||24|$, $|34||34|$,
$|23||14|$, $|14||23|$, $|12||34|$, $|34||12|$, $|13||24|$, $|24||13|$
other determinants being of the wrong symmetry.


\bibliographystyle{unsrt}
\bibliography{WFN1,WFN1_alt}

\end{document}
